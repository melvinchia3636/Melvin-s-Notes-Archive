% chktex-file 29
% chktex-file 13
\documentclass{report}
\usepackage{setspace}
\usepackage[a4paper, total={7in, 9in}]{geometry}
\usepackage[fleqn]{amsmath}
\usepackage{empheq}
\usepackage{amssymb}
\usepackage{amsthm}
\usepackage{gensymb}
\usepackage[fleqn]{cases}
\usepackage{multicol}
\usepackage{color}
\usepackage{stix}
\usepackage{chngcntr}
\usepackage{tikz}
\usetikzlibrary{calc,matrix}

\counterwithout{equation}{chapter}
\setlength{\columnseprule}{1pt}
\setlength{\columnsep}{24pt}
\setcounter{chapter}{11}
\hfuzz=100pt

\newtheorem{theorem}{Theorem}

\begin{document}
\newcommand{\sol}[1]{

    \noindent \textbf{Sol.}
}
\newcommand{\prooff}[1]{

    \noindent \textbf{Proof.}
}
\newenvironment{amatrix}[1]{%
    \left(\begin{array}{@{}*{#1}{c}|c@{}}
        }{%
    \end{array}\right)
}
\begin{titlepage}
    \raggedleft{}
    \rule{1pt}{\textheight}
    \hspace{0.02\textwidth}
    \parbox[b]{0.75\textwidth}{

    {\Huge\bfseries Solution Book of \\[0.5\baselineskip] Mathematic}\\[2\baselineskip]
    {\large\textit{Ssnior 2 Part I}}\\[4\baselineskip]
    {\Large\textsc{MELVIN CHIA}}

    \vspace{0.5\textheight}

    {\noindent Written on 9 October 2022}\\[\baselineskip]
    }

\end{titlepage}

\doublespacing{}
\tableofcontents
\singlespacing{}
\newpage

\begin{multicols}{2}
    Gauss elimination can also be used to find the inverse of a matrix.
    Let $A = \begin{pmatrix} a_{11} & a_{12} & a_{13} \\ a_{21} & a_{22} & a_{23} \\ a_{31} & a_{32} & a_{33} \end{pmatrix}$ be a invertible matrix, that is, $|A| \neq 0$. Now we arrange the matrix $A$ and the identity matrix $I$ into a 3 by 6 augmented matrix $A|I$ as follows:
    \[
        \left(\begin{array}{ccc|ccc}
                a_{11} & a_{12} & a_{13} & 1 & 0 & 0 \\
                a_{21} & a_{22} & a_{23} & 0 & 1 & 0 \\
                a_{31} & a_{32} & a_{33} & 0 & 0 & 1
            \end{array}\right)
    \]
    We then apply Gauss elimination to the augmented matrix $A|I$ to obtain the
    following matrix such that the left hand side of this matrix become an identity
    matrix:
    \[
        \left(\begin{array}{ccc|ccc}
                1 & 0 & 0 & b_{11} & b_{12} & b_{13} \\
                0 & 1 & 0 & b_{21} & b_{22} & b_{23} \\
                0 & 0 & 1 & b_{31} & b_{32} & b_{33}
            \end{array}\right)
    \]
    where $b_{ij}$ are constants, the right hand side of the augmented matrix is
    the inverse of $A$:
    \[
        A^{-1} = \begin{pmatrix} b_{11} & b_{12} & b_{13} \\ b_{21} & b_{22} & b_{23} \\ b_{31} & b_{32} & b_{33} \end{pmatrix}
    \]

    \subsection{Practice 15}

    Using the method of Gauss elimination, find the inverse of $\begin{pmatrix}
            1 & 1 & 1  \\
            1 & 2 & 3  \\
            2 & 3 & -4
        \end{pmatrix}$.
    \sol{}
    \begin{flalign*}
         & \left(\begin{array}{ccc|ccc}
                     1 & 1 & 1  & 1 & 0 & 0 \\
                     1 & 2 & 3  & 0 & 1 & 0 \\
                     2 & 3 & -4 & 0 & 0 & 1
                 \end{array}\right)                                 \\
        \xrightarrow[R_3 \rightarrow R_3 - 2R_1]{R_2 \rightarrow R_2 - R_1}
         & \left(\begin{array}{ccc|ccc}
                         1 & 1 & 1  & 1  & 0 & 0 \\
                         0 & 1 & 2  & -1 & 1 & 0 \\
                         0 & 1 & -6 & -2 & 0 & 1
                     \end{array}\right)                                \\
        \xrightarrow[R_3 \rightarrow R_3 - R_2]{R_1 \rightarrow R_1 - R_2}
         & \left(\begin{array}{ccc|ccc}
                         1 & 0 & -1 & 2  & -1 & 0 \\
                         0 & 1 & 2  & -1 & 1  & 0 \\
                         0 & 0 & -8 & -1 & -1 & 1
                     \end{array}\right)                               \\
        \xrightarrow{R_3 \rightarrow -\frac{1}{8}R_3}
         & \left(\begin{array}{ccc|ccc}
                         1 & 0 & -1 & 2           & -1          & 0            \\
                         0 & 1 & 2  & -1          & 1           & 0            \\
                         0 & 0 & 1  & \frac{1}{8} & \frac{1}{8} & -\frac{1}{8}
                     \end{array}\right)  \\
        \xrightarrow[R_1 \rightarrow R_1 + R_3]{R_2 \rightarrow R_2 - 2R_3}
         & \left(\begin{array}{ccc|ccc}
                         1 & 0 & 0 & \frac{17}{8} & -\frac{7}{8} & -\frac{1}{8} \\
                         0 & 1 & 0 & -\frac{5}{4} & \frac{3}{4}  & \frac{1}{4}  \\
                         0 & 0 & 1 & \frac{1}{8}  & \frac{1}{8}  & -\frac{1}{8}
                     \end{array}\right) \\
    \end{flalign*}
    \begin{flalign*}
        \therefore\ A^{-1} & = \begin{pmatrix} \frac{17}{8} & - \frac{7}{8} & -\frac{1}{8} \\ -\frac{5}{4} & \frac{3}{4} & \frac{1}{4} \\ \frac{1}{8} & \frac{1}{8} & -\frac{1}{8} \end{pmatrix}
    \end{flalign*}

    \subsection{Exercise 14.7}

    Solve the following system of linear equations using the method of Gauss
    elimination:

    \begin{enumerate}
        \item $\begin{cases}
                      3x - y - 14 = 0 \\
                      2y + z - 5 = 0  \\
                      x - 5z + 10 = 0
                  \end{cases}$
              \sol{}
              \begin{flalign*}
                                & \begin{cases}
                                      3x - y = 14 \\
                                      2y + z = 5  \\
                                      x - 5z = -10
                                  \end{cases} \\
                                & \begin{amatrix}{3}
                                      3 & -1 & 0 & 14 \\
                                      0 & 2 & 1 & 5 \\
                                      1 & 0 & -5 & -10
                                  \end{amatrix}              \\
                  \xrightarrow{R_1 \rightarrow R_1 - 3R_3}
                                & \begin{amatrix}{3}
                                      0 & -1 & 15 & 44 \\
                                      0 & 2 & 1 & 5 \\
                                      1 & 0 & -5 & -10
                                  \end{amatrix}              \\
                  \xrightarrow{R_2 \rightarrow R_2 + 2R_1}
                                & \begin{amatrix}{3}
                                      0 & -1 & 15 & 44 \\
                                      0 & 0 & 31 & 93 \\
                                      1 & 0 & -5 & -10
                                  \end{amatrix}              \\
                  \xrightarrow{R_2 \rightarrow \frac{1}{31}R_2}
                                & \begin{amatrix}{3}
                                      0 & -1 & 15 & 44 \\
                                      0 & 0 & 1 & 3 \\
                                      1 & 0 & -5 & -10
                                  \end{amatrix}              \\
                  \xrightarrow[R_1 \rightarrow R_1 - 15R_2]{R_3 \rightarrow R_3 + 5R_2}
                                & \begin{amatrix}{3}
                                      0 & -1 & 0 & -1 \\
                                      0 & 0 & 1 & 3 \\
                                      1 & 0 & 0 & 5
                                  \end{amatrix}              \\
                  \xrightarrow[R_1 \rightarrow -R_1]{R_3 \leftrightarrow R_2}
                                & \begin{amatrix}{3}
                                      0 & 1 & 0 & 1 \\
                                      1 & 0 & 0 & 5 \\
                                      0 & 0 & 1 & 3 \\
                                  \end{amatrix}              \\
                  \xrightarrow{R_1 \leftrightarrow R_2}
                                & \begin{amatrix}{3}
                                      1 & 0 & 0 & 5 \\
                                      0 & 1 & 0 & 1 \\
                                      0 & 0 & 1 & 3 \\
                                  \end{amatrix}              \\
                  \therefore\ x & = 5,\ y = 1,\ z = 3
              \end{flalign*}

        \item $\begin{cases}
                      x + y + z = 6    \\
                      x + 2y + 3z = 10 \\
                      2x + 3y - 4z = 8
                  \end{cases}$
              \sol{}
              \begin{flalign*}
                                & \begin{amatrix}{3}
                                      1 & 1 & 1 & 6 \\
                                      1 & 2 & 3 & 10 \\
                                      2 & 3 & -4 & 8
                                  \end{amatrix}   \\
                  \xrightarrow[R_3 \rightarrow R_3 - 2R_1]{R_2 \rightarrow R_2 - R_1}
                                & \begin{amatrix}{3}
                                      1 & 1 & 1 & 6 \\
                                      0 & 1 & 2 & 4 \\
                                      0 & 1 & -6 & -4
                                  \end{amatrix}   \\
                  \xrightarrow[R_3 \rightarrow R_3 - R_2]{R_1 \rightarrow R_1 - R_2}
                                & \begin{amatrix}{3}
                                      1 & 0 & -1 & 2 \\
                                      0 & 1 & 2 & 4 \\
                                      0 & 0 & -8 & -8
                                  \end{amatrix}   \\
                  \xrightarrow{R_3 \rightarrow -\frac{1}{8}R_3}
                                & \begin{amatrix}{3}
                                      1 & 0 & -1 & 2 \\
                                      0 & 1 & 2 & 4 \\
                                      0 & 0 & 1 & 1
                                  \end{amatrix}   \\
                  \xrightarrow[R_1 \rightarrow R_1 + R_3]{R_2 \rightarrow R_2 - 2R_3}
                                & \begin{amatrix}{3}
                                      1 & 0 & 0 & 3 \\
                                      0 & 1 & 0 & 2 \\
                                      0 & 0 & 1 & 1
                                  \end{amatrix}   \\
                  \therefore\ x & = 3,\ y = 2,\ z = 1
              \end{flalign*}

        \item $\begin{cases}
                      -x + y + z = 5   \\
                      2x - 7y + 4z = 1 \\
                      2x - 5y + 3z = -2
                  \end{cases}$
              \sol{}
              \begin{flalign*}
                                & \begin{amatrix}{3}
                                      -1 & 1 & 1 & 5 \\
                                      2 & -7 & 4 & 1 \\
                                      2 & -5 & 3 & -2
                                  \end{amatrix}     \\
                  \xrightarrow[R_2 \rightarrow R_2 + 2R_1]{R_3 \rightarrow R_3 + 2R_1}
                                & \begin{amatrix}{3}
                                      -1 & 1 & 1 & 5 \\
                                      0 & -5 & 6 & 11 \\
                                      0 & -3 & 5 & 8
                                  \end{amatrix}
                  \\
                  \xrightarrow{R_2 \rightarrow R_2 - R3}
                                & \begin{amatrix}{3}
                                      -1 & 1 & 1 & 5 \\
                                      0 & -2 & 1 & 3 \\
                                      0 & -3 & 5 & 8
                                  \end{amatrix}     \\
                  \xrightarrow[R_1 \rightarrow R_1 - R_2]{R_3 \rightarrow R_3 - 5R_2}
                                & \begin{amatrix}{3}
                                      -1 & 3 & 0 & 2 \\
                                      0 & -2 & 1 & 3 \\
                                      0 & 7 & 0 & -7
                                  \end{amatrix}     \\
                  \xrightarrow[R_1 \rightarrow -R_1]{R_3 \rightarrow \frac{1}{7}R_3}
                                & \begin{amatrix}{3}
                                      1 & -3 & 0 & -2 \\
                                      0 & -2 & 1 & 3 \\
                                      0 & 1 & 0 & -1
                                  \end{amatrix}     \\
                  \xrightarrow[R_1 \rightarrow R_1 + 3R_3]{R_2 \rightarrow R_2 + 2R_3}
                                & \begin{amatrix}{3}
                                      1 & 0 & 0 & -5 \\
                                      0 & 0 & 1 & 1 \\
                                      0 & 1 & 0 & -1
                                  \end{amatrix}     \\
                  \xrightarrow{R_2 \leftrightarrow R_3}
                                & \begin{amatrix}{3}
                                      1 & 0 & 0 & -5 \\
                                      0 & 1 & 0 & -1 \\
                                      0 & 0 & 1 & 1
                                  \end{amatrix}     \\
                  \therefore\ x & = -5,\ y = -1,\ z = 1
              \end{flalign*}

        \item $\begin{cases}
                      4x - y - 7z = 0 \\
                      5x - 2y - z = 1 \\
                      3x + 3y + 5z = 2
                  \end{cases}$
              \sol{}
              \begin{flalign*}
                                & \begin{amatrix}{3}
                                      4 & -1 & -7 & 0 \\
                                      5 & -2 & -1 & 1 \\
                                      3 & 3 & 5 & 2
                                  \end{amatrix}                                 \\
                  \xrightarrow{R_2 \rightarrow R_2 - 2R_1}
                                & \begin{amatrix}{3}
                                      4 & -1 & -7 & 0 \\
                                      -3 & 0 & 13 & 1 \\
                                      3 & 3 & 5 & 2
                                  \end{amatrix}                                 \\
                  \xrightarrow{R_3 \rightarrow R_3 + R_2}
                                & \begin{amatrix}{3}
                                      4 & -1 & -7 & 0 \\
                                      -3 & 0 & 13 & 1 \\
                                      0 & 3 & 18 & 3
                                  \end{amatrix}                                 \\
                  \xrightarrow{R_3 \rightarrow \frac{1}{3}R_3}
                                & \begin{amatrix}{3}
                                      4 & -1 & -7 & 0 \\
                                      -3 & 0 & 13 & 1 \\
                                      0 & 1 & 6 & 1
                                  \end{amatrix}                                 \\
                  \xrightarrow{R_1 \rightarrow R_1 + R_3}
                                & \begin{amatrix}{3}
                                      4 & 0 & -1 & 1 \\
                                      -3 & 0 & 13 & 1 \\
                                      0 & 1 & 6 & 1
                                  \end{amatrix}                                 \\
                  \xrightarrow{R_3 \rightarrow 4R_3}
                                & \begin{amatrix}{3}
                                      4 & 0 & -1 & 1 \\
                                      -12 & 0 & 52 & 4 \\
                                      0 & 1 & 6 & 1
                                  \end{amatrix}                                 \\
                  \xrightarrow{R_2 \rightarrow R_2 + 3R_1}
                                & \begin{amatrix}{3}
                                      4 & 0 & -1 & 1 \\
                                      0 & 0 & 49 & 7 \\
                                      0 & 1 & 6 & 1
                                  \end{amatrix}                                 \\
                  \xrightarrow{R_2 \rightarrow \frac{1}{49}R_2}
                                & \begin{amatrix}{3}
                                      4 & 0 & -1 & 1 \\
                                      0 & 0 & 1 & \frac{1}{7} \\
                                      0 & 1 & 6 & 1
                                  \end{amatrix}                           \\
                  \xrightarrow[R_3 \rightarrow R_3 - 6R_2]{R_1 \rightarrow R_1 + R_2}
                                & \begin{amatrix}{3}
                                      4 & 0 & 0 & \frac{8}{7} \\
                                      0 & 0 & 1 & \frac{1}{7} \\
                                      0 & 1 & 0 & \frac{1}{7}
                                  \end{amatrix}                           \\
                  \xrightarrow[R_1 \rightarrow \frac{1}{4}R_1]{R_2 \leftrightarrow R_3}
                                & \begin{amatrix}{3}
                                      1 & 0 & 0 & \frac{2}{7} \\
                                      0 & 1 & 0 & \frac{1}{7} \\
                                      0 & 0 & 1 & \frac{1}{7}
                                  \end{amatrix}                           \\
                  \therefore\ x & = \frac{2}{7},\ y = \frac{1}{7},\ z = \frac{1}{7}
              \end{flalign*}
    \end{enumerate}
\end{multicols}
\end{document}