\documentclass{report}
\usepackage[a4paper, total={7in, 9in}]{geometry}
\usepackage[fleqn]{amsmath}
\usepackage{amssymb}
\usepackage[fleqn]{cases}
\usepackage{multicol}
\usepackage{color}
\setlength{\columnseprule}{1pt}
\def\columnseprulecolor{\color{black}}

\title{Senior 1 Science Stream Math}
\author{Melvin Chia}

\begin{document}

\maketitle

\begin{multicols}{2}
\chapter{Trigonometry}

\section{Trigonometric Equation}

\begin{enumerate}

\item $\sin x = -\frac{1}{\sqrt{2}}$

\begin{align*}
\textbf{Sol.}&\because 0 \leq x \leq 2\pi \\
&\because \sin x < 0 \\
&\therefore x\ is\ in\ the\ 3rd\ or\ 4th\ quadrant \\
&\because The\ reference\ angle\ of \ x \ is \ \frac{\pi}{4} \\
&\therefore x = \frac{5\pi}{4}, \frac{7\pi}{4}
\end{align*}

\item $\cos x = \frac{1}{\sqrt{2}}$

\begin{align*}
\textbf{Sol.}&\because 0 \leq x \leq 2\pi \\
&\because \cos x > 0 \\
&\therefore x\ is\ in\ the\ 1st\ or\ 4th\ quadrant \\
&\because The\ reference\ angle\ of \ x \ is \ \frac{\pi}{4} \\
&\therefore x = \frac{\pi}{4}, \frac{7\pi}{4}
\end{align*}

\item $\tan x = -\sqrt{3}$

\begin{align*}
\textbf{Sol.}&\because 0 \leq x \leq 2\pi \\
&\because \tan x < 0 \\
&\therefore x\ is\ in\ the\ 2nd\ or\ 4rd\ quadrant \\
&\because The\ reference\ angle\ of \ x \ is \ \frac{\pi}{3} \\
&\therefore x = \frac{2\pi}{3}, \frac{5\pi}{3}
\end{align*}

\item $2\sin x = \sqrt{12}\cos x$

\begin{align*}
  \textbf{Sol.}\frac{\sin x}{\cos x} &= \frac{\sqrt{12}}{2} \\
  \tan x &= \frac{2\sqrt{3}}{2} \\
  \tan x &= \sqrt{3}\\
  &\because 0 \leq x \leq 2\pi \\
  &\because \tan x > 0 \\
  &\therefore x\ is\ in\ the\ 1st\ or\ 4th\ quadrant \\
  &\because The\ reference\ angle\ of \ x \ is \ \frac{\pi}{3} \\
  &\therefore x = \frac{\pi}{3}, \frac{4\pi}{3}
\end{align*}

\item $2\sin \frac{2x}{3} = 1$

\begin{align*}
  \textbf{Sol.}&\sin \frac{2x}{3} = \frac{1}{2} \\
  &\because 0 \leq x \leq 2\pi \\
  &\therefore 0 \leq \frac{2x}{3} \leq \frac{4\pi}{3} \\
  &\because \sin \frac{2x}{3} > 0 \\
  &\therefore x\ is\ in\ the\ 1st\ or\ 2nd\ quadrant \\
  &\because The\ reference\ angle\ of \ \frac{2\pi}{3} \ is \ \frac{\pi}{6} \\
  &\therefore \frac{2x}{3} = \frac{\pi}{6}, \frac{5\pi}{6}\\
  &\therefore x = \frac{\pi}{4}, \frac{5\pi}{4}
\end{align*}

\item $\cos^2 x - 2\sin x + 2 = 0$

\begin{align*}
  \textbf{Sol.}(1 - \sin^2 x) - 2\sin x + 2 &= 0 \\
  - \sin^2 x - 2\sin x + 3 &= 0 \\
  \sin^2 x + 2\sin x - 3 &= 0 \\
  (\sin x + 1)(\sin x - 3) &= 0 \\
  \sin x &= -1, 3(invalid) \\
  &\because 0 \leq x \leq 2\pi \\
  &\therefore x = \frac{3\pi}{2}
\end{align*}

\end{enumerate}
\end{multicols}

\end{document}
