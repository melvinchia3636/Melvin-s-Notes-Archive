\documentclass{report}
\usepackage[a4paper, total={7.5in, 10in}]{geometry}
\usepackage[fleqn]{amsmath}
\usepackage{amssymb}
\usepackage{amsthm}
\usepackage{enumitem}
\usepackage[]{mdframed}
\usepackage{multicol}
\usepackage{thmtools}
\usepackage{graphicx}
\usepackage{tikz}
\usepackage{tipa}
\usepackage{array, makecell, cellspace}
\usepackage{bigints}
\usepackage[export]{adjustbox}
\usepackage{accents}
\setlength{\cellspacetoplimit}{13.2ex}
\setlength{\cellspacebottomlimit}{13.2ex}

\usepackage{ifxetex}

\ifxetex
    \usepackage{substitutefont}
    \substitutefont{T3}{\rmdefault}{cmr}
\fi

\usepackage{fontspec}
\setmainfont[Mapping=tex-text]{Georgia}

\title{Praktis 8\\Vectors}
\author{Melvin Chia}

\newcommand\perm[2][^n]{\prescript{#1\mkern-2.5mu}{}P_{#2}}
\newcommand\permtwo[2][^n]{{}_{#1}P_{#2}}
\newcommand\comb[2][^n]{{}_{#1}C_{#2}}
\newcommand\combtwo[2][^n]{\prescript{#1\mkern-2.5mu}{}C_{#2}}
\renewcommand{\vec}[1]{\underaccent{\tilde}{#1}}

\newcommand{\sol}[1]{

    \noindent \textbf{Sol.}
}
\newcommand{\prooff}[1]{

    \noindent \textbf{Proof.}
}

\newcommand{\arc}[1]{{%
            \setbox9=\hbox{#1}%
            \ooalign{\resizebox{\wd9}{\height}{\texttoptiebar{\phantom{A}}}\cr#1}}}

\def\eos{\quad\hbox{\rlap{\hbox{\vrule depth 1.5pt height 2.6mm width 0.2mm \hskip 1mm \vrule height 2.6mm width 0.2mm}}{\vbox{\hrule height 0.2mm width 1.4mm \vskip 2.8mm \hrule depth 1.5pt height -0.35mm width 1.2mm}}}}

\counterwithout{equation}{chapter}
\setlength{\columnseprule}{1pt}
\setlength{\columnsep}{24pt}
\hfuzz=100pt
\setcounter{chapter}{8}

\begin{document}
\maketitle

\begin{multicols*}{2}
    \noindent\Large{\underline{\textbf{Praktis Formatif}}}
    \normalsize
    \section{Vectors}

    \begin{enumerate}
        \item Vectors $\vec{x}$ and $\vec{y}$ are non-zero and not parallel. Given that
              $(2m+3)\vec{x} + (5-n)\vec{y} = 0$, find the value of $m$ and $n$.

        \item Given that points $R$, $S$ and $T$ lie on a straight line. If
              $|\overrightarrow{RS}| = 15\textit{ units}$ and $|\overrightarrow{RT}| =
                  40\textit{ units}$, express $\overrightarrow{RS}$ in terms of
              $\overrightarrow{ST}$.

        \item Given that $\overrightarrow{EF} = (p - 4)\vec{a} + 6q\vec{b}$ and
              $\overrightarrow{GH} = \vec{a} + 2\vec{b}$. If line $EF$ and line $GH$ are
              parallel, express $p$ in terms of $q$.

        \item GIven that $\overrightarrow{AB} = 9\vec{p} - 12\vec{q}$ and
              $\overrightarrow{BC} = 6\vec{p} + (5-m)\vec{q}$ where $m$ is a constant. If
              points $A$, $B$, and $C$ are collinear, find
              \begin{enumerate}
                  \item the value of $m$,

                  \item the ratio of $AB:BC$.
              \end{enumerate}
    \end{enumerate}

    \section{Addition and Subtraction of Vectors}

    \begin{enumerate}
        \setcounter{enumi}{4}
        \item In the diagram in the answer space, $\overrightarrow{OX} = \vec{x}$ and
              $\overrightarrow{OY} = \vec{y}$. On the same diagram,
              \begin{enumerate}
                  \item draw the vector $\overrightarrow{OU}$ such that $\overrightarrow{OU} = 3\vec{x}
                            - 2\vec{y}$,
                  \item  mark point $V$ such that $\overrightarrow{UV} = 4\vec{y} - \vec{x}$.
              \end{enumerate}

        \item Given that $\vec{a} = 3\vec{p} + 4\vec{q}$, $\vec{b} = 2\vec{p} - \vec{q}$, and
              $\vec{c} = m\vec{p} + (m-n)\vec{q}$, where $m$ and $n$ are constants. Find the
              value of $m$ and $n$ when $\vec{c} = 4\vec{a} - 2\vec{b}$.

        \item In the following diagram, $OPQR$ is a trapezium where $PQ$ is parallel to $OR$
              adn $4PQ = 3OR$.

              Given that $\overrightarrow{OP} = \vec{p}$, and $\overrightarrow{OR} =
                  4\vec{r}$, express in terms of $\vec{p}$ and $\vec{r}$,
              \begin{enumerate}
                  \item $\overrightarrow{PR}$,
                  \item $\overrightarrow{RQ}$.
              \end{enumerate}

        \item The following diagram shows a regular hexagon $OPQRST$ with origin $O$,
              $\overrightarrow{OQ} = \vec{q}$ and $\overrightarrow{OR} = \vec{r}$.

              Find in terms of $\vec{q}$ and $\vec{r}$,
              \begin{enumerate}
                  \item $\overrightarrow{RQ}$,
                  \item $\overrightarrow{OS}$.
              \end{enumerate}

        \item The following diagram shows a trapezium where $PQ$ is parallel to $SR$.

              Given that $\overrightarrow{PQ} = h\vec{a}$, $\overrightarrow{RS} = k\vec{a}$,
              $\overrightarrow{SP} = h\vec{b}$, and $\overrightarrow{RQ} = 2\vec{a} +
                  (k+8)\vec{b}$. Find the value of $h$ and $k$.

        \item The following diagram shows a triangle $OPR$ and the point $Q$ lies on the
              straight line $PR$.

              It is given that $\overrightarrow{PQ}:\overrightarrow{QR} = 1:2$. Express
              $\overrightarrow{OQ}$, int erms of $\vec{a}$ and $\vec{b}$.

        \item The following diagram shows a rectangle $ABCD$ and $BED$ is a straight line.

              Given that $\overrightarrow{AB} = 10\vec{p}$, $\overrightarrow{BC} = 6\vec{q}$,
              and $\overrightarrow{BE} = 3\overrightarrow{ED}$. Express each of the following
              vectors in terms of $\vec{p}$ and $\vec{q}$.
              \begin{enumerate}
                  \item $\overrightarrow{BD}$,
                  \item $\overrightarrow{EC}$,
              \end{enumerate}

        \item The following diagram shows a triangle $OAB$.

              Given that $CS = 2OC$, $D$ is the midpoint of $AB$, $OE:ED = 2:1$,
              $\overrightarrow{OA} = \vec{a}$ and $\overrightarrow{OB} = \vec{b}$.
              \begin{enumerate}
                  \item Express $\overrightarrow{OD}$ in terms of $\vec{a}$ and $\vec{b}$.
                  \item Find the ratio of $CE:OB$.
              \end{enumerate}

        \item Given that $\overrightarrow{OP} = -5\vec{x} + 10\vec{y}$, $\overrightarrow{OQ}
                  = 5\vec{x} + 8\vec{y}$, and $\overrightarrow{OR} = (m-1)\vec{x} + 7\vec{y}$,
              where $m$ is a constant.
              \begin{enumerate}
                  \item Find
                        \begin{enumerate}
                            \item $\overrightarrow{PQ}$, in terms of $\vec{x}$ and $\vec{y}$,
                            \item $\overrightarrow{PR}$, in terms of m, $\vec{x}$ and $\vec{y}$.
                        \end{enumerate}
                  \item If the points $P$, $Q$, and $R$ are collinear, find hte value of $m$.
              \end{enumerate}

    \end{enumerate}

    \section{Vectors in a Cartesian Plane}

    \begin{enumerate}

        \setcounter{enumi}{13}

        \item The following diagram shows two vectors, $PO$ and $QO$.

              Given that $\vec{QP} = m\vec{i} + n\vec{j}$. Find the value of $m$ and $n$.

        \item The following diagram shows a parallelogram $ABCD$ drawn on a Cartesian plane
              where $E$ is the midpoint of $BD$.

              Given that $\overrightarrow{AB} = 3\vec{i}+2\vec{j}$ and $\vec{BC} = 7\vec{i} -
                  6\vec{j}$. Find
              \begin{enumerate}
                  \item $\overrightarrow{ED}$,
                  \item $|\overrightarrow{EC}|.$
              \end{enumerate}

        \item Given that points $A(2, -1)$ and $B(5, 3)$ lie on a Cartesian plane.
              \begin{enumerate}
                  \item Express $\overrightarrow{AB}$ in the form of $\begin{pmatrix}
                                x \\y
                            \end{pmatrix}$.
                  \item Find the unit vector in the direction of $\overrightarrow{AB}$.
              \end{enumerate}

        \item Given that $\overrightarrow{OM} = \begin{pmatrix}
                      -5 \\k
                  \end{pmatrix}$ and $\overrightarrow{ON} = \begin{pmatrix}
                      3 \\4
                  \end{pmatrix}$, find the possible values of $k$ if $|\overrightarrow{MN} = 10$\textit{ units}.

        \item Given the vectors $\vec{a} = -7i - m\vec{j}$, $\vec{b} = 8\vec{i} - \vec{j}$
              and $\vec{c} = -10\vec{i} + 6\vec{j}$. If vector $\vec{a} - \vec{b}$ is
              parallel to vector $\vec{c}$, find the value of the constant $m$.

        \item Given $A(2, -5)$, $B(3, 4)$ and $C(p, q)$. Find the value of $p$ and $q$ such
              that $\overrightarrow{AB} - 2\overrightarrow{BC} = 9\vec{i} - 5\vec{j}$.

        \item Given the vectors $\begin{pmatrix}
                      1 \\-6
                  \end{pmatrix}$, $\overrightarrow{OQ} = \begin{pmatrix}
                      3 \\5
                  \end{pmatrix}$, $\overrightarrow{OR} = \begin{pmatrix}
                      2 \\7
                  \end{pmatrix}$, and $\overrightarrow{OS} = \begin{pmatrix}
                      m \\2
                  \end{pmatrix}$, find
              \begin{enumerate}
                  \item vector $\overrightarrow{QR}$,
                  \item the value of $m$ when $\overrightarrow{PS}$ is parallel to
                        $\overrightarrow{QR}$.
                  \item the values of $m$ such that $|\overrightarrow{OS}| = 2|\overrightarrow{QR}|$.
              \end{enumerate}
    \end{enumerate}

\end{multicols*}

\end{document}