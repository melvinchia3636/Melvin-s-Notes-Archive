\documentclass[UTF8]{ctexart}
\usepackage{ctex}
\setlength{\parskip}{\baselineskip}%
\usepackage{amsmath}
\usepackage{amsfonts,stmaryrd,amssymb} % Math packages

\usepackage{enumerate} % Custom item numbers for enumerations
\usepackage{fontawesome}
\usepackage{setspace}
\usepackage{hyperref}
\usepackage{enumitem}
\usepackage{multicol}
\usepackage{xhfill}
\usepackage[p,osf]{cochineal}
\usepackage[scale=.95,type1]{cabin}
\usepackage[cochineal,bigdelims,cmintegrals,vvarbb]{newtxmath}
\usepackage[zerostyle=c,scaled=.94]{newtxtt}
\usepackage[cal=boondoxo]{mathalfa}
\usepackage[export]{adjustbox}
\usepackage{vwcol}  
\usepackage{fancyhdr}
\DeclareSymbolFont{yhlargesymbols}{OMX}{yhex}{m}{n}
\DeclareMathAccent{\wideparen}{\mathord}{yhlargesymbols}{"F3}

\hypersetup{
	colorlinks=false,
	linkcolor=black,
	filecolor=black,      
	urlcolor=black,
	pdftitle={Overleaf Example},
	pdfpagemode=FullScreen,
	urlbordercolor=white,
}

\urlstyle{same}


	
\newenvironment{cequation}{
	\makeatletter
	\setbool{@fleqn}{false}
	\makeatother
	\begin{equation*}
		}{\end{equation*}}
		
\newcommand{\sol}{\noindent\textbf{Solution:} }
%----------------------------------------------------------------------------------------

\newcommand{\exercise}[1]{%
	\subsection*{\faPencil\ \ Exercise #1\hspace{0.5em}\xrfill[0.175\baselineskip]{1pt}}
}

\newcommand{\practice}[1]{%
	\subsection*{\faFlag\ \ Practice #1\hspace{0.5em}\xrfill[0.175\baselineskip]{1pt}}
}

\newcommand{\revision}[1]{%
	\section*{\faGears\ \ Revision Exercise #1\hspace{0.5em}\xrfill[0.175\baselineskip]{1pt}}
}

\usepackage[ruled]{algorithm2e} % Algorithms

\usepackage[framemethod=tikz]{mdframed} % Allows defining custom boxed/framed environments

\usepackage{listings} % File listings, with syntax highlighting
\lstset{
	basicstyle=\ttfamily, % Typeset listings in monospace font
}

%----------------------------------------------------------------------------------------
%	DOCUMENT MARGINS
%----------------------------------------------------------------------------------------

\usepackage{geometry} % Required for adjusting page dimensions and margins

\geometry{
	paper=a4paper, % Paper size, change to letterpaper for US letter size
	top=2.5cm, % Top margin
	bottom=3cm, % Bottom margin
	left=2.5cm, % Left margin
	right=2.5cm, % Right margin
	headheight=14pt, % Header height
	footskip=1.5cm, % Space from the bottom margin to the baseline of the footer
	headsep=1.2cm, % Space from the top margin to the baseline of the header
	%showframe, % Uncomment to show how the type block is set on the page
}

%----------------------------------------------------------------------------------------
%	FONTS
%----------------------------------------------------------------------------------------

\usepackage[utf8]{inputenc} % Required for inputting international characters
\usepackage[T1]{fontenc} % Output font encoding for international characters

%----------------------------------------------------------------------------------------
%	COMMAND LINE ENVIRONMENT
%----------------------------------------------------------------------------------------

% Usage:
% \begin{commandline}
%	\begin{verbatim}
%		$ ls
%		
%		Applications	Desktop	...
%	\end{verbatim}
% \end{commandline}

\mdfdefinestyle{commandline}{
	leftmargin=10pt,
	rightmargin=10pt,
	innerleftmargin=15pt,
	middlelinecolor=black!50!white,
	middlelinewidth=2pt,
	frametitlerule=false,
	backgroundcolor=black!5!white,
	frametitle={Command Line},
	frametitlefont={\normalfont\sffamily\color{white}\hspace{-1em}},
	frametitlebackgroundcolor=black!50!white,
	nobreak,
}

% Define a custom environment for command-line snapshots
\newenvironment{commandline}{
	\medskip
	\begin{mdframed}[style=commandline]
		}{
	\end{mdframed}
	\medskip
}

%----------------------------------------------------------------------------------------
%	FILE CONTENTS ENVIRONMENT
%----------------------------------------------------------------------------------------

% Usage:
% \begin{file}[optional filename, defaults to "File"]
%	File contents, for example, with a listings environment
% \end{file}

\mdfdefinestyle{file}{
	innertopmargin=1.6\baselineskip,
	innerbottommargin=0.8\baselineskip,
	topline=false, bottomline=false,
	leftline=false, rightline=false,
	leftmargin=2cm,
	rightmargin=2cm,
	singleextra={%
		\draw[fill=black!10!white](P)++(0,-1.2em)rectangle(P-|O);
		\node[anchor=north west]
		at(P-|O){\ttfamily\mdfilename};
		%
		\def\l{3em}
		\draw(O-|P)++(-\l,0)--++(\l,\l)--(P)--(P-|O)--(O)--cycle;
		\draw(O-|P)++(-\l,0)--++(0,\l)--++(\l,0);
	},
	nobreak,
}

% Define a custom environment for file contents
\newenvironment{file}[1][File]{ % Set the default filename to "File"
	\medskip
	\newcommand{\mdfilename}{#1}
	\begin{mdframed}[style=file]
		}{
	\end{mdframed}
	\medskip
}

%----------------------------------------------------------------------------------------
%	NUMBERED QUESTIONS ENVIRONMENT
%----------------------------------------------------------------------------------------

% Usage:
% \begin{question}[optional title]
%	Question contents
% \end{question}

\mdfdefinestyle{question}{
	innertopmargin=1.2\baselineskip,
	innerbottommargin=0.8\baselineskip,
	roundcorner=5pt,
	nobreak,
	singleextra={%
		\draw(P-|O)node[xshift=1em,anchor=west,fill=white,draw,rounded corners=3pt]{%
			\faCaretRight\ \textbf{Example \theQuestion\questionTitle}};
	},
}

\newcounter{Question} % Stores the current question number that gets iterated with each new question

% Define a custom environment for numbered questions
\newenvironment{question}[1][\unskip]{
	\bigskip
	\stepcounter{Question}
	\newcommand{\questionTitle}{~#1}
	\begin{mdframed}[style=question]
		}{
	\end{mdframed}
	\medskip
}

%----------------------------------------------------------------------------------------
%	SOLUTIONS ENVIRONMENT
%----------------------------------------------------------------------------------------

% Usage:
% \begin{solution}
%	Solution contents
% \end{solution}

\mdfdefinestyle{solution}{
	innertopmargin=1.2\baselineskip,
	innerbottommargin=0.8\baselineskip,
	roundcorner=5pt,
	nobreak,
	singleextra={%
		\draw(P-|O)node[xshift=1em,anchor=west,fill=white,draw,rounded corners=5pt]{解};
	},
}

% Define a custom environment for solutions
\newenvironment{solution}{
	\begin{mdframed}[style=solution]
		}{
	\end{mdframed}
}

%----------------------------------------------------------------------------------------
%	WARNING TEXT ENVIRONMENT
%----------------------------------------------------------------------------------------

% Usage:
% \begin{warn}[optional title, defaults to "Warning:"]
%	Contents
% \end{warn}

\mdfdefinestyle{warning}{
	topline=false, bottomline=false,
	leftline=false, rightline=false,
	nobreak,
	singleextra={%
		\draw(P-|O)++(-0.5em,0)node(tmp1){};
		\draw(P-|O)++(0.5em,0)node(tmp2){};
		\fill[black,rotate around={45:(P-|O)}](tmp1)rectangle(tmp2);
		\node at(P-|O){\color{white}\scriptsize\bf !};
		\draw[very thick](P-|O)++(0,-1em)--(O);%--(O-|P);
	}
}

% Define a custom environment for warning text
\newenvironment{warn}[1][Warning:]{ % Set the default warning to "Warning:"
	\medskip
	\begin{mdframed}[style=warning]
		\noindent{\textbf{#1}}
		}{
	\end{mdframed}
	\vspace{-0.5cm}
}

%----------------------------------------------------------------------------------------
%	INFORMATION ENVIRONMENT
%----------------------------------------------------------------------------------------

% Usage:
% \begin{info}[optional title, defaults to "Info:"]
% 	contents
% 	\end{info}

\mdfdefinestyle{info}{%
	topline=false, bottomline=false,
	leftline=false, rightline=false,
	nobreak,
	singleextra={%
		\fill[black](P-|O)circle[radius=0.6em];
		\node at(P-|O){\color{white}\scriptsize\bf \faInfo};
		\draw[very thick](P-|O)++(0,-0.8em)--(O);%--(O-|P);
	}
}

% Define a custom environment for information
\newenvironment{info}[1][Info:]{ % Set the default title to "Info:"
	\medskip
	\begin{mdframed}[style=info]
		\noindent{\textbf{#1}}
		}{
	\end{mdframed}
	\vspace{-0.5cm}
	
}

\mdfdefinestyle{explore}{%
	topline=false, bottomline=false,
	leftline=false, rightline=false,
	nobreak,
	singleextra={%
		\fill[black](P-|O)circle[radius=0.6em];
		\node at(P-|O){\color{white}\scriptsize\bf \faFlask};
		\draw[very thick](P-|O)++(0,-0.8em)--(O);%--(O-|P);
	}
}

% Define a custom environment for warning text
\newenvironment{explore}[1][Exploration Activity:]{ % Set the default warning to "Warning:"
	\medskip
	\begin{mdframed}[style=explore]
		\noindent{\large\textbf{#1}}
		}{
	\end{mdframed}
	\vspace{-0.5cm}
}

\mdfdefinestyle{think}{%
	topline=false, bottomline=false,
	leftline=false, rightline=false,
	nobreak,
	singleextra={%
		\fill[black](P-|O)circle[radius=0.6em];
		\node at(P-|O){\color{white}\scriptsize\bf \faQuestion};
		\draw[very thick](P-|O)++(0,-0.8em)--(O);%--(O-|P);
	}
}

% Define a custom environment for warning text
\newenvironment{think}[1][Think about It:]{ % Set the default warning to "Warning:"
	\medskip
	\begin{mdframed}[style=think]
		\noindent{\large\textbf{#1}}
		}{
	\end{mdframed}
	\vspace{-0.5cm}
}



\newenvironment{cequation}{
    \makeatletter
    \setbool{@fleqn}{false}
    \makeatother
    \begin{equation*}
        }{\end{equation*}}

\title{2024年高三上半年数学统测复习} % Title of the assignment

\author{Melvin Chia} % Author name and email address

\date{高三商1 --- 2024年3月29日} % University, school and/or department name(s) and a date

%----------------------------------------------------------------------------------------

\begin{document}

\maketitle % Print the title

\section{函数} % Numbered section

\subsection{啥是函数?}

\textbf{函数(Function)}是一种特殊的关系,它把一个集合的元素对应到另一个集合的元素上。函数的\textbf{定义域(Domain)}是指所有可能输入/\textbf{原像(Preimage)}的集合,函数的\textbf{值域(Range)}是指所有可能输出/\textbf{映像(Image)}的集合。

打个比方,全世界每个国家都有首都,那么我们可以把每个国家和它的首都对应起来,这就是一个函数。国家就是定义域,首都就是值域。

\begin{info}[函数的定义]
    设$X$和$Y$是两个非空集合,如果存在一个从$X$到$Y$的对应关系$f$,使得对于$X$中的任意一个元素$x$,都有唯一的$Y$中的元素$y$与之对应,那么称$f$为从$X$到$Y$的一个函数,记作$f:X\to Y$。
\end{info}
但是,正常情况下,一个国家只有一个首都。所以,如果有哪个国家经过这个对应关系对应到了两个首都,那么这个对应关系就不是函数。

\begin{warn}[啥情况不是函数?]
    如果对于$X$中的某个元素$x$,存在两个$Y$中的元素$y_1$和$y_2$与之对应,那么这个对应关系就不是函数。
\end{warn}

还有,有的国家没有首都,这也是不行的。所以,如果有哪个国家没有对应的首都,那么这个对应关系也不是函数。

\begin{warn}[啥情况不是函数?]
    如果$X$中的某个元素$x$没有对应的$Y$中的元素与之对应,那么这个对应关系就不是函数。
\end{warn}

\subsection{函数的表示}

函数可以用很多种方式来表示,最常见的就是用公式来表示。拿上面的例子来说,国家与首都对应的关系可以表示为:
\begin{cequation}
    f(\text{国家}) = \text{该国家的首都}
\end{cequation}
但是,在数学里,我们一般用字母来表示函数,所以这个函数可以表示为:
\begin{cequation}
    f(x) = y
\end{cequation}
其中,$x$表示国家,$y$表示该国家的首都首都。

如果我们要限制函数的定义域,那么我们可以这样表示:
\begin{cequation}
    f: A \to B,\ f(x) = y
\end{cequation}
这表示函数$f$的定义域是$A$,对应域是$B$。至于定义域和对应域是什么,开头我们已经概括了,并缺会在下一节细讲。

打个比方,如果我们要限制函数$f$的定义域是所有亚洲国家的集合,值域是城市的集合,那么我们可以在前面加多一些东西
\begin{cequation}
    f: \text{亚洲国家} \to \text{城市},\ f(\text{国家}) = \text{该国家的首都}
\end{cequation}
好啦,现在我们来看专业一点的东西。如果我们不用国家和首都来举例,而是用数学里的东西,会发生什么事情呢?

作为纯商科学生的我们,经济学都学过需求函数对吧?那就是一个函数,它把商品的价格对应到了商品的需求量上。经济学课本第一册告诉我们,需求函数的一般形式是:
\begin{cequation}
     Q_d(P) = a - bP
\end{cequation}
或者是,如果我们要限制需求函数的定义域是所有正数,值域是所有实数,那么我们可以表示为:
\begin{cequation}
    Q_d: \mathbb{R}^+ \to \mathbb{R},\ Q_d(P) = a - bP
\end{cequation}
毕竟,价格不能是负数,对吧?

\begin{question}
	已知函数$f(x) = 2x^2 - 3x + 1$,求$f(2)$。
\end{question}

\begin{solution}
    将$x = 2$代入$f(x)$,得到
    \begin{align*}
        f(2) &= 2 \cdot 2^2 - 3 \cdot 2 + 1 &\\
        &= 2 \cdot 4 - 6 + 1\\
        &= 3
    \end{align*}
\end{solution}

\begin{question}
    已知函数$f(x) = 2x^2 - 3x + 1$,求$f(x) = 0$的解。
\end{question}

\begin{solution}
    由$f(x) = 2x^2 - 3x + 1 = 0$,得到
    \begin{align*}
        2x^2 - 3x + 1 &= 0 &\\
        (2x - 1)(x - 1) &= 0
    \end{align*}
    所以,$x = 1$或$x = \dfrac{1}{2}$。
\end{solution}

\begin{question}
    已知函数$f(x) = x^2 - 3x + 1$,求$f(x) = -x$的解。
\end{question}

\begin{solution}
    由$f(x) = x^2 - 3x + 1 = -x$,得到
    \begin{align*}
        x^2 - 3x + 1 &= -x &\\
        x^2 - 2x + 1 &= 0\\
        (x - 1)^2 &= 0
    \end{align*}
    所以,$x = 1$。
\end{solution}

\begin{question}
    已知函数$f(x) = \begin{cases}
        2x - 1, & x \geq 0\\
        x^2 + 1, & x < 0
    \end{cases}$,求$f(1)$。
\end{question}

\begin{solution}
    由于$1 \geq 0$,所以$f(1) = 2 \cdot 1 - 1 = 1$。
\end{solution}


\subsection{定义域、对应域和值域} 

我们继续拿国家和首都来当例子好了。给定一个函数:

\vspace{-0.5cm}
\begin{cequation}
    f: \text{亚洲国家} \to \text{城市},\ f(\text{国家}) = \text{该国家的首都}
\end{cequation}

如果我叫你求$f(\text{中国})$,你会怎么做呢?很简单,你只要知道中国的首都是北京,然后你就可以得到$f(\text{中国}) = \text{北京}$。

但是,如果我叫你求$f(\text{美国})$,这个时候你就会发现,你得不到答案。因为美国并不是亚洲国家,它不在函数的定义域里,所以它没有对应的值。

因此,函数的\textbf{定义域(Domain)}是指所有可能输入的集合。在上面的例子里,亚洲国家就是函数的定义域。

\begin{info}[定义域]
    函数的定义域是指所有可能输入的集合, 记作$D_f$。
\end{info}

再来看看对应域。对应域是什么呢?上面的例子里,城市就是函数的\textbf{对应域(Codomain)}。对应域里的元素不一定需要是和定义域里的元素的映射。比如,对应域里可以包含一些不是首都的城市。

世界上有千千万万座城市,但是不是每座城市都是一个国家的首都,更何况我们这个函数只对应亚洲国家的首都,比如新山虽然是一座城市,但却不是任何国家的首都。所以,函数的\textbf{值域(Range)}是指所有可能输出的集合。在上面的例子里,所有亚洲国家的首都就是函数的值域。

\begin{info}[值域]
    函数的值域是指所有可能输出的集合, 记作$R_f$。
\end{info}

所以,总的来说,
\begin{info}[函数的定义域、对应域和值域之间]
    给定一个函数$f: A \to B$,则其中$A$是函数的定义域,$B$是函数的对应域。函数的值域是$B$中所有可能的映像的集合。
\end{info}

举个数学的例子,如果我们有一个函数$f: \mathbb{R} \to \mathbb{R}$,它的定义域是所有实数,对应域是所有实数,那么它的值域是什么呢?这个时候,我们就需要看这个函数的公式了。

我们假设这个函数的公式是$f(x) = x^2$,那么这个函数的值域是什么呢?如果你仔细思考,你会发现,无论$x$是正数还是负数,$x^2$都会是正数。所以,这个函数的值域是所有正数的集合。因此,

\vspace{-0.5cm}
\begin{cequation}
    D_f = \mathbb{R},\ R_f = \{y \in \mathbb{R} | y \geq 0\}
\end{cequation}

在考试时,有时候会有一些题目会让你求函数的定义域和值域,即函数输入值( $x$)的有效范围和输出值( $y$)的可能范围。这个时候,你就需要根据函数的公式来判断。

\begin{question}[(平方根函数)]
    已知函数$f(x) = \sqrt{4 - x^2}$,求$f(x)$的定义域和值域。
\end{question}

\begin{solution}
    由于$\sqrt{4 - x^2}$的根号里面不能是负数,所以$4 - x^2 \geq 0$,即$x^2 \leq 4$。所以,$-2 \leq x \leq 2$,即
    \[
        D_f = [-2, 2]
    \]
    又因为对于所有$x$,$\sqrt{4 - x^2} \geq 0$,所以
    \[
        R_f = \{y \in \mathbb{R} | y \geq 0\}
    \]
\end{solution}

\begin{question}[(绝对值函数)]
    已知函数$f(x) = |x - 2|$,求$f(x)$的定义域和值域。
\end{question}

\begin{solution}
    若将任意实数$x$代入$f(x)$,则$|x - 2| \geq 0$。所以,
    \[
        D_f = \mathbb{R}
    \]
    又因为对于所有$x$,$|x - 2| \geq 0$,所以
    \[
        R_f = \{y \in \mathbb{R} | y \geq 0\}
    \]
\end{solution}

\begin{question}[(分数函数)]
    已知函数$f(x) = \dfrac{1}{x - 2}$,求$f(x)$的定义域和值域。
\end{question}

\begin{solution}
    由于分母不能为零,所以$x - 2 \neq 0$,即$x \neq 2$。所以,
    \[
        D_f = \{x \in \mathbb{R} | x \neq 2\}
    \]
    又因为对于所有$x$,$\dfrac{1}{x - 2} \neq 0$,所以
    \[
        R_f = \{y \in \mathbb{R} | y \neq 0\}
    \]
\end{solution}

\begin{question}[(增函数)]
    已知函数$f(x) = x^2 + 3$,求$f(x)$的定义域和值域。
\end{question}

\begin{solution}
    由于$x^2 + 3$对于所有$x$都有定义,所以
    \[
        D_f = \mathbb{R}
    \]
    又因为对于所有$x$,$x^2 \geq 0$,两侧加3得 $x^2 + 3 \geq 3$,所以
    \[
        R_f = \{y \in \mathbb{R} | y \geq 3\}
    \]
\end{solution}

\begin{question}[(一元二次函数)]
    已知函数$f(x) = x^2 - 3x + 1$,求$f(x)$的定义域和值域。
\end{question}

\begin{solution}
    由于$x^2 - 2x + 10$对于所有$x$都有定义,所以
    \[
        D_f = \mathbb{R}
    \]
    由于此二次函数是开口向上的,所以它的最小值是在顶点处。要找到顶点,我们可以用配方法:
\begin{align*}
    f(x) &= x^2 - 2x + 10\\
    &= x^2 - 2x + 1 - 1 + 10\\
    &= (x - 1)^2 - 1 + 10\\
    &= (x - 1)^2 + 9
\end{align*}
因此可得顶点为$(1, 9)$,所以f(x)的最小值为9,即
\[
    R_f = \{y \in \mathbb{R} | y \geq 9\}
\]

\end{solution}

\begin{info}[区间表示法]
    有时候,我们会用中括弧和小括弧而不是不等式来表达区间(两个数之间的任意数字的集合)。中括弧表示闭区间,小括弧表示开区间。比如,$[a, b]$表示闭区间,即$a \leq x \leq b$;$(a, b)$表示开区间,即$a < x < b$。如果是无穷大,我们用$\infty$表示。比如,$(-\infty, 2]$表示$x \leq 2$。注意,无穷大是开区间。
\end{info}

\newpage
\begin{question}[(已知定义域的一元二次函数)]
    已知函数$f(x) = x^2 - 4x - 8$,且$-1 \leq x \leq 3$,求$f(x)$的值域。
\end{question}

\begin{solution}
    从题目可知,$-1 \leq x \leq 3$,所以
    \[
        D_f = [-1, 3]
    \]
    由于此二次函数是开口向上的,所以它的最小值是在顶点处。要找到顶点,我们可以用配方法:
\begin{align*}
    f(x) &= x^2 - 4x - 8\\
    &= x^2 - 4x + 4 - 4 - 8\\
    &= (x - 2)^2 - 12
\end{align*}
因此可得顶点为$(2, -12)$,所以f(x)的最小值为-12。

现在我们来看最大值。由于$x^2 - 4x - 8$是一个开口向上的二次函数,所以它的最大值必定在两个端点处。我们已经找到了顶点处的最小值,所以我们只需要找两个端点处的值,然后比较大小即可。即,比较$f(-1)$和$f(3)$的大小。
\begin{align*}
    f(-1) &= (-1)^2 - 4 \cdot (-1) - 8= -3\\
    f(3) &= 3^2 - 4 \cdot 3 - 8 = -11
\end{align*}
由于$f(-1) < f(3)$,所以$f(x)$的最大值为-3,即
\[
    R_f = \{y \in \mathbb{R} | -12 \leq y \leq -3\}
\]
\end{solution}

\begin{info}[常用集合符号]
    \begin{enumerate}
        \item $\mathbb{N}$:自然数(非负整数)的集合,即$\{1, 2, 3, \ldots\}$
        \item   $\mathbb{Z}$:整数,即$\{\ldots, -2, -1, 0, 1, 2, \ldots\}$
        \item  $\mathbb{Q}$:有理数,可以被写成分数的数,即$\left\{\dfrac{a}{b} | a, b \in \mathbb{Z}, b \neq 0\right\}$
        \item $\mathbb{R}$:实数,包括整数、分数、无限循环小数、圆周率等等
        \item $\mathbb{R}^+$:正实数,即$x > 0$
        \item $\mathbb{R}^-$:负实数,即$x < 0$
        \item $\mathbb{I}$:虚数,即取平方根后得到负数的数
        \item $\mathbb{C}$:复数,由实数和虚数组成的数,即$\{a + bi | a, b \in \mathbb{R}\}$
    \end{enumerate}
\end{info}

\subsection{合成函数}

\textbf{合成函数(Composite Function)}是指把一个函数的输出作为另一个函数的输入。比如,我们有两个函数$f(x)$和$g(x)$,那么它们的合成函数就是$f(g(x))$。

打个比方,我们有两个函数,一个是把国家对应到首都的函数$f(x)$,另一个是把首都对应到人口的函数$g(x)$。那么,如果我们想知道一个国家首都的人口,我们就可以先用$f(x)$得到这个国家的首都,再把这个首都代入$g(x)$,就可以得到这个国家首都的人口,即

\vspace{-0.5em}
\begin{cequation}\text{国家}\ \xrightarrow{\displaystyle\text{某国的首都}} \text{该国家的首都}\ \xrightarrow{\displaystyle\text{某城市的人口} } \text{该国家首都的人口}\end{cequation}

再来拿数学运算作为例子,假设我们有两个函数,一个是把一个数加上1的函数$f(x) = x + 1$,另一个是把一个数乘以2的函数$g(x) = 2x$。那么,如果我们想知道一个数加1后乘以2的结果,我们就可以先用$f(x)$得到这个数加1的结果,再把这个结果代入$g(x)$,就可以得到这个数加1后乘以2的结果,即

\vspace{-0.5em}
\begin{cequation}x\ \xrightarrow{\displaystyle\text{加1}} x + 1\ \xrightarrow{\displaystyle\text{乘2}} 2(x + 1)\end{cequation}

继续那上面的例子,如果我们设“加一”函数为$f(x) = x+1$,“乘二”函数为$g(x) = 2x$,那么我们可以得到合成函数$f(g(x))$。这个合成函数的意思是,先把$x$乘以2,再把这个结果加1。所以,$f(g(x)) = 2x + 1$。

合成函数不只是两个函数的合成,它可以是任意多个函数的合成。比如,我们有三个函数$f(x)$,$g(x)$和$h(x)$,那么它们的合成函数就是$f(g(h(x)))$。需要注意的是,合成函数的顺序是从括弧里面开始的,即先计算$h(x)$,再计算$g(h(x))$,最后计算$f(g(h(x)))$。因此,$f(g(x))$和$g(f(x))$是不一样的。

如果今天给你10个函数组成的合成函数,要写一大堆括号,你会不会觉得很麻烦?所以,我们可以用一个更简单的方法来表示合成函数。比如,我们有三个函数$f(x)$,$g(x)$和$h(x)$,其合成函数 $f(g(h(x)))$ 可以表示为$(f \circ g \circ h)(x)$。当然,如果你用圈圈表示的话,它的运算顺序是从右往左的。同理,$(f \circ g)(x)$ 和 $(g \circ f)(x)$ 是不一样的。

当然,一个函数也可以和自己合成。比如,我们有一个函数将一个数加1,那么这个函数和自己合成就是将一个数加1后再加1,即$(f \circ f)(x) = f(f(x)) = f(x + 1) = (x + 1) + 1 = x + 2$。

但是,如果你今天要讲一个函数和自己合成100次,你不可能写100个括号,或是100个圈圈,对吧?所以,我们可以用一个更简单的方法来表示。比如,$(f \circ f \circ f \circ f \circ f)(x)$ 可以简写为 $f^5(x)$。

\begin{info}[函数和自己合成]
    对于所有的正整数$n$,$f^n(x)$ 表示将函数 $f(x)$ 和自己合成 $n$ 次。
\end{info}

\begin{question}
    已知函数$f(x) = 2x + 1$,$g(x) = x^2$,求$f(g(x))$。
\end{question}

\begin{solution}
    \vspace{-1em}
    \begin{align*}
        f(g(x)) &= f(x^2)\\
        &= 2x^2 + 1
    \end{align*}
\end{solution}

\begin{question}
    已知函数$f(x) = x^2$,$g(x) = \sqrt{x}$,求$(f \circ g)(x)$。
\end{question}

\begin{solution}
    \vspace{-1em}
    \begin{align*}
        (f \circ g)(x) &= f(g(x))\\
        &= f(\sqrt{x})\\
        &= (\sqrt{x})^2\\
        &= x
    \end{align*}
\end{solution}

\begin{question}
    已知函数$f(x) = 2x + 1$,$g(x) = x^2 - 4$,求$(f \circ g)(x)$ 及 $(g \circ f)(x)$。
\end{question}

\begin{solution}
    \vspace{-1em}
    \begin{align*}
        (f \circ g)(x) &= f(g(x))\\
        &= f(x^2 - 4)\\
        &= 2(x^2 - 4) + 1\\
        &= 2x^2 - 8 + 1\\
        &= 2x^2 - 7\\
        \\
        (g \circ f)(x) &= g(f(x))\\
        &= g(2x + 1)\\
        &= (2x + 1)^2 - 4\\
        &= 4x^2 + 4x + 1 - 4\\
        &= 4x^2 + 4x - 3
    \end{align*}
\end{solution}

\newpage
有时候,题目会给你一个函数$f(x)$和它与另外一个函数$g(x)$的合成函数。这时候会有两种情况,一种是其合成函数为$f(g(x))$,另一种是其合成函数为$g(f(x))$。两种情况都有不一样的解法,所以在解题时要注意。

\begin{question}
    已知函数$f(x) = x+3$,$f(g(x)) = x^2 + 6x + 10$,求$g(x)$。
\end{question}

\begin{solution}
    \vspace{-1em}
    \begin{align*}
        f(g(x)) &= f(x^2 + 3)\\
    g(x) + 3 &= x^2 + 6x + 10\\
    g(x) &= x^2 + 6x + 7
    \end{align*}
\end{solution}

\begin{question}
    已知一函数的定义是$f(x) = x - 3$,另一函数$g(x)$满足$g(f(x)) = 4x^2 - 20x + 25$,求$g(x)$。
\end{question}

\begin{solution}
    设$y = f(x) = x - 3$,移项后得到$x = y + 3$。
    \begin{align*}
        g(f(x)) &= 4x^2 - 20x + 25\\
        g(y) &= 4(y + 3)^2 - 20(y + 3) + 25\\
        &= 4(y^2 + 6y + 9) - 20y - 60 + 25\\
        &= 4y^2 + 24y + 36 - 20y - 35\\
        &= 4y^2 + 4y + 1\\
        &= (2y + 1)^2
    \end{align*}
    将$y$换回$x$,得到$g(x) = (2x + 1)^2$。
\end{solution}

\newpage

\subsection{一对一函数、映成函数、一一映成函数}

我们继续拿国家和首都来当例,如果我们有一个函数$f(x)$,它把国家对应到首都,那么这个函数是不是一对一的呢?通常来说,一个首都不可能同时是两个国家的首都,对吧?所以,这个函数是一对一的,或者说将国家映射到首都的函数是一个\textbf{一对一函数(One-to-One Function)}。

\begin{info}[一对一函数的定义]
    如果对于函数$f: A \to B$,对于$B$中的任意一个元素$y$,最多只有一個的$A$中的元素$x$与之对应,那么称$f$是一个一对一函数。
\end{info}

那什么函数不是一对一函数呢?如果我们有一个函数$f(x)$,它把高三商1的学生对应到他们数学统测的分数水平\{$A$, $B$, $C$, $D$, $F$\},那么这个函数就不是一对一的,因为肯定有不止一个学生在数学统测中拿到$A$。

现在我们来看看什么是\textbf{映成函数(Onto Function)}。我们先来看看映成函数的定义:

\begin{info}[映成函数的定义]
    如果对于函数$f: A \to B$,对于$B$中的任意一个元素$y$,至少有一个$A$中的元素$x$与之对应,那么称$f$是一个映成函数。
\end{info}

如果我们有一个函数$f(x)$,它把一本教科书对应到它所属的学科,那么这个函数就是一个映成函数,因为每一本教科书都属于某一个学科,或者说每一个学科都至少有一本教科书。

最后,我们来看看什么是\textbf{一一映成函数(One-to-One Onto Function)}。一一映成函数是一对一函数和映成函数的结合,它既是一对一函数,又是映成函数。也就是说,把它们的定义结合起来,就是一一映成函数的定义:

\begin{info}[一一映成函数的定义]
    如果对于函数$f: A \to B$,对于$B$中的任意一个元素$y$,恰好只有一个$A$中的元素$x$与之对应,那么称$f$是一个一一映成函数。
\end{info}

比如说,如果我们有一个函数$f(x)$,它把每个人对应到他们的身份证号,那么这个函数就是一个一一映成函数,因为每个人都有一个唯一的身份证号,而每个身份证号也只对应一个人。

那数学的例子来说,如果我们有一个函数$f(x) = x + 1$,那么这个函数是不是一对一函数呢?是不是映成函数呢?是不是一一映成函数呢?我们来看看。

首先,对于任何一个$x$,$f(x)$都是该数加1,不可能有两个数加1后得到同一个数,所以这个函数是一对一函数。其次,对于任何一个数$y$,只要$x = y - 1$,就可以得到$y$,所以这个函数是映成函数。最后,这个函数是一对一函数,也是映成函数,所以它是一一映成函数。

再来,如果我们有一个函数$f(x) = x^2$,同一个数字,正负号不同的平方是一样的,所以这个函数不是一对一函数。其次,对于任何一个数$y$,只要$x = \sqrt{y}$或$x = -\sqrt{y}$,就可以得到$y$,所以这个函数是映成函数。

\subsection{反函数}

我们依旧拿国家和首都来当例子。如果我们有一个函数$f(x)$,它把国家对应到首都,那么我们可以定义一个反函数$f^{-1}(x)$,它把首都对应到国家。比如,如果$f(\text{中国}) = \text{北京}$,那么$f^{-1}(\text{北京}) = \text{中国}$。

\begin{info}[反函数的定义]
    如果函数$f: A \to B$是一个一一映成函数,那么存在一个函数$f^{-1}: B \to A$,使得$x$,$f^{-1}(f(x)) = x$,$f(f^{-1}(x)) = x$。
\end{info}

为什么一定要是一一映成函数才能有反函数呢?因为如果一个函数不是一一映成函数,那么就会有两个不同的$x$对应到同一个$y$,那么反函数就不知道应该把$y$对应到哪个$x$了。拿上面举过的例子,若一个函数$f(x)$,它把高三商1的学生对应到他们数学统测的分数水平\{$A$, $B$, $C$, $D$, $F$\},这个函数就不是一对一的,因为肯定有不止一个学生在数学统测中拿到$A$。所以,如果硬要定义一个反函数$f^{-1}(x)$将分数水平对应到高三商1的学生,那么如果你输入$A$,这个反函数就会对应到不止一个学生,这就不符合函数的定义了。

\begin{question}
    已知函数$f(x) = 2x + 1$,求$f^{-1}(x)$。
\end{question}

\begin{solution}
    首先,我们令$y = f^{-1}(x)$,那么$f(y) = x$。所以,
    \begin{align*}
        f(y) &= 2y + 1 = x\\
        2y &= x - 1\\
        y &= \dfrac{x - 1}{2}
    \end{align*}
\end{solution}

\begin{question}
    已知函数$f(x) = \dfrac{1}{x - 2}$,$x \neq 2$,求$f^{-1}(x)$。
\end{question}

\begin{solution}
    首先,我们令$y = f^{-1}(x)$,那么$f(y) = x$。所以,
    \begin{align*}
        f(y) &= \dfrac{1}{y - 2} = x\\
        y - 2 &= \dfrac{1}{x}\\
        y &= \dfrac{1}{x} + 2
    \end{align*}
\end{solution}

\subsection*{练习题}

\subsubsection*{定义域和值域}

\begin{enumerate}
    \item 若$f(x) = x + \dfrac{1}{x}$, $x$ 为非零实数,则$f(x)$的值域是什么?
    \item 试求函数$f(x) = \sqrt{x^2 - 1}{x^2 + 1}$的值域,式中$x \in \mathbb{R}$。
    \item 若$f(x) = |2x - 3| + 1$,式中$0 \leq x \leq 4$,求$f(x)$的值域。
    \item 函数$y = -2x^2 + 6x - 9$的值域是什么?
    \item 已知函数$f: \mathbb{R} \to \mathbb{R}$,$f(x) = 2x^2 - 1$,$x \in (-1,3)$。求$f(x)$的值域。
    \item 求函数$f(x) = \dfrac{\sqrt{3x - 1}}{x - 1}$的定义域。
    \item 求函数$y = x^2 - 6x - 27$,$x \in \mathbb{R}$的值域。
    \item 已知$f(x) = \dfrac{2x + k}{k - x}$,且$f(1) = 2$。当$x$取何值时,$f(x)$无意义?
    \item 求实数函数$f(x) = \dfrac{1}{\sqrt{x+2}}$的定义域。
    \item 求函数$f(x) = \dfrac{\sqrt{x + 1}}{x}$的定义域。
    \item 已知函数$f(x) = x^2 - 4$的定义域为$[-1, 3]$,求$f(x)$的值域。
    \item 函数$f(x)$定义成$f(x) = (x - 1)^2$,求$f(x)$的值域。
\end{enumerate}

\subsubsection*{合成函数}

\begin{enumerate}
    \item 已知函数$f(x) = x^2 - 2x + 3$,求$f(x-1)$。
    \item 已知函数$f(x+1) = 2x^2 + 5x + 4$,求$f(3)$。
    \item 设$f(x-1) = x^2 - 3x + 2$, 则$f(2)=$ ?
    \item 已知函数$f(x) = 3x - 1$及$g(3x + 1) = f(x+2)$,求$g(x)$。
    \item 若$g(x) = x - 2$, $(f \circ g)(x) = 2x^2 - 5x + 7$,求$f(x)$。
    \item 若 $f: \mathbb{R} \rightarrow \mathbb{R}$ 定义成 $x \rightarrow 3 x-4$, 求另一函数 $g: \mathbb{R} \rightarrow \mathbb{R}$ 使得 $g \circ f: x \rightarrow 9 x^2+24 x+22$
    \item 设 $f: x \rightarrow 2 x+3$ 及 $g: x \rightarrow \dfrac{x}{x+1}$, 式中 $x \neq-1$, 求 $f \circ g$。
    \item 一函数定义成 $f: x \rightarrow x+1$ 。另一函数 $g$ 使到 $g \circ f: x \rightarrow 3 x^2+6 x$。求函数 $g$。
    \item 一函数 $g: \mathbb{R} \rightarrow \mathbb{R}$ 定义成 $g(x)=3 x+1$, 另有一函数 $f$ 使得 $f \circ g(x)=9 x^2+6 x$ 。求函数 $f$。
    \item 已知 $h(x)=x+5$ 及 $g(x)=x^2+8$。若 $g \circ h(a)=g(3)$, 求 $a$ 的值。
    \item 二函数 $f, g$ 定义成 $f(x)=x^2$ 及 $g(x)=x+1$, 试求 $f \circ g(2)$ 之值。
    \item 一函数 $f$ 的定义是 $f: x \rightarrow x-3$, 另一函数 $g$ 则使到 $g \circ f: x \rightarrow 4 x^2-20 x+25$, 那么 $g(x)$ 为
    \item 已知 $f: x \rightarrow 2 x-1$ 及 $g: x \rightarrow \dfrac{x}{2}$, 求 $g \circ f$ 。
    \item 已知 $f: x \rightarrow x+3$ 及 $f \circ g: x \rightarrow x^2+6 x+10$, 求 $g$ 。
    \item 对于任意实数$x$而言,若$f(x) = 2x _ 1$,且$f[g(x)] = x^2 + x + g(x)$,则$g(x) =$ ?
    \item 一函数 $f$ 定义成 $f(3 x-1)=9 x^2+3 x+5$, 则 $f(x)$ 的表达式为?
    \item 假设函数 $f$ 与 $g$ 分别为 $f(x)=2 x+1, g(x)=x^2-4$ 。求合成函数
    $g \circ f$ 和 $f \circ g$。
    \item 设 $f(x)=3 x+2, g(x)=a x+b$, 求合成函数 $f \circ g$ 及 $g \circ f \circ$ 求出条件使得 $f \circ g=g \circ f$。
    \item 设 $f(x)=3 x-2, g(x)=2 x^2+1, h(x)=a x+b$,求合成函数 $f \circ g$ 及 $g \circ f$。若 $(f \circ g \circ h)(x)=6 x^2+12 x+7$, 求 $a, b$ 之值。
    \item 设 $f(x)=x+1, g(x)=x^2-4 x+1$,试求合成函数 $f \circ g$ 及 $g \circ f$。据此,试求使 $f \circ g(x)=g \circ f(x)$ 之 $x$ 值。
    \item 函数 $f$ 与 $g$ 之定义为:
    $f: x \rightarrow x^2+2 x$ 及 $g: x \rightarrow 3 x-2$ 。
    求 $f(2), g(2),(f \circ g)(2)$ 及 $(g \circ f)(2)$ 之值。
    \item 若 $f: x \rightarrow x+2$ 及 $g: x \rightarrow x^2-2 x+1$, 求合成函数 $g \circ f$ 及 $f \circ g$ 。从而求 $g \circ f(2)$ 及 $f \circ g(2)$ 之值。
    \item 已知 $f: x \rightarrow 2 x-3$ 及 $f \circ g: x \rightarrow 6 x^2+10 x-15, x \in \mathbb{R}$, 求 $g(x)$。
    \item 已知函数 $f: \mathbb{R}^{+} \rightarrow \mathbb{R}$ 定义成 $f(x)=\log x$, 函数 $g: S \rightarrow \mathbb{R}$ 定义成 $g(x)=\sqrt{2 x-3}-1$ 。试求 $\mathbb{R}$ 的最大子集 $S$, 使得 $f \circ g$ 有意义。
\end{enumerate}

\subsubsection*{反函数}

\begin{enumerate}
    \item 已知$f(1-2x) = 4x + 7$,求$f^{-1}(-5)$的值。
    \item 已知$f(2x - 1) = \dfrac{x - 2}{4x + 3}$,求$f^{-1}(3)$的值。
    \item 已知函数$g:\mathbb{R} \to \mathbb{R}$定义成$g(x) = x - 2$,$f:[0, \infty) \to (-\infty, 3]$为一函数使得$(g\circ f^{-1})(x) = \dfrac{x^2 - 6x + 1}{4}$,求$f(9)$的值。
    \item 已知 $f: x \rightarrow \dfrac{a x+3}{4 x+5}$ 。若 $f^{-1}=f$, 则 $a$ 之值为?
    \item 已知 $x>2$ 时, 函数 $f(x)=\dfrac{2 x-3}{x+a}$ 的反函数就是 $f(x)$ 本身, 则 $a$ 的值是?
    \item 如果 $f: x \rightarrow x-3, g: x \rightarrow x^2-3$, 求 $\left(f^{-1} \circ g\right)(x)$。
    \item 已知函数 $f: \mathbb{R} \rightarrow \mathbb{R}$ 定义成 $f(x)=a x+b,-10 \leq x \leq 10$ 其中 $a$ 与 $b$ 为常数。若 $f(4)=3$及 $f(-2)=6$, 求 $f^{-1}(-1)$ 的值。
    \item 已知 $f^{-1}: x \rightarrow \dfrac{2 x-1}{x+3}, x \neq-3$, 求函数 $f$.
   \item 已知 $f(x)=2 x^2+7$ 及 $(g \circ f)(x)=4 x^4+22 x^2+29$, 求 $g(x)$。
    \item 若 $f: x \rightarrow 4 x+7$, 且 $g: x \rightarrow \dfrac{2 x+1}{x-1}$, 求 $f^{-1}(x)$,$g^{-1}(x)$,$g^{-1} \circ f^{-1}(3)$。
    \item 设 $f: x \rightarrow 2 x+3$ 及 $g: x \rightarrow \frac{x}{x+1}$, 式中 $x \neq-1$, 求 $g^{-1}$ 。
    \item 如果函数 $f: \mathbb{R} \rightarrow \mathbb{R}$ 定义成 $f(x+5)=x^3+7 x+8$, 求$f(8)$及$f^{-1}(8)$
    4. 已知函数 $f(x)=\dfrac{2 x-3}{x+1}, x \neq-1$。求其反函数的解析式 $f^{-1}(x)$.
    \item 若 $f: x \rightarrow 3+\log (x-2)$, 则 $f^{-1}(3)=$?
    \item 若 $f(2 x-1)=x+1$, 则 $f^{-1}(x)=$
    \item 已知 $f \circ g: x \rightarrow 3 x^2-1, x \in \mathbb{R}$ 且 $g^{-1} \circ f: x \rightarrow \sqrt{5 x+2}, x \in\left\{x \left\lvert\, x \geq-\frac{2}{5}\right.\right\}$, 求 $f^{-1} \circ f^{-1}(0)$的值。
    \item 若函数 $f: \mathbb{R} \rightarrow \mathbb{R}$ 满足 $f(2 x-1)=5 x+2$, 求 $f^{-1}(3 x+1)$。
    \item 设 $f(x)=\dfrac{10^x-10^{-x}}{10^x+10^{-x}}$, 试求其反函数 $f^{-1}$。
    \item 函数 $f$ 与 $g$ 之定义为$f: x \rightarrow 5 x+3$;$g: x \rightarrow 2 x-7$。求$f \circ g$ 与 $(f \circ g)^{-1}$; $f^{-1}, g^{-1}$ 与 $g^{-1} \circ f^{-1}$。
    \item 已知函数 $f$ 定义成 $f: x \rightarrow 3-\dfrac{x}{4}, x \in \mathbb{R}$。如果 $g \circ f^{-1}: x \rightarrow 2-5 x-3 x^2$, 求函数 $g$。
\end{enumerate}

\end{document}
