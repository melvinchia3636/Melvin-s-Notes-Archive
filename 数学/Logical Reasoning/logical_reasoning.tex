\documentclass{report}

\usepackage[fleqn]{amsmath}
\usepackage{amssymb}
\usepackage{setspace}
\usepackage{enumitem}
\usepackage{fontspec}
\usepackage{titlesec}
\usepackage{nicematrix}
\usepackage[total={6.6in,9.2in}]{geometry}

\setmainfont{Times New Roman}

\title{\Huge{\textbf{Logical Reasoning}}}
\author{Melvin Chia}
\date{18 June 2023}

\setcounter{chapter}{10}

\titleformat{\chapter}[display]
{\normalfont\huge\bfseries}{\chaptertitlename\ \thechapter}{20pt}{\Huge}

% this alters "before" spacing (the second length argument) to 0
\titlespacing*{\chapter}{0pt}{0pt}{40pt}

\begin{document}
\maketitle

\onehalfspacing

\chapter{Logical Reasoning}

\section{Logic}

\textbf{Logic} is a branch of science that studies the way we reason and its patterns. When we are reasoning, We need to use concepts, make judgments, and make inferences. Concept, judgment, and inference are the three basic elements of the thinking process. They are interconnected and follow a certain pattern.

The science of logic emerged over 2000 years ago. Ancient philosophers already
started to study the formation of thinking and its patterns from a long time
ago. Aristotle from ancient Greek was the first to systematically study logic,
thus he is known as the father of classical logic. In the 17th century, Leibniz
from Germany was the first to put forward the idea of using symbolic operations
to study logic problems. This has led to the emergence of mathematical logic, a
branch of logic that uses mathematical methods to study logic problems. In the
19th century, Boole, a British mathematician, created a fringe science that is
in between algebra and logic, known as \textbf{Logical algebra} (also known as
\text{Boolean algebra}). Since the 20th century, mathematical logic has
received greater and deeper development. It describes and studies logic more
accurately and mathematically. It has provided a meaningful tool and method for
the study of the foundation of mathematics. Its research on computerizing,
programming, and mechanizing the thinking process has also become the
theoretical basis of computer science.

Human society is evolving into an age of information. The popularization of
computers has led to the digitalization of science and technology, the
mechanization of human thinking, and computerization is increasing day by day.
Mathematics as a logically rigorous basic science, is playing an increasingly
important role, and logic as one of the foundations of mathematics is also
increasingly valued by people.

\section{Proposition}

\textbf{Proposition} is a declarative sentence that is used to express a certain judgment. For example,
\begin{enumerate}[label=(\alph*)]
    \item The sum of the interior angles of a triangle is $180^\circ$.
    \item $\sqrt{2}$ is not a rational number.
    \item $(a+b)^2$ is equal to $a^2 + 2ab + b^2$. (That is, the equation $a^2 + 2ab + b^2 = (a+b)^2$)
    \item Two lines that are perpendicular to the same plane are parallel to each other.
    \item The equation $x^2 + 4x + 5 = 0$ has two real roots.
    \item $\sin^2 x - \cos^2 x = 2$.
\end{enumerate}
Some of these sentences are true, some are false. The sentences (a), (b), (c), (d) above are true, while (e) and (f) are false.

The sentences that can be judged as true or false are called
\textbf{propositions}. All the six sentences above are propositions.

Some sentences cannot be judged as true or false, these kinds of uncertain
sentences are not propositions. For example,
\begin{enumerate}[label=(\alph*), start=7]
    \item The two base angles of $\triangle ABC$ are equal.
    \item $a$ is the smallest among the three numbers $a$, $b$, and $c$.
\end{enumerate}
Since $\triangle ABC$ and the number $a$, $b$, and $c$ are not specified, the sentence (g) and (h) cannot be judged as true or false.

The proposition that is true is called a \textbf{true proposition}, and the
proposition that is false is called a \textbf{false proposition}. Generally, we
use small letters $p$, $q$, $r$, $s$, $\cdots$ to represent propositions. For
example, below are 4 propositions represented by $p$, $q$, $r$, and $s$
respectively.
\begin{flalign*}
    p: & \ \sin^2 x + \cos^2 x = 1                    \\
    q: & \ \text{When } x \in \mathbb{R},\ x^2 \geq 0 \\
    r: & \ 3\sin x = 4                                \\
    s: & \ \emptyset \in \{0\}
\end{flalign*}
Among the propositions above, $p$ and $q$ are true propositions, while $r$ and $s$ are false propositions. The true or false of a proposition is called the \textbf{truth value} of the proposition. We stipulate that the truth value of a true proposition is 1, and the truth value of a false proposition is 0. For example, the propositions $p$ and $q$ above are true, denoted as $p = 1$ and $q = 1$, while the propositions $r$ and $s$ are false, denoted as $r = 0$ and $s = 0$.

\vspace{0.5cm}
\begin{enumerate}[label=\textbf{Example \arabic*}, leftmargin=*]
    \item State whether the following sentences are propositions. If it is a proposition,
          state whether it is true or false and give the reason to your answer.
          \begin{flalign*}
              p: & \ \text{The square of any number is not less than zero.}                            & \\
              q: & \ \text{The parabola $y = x^2 + 1$ has no point of intersection with the $x$-axis.}   \\
              r: & \ x-y = 0.                                                                            \\
              s: & \ \text{At least two interior angles of a triangle are acute angles.}                 \\
              t: & \ \text{For any real number $x$, $2x + 1 > x$.}
          \end{flalign*}
\end{enumerate}
\begin{enumerate}[label=\textbf{Sol.}]
    \item \begin{enumerate}[label=]
              \item $p$ is a true proposition. The square of any number is non-negative.
              \item $q$ is a true proposition. The parabola $y = x^2 + 1$ is on top of the $x$-axis and its vertex is $(0, 1)$.
              \item $r$ is not a proposition. For any $x$ and $y$, we cannot tell whether $x-y$ is equal to 0.
              \item $s$ is a true proposition. The sum of the interior angles of a triangle is $180^\circ$, there cannot be more two obtuse angle or straight angle at the same time.
              \item $t$ is a false proposition. For example, $2(-1) + 1 < -2$.
          \end{enumerate}
\end{enumerate}
\newpage
\begin{enumerate}[label=\textbf{Example \arabic*}, leftmargin=*, start=2]
    \item Write down the truth value of the following propositions.
          \begin{flalign*}
              p: & \ \text{For any real number $x$, $x < x + 1$.}                 & \\
              q: & \ \text{For any real number $a$, if $a^3 > 0$, then $a > 0$.}    \\
              r: & \ \text{The period of the function $y = \sin x$ is $\pi$.}       \\
              s: & \ \text{The equation $2\sin x - \cos x = 4$ has no solution.}    \\
              t: & \ \text{The line $3x - 4y + 1 = 0$ passes through the origin.}
          \end{flalign*}
\end{enumerate}
\begin{enumerate}[label=\textbf{Sol.}]
    \item \begin{enumerate}[label=]
              \item $p = 1 \qquad q = 1 \qquad r = 0 \qquad s = 1 \qquad t = 0$
          \end{enumerate}
\end{enumerate}

\subsection*{Exercise 11a}
\begin{enumerate}
    \item State whether the following sentences are propositions.
          \begin{enumerate}[label=, leftmargin=*]
              \item $p$: The equation $x^2 - 5x + 6 = 0$ has two positive real roots.
              \item $q$: $x + 5 = y + 3$.
              \item $r$: THe line $y = 3x + b$ and the line $y = 3x - b$ ($b \neq 0$) are parallel to each other.
              \item $s$: $\triangle ABC \cong \triangle A'B'C'$.
              \item $t$: 5 is the greatest common factor of 25 and 30.
              \item $u$: The probability of a sure event is 1, and the probability of an impossible event is 0.
          \end{enumerate}
    \item Write down the truth value of the following propositions, and give a
          counterexample for the false propositions.
          \begin{enumerate}[label=, leftmargin=*]
              \item $p$: All the even numbers are not prime numbers.
              \item $q$: When $x \in \mathbb{R}$, $x^2 + x + 1$ is always greater than 0.
              \item $r$: The maximum value of the function $y = ax^2 + bx + c$ ($a \neq 0$) is $\dfrac{4ac - b^2}{4a}$.
              \item $s$: The equation $\sin x < \sin 2x$ is true for any real number $x$.
              \item $t$: The solution set of the equation $\sin x = \cos x$ is $\left\{x | x = \dfrac{\pi}{4} + 2k\pi, k \in \mathbb{Z}\right\}$.
              \item $u$: The value of the function $y = 2^x$ is always greater than 0, where $x$ is any real number.
          \end{enumerate}
\end{enumerate}

\section*{Compound Propositions}

Lets consider the following propositions:
\begin{enumerate}[label=]
    \item $p$: 4 is a factor of 8.
    \item $q$: 4 is a factor of 12.
    \item $r$: 4 is not a factor of 8.
    \item $s$: 4 is a factor of 8 and is a factor of 12.
    \item $t$: 4 is a factor of 8 or is a factor of 12.
\end{enumerate}

Among these propositions, some of them are true propositions, while some of
them are false propositions. In the proposition $r$, $s$, and $t$, they contain
the words "not", "and", and "or" respectively. These words are called
\textbf{logical connectives}.

Propositions that do not contain any logical connectives are called
\textbf{simple propositions}. For example, the propositions $p$ and $q$ above
are simple propositions.

Propositions that are formed by connecting simple propositions using logical
connectives are called \textbf{compound propositions}. For example, the
propositions $r$, $s$, and $t$ above are compound propositions.

\subsection*{Inverse Proposition and its Truth Table}

Let $p$ be a proposition. Adding a logical connective "not" to $p$ gives us a
new proposition, that is, the inverse proposition of $p$, denoted by $\sim p$,
read as "not $p$".

The meaning of the inverse proposition $\sim p$ is the \textbf{negation} of the
original proposition $p$.

\vspace{0.5cm}
\begin{enumerate}[label=\textbf{Example \arabic*}, leftmargin=*]
    \item Write down the inverse proposition of the following propositions:
          \begin{enumerate}[label=, leftmargin=*]
              \item $p$: $2 + 2 = 4$.
              \item $q$: 25 is a multiple of 5.
              \item $r$: The equation $y = 3x^3 - x$ is an odd function.
              \item $s$: The square of 15 is 235.
              \item $t$: The base number $a$ of $\log_a x$ can be a negative number.
          \end{enumerate}
\end{enumerate}
\begin{enumerate}[label=\textbf{Sol.}]
    \item \begin{enumerate}[label=, leftmargin=*]
              \item $\sim p$: $2 + 2 \neq 4$.
              \item $\sim q$: 25 is not a multiple of 5.
              \item $\sim r$: The equation $y = 3x^3 - x$ is not an odd function.
              \item $\sim s$: The square of 15 is not 235.
              \item $\sim t$: The base number $a$ of $\log_a x$ cannot be a negative number.
          \end{enumerate}
\end{enumerate}

Apparently, if $p$ is a true proposition, then $\sim p$ is a false proposition.
If $p$ is a false proposition, then $\sim p$ is a true proposition. The truth
value of $p$ and $\sim p$ are opposite to each other, and their truth table is
as follows:
\begin{center}
    \begin{NiceTabular}{|c|c|}[code-before = \rowcolor{lightgray}{1}, hvlines]
        $p$ & $\sim p$ \\
        1   & 0        \\
        0   & 1        \\
    \end{NiceTabular}
\end{center}

\end{document}