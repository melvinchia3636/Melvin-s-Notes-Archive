\documentclass{report}
\usepackage[a4paper,margin=1in]{geometry}
\usepackage{amsmath,amssymb}
\usepackage{multirow}
\usepackage{setspace}
\usepackage{enumitem}

\newcommand{\sol}{

    \vspace{0.5em}\textbf{Solution:}\vspace{0.5em}}

\begin{document}
\onehalfspacing
\begin{enumerate}

    \item Which of the following is not a risk?
          \begin{enumerate}[label=(\Alph*)]
              \item Road accident
              \item Fire
              \item Theft
              \item Computer damage
          \end{enumerate}
          \sol{} (D)

          Unless that is a super computer lol. \vspace{0.5cm}

    \item Which of the following can claim compensation from life insurance?
          \begin{enumerate}[label=(\Alph*)]
              \item Cancer
              \item House on fire
              \item Car broken down at road
              \item Loss of wallet
          \end{enumerate}
          \sol{} (A)

          Life insurance is for life, not for other things. \vspace{0.5cm}

    \item The table below shows the premiums for a travel insurance to Europe that
          offered by Najwa Insurance Company.

          \begin{tabular}{|c|c|c|c|}
              \hline \begin{tabular}{c}
                         Number \\
                         of days
                     \end{tabular} & \begin{tabular}{c}
                                         Policyholder \\
                                         (RM)
                                     \end{tabular} & \begin{tabular}{c}
                                                         Policyholder \\
                                                         and spouse   \\
                                                         (RM)
                                                     \end{tabular} & \begin{tabular}{c}
                                                                         Group     \\
                                                                         (Maximum  \\
                                                                         4 people) \\
                                                                         (RM)
                                                                     \end{tabular}      \\
              \hline $1-5$              & 388                & 650                             & 1050 \\
              \hline $6-10$             & 540                & 920                             & 1800 \\
              \hline $11-15$            & 780                & 1340                            & 2500 \\
              \hline
          \end{tabular}

          Nizam's family of 6 people consisting of a spouse and 4 children want to travel
          to Europe from 21 January 2021 to 28 January 2021. He wants to buy a travel
          insurance. What is the minimum amount of the premium?
          \begin{enumerate}[label=(\Alph*)]
              \item RM1 840
              \item RM2 720
              \item RM3 240
              \item RM3 600
          \end{enumerate}
          \sol{} (B)

          Nizam has to buy a 6 - 10 days policy fro policy holder and spouse as well as a
          group policy for 4 people. Hence, the minimum amount of the premium is $1800 +
              920 = \text{RM2 720}$. \vspace{0.5cm}

    \item The table below shows the premium rates for every RM1 000 face value of a life
          insurance offered by Samuel Insurance Company.

          \begin{tabular}{|c|c|c|c|c|}
              \hline \multirow{2}{*}{ Age } & \multicolumn{2}{|c|}{ Male } & \multicolumn{2}{c|}{ Female }                              \\
              \cline { 2 - 5 }              & \begin{tabular}{c}
                                                  Non- \\
                                                  smoker
                                              \end{tabular}           & Smoker                        & \begin{tabular}{c}
                                                                                                            Non- \\
                                                                                                            smoker
                                                                                                        \end{tabular} & Smoker          \\
              \hline 29                     & 2.050                        & 2.237                         & 1.785              & 1.961 \\
              \hline 30                     & 2.092                        & 2.318                         & 1.720              & 2.054 \\
              \hline 31                     & 2.141                        & 2.405                         & 1.758              & 2.148 \\
              \hline 32                     & 2.185                        & 2.497                         & 1.790              & 2.239 \\
              \hline 33                     & 2.223                        & 2.586                         & 1.826              & 2.317 \\
              \hline
          \end{tabular}

          Mr Raji who is 32 years old wants to get a coverage of RM95 000. What is the
          premium for Mr Raji's life insurance if he does not smoke?
          \begin{enumerate}[label=(\Alph*)]
              \item RM170.05
              \item RM207.58
              \item RM212.71
              \item RM237.22
          \end{enumerate}
          \sol{} (B)

          According to the table, the premium rate for a 32 years old male non-smoker is
          RM2.185 per RM1 000 face value. Hence, the premium for Mr Raji's life insurance
          is $\dfrac{95 000}{1000} \times 2.185 = \text{RM207.58}$. \vspace{0.5cm}

    \item Madam Hendon wants to buy a motor insurance for her car in Peninsular Malaysia.
          The following shows the information about her car.

          \begin{tabular}{|ll|}
              \hline Age of vehicle & $: 2$ years          \\
              Engine capacity       & $: 1800 \mathrm{cc}$ \\
              NCD                   & $: 25 \%$            \\
              Sum insured           & $:$ RM85 000         \\
              \hline
          \end{tabular}

          Calculate the gross premium for Madam Hendon's car under the comprehensive
          policy.
          \begin{enumerate}[label=(\Alph*)]
              \item RM1 472.54
              \item RM1 500.00
              \item RM1 892.33
              \item RM2 017.60
          \end{enumerate}

          \sol{} (C)

          According to the premium rates under the Motor Tariff, the basic premium for
          the first RM1000 is RM339.10.

          The basic premium for the balance is $\dfrac{85 000 - 1000}{1000} \times 26 =
              \text{RM} 2 184.00$ The total basic premium is $339.10 + 2 184.00 = \text{RM} 2
              523.10$. The NCD discount is $2 523.10 \times 25 \% = \text{RM} 630.775$. The
          gross premium is $2 523.10 - 630.775 = \text{RM} 1 892.33$. \newpage

    \item Motor insurance for Madam Lee's car has a deductible provision of RM450. Madam
          Lee has suffered an accident that causes a loss of RM1 680. Determine the
          amount of compensation that can be claimed by Madam Lee.
          \begin{enumerate}[label=(\Alph*)]
              \item RM450
              \item RM1 230
              \item RM1 680
              \item RM2 130
          \end{enumerate}
          \sol{} (B)

          The amount of compensation that can be claimed by Madam Lee is $1 680 - 450 =
              \text{RM} 1 230$. \vspace{0.5cm}
    \item Hui Ling has bought a property insurance that has a deductible provision of
          RM900. She suffered losses in three months as shown in the following table.

          \begin{tabular}{|c|c|}
              \hline Month   & Loss (RM) \\
              \hline January & 480       \\
              \hline April   & 1220      \\
              \hline July    & 925       \\
              \hline
          \end{tabular}

          Determine the amount of compensation can be claimed by Hui Ling in April.
          \begin{enumerate}[label=(\Alph*)]
              \item RM320
              \item RM740
              \item 12
              \item 23
          \end{enumerate}
          \sol{} (A)

          The amount of compensation that can be claimed by Hui Ling in April is $1 220 -
              900 = \text{RM} 320$. \vspace{0.5cm}

    \item Chin Siong has a basic medical insurance with an annual limit of RM100 000. The
          amount of deductible borne by Chin Siong is RM1 000. He has made a treatment in
          a hospital with a medical cost of RM95 000. Calculate the amount of
          compensation that will be paid to Chin Siong.
          \begin{enumerate}[label=(\Alph*)]
              \item RM79 000
              \item RM80 000
              \item RM94 000
              \item RM95 000
          \end{enumerate}
          \sol{} (C)

          The amount of compensation that will be paid to Chin Siong is $95 000 - 1 000 =
              \text{RM} 94 000$. \vspace{0.5cm}

          \newpage
    \item Insurable value of Madam Low's house is RM185000. She bought a fire insurance
          that has a co-insurance provision to insure $75 \%$ of the insurable value of
          her house and a deductible of RM2 400. Madam Low's house suffered a loss of
          RM12 000. Calculate the amount of compensation that will be received by Madam
          Low if she insures her house with RM110 000.
          \begin{enumerate}[label=(\Alph*)]
              \item RM4 113.51
              \item RM5 113.51
              \item RM6 113.51
              \item RM7 113.51
          \end{enumerate}
          \sol{} (D)

          The amount of required insurance is $185 000 \times 75 \% = \text{RM} 138 750$.
          The amount of insured value is $110 000 < 138 750$. Hence, the amount of
          compensation that will be received by Madam Low is $\dfrac{110 000}{138 750}
              \times 12 000 - 2 400 = \text{RM} 7 113.51$. \vspace{0.5cm}

    \item Mr Halim is a policyholder for a major medical insurance with a deductible
          provision of RM600 and an 80/20 co-insurance percentage participation clause.
          If Mr Halim's treatment cost is RM7 800 , what is the treatment cost borne by
          insurance company?
          \begin{enumerate}
              \item RM5 760
              \item RM6 240
              \item RM7 200
              \item RM7 800
          \end{enumerate}
          \sol{} (A)

          The treatment cost after deductible is $7 800 - 600 = \text{RM} 7 200$. The
          treatment cost borne by insurance company is $7 200 \times 80 \% = \text{RM} 5
              760$. \vspace{0.5cm}

    \item The table below shows the premium rates for every RM1 000 face value of a life
          insurance offered by an insurance company.

          \begin{tabular}{|c|c|c|c|c|}
              \hline \multirow{2}{*}{ Plan } & \multicolumn{2}{|c|}{$\mathbf{3 0 - 3 4}$ years old } & \multicolumn{2}{c|}{$\mathbf{3 5 - 4 0}$ years old }                           \\
              \cline { 2 - 5 }               & \begin{tabular}{c}
                                                   Non- \\
                                                   smoker
                                               \end{tabular}                                    & Smoker                                               & \begin{tabular}{c}
                                                                                                                                                             Non- \\
                                                                                                                                                             smoker
                                                                                                                                                         \end{tabular} & Smoker       \\
              \hline \begin{tabular}{c}
                         5-year \\
                         term
                     \end{tabular}      & RM6.28                                                & RM9.50                                               & RM5.70             & RM8.39  \\
              \hline \begin{tabular}{c}
                         10-year \\
                         term
                     \end{tabular}      & RM7.80                                                & RM11.24                                              & RM6.95             & RM10.16 \\
              \hline
          \end{tabular}

          Calculate the annual premium needed to be paid by each of the following
          policyholders.
          \begin{enumerate}
              \item Madam Norhafizah is 38 years old and a nonsmoker. She wants to buy a life
                    insurance with a coverage of RM230 000 for 10 years. \sol{}

                    According to the table, the premium rate for Madam Norhafizah is RM6.95. Hence,
                    the annual premium needed to be paid by Madam Norhafizah is
                    $\dfrac{230000}{1000} \times 6.95 = \text{RM} 1 598.50$.

              \item Mr Zaini is 32 years old and a smoker. He wants to buy a life insurance with a
                    coverage of RM65 000 and add on a critical illness policy for 5 years. The
                    critical illness policy offered has a coverage of $40 \%$ of basic face value
                    and the premium rate for every RM1 000 is RM3.75. \sol{}

                    According to the table, the premium rate for Mr Zaini is RM9.50. Hence, the
                    basic premium is $\dfrac{65000}{1000} \times 9.50 = \text{RM} 617.50$.

                    The amount covered by the critical illness policy is $65000 \times 40 \% =
                        \text{RM} 26 000$. Hence, the premium for the critical illness policy is
                    $\dfrac{26000}{1000} \times 3.75 = \text{RM} 97.50$.

                    The total annual premium needed to be paid by Mr Zaini is $617.50 + 97.50 =
                        \text{RM} 715.00$.
          \end{enumerate}
          \vspace{0.5cm}

    \item Siti has a car with an engine capacity of $1500 \mathrm{cc}$. She wants to buy
          a motor insurance for her car. The sum insured is RM34000. If she lives in
          Kuching, Sarawak and her car has a NCD of $55 \%$. Calculate the gross premium
          of Siti's car under the following policies.
          \begin{enumerate}
              \item Comprehensive \sol{}

                    According to the premium rates under the Motor Tariff, the basic premium for
                    the first RM1000 is RM220.00.

                    The basic premium for the balance is
                    \begin{align*}
                        \frac{34000 - 1000}{1000} \times 20.30 & = \text{RM} 669.90
                    \end{align*}
                    The total basic premium is $220.00 + 669.90 = \text{RM} 889.90$.

                    The NCD discount is $889.90 \times 55 \% = \text{RM} 489.445$.

                    The gross premium is $889.90 - 489.445 = \text{RM} 400.46$. \vspace{0.5cm}

              \item Third party, fire and theft \sol{}

                    The gross premium for third party, fire and theft is $75 \%$ of the
                    comprehensive policy's gross premium, which is $75\% \times 400.46 = \text{RM}
                        300.34$.
          \end{enumerate}
          \vspace{0.5cm}

    \item Mr Zaki has bought a property insurance for his farm. The property insurance
          has a deductible provision of RM650. He has suffered a loss at his farm in
          three consecutive months as shown in the following table.

          \begin{tabular}{|c|c|}
              \hline Month & Loss (RM) \\
              \hline May   & 430       \\
              \hline June  & 760       \\
              \hline July  & 1005      \\
              \hline
          \end{tabular}

          Determine the amount of compensation that can be claimed for the loss in each
          month.

          \sol{}

          In May, the loss is less than the deductible. Hence, the compensation cannot be
          claimed. In June, the loss is $760 - 650 = \text{RM} 110$. In July, the loss is
          $1005 - 650 = \text{RM} 355$.

    \item \begin{enumerate}
              \item Health insurance of Madam Lum has a deductible provision of RM780 per year. She
                    has made a treatment in three consecutive weeks in a private hospital. The
                    table below shows the treatment costs in the three weeks.

                    \begin{tabular}{|c|c|}
                        \hline Week   & Treatment cost (RM) \\
                        \hline First  & 500                 \\
                        \hline Second & $x$                 \\
                        \hline Third  & 210                 \\
                        \hline
                    \end{tabular}

                    If the total amount of compensation claimed by Madam Lum is RM230, what is the
                    treatment cost in the second week? \sol{}
                    \begin{align*}
                        500 + x + 210 - 780 & = 230                   \\
                        x                   & = 230 - 500 - 210 + 780 \\
                                            & = \text{RM} 300
                    \end{align*}

              \item Chan has suceed to claim a compensation of RM9 600 from the insurance company
                    upon the treatment cost. His medical insurance has an annual limit of RM130
                    000. The amount of deductible borne by Chan is RM300. What is the total cost of
                    his treatment? \sol{}

                    The amount of compensation claimed by Chan is $9 600 + 300 = \text{RM} 9 900$.
          \end{enumerate}

    \item Insurable value of Mr Phua's house is RM450 000. He has bought a burglary
          insurance that has a coinsurance provision to insure $85 \%$ of the insurable
          value and a deductible of RM600. Mr Phua's house has suffered a loss of RM8 800
          due to the theft. Calculate the amount of compensation that will be received by
          Mr Phua if he insures his house at
          \begin{enumerate}
              \item an amount of required insurance, \sol{}

                    The amount of required insurance is $450 000 \times 85 \% = \text{RM} 382 500$.

                    The amount of loss is less than the amount of reuqiured insurance. Hence, the
                    amount of compensation that will be received by Mr Phua is $8 800 - 600 =
                        \text{RM} 8 200$.

              \item a sum of RM300 000. \sol{}

                    The amount of insured value is $300 000 < 382 500$. Hence, the amount of
                    compensation is
                    \begin{align*}
                        \frac{300 000}{382 500} \times 8 800 - 600 & = \text{RM} 6 301.96
                    \end{align*}
          \end{enumerate}
    \item Azura is a policyholder of a major medical insurance with a deductible
          provision of $\mathrm{RM}x$ and a 75/25 co-insurance percentage participation
          clause. If Azura's treatment cost is RM5 800 and the treatment cost borne by
          Azura herself is RM1 825, find the value of $x$.
          \begin{align*}
              (5800 - x) \times 25\% + x & = 1825          \\
              1450 + 75\% \times x       & = 1825          \\
              x                          & = \text{RM} 500
          \end{align*}

\end{enumerate}
\end{document}