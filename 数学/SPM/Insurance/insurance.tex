\documentclass{report}
\usepackage[a4paper,margin=1in]{geometry}
\usepackage{amsmath,amssymb}
\usepackage{multirow}
\usepackage{setspace}

\newcommand{\sol}{

    \vspace{0.5em}\textbf{Solution:}\vspace{0.5em}}

\begin{document}
\onehalfspacing
\begin{enumerate}
    \item State three examples of risks that may be faced in our daily life. \sol{}
          \begin{enumerate}
              \item Getting injured when travelling
              \item Car accident
              \item House being burned down by fire
          \end{enumerate}
          \vspace{0.5cm}

    \item State one disaster covered by
          \begin{enumerate}
              \item life insurance
              \item fire insurance
              \item health insurance
          \end{enumerate}
          \sol{}
          \begin{enumerate}
              \item Death
              \item Property damage due to fire
              \item Staying in hospital due to injury
          \end{enumerate}
          \vspace{0.5cm}

    \item The table below shows the premiums for a travel insurance offered by SJM
          Insurance Company for traveling to Japan and Australia.

          \begin{tabular}{|c|c|c|c|c|c|c|}
              \hline \multirow{2}{*}{\begin{tabular}{c}
                                             Number \\
                                             of days
                                         \end{tabular}} & \multicolumn{2}{|c|}{\begin{tabular}{c}
                                                                                   Policyholder \\
                                                                                   (RM)
                                                                               \end{tabular}} & \multicolumn{2}{|c|}{\begin{tabular}{c}
                                                                                                                         Policyholder and \\
                                                                                                                         spouse (RM)
                                                                                                                     \end{tabular}} & \multicolumn{2}{|c|}{\begin{tabular}{c}
                                                                                                                                                               Family             \\
                                                                                                                                                               (Maximum 6 people) \\
                                                                                                                                                               (RM)
                                                                                                                                                           \end{tabular}}                           \\
              \cline { 2 - 7 }                          & Japan                                   & Australia                               & Japan                                   & Australia & Japan & Australia \\
              \hline $1-4$                              & 62                                      & 85                                      & 135                                     & 140       & 288   & 380       \\
              \hline $5-8$                              & 88                                      & 112                                     & 164                                     & 175       & 360   & 480       \\
              \hline $9-12$                             & 110                                     & 140                                     & 185                                     & 198       & 390   & 530       \\
              \hline $13-16$                            & 150                                     & 175                                     & 210                                     & 210       & 460   & 600       \\
              \hline \begin{tabular}{l}
                         Annual
                         premium \\
                         (18-65
                         years
                         old)
                     \end{tabular}                 & 540                                     & 720                                     & -                                       & -         & -     & -              \\
              \hline
          \end{tabular}
          \begin{enumerate}
              \item State the factors that cause the difference in premiums for travel insurance.
                    \sol{}

                    \begin{enumerate}
                        \item The number of days of travel
                        \item The destination of travel
                        \item The number of people insured
                    \end{enumerate}
                    \vspace{0.5cm}

              \item Mr Faud wants to travel to Japan with his wife and two children from 6 January
                    2021 to 16 January 2021. Which package is suitable for $\mathrm{Mr}$ Faud?
                    \sol{}

                    Mr Faud should choose the 9 - 12 days family package for Japan. The total
                    premium is RM390. \vspace{0.5cm}

              \item Hui Ling wants to travel to Australia for 5 days every month to do an
                    investigation. Which package is suitable for Hui Ling? Give your reason. \sol{}

                    Hui Ling should choose the annual premium package for Australia. The total
                    premium is RM720. \vspace{0.5cm}
          \end{enumerate}

    \item The table below shows the premium rates for every RM1 000 face value of a life
          insurance offered by Deva Insurance Company.

          \begin{tabular}{|c|c|c|c|c|}
              \hline \multirow{2}{*}{ Age } & \multicolumn{2}{|c|}{ Male } & \multicolumn{2}{c|}{ Female }                              \\
              \cline { 2 - 5 }              & \begin{tabular}{c}
                                                  Non- \\
                                                  smoker
                                              \end{tabular}           & Smoker                        & \begin{tabular}{c}
                                                                                                            Non- \\
                                                                                                            smoker
                                                                                                        \end{tabular} & Smoker          \\
              \hline 25                     & 1.752                        & 1.932                         & 1.412              & 1.681 \\
              \hline 26                     & 1.784                        & 1.985                         & 1.450              & 1.728 \\
              \hline 27                     & 1.810                        & 2.046                         & 1.483              & 1.782 \\
              \hline 28                     & 1.837                        & 2.101                         & 1.510              & 1.829 \\
              \hline 29                     & 1.870                        & 2.153                         & 1.544              & 1.867 \\
              \hline
          \end{tabular}

          Calculate the annual premium needed to be paid by each of the following
          policyholders.
          \begin{enumerate}
              \item Ms Gui is 29 years old and a smoker. She wants to get a life insurance coverage
                    of RM130 000. \sol{}

                    According to the table, the premium rate for Ms Gui is 1.867
                    \begin{align*}
                        \text{Annual premium} & = \frac{\text{Face value}}{1000} \times \text{Premium rate} \\
                                              & = \frac{130 000}{1000} \times 1.867                         \\
                                              & = \text{RM } 242.71
                    \end{align*}

              \item Mr Sahrin is 26 years old and does not smoke. He wants to get a life insurance
                    coverage of RM80 000 and add on a critical illness policy. Deva Insurance
                    Company has offered a critical illness policy to Mr Sahrin with a coverage of
                    $30 \%$ of basic face value and the premium rate for every RM1 000 is RM1.128.
                    \sol{}

                    According to the table, the premium rate for Mr Sahrin is 1.784

                    The amount coverage for critical illness is $30 \%$ of basic face value, which
                    is
                    \begin{align*}
                        \text{Amount coverage} & = \frac{30}{100} \times 80 000 \\
                                               & = \text{RM } 24 000
                    \end{align*}
                    \begin{align*}
                        \text{Annual premium} & = \text{Annual basic premium} + \text{Additional critical illness premium} \\
                                              & = \frac{80 000}{1000} \times 1.784  + \frac{24 000}{1000} \times 1.128     \\
                                              & = \text{RM } 169.80
                    \end{align*}
          \end{enumerate}

    \item Mr Fazli wants to buy a motor insurance for his car in Sabah. The following
          shows the information of his car.

          \begin{tabular}{|lll|}
              \hline Age of vehicle & $:$ & 5 years            \\
              Engine capacity       & $:$ & $1600 \mathrm{cc}$ \\
              NCD                   & $:$ & $45 \%$            \\
              Sum insured           & $:$ & RM52 000           \\
              \hline
          \end{tabular}

          Calculate the gross premium for Mr Fazli's car under each of the following
          policies.
          \begin{enumerate}
              \item Comprehensive \sol{}

                    According to the premium rate table under the Motor Tariff, the basic premium
                    for the first RM 1000 is RM 220.50. The basic premium for the balance is
                    \begin{align*}
                        \text{Basic premium} & = \frac{52 000 - 1000}{1000} \times 20.30 \\
                                             & = \text{RM } 1 035.30
                    \end{align*}
                    \vspace{-2em}
                    \begin{align*}
                        \text{Total basic premium} & = 220.00 + 1 035.30   \\
                                                   & = \text{RM } 1 255.30
                    \end{align*}
                    \vspace{-2em}
                    \begin{align*}
                        \text{NCD 45\%} & = \frac{45}{100} \times 1 255.30 \\
                                        & = \text{RM } 564.885
                    \end{align*}
                    \vspace{-2em}
                    \begin{align*}
                        \text{Gross premium} & = 1 255.80 - 564.885 \\
                                             & = \text{RM } 690.42
                    \end{align*}

              \item Third party \sol{}

                    According to the premium rate table under the Motor Tariff, the basic premium
                    is RM 75.60.
                    \begin{align*}
                        \text{NCD 45\%}      & = \frac{45}{100} \times 75.60 \\
                                             & = \text{RM } 34.02            \\
                        \text{Gross premium} & = 75.60 - 34.02               \\
                                             & = \text{RM } 41.58
                    \end{align*}

              \item Third party, fire and theft \sol{}
                    \begin{align*}
                        \text{Gross premium} & = \text{75\% of comprehensive policy's gross premium} \\
                                             & = \frac{75}{100} \times 690.42                        \\
                                             & = \text{RM } 517.82
                    \end{align*}
          \end{enumerate}

    \item Determine whether each of the following policyholders can claim the
          compensation from the loss suffered. Hence, state the amount of compensation
          that can be claimed.
          \begin{enumerate}
              \item Motor insurance for Tsu Chin's car has a deductible provision of RM300. Tsu
                    Chin has suffered an accident that causes a loss of RM648. \sol{}

                    The amount of loss exceeds the deductible amount. Therefore, Tsu Chin can claim
                    the compensation. The amount of compensation that can be claimed is $648 - 300
                        = \text{RM } 348$.

              \item Motor insurance for Madam Gayah's car has a deductible provision of RM500.
                    Madam Gayah has suffered an accident that causes a loss of RM290. \sol{}

                    The amount of loss does not exceed the deductible amount. Therefore, Madam
                    Gayah cannot claim the compensation.
          \end{enumerate}

    \item Mariana bought a motor insurance for her car with a deductible provision of
          RM400. Due to the flood, Mariana has suffered losses in three months as shown
          in the following table.

          \begin{tabular}{|c|c|}
              \hline Month    & Loss (RM) \\
              \hline January  & 500       \\
              \hline May      & 340       \\
              \hline November & 875       \\
              \hline
          \end{tabular}

          Determine whether the loss can be claimed in each month. Hence, state the total
          amount of compensation that can be claimed. \sol{}

          \begin{enumerate}
              \item January \sol{}

                    The amount of loss exceeds the deductible amount. Therefore, Mariana can claim
                    the compensation. The amount of compensation that can be claimed is $500 - 400
                        = \text{RM } 100$.

              \item May \sol{}

                    The amount of loss does not exceed the deductible amount. Therefore, Mariana
                    cannot claim the compensation.

              \item November \sol{}

                    The amount of loss exceeds the deductible amount. Therefore, Mariana can claim
                    the compensation. The amount of compensation that can be claimed is $875 - 400
                        = \text{RM } 475$.

              \item Total amount of compensation \sol{}

                    The total amount of compensation that can be claimed is $100 + 475 = \text{RM }
                        575$.
          \end{enumerate}`'

    \item Calculate the amount of compensation that will be paid to each of the following
          health insurance policyholders.
          \begin{enumerate}
              \item Health insurance of Mr Suman has a deductible provision of RM450 per year. He
                    has made a treatment in three consecutive months for his illness in a private
                    hospital. The table below shows the treatment costs in the three months.

                    \begin{tabular}{|c|c|}
                        \hline Month     & Treatment cost (RM) \\
                        \hline July      & 830                 \\
                        \hline August    & 360                 \\
                        \hline September & 250                 \\
                        \hline
                    \end{tabular}
                    \sol{}
                    \begin{align*}
                        \text{Accumulated treatment cost} & = 830 + 360 + 250  \\
                                                          & = \text{RM } 1 440 \\
                        \text{Amount of compensation}     & = 1 440 - 450      \\
                                                          & = \text{RM } 990
                    \end{align*}
              \item Agatha has a medical insurance with an annual limit of RM50 000. The amount of
                    deductible borne by Agatha is RM1 200 per year. She has been treated at a
                    specialist hospital with a medical cost of RM40 000. \sol{}
                    \begin{align*}
                        \text{Amount of compensation} & = 40 000 - 1 200    \\
                                                      & = \text{RM } 38 800
                    \end{align*}

          \end{enumerate}
          \vfill\null

    \item Insurable value of Madam Loke's house is RM360 000. She has bought a fire
          insurance that has a co-insurance provision to insure $90 \%$ of the insurable
          value of her house and a deductible of RM1 600. Madam Loke's house caught fire
          and the amount of loss is RM67 000. Calculate the amount of compensation that
          will be received by Madam Loke if she insures her house at
          \begin{enumerate}
              \item an amount of required insurance, \sol{}
                    \begin{align*}
                        \text{Amount of required insurance} & = 90 \% \times 360 000 \\
                                                            & = \text{RM } 324 000
                    \end{align*}
                    The loss does not exceed the deductible amount. Therefore,
                    \begin{align*}
                        \text{Amount of compensation} & = 67 000 - 1 600    \\
                                                      & = \text{RM } 65 400
                    \end{align*}
                    \vfill\null

                    \newpage
              \item a sum of RM180 000. \sol{}

                    The amount of insured value is less than the required insurance. Therefore,
                    \begin{align*}
                        \text{Amount of compensation} & = \frac{\text{Amount of insured value}}{\text{Amount of required insurance}} \times \text{Amount of loss} - \text{Deductible} \\
                                                      & = \frac{180 000}{324 000} \times 67 000 - 1 600                                                                               \\
                                                      & = \text{RM } 35 622.22
                    \end{align*}
          \end{enumerate}

    \item Awang Farid has a major medical insurance policy with a deductible provision of
          RM1 000 and a 90/10 co-insurance percentage participation clause. If the
          treatment cost of Awang Farid is RM26 400, calculate the amount of treatment
          costs borne by insurance company and Awang Farid respectively. \sol{}
          \begin{align*}
              \text{Treatment cost after deductible}           & = 26 400 - 1 000      \\
                                                               & = \text{RM } 25 400   \\
              \text{Treatment cost borne by insurance company} & = 90 \% \times 25 400 \\
                                                               & = \text{RM } 22 860   \\
              \text{Treatment cost borne by Awang Farid}       & = 10 \% \times 25 400 \\
                                                               & = \text{RM } 2 540
          \end{align*}
\end{enumerate}
\end{document}