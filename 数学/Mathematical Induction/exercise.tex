\documentclass{report}

\usepackage{amsmath}
\usepackage{amssymb}
\usepackage{setspace}
\usepackage{enumitem}
\usepackage{fontspec}
\usepackage[total={6.6in,9.2in}]{geometry}

\setmainfont{Times New Roman}

\title{\Huge{\textbf{Mathematical Induction}}}
\author{Melvin Chia}
\date{10 June 2023}

\begin{document}
\onehalfspacing

\subsection*{Exercise 1a}
Use mathematical induction to prove the following statements (1 - 7).
\begin{enumerate}
    \item $1 + 2 + 3 + \cdots + n = \dfrac{n(n + 1)}{2}$
          \vfill
    \item $1 \cdot 2 + 2 \cdot 3 + 3 \cdot 4 + \cdots + n(n + 1) = \dfrac{1}{3}n(n + 1)(n + 2)$
          \vfill
          \newpage

    \item $1^2 + 3^2 + 5^2 + \cdots + (2n - 1)^2 = \dfrac{n(4n^2 - 1)}{3}$
          \vfill
    \item $1 \cdot 4 + 2 \cdot 7 + 3 \cdot 10 + \cdots + n(3n + 1) = n(n + 1)^2$
          \vfill
          \newpage
    \item $2 \cdot 2 + 3 \cdot 2^2 + 4 \cdot 2^3 + \cdots + (n + 1) \cdot 2^n = n \cdot 2^{n + 1}$
          \vfill
    \item $\dfrac{1}{1 \cdot 3} + \dfrac{1}{3 \cdot 5} + \dfrac{1}{5 \cdot 7} + \cdots + \dfrac{1}{(2n - 1)(2n + 1)} = \dfrac{n}{2n + 1}$
          \vfill
          \newpage
\end{enumerate}

\newpage
\section*{Exercise 1b}
Prove the following statements using the method of mathematical induction:
\begin{enumerate}
    \item $-1 + 3 - 5 + \cdots + (-1)^n(2n - 1) = (-1)^n \cdot n$
          \vfill
    \item $\sum (5n - 1) = \dfrac{n(5n + 3)}{2}$, $n \in \mathbb{N}$
          \vfill
          \newpage
    \item $\sum 3^{n-1} = \dfrac{3^n - 1}{2}$, $n \in \mathbb{N}$
          \vfill
    \item $2^n > n^2$, $n > 4$ and $n \in \mathbb{N}$
          \vfill
          \newpage
    \item $2^n + 2 > n^2$, $n \in \mathbb{N}$
          \vfill
    \item The sum of the interior angles of a polygon with $n$ sides is $(n-2)\pi$, $n
              \geq 3$. \vfill \newpage
    \item $(a^n - b^n)$ is divisible by $(a - b)$.
          \vfill
    \item $x^{n+2} + (x+1)^{2n + 1}$ is divisible by $x^2 + x + 1$, $n \geq 0$ and $n \in
              \mathbb{Z}$.
          \vfill
          \newpage
    \item $x^n + 5n$ ($n \in \mathbb{N}$) is divisible by $6$.
          \vfill
    \item The sum of the cube of three consecutive integers is divisible by $9$. \vfill
          \newpage
    \item For all natural number $n$, $9^n - 8n - 1$ is a multiple of $64$, $n \geq 2$.
          \vfill
    \item Determine the general formula for the following, and prove it using the method
          of mathematical induction.
          \begin{flalign*}
              1\ \ \ \ \ \ \ \ \ \ \ \ \ \ \  & = 1   \\
              3 + 5\ \ \ \ \ \ \ \ \ \ \ \    & = 8   \\
              7 + 9 + 11\ \ \ \ \ \ \ \       & = 27  \\
              13 + 15 + 17 + 19\ \ \ \        & = 64  \\
              21 + 23 + 25 + 27 + 29          & = 125 \\
          \end{flalign*}
          \vfill
\end{enumerate}
\end{document}