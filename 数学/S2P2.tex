% chktex-file 2% chktex-file 29
% chktex-file 13
\documentclass{report}
\usepackage{setspace}
\usepackage[a4paper, total={7in, 10in}]{geometry}
\usepackage[fleqn]{amsmath}
\usepackage{empheq}
\usepackage{amssymb}
\usepackage{amsthm}
\usepackage{gensymb}
\usepackage[fleqn]{cases}
\usepackage{multicol}
\usepackage{color}
\usepackage{stix}
\usepackage{chngcntr}
\usepackage{tikz}
\usepackage{enumitem}
\usepackage{pgfplots}
\usepackage{etoolbox}
\usepackage{tikz-3dplot}
\usepackage{tkz-euclide}
\usepackage{graphicx}
\usepackage{enumitem}

\def\nswe#1#2#3{#1\,$#2^\circ\,#3'$}
\graphicspath{ {./assets/} }
\usetikzlibrary{calc,matrix,arrows}
\usetikzlibrary{decorations.pathmorphing,patterns, calligraphy, perspective,backgrounds}

\tikzset{
  right angle quadrant/.code={
      \pgfmathsetmacro\quadranta{{1,1,-1,-1}[#1-1]}     % Arrays for selecting quadrant
      \pgfmathsetmacro\quadrantb{{1,-1,-1,1}[#1-1]}},
  right angle quadrant=1, % Make sure it is set, even if not called explicitly
  right angle length/.code={\def\rightanglelength{#1}},   % Length of symbol
  right angle length=2ex, % Make sure it is set...
  right angle symbol/.style n args={3}{
      insert path={
          let \p0 = ($(#1)!(#3)!(#2)$) in     % Intersection
          let \p1 = ($(\p0)!\quadranta*\rightanglelength!(#3)$), % Point on base line
          \p2 = ($(\p0)!\quadrantb*\rightanglelength!(#2)$) in % Point on perpendicular line
          let \p3 = ($(\p1)+(\p2)-(\p0)$) in  % Corner point of symbol
          (\p1) -- (\p3) -- (\p2)
        }
    }
}

\counterwithout{equation}{chapter}
\setlength{\columnseprule}{1pt}
\setlength{\columnsep}{24pt}
\setcounter{chapter}{17}
\hfuzz=100pt

\newcommand{\pgfplotsdrawaxis}{\pgfplots@draw@axis}
\makeatother
\pgfplotsset{only axis on top/.style={axis on top=false, after end axis/.code={
          \pgfplotsset{axis line style=opaque, ticklabel style=opaque, tick style={thick,opaque},
            grid=none}\pgfplotsdrawaxis}}}

\newtheorem{theorem}{Theorem}

\begin{document}
\makeatletter
\newcommand{\newparallel}{\mathrel{\mathpalette\new@parallel\relax}}
\newcommand{\new@parallel}[2]{%
  \begingroup
  \sbox\z@{$#1T$}% get the height of an uppercase letter
  \resizebox{!}{\ht\z@}{\raisebox{\depth}{$\m@th#1/\mkern-5mu/$}}%
  \endgroup
}
\makeatother

\newcommand{\planelineinter}[5]% a, b, c, p as {a_x,a_y,a_z}, coordinate name
{   \foreach \a [count=\k] in {#1}
    { \ifthenelse{\k=1}{\xdef\tempxa{\a}}
      \ifthenelse{\k=2}{\xdef\tempya{\a}}
      \ifthenelse{\k=3}{\xdef\tempza{\a}}
    }
  \foreach \b [count=\k] in {#2}
    { \ifthenelse{\k=1}{\xdef\tempxb{\b}}
      \ifthenelse{\k=2}{\xdef\tempyb{\b}}
      \ifthenelse{\k=3}{\xdef\tempzb{\b}}
    }
  \foreach \c [count=\k] in {#3}
    { \ifthenelse{\k=1}{\xdef\tempxc{\c}}
      \ifthenelse{\k=2}{\xdef\tempyc{\c}}
      \ifthenelse{\k=3}{\xdef\tempzc{\c}}
    }
  \foreach \p [count=\k] in {#4}
    { \ifthenelse{\k=1}{\xdef\tempxp{\p}}
      \ifthenelse{\k=2}{\xdef\tempyp{\p}}
      \ifthenelse{\k=3}{\xdef\tempzp{\p}}
    }
  \pgfmathsetmacro{\abx}{\tempxb-\tempxa}
  \pgfmathsetmacro{\aby}{\tempyb-\tempya}
  \pgfmathsetmacro{\abz}{\tempzb-\tempza}
  \pgfmathsetmacro{\acx}{\tempxc-\tempxa}
  \pgfmathsetmacro{\acy}{\tempyc-\tempya}
  \pgfmathsetmacro{\acz}{\tempzc-\tempza}
  \pgfmathsetmacro{\nx}{\aby*\acz-\abz*\acy}
  \pgfmathsetmacro{\ny}{\abz*\acx-\abx*\acz}
  \pgfmathsetmacro{\nz}{\abx*\acy-\aby*\acx}
  \pgfmathsetmacro{\d}{(\nx+\ny+\nz)/(\nx*\tempxp+\ny*\tempyp+\nz*\tempzp)}
  \path (0,0,0) -- (#4) coordinate[pos=\d] (#5);
}

% golden ratio and inverse golden ratio
\pgfmathsetmacro{\gr}{(1+sqrt(5))/2}
\pgfmathsetmacro{\igr}{2/(1+sqrt(5))}

%choose axis angles
\newcommand{\xangle}{0}
\newcommand{\yangle}{90}
\newcommand{\zangle}{225}

%choose axis lengths
\newcommand{\xlength}{1}
\newcommand{\ylength}{1}
\newcommand{\zlength}{0.5}

\pgfmathsetmacro{\xx}{\xlength*cos(\xangle)}
\pgfmathsetmacro{\xy}{\xlength*sin(\xangle)}
\pgfmathsetmacro{\yx}{\ylength*cos(\yangle)}
\pgfmathsetmacro{\yy}{\ylength*sin(\yangle)}
\pgfmathsetmacro{\zx}{\zlength*cos(\zangle)}
\pgfmathsetmacro{\zy}{\zlength*sin(\zangle)}

\newcommand{\sol}[1]{

  \noindent \textbf{Sol.}
}
\newcommand{\prooff}[1]{

  \noindent \textbf{Proof.}
}
\newcommand\m[1]{\begin{pmatrix}#1\end{pmatrix}}
\newcommand\vm[1]{\begin{vmatrix}#1\end{vmatrix}}
\newenvironment{amatrix}[1]{%
  \left(\begin{array}{@{}*{#1}{c}|c@{}}
    }{%
  \end{array}\right)
}
\newenvironment{cequation}{
  \makeatletter
  \setbool{@fleqn}{false}
  \makeatother
  \begin{equation*}
    }{\end{equation*}}

\begin{titlepage}
  \raggedleft{}
  \rule{1pt}{\textheight}
  \hspace{0.02\textwidth}
  \parbox[b]{0.75\textwidth}{

  {\fontsize{40}{60}\selectfont\bfseries Mathematics}\\[2\baselineskip]
  {\huge\textit{Senior 2 Part II}}\\[4\baselineskip]
  {\Large\textsc{Melvin Chia}}

  \vspace{0.5\textheight}

  {\noindent Started on 1 January 2023}\\[\baselineskip]
  {\noindent Finished on ...}\\[\baselineskip]}

\end{titlepage}

\doublespacing{}
\tableofcontents
\singlespacing{}
\newpage

\begin{multicols}{2}
  \setstretch{1.25}
  \chapter{Statistics}

  \section{Basic Concepts}

  Statistics mainly study how to collect, organize, summarize, and interpret
  data. It is a branch of mathematics that deals with the collection, analysis,
  interpretation, and presentation of data. It is used to answer questions about
  the data and to make decisions based on the data.

  \subsection*{Population and Sample}

  In statistics, a population is the entire group of individuals that we are
  studying, and the units that form a population are called individuals or
  elements. A sample is a subset of the population. The number of elements in a
  sample is called the sample size. For example: select 20 of the 4,000 senior
  high school mathematics UEC exam papers and record their scores:
  \begin{flalign*}
    72 \qquad 80 \qquad 96 \qquad 20 \qquad 42 \\
    75 \qquad 60 \qquad 92 \qquad 18 \qquad 53 \\
    82 \qquad 77 \qquad 53 \qquad 29 \qquad 34 \\
    57 \qquad 79 \qquad 82 \qquad 90 \qquad 41
  \end{flalign*}
  Here, the population is the 4,000 scores, each of which is an element of the population. The sample is the 20 scores, the sample size is 20.

  \subsection*{Census and Sample Survey}

  The way of surveying can be divided into two types: census and sample survey. A
  census is a survey in which every element of the population is included in the
  sample. For example: national census. The data collected in a census is more
  accurate and reliable, but it is very expensive and time-consuming.

  A sample survey is a survey in which only a part of the population is included
  in the sample. Researchers can use a sample survey to estimate the
  characteristics of the population. For example: a light bulb manufacturer
  produces a lot of light bulbs, thus it is impossible to test every single light
  bulb. The manufacturer can randomly select a sample of light bulbs and test
  them.

  \section{Data Processing}

  Data that are collected must be processed before they can be analyzed.

  \subsection*{Frequency Distribution}

  When the possible values of a dataset are not too many, we can use a frequency
  distribution table to organize the data. The frequency distribution table is a
  table that shows the frequency of each value in a dataset. The frequency of a
  value is the number of times that value appears in the dataset.

  When there are too many possible values, we must group the values into classes.
  Before grouping the values, we must first determine the range of the values,
  aka the difference between the largest and smallest values, then determine the
  number of classes. The number of classes should be determined according to the
  purpose of the study and the identity of the data. After classifying the data,
  the range of each group is called the class interval. Typically, the class
  interval is the same for all classes, and must be greater than the number of
  classes divided by the range of the data. After the number and interval of the
  classes are determined, we can arrange the frequency of each class in a
  frequency distribution table.

  Take 100 sample from a population of some kind of component, their weight (in
  $g$), are as below:

  \begin{flalign*}
    1.36 & \qquad 1.49 \qquad 1.43 \qquad 1.41 \qquad 1.37 \qquad 1.40 \\
    1.32 & \qquad 1.42 \qquad 1.47 \qquad 1.39 \qquad 1.41 \qquad 1.36 \\
    1.40 & \qquad 1.34 \qquad 1.42 \qquad 1.42 \qquad 1.45 \qquad 1.35 \\
    1.42 & \qquad 1.39 \qquad 1.44 \qquad 1.42 \qquad 1.39 \qquad 1.42 \\
    1.42 & \qquad 1.30 \qquad 1.34 \qquad 1.42 \qquad 1.37 \qquad 1.36 \\
    1.37 & \qquad 1.34 \qquad 1.37 \qquad 1.37 \qquad 1.44 \qquad 1.45 \\
    1.32 & \qquad 1.48 \qquad 1.40 \qquad 1.45 \qquad 1.39 \qquad 1.46 \\
    1.39 & \qquad 1.53 \qquad 1.36 \qquad 1.48 \qquad 1.40 \qquad 1.39 \\
    1.38 & \qquad 1.40 \qquad 1.36 \qquad 1.45 \qquad 1.50 \qquad 1.43 \\
    1.38 & \qquad 1.43 \qquad 1.41 \qquad 1.48 \qquad 1.39 \qquad 1.45
  \end{flalign*}
  \begin{flalign*}
    1.37 & \qquad 1.37 \qquad 1.39 \qquad 1.45 \qquad 1.31 \qquad 1.41 \\
    1.44 & \qquad 1.44 \qquad 1.42 \qquad 1.47 \qquad 1.35 \qquad 1.36 \\
    1.39 & \qquad 1.40 \qquad 1.38 \qquad 1.35 \qquad 1.38 \qquad 1.43 \\
    1.42 & \qquad 1.42 \qquad 1.42 \qquad 1.40 \qquad 1.41 \qquad 1.37 \\
    1.46 & \qquad 1.36 \qquad 1.37 \qquad 1.27 \qquad 1.37 \qquad 1.38 \\
    1.42 & \qquad 1.34 \qquad 1.43 \qquad 1.42 \qquad 1.41 \qquad 1.41 \\
    1.44 & \qquad 1.48 \qquad 1.55 \qquad 1.39
  \end{flalign*}

  In the dataset above, the minimum value is $1.27$ and the maximum value is
  $1.55$.

  $\therefore $ The range of the data is $1.55 - 1.27 = 0.28$.

  If we classify the data into 10 classes, then the class interval must be
  greater than $\frac{0.28}{10} = 0.028$. Thus, we can use a class interval of
  $0.03$.

  Let the lower limit of the first class be $1.27$, then the lower limit of the
  second class is $1.27 + 0.03 = 1.30$.

  Since all the values in the dataset are of 2 decimal places, the upper limit of
  the first class is should be $1.29$. By the same logic, we can get all the
  classes: $1.27 - 1.29$, $1.30 - 1.32$, $\cdots$, $1.54 - 1.56$.

  Now we can arrange the data into the frequency distribution table:

  \begin{center}
    \begin{tabular}{|c|c|}
      \hline
      Weight $m$($g$) & Frequency \\
      \hline
      $1.27 - 1.29$   & 1         \\
      $1.30 - 1.32$   & 4         \\
      $1.33 - 1.35$   & 7         \\
      $1.36 - 1.38$   & 22        \\
      $1.39 - 1.41$   & 24        \\
      $1.42 - 1.44$   & 24        \\
      $1.45 - 1.47$   & 10        \\
      $1.48 - 1.50$   & 6         \\
      $1.51 - 1.53$   & 1         \\
      $1.54 - 1.56$   & 1         \\
      \hline
    \end{tabular}
  \end{center}

  In the example above, we assume that the weight of the components is accurate
  to 2 decimal places. Hence, if a component has a weight of $1.443g$, it is
  rounded to $1.44g$, thus it belongs to the class $1.42 - 1.44$. Hence, the
  actual range of the first class $1.27 - 1.29$ is $1.265 \leq m < 1.295$,
  written as $1.265 - 1.295$, while $1.265$ and $1.295$ are the boundaries of the
  first class, $1.265$ is the lower boundary and $1.295$ is the upper boundary.
  The mean of the lower boundary and upper boundary of a class is called the
  class midpoint. For example, the class midpoint of the first class is
  $\frac{1.265 + 1.295}{2} = 1.28$.

  When we are analyzing the data data that have been classified into classes, the
  midpoint of each class is used as the representative value of the class. Thus,
  we should try our best to make the data-intensive place the group midpoint when
  choosing the class interval and boundaries, so that the data can be analyzed
  more precisely.

  The distribution of frequency can be represented by a histogram or a frequency
  polygon.

  The histogram is a row of continuous bars, the bottom side of each bar on the
  x-axis. For unclassified data, the bottom side of each bar is marked with the
  values, while the height of each bar is the frequency of the corresponding
  value. For classified data, the bottom side of each bar is marked with the
  boundaries of the corresponding class, while the area of each bar must be
  proportional to the frequency of the corresponding class. When the class
  interval of each class is the same, we can use the frequency of each class as
  the height of the bar.

  The frequency polygon is a continuous line graph, the x-axis is the midpoint of
  each class, and the y-axis is the frequency of each class. To draw a frequency
  polygon, we plot each point, including the point before the first class and the
  point after the last class that uses $0$ as their frequency, and then connect
  the points with a continuous line.

  \subsection*{Accumulative Frequency Distribution}

  \section{Central Tendency}

  \section{Measures of Dispersion}

  \section{Coefficient of Variation}

  \section{Correlation and Correlation Coefficient}

  \section{Statistical Index}

  \chapter{Permutations and Combinations}

  \section{Addition and Multiplication Principles}

  \section{Permutations and Permutation Formula}

  \section{Circular Permutations}

  \section{Full Permutations of Inexactly Distinct Elements}

  \section{Permutations with Repetition}

  \section{Combinations and Combination Formula}

  \chapter{Bionomial Theorem}

  \section{Bionomial Theorem when $n$ is a Natural Number}

  \section{General Form of Bionomial Expansion}

  \chapter{Probability}

  \section{Sample Space and Events}

  \section{Definition of Probability}

  \section{Addition Rule}

  \section{Multiplication Rule}

  \section{Mathematical Expectation}

  \section{Normal Distribution}
\end{multicols}

\end{document}