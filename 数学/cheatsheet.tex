% chktex-file 2% chktex-file 29
% chktex-file 13
\documentclass{report}
\usepackage{setspace}
\usepackage[a4paper, total={7in, 10in}]{geometry}
\usepackage[fleqn]{amsmath}
\usepackage{empheq}
\usepackage{amssymb}
\usepackage{amsthm}
\usepackage{gensymb}
\usepackage[fleqn]{cases}
\usepackage{multicol}
\usepackage{color}
\usepackage{stix}
\usepackage{chngcntr}
\usepackage{tikz}
\usepackage{enumitem}
\usepackage{pgfplots}
\usepackage{etoolbox}
\usepackage{tikz-3dplot}
\usepackage{tkz-euclide}
\usepackage{graphicx}
\usepackage{enumitem}

\def\nswe#1#2#3{#1\,$#2^\circ\,#3'$}
\graphicspath{ {./assets/} }
\usetikzlibrary{calc,matrix,arrows}
\usetikzlibrary{decorations.pathmorphing,patterns, calligraphy, perspective,backgrounds}

\tikzset{
    right angle quadrant/.code={
            \pgfmathsetmacro\quadranta{{1,1,-1,-1}[#1-1]}     % Arrays for selecting quadrant
            \pgfmathsetmacro\quadrantb{{1,-1,-1,1}[#1-1]}},
    right angle quadrant=1, % Make sure it is set, even if not called explicitly
    right angle length/.code={\def\rightanglelength{#1}},   % Length of symbol
    right angle length=2ex, % Make sure it is set...
    right angle symbol/.style n args={3}{
            insert path={
                    let \p0 = ($(#1)!(#3)!(#2)$) in     % Intersection
                    let \p1 = ($(\p0)!\quadranta*\rightanglelength!(#3)$), % Point on base line
                    \p2 = ($(\p0)!\quadrantb*\rightanglelength!(#2)$) in % Point on perpendicular line
                    let \p3 = ($(\p1)+(\p2)-(\p0)$) in  % Corner point of symbol
                    (\p1) -- (\p3) -- (\p2)
                }
        }
}

\counterwithout{equation}{chapter}
\setlength{\columnseprule}{1pt}
\setlength{\columnsep}{24pt}
\setcounter{chapter}{16}
\hfuzz=100pt

\newcommand{\pgfplotsdrawaxis}{\pgfplots@draw@axis}
\makeatother
\pgfplotsset{only axis on top/.style={axis on top=false, after end axis/.code={
                    \pgfplotsset{axis line style=opaque, ticklabel style=opaque, tick style={thick,opaque},
                        grid=none}\pgfplotsdrawaxis}}}

\newtheorem{theorem}{Theorem}

\begin{document}\makeatletter
\newcommand{\newparallel}{\mathrel{\mathpalette\new@parallel\relax}}
\newcommand{\new@parallel}[2]{%
    \begingroup
    \sbox\z@{$#1T$}% get the height of an uppercase letter
    \resizebox{!}{\ht\z@}{\raisebox{\depth}{$\m@th#1/\mkern-5mu/$}}%
    \endgroup
}
\makeatother

\newcommand{\planelineinter}[5]% a, b, c, p as {a_x,a_y,a_z}, coordinate name
{   \foreach \a [count=\k] in {#1}
        { \ifthenelse{\k=1}{\xdef\tempxa{\a}}
            \ifthenelse{\k=2}{\xdef\tempya{\a}}
            \ifthenelse{\k=3}{\xdef\tempza{\a}}
        }
    \foreach \b [count=\k] in {#2}
        { \ifthenelse{\k=1}{\xdef\tempxb{\b}}
            \ifthenelse{\k=2}{\xdef\tempyb{\b}}
            \ifthenelse{\k=3}{\xdef\tempzb{\b}}
        }
    \foreach \c [count=\k] in {#3}
        { \ifthenelse{\k=1}{\xdef\tempxc{\c}}
            \ifthenelse{\k=2}{\xdef\tempyc{\c}}
            \ifthenelse{\k=3}{\xdef\tempzc{\c}}
        }
    \foreach \p [count=\k] in {#4}
        { \ifthenelse{\k=1}{\xdef\tempxp{\p}}
            \ifthenelse{\k=2}{\xdef\tempyp{\p}}
            \ifthenelse{\k=3}{\xdef\tempzp{\p}}
        }
    \pgfmathsetmacro{\abx}{\tempxb-\tempxa}
    \pgfmathsetmacro{\aby}{\tempyb-\tempya}
    \pgfmathsetmacro{\abz}{\tempzb-\tempza}
    \pgfmathsetmacro{\acx}{\tempxc-\tempxa}
    \pgfmathsetmacro{\acy}{\tempyc-\tempya}
    \pgfmathsetmacro{\acz}{\tempzc-\tempza}
    \pgfmathsetmacro{\nx}{\aby*\acz-\abz*\acy}
    \pgfmathsetmacro{\ny}{\abz*\acx-\abx*\acz}
    \pgfmathsetmacro{\nz}{\abx*\acy-\aby*\acx}
    \pgfmathsetmacro{\d}{(\nx+\ny+\nz)/(\nx*\tempxp+\ny*\tempyp+\nz*\tempzp)}
    \path (0,0,0) -- (#4) coordinate[pos=\d] (#5);
}

% golden ratio and inverse golden ratio
\pgfmathsetmacro{\gr}{(1+sqrt(5))/2}
\pgfmathsetmacro{\igr}{2/(1+sqrt(5))}

%choose axis angles
\newcommand{\xangle}{0}
\newcommand{\yangle}{90}
\newcommand{\zangle}{225}

%choose axis lengths
\newcommand{\xlength}{1}
\newcommand{\ylength}{1}
\newcommand{\zlength}{0.5}

\pgfmathsetmacro{\xx}{\xlength*cos(\xangle)}
\pgfmathsetmacro{\xy}{\xlength*sin(\xangle)}
\pgfmathsetmacro{\yx}{\ylength*cos(\yangle)}
\pgfmathsetmacro{\yy}{\ylength*sin(\yangle)}
\pgfmathsetmacro{\zx}{\zlength*cos(\zangle)}
\pgfmathsetmacro{\zy}{\zlength*sin(\zangle)}

\newcommand{\sol}[1]{

    \noindent \textbf{Sol.}
}
\newcommand{\prooff}[1]{

    \noindent \textbf{Proof.}
}
\newcommand\m[1]{\begin{pmatrix}#1\end{pmatrix}}
\newcommand\vm[1]{\begin{vmatrix}#1\end{vmatrix}}
\newenvironment{amatrix}[1]{%
    \left(\begin{array}{@{}*{#1}{c}|c@{}}
        }{%
    \end{array}\right)
}
\newenvironment{cequation}{
    \makeatletter
    \setbool{@fleqn}{false}
    \makeatother
    \begin{equation*}
        }{\end{equation*}}

\begin{titlepage}
    \raggedleft{}
    \rule{1pt}{\textheight}
    \hspace{0.02\textwidth}
    \parbox[b]{0.75\textwidth}{

    {\Huge\bfseries Solution Book of \\[0.5\baselineskip] Mathematic}\\[2\baselineskip]
    {\large\textit{Senior 2 Part I}}\\[4\baselineskip]
    {\Large\textsc{MELVIN CHIA}}

    \vspace{0.5\textheight}

    {\noindent Started on 9 October 2022}\\[\baselineskip]
    {\noindent Finished on 28 December 2022}\\[\baselineskip]}

\end{titlepage}

\doublespacing{}
\tableofcontents
\singlespacing{}
\newpage

\chapter*{Introduction}
\addcontentsline{toc}{chapter}{Introduction} \markboth{INTRODUCTION}{}

\doublespacing{}
\section*{Why this book?}

Back in October 2022, I decided to complete every single question inside the
Senior 2 Mathematics Part I textbook published by DongZong. You might wonder
why I decided to do this? The short answer is: I love Math. The long answer is?

Well, you see, the year 2022 has almost come to an end. There's only a year
left before I sit on the SPM examination. There's really no more time to waste,
so I decided to complete the DongZong textbook in advance so that I can focus
on studying for the SPM for the entire first half year of the year 2023. The
syllabus for the SPM is completely different as DongZong, especially in Math,
and school doesn't really put in enough time and effort to teach us the SPM
syllabus (I've heard one of senior 2 friend saying that the school only gives
you few exercises paper, which is obviously insufficient), so the only way or
be to achieve a good grade in the SPM is to study on my own.

Back when I started grinding the questions, I was using a traditional books and
pen. Just a few practices later, I stumbled upon a markdown language called
LaTeX that allows me to typeset my notes in a professional manner. I was so
amazed by the power of LaTeX that I decided to use it throughout my entire
grinding journey and print it out as a book after completing the entire
textbook.

\section*{Disclaimer}

This book is just my own solution and notes to the textbook, all the solutions
it is not guaranteed to be correct, and there might be some missing stuff that
I've forgotten to add into it. I am not responsible for any consequences caused
by using this book. And please, don't copy my solutions, it's not going to help
you in the long run.

\section*{Acknowledgements}

I would like to thank myself for wasting my entire year-end holiday to complete
this book. I would also like to thank my parents for sponsoring the printing of
this book. Special tahnks to my friends for chatting with me and eliminate part
of my boredom. Thanks to KUMON for building a good foundation for me in Math.

\section*{Timelapse Video}

I've recorded timelapse videos of me solving the questions. Scan the QR code on
the next page to watch the videos.

\singlespacing{}

\newpage

\topskip0pt
\vspace*{\fill}
\begin{center}
    \includegraphics[scale=0.1]{qr-code.png}
\end{center}
\vskip0.5cm
\centering \huge{Timelapse of me grinding the questions}

\vspace*{\fill}

\newpage

\end{document}