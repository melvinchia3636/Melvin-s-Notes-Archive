\documentclass{report}
\usepackage[total={6.5in,9in}]{geometry}

\usepackage[fleqn]{amsmath}
\usepackage{amssymb}
\usepackage{enumitem}
\usepackage{multicol}

\begin{document}
\section*{Exercise 13e - Applications of Differential Equation}
\begin{enumerate}[leftmargin=*]
    \item A the rate of change of a function $y$ with respect to $x$ is $\dfrac{1}{3}y$.
          When $x = -1$, $y = 4$, find the relationship between $x$ and $y$.

          \textbf{Sol.}
          \begin{flalign*}
              \dfrac{dy}{dx}  & = \dfrac{1}{3}y        \\
              \dfrac{dy}{y}   & = \dfrac{1}{3}dx       \\
              \ln\vert y\vert & = \dfrac{1}{3}x + C    \\
              y               & = Ce^{\frac{1}{3}x}    \\
              y(-1)           & = 4                    \\
              4               & = Ce^{-\frac{1}{3}}    \\
              C               & = 4e^{\frac{1}{3}}     \\
                              & \approx 5.58           \\
              y               & = 5.58e^{\frac{1}{3}x}
          \end{flalign*}

    \item A the rate of change of a function $y$ with respect to $x$ is $2 - y$. When $x
              = 0$, $y = 8$, find the relationship between $x$ and $y$.

          \textbf{Sol.}
          \begin{flalign*}
              \dfrac{dy}{dx}      & = 2 - y       \\
              \dfrac{dy}{y-2}     & = -dx         \\
              \ln\vert y - 2\vert & = -x + C      \\
              y - 2               & = Ce^{-x}     \\
              y                   & = 2 + Ce^{-x} \\
              y(0)                & = 8           \\
              8                   & = 2 + C       \\
              C                   & = 6           \\
              y                   & = 2 + 6e^{-x}
          \end{flalign*}

    \item The gradient of tangent of a curve at any point is $xy$, and the curve passes
          through $(0, 1)$, find the equation of the curve.

          \textbf{Sol.}
          \begin{flalign*}
              \dfrac{dy}{dx}  & = xy                  \\
              \dfrac{dy}{y}   & = xdx                 \\
              \ln\vert y\vert & = \dfrac{1}{2}x^2 + C \\
              y               & = Ce^{\frac{1}{2}x^2} \\
              y(0)            & = 1                   \\
              C               & = 1                   \\
              y               & = e^{\frac{1}{2}x^2}
          \end{flalign*}
    \item The decay of a substance is proportional to the amount of the substance
          present. When $t = 0$, the amount of the substance is $9$g, it's also given
          that the substance had reduced by 1g after the first hour passed. Find the
          amount of time needed for the substance to be reduced by 5g.

          \textbf{Sol.}
          \begin{flalign*}
              \dfrac{dR}{dt}  & = -kR                           \\
              \dfrac{dR}{R}   & = -kdt                          \\
              \ln\vert R\vert & = -kt + C                       \\
              R               & = Ce^{-kt}                      \\
              R(0)            & = 9                             \\
              C               & = 9                             \\
              R               & = 9e^{-kt}                      \\
              R(1)            & = 8                             \\
              8               & = 9e^{-k}                       \\
              k               & = \ln\frac{9}{8} \approx 0.1178 \\
              R               & = 9e^{-0.1178t}                 \\
              4               & = 9e^{-0.1178t}                 \\
              t               & \approx 6.9
          \end{flalign*}
          $\therefore$ The amount of time needed for the substance to be reduced by 5g is about 6.9 hours.

    \item Let the population growth rate of a country is directly proportional to the
          population of the country at that year. Given that in the year $1951$, its
          population is $1.5$ million, and in the year $1970$, its population is $2$
          million, find
          \begin{enumerate}
              \item The population of the country in the year $2000$.

                    \textbf{Sol.}
                    \begin{flalign*}
                        \dfrac{dP}{dt}  & = kP                                                                        \\
                        \dfrac{dP}{P}   & = kdt                                                                       \\
                        \ln\vert P\vert & = kt + C                                                                    \\
                        P               & = Ce^{kt}                                                                   \\
                        Ce^{1950k}      & = 1.5\ \cdots (1)                                                           \\
                        Ce^{1970k}      & = 2\ \cdots (2)                                                             \\
                        (2) \div (1)    & \Rightarrow e^{20k} = \dfrac{4}{3}                                          \\
                        k               & = \dfrac{1}{20}\ln\dfrac{4}{3}                                              \\
                        C               & = 2e^{-1970\times\frac{1}{20}\ln\frac{4}{3}}                                \\
                        P               & = 2e^{-1970\times\frac{1}{20}\ln\frac{4}{3}}e^{\frac{1}{19}\ln\frac{4}{3}t} \\
                                        & = 2e^{\frac{1}{20}\ln\frac{4}{3}(t - 1970)}                                 \\
                        P(2000)         & = 2e^{\frac{1}{20}\ln\frac{4}{3}(2000 - 1970)}                              \\
                                        & \approx 3.0792
                    \end{flalign*}
                    $\therefore$ The population of the country in the year $2000$ is about $307.9$ million.

              \item The year (approximate value) when the population of the country reaches 100
                    million.

                    \textbf{Sol.}
                    \begin{flalign*}
                        P                                        & = 2e^{\frac{1}{20}\ln\frac{4}{3}(t - 1970)} \\
                        10                                       & = 2e^{\frac{1}{20}\ln\frac{4}{3}(t - 1970)} \\
                        e^{\frac{1}{20}\ln\frac{4}{3}(t - 1970)} & = 5                                         \\
                        \frac{1}{20}\ln\frac{4}{3}(t - 1970)     & = \ln 5                                     \\
                        \ln\frac{4}{3}(t - 1970)                 & = 20\ln 5                                   \\
                        t - 1970                                 & = \dfrac{20\ln 5}{\ln\dfrac{4}{3}}          \\
                        t                                        & = 1970 + \dfrac{20\ln 5}{\ln\dfrac{4}{3}}   \\
                                                                 & \approx 2081.89
                    \end{flalign*}
                    $\therefore$ The country will reach 100 million population in the year 2082.
          \end{enumerate}

    \item The decaying speed of a mothball is that it's decayed to half its original
          volume every three weeks. At the beginning, its volume is 1cm$^3$. When its
          volume become 0.1cm$^3$, the mothball loss its effectiveness. Find the time of
          effectiveness of the mothball.

          \textbf{Sol.}
          \begin{flalign*}
              -\dfrac{dV}{dt}        & = kV                     \\
              \dfrac{dV}{V}          & = -kdt                   \\
              \ln\vert V\vert        & = -kt + C                \\
              V                      & = Ce^{-kt}               \\
              V(0)                   & = 1                      \\
              C                      & = 1                      \\
              V                      & = e^{-kt}                \\
              V(3)                   & = 0.5                    \\
              0.5                    & = e^{-3k}                \\
              k                      & = \dfrac{1}{3}\ln 2      \\
              V                      & = e^{-\frac{1}{3}t\ln 2} \\
              e^{-\frac{1}{3}t\ln 2} & = 0.1                    \\
              -\frac{1}{3}t\ln 2     & = \ln 0.1                \\
              t                      & = \dfrac{3\ln 10}{\ln 2} \\
                                     & \approx 9.96
          \end{flalign*}
          $\therefore$ The mothball loss its effectiveness in about 10 weeks.
          \newpage

    \item The atmospheric pressure $p$ of each point above the earth is a function of its
          altitude $h$. Now it's given that when $h = 0$, $p = 1.033 \times 10^5$
          N/m$^2$, and when $h = 3048$m, $p = 6.88 \times 10^4$ N/m$^2$. Find the
          atmosperic pressure $p$ when $h = 2000$m.

          \textbf{Sol.}
          \begin{flalign*}
              -\dfrac{dp}{dh}             & = kh                                                               \\
              \dfrac{dp}{p}               & = -kdh                                                             \\
              \ln\vert p\vert             & = -kh + C                                                          \\
              p                           & = Ce^{-kh}                                                         \\
              p(0)                        & = 1.033 \times 10^5                                                \\
              C                           & = 1.033 \times 10^5                                                \\
              p                           & = 1.033 \times 10^5e^{-kh}                                         \\
              p(3048)                     & = 6.88 \times 10^4                                                 \\
              1.033 \times 10^5e^{-3048k} & = 6.88 \times 10^4                                                 \\
              e^{-3048k}                  & = \dfrac{688}{1033}                                                \\
              -3048k                      & = \ln\dfrac{688}{1033}                                             \\
              k                           & = -\dfrac{1}{3048}\ln\dfrac{688}{1033}                             \\
              p                           & = 1.033 \times 10^5e^{\frac{1}{3048}h\ln\frac{688}{1033}}          \\
              p(2000)                     & = 1.033 \times 10^5e^{\frac{1}{3048}\times2000\ln\frac{688}{1033}} \\
                                          & \approx 79118.7                                                    \\
                                          & \approx 7.91 \times 10^4
          \end{flalign*}
          $\therefore$ The atmospheric pressure when $h = 2000$m is about $7.91 \times 10^4$ N/m$^2$.

    \item A boat is sailing on still water. The the deceleration produced by the
          resistance of the water is directly proportional to the speed of the boat.
          Prove that the speed of the boat $t$ seconds after its engine has stopped is $v
              = v_0e^{-kt}$, where $v_0$ is the speed of the boat when the boat engine stop,
          and $k$ is a constant of proportionality.

          \textbf{Proof.}
          \begin{flalign*}
              -\dfrac{dv}{dt} & = kv                            \\
              \dfrac{dv}{v}   & = -kdt                          \\
              \ln\vert v\vert & = -kt + C                       \\
              v               & = Ce^{-kt}                      \\
              v(0)            & = v_0                           \\
              C               & = v_0                           \\
              v               & = v_0e^{-kt} \quad \blacksquare
          \end{flalign*}

          \newpage
    \item A capacitor is releasing its charge, and the rate of change of $V$ with respect
          to the time is proportional to $V$, while $V$ decreases as the time increases.
          Now, it's given that the constant of proportionality is $k = 40$, if $V$ is
          decreased to $10\%$ of its original value, find $t$.

          \textbf{Sol.}
          \begin{flalign*}
              \dfrac{dV}{dt}  & = -40V               \\
              \dfrac{dV}{V}   & = -40dt              \\
              \ln\vert V\vert & = -40t + C           \\
              V               & = Ce^{-40t}          \\
              V(0)            & = 1                  \\
              C               & = 1                  \\
              V               & = e^{-40t}           \\
              e^{-40t}        & = 0.1                \\
              -40t            & = \ln 0.1            \\
              t               & = \dfrac{\ln 10}{40} \\
                              & \approx 0.058
          \end{flalign*}

    \item The decay of a radioactive substance has been found to be proportional to the
          amount of the substance present $R$. The experiments show that the substance is
          reduced to half of its original amount $R_0$ in 1,600 years. Find the
          relationship between the amount of the substance $R$ and the time $t$.

          \textbf{Sol.}
          \begin{flalign*}
              \dfrac{dR}{dt}  & = -kR                         \\
              \dfrac{dR}{R}   & = -kdt                        \\
              \ln\vert R\vert & = -kt + C                     \\
              R               & = Ce^{-kt}                    \\
              R(0)            & = R_0                         \\
              C               & = R_0                         \\
              R               & = R_0e^{-kt}                  \\
              R(1600)         & = \dfrac{1}{2}R_0             \\
              R_0e^{-1600k}   & = \dfrac{1}{2}R_0             \\
              e^{-1600k}      & = \dfrac{1}{2}                \\
              -1600k          & = \ln\dfrac{1}{2}             \\
              k               & = \dfrac{\ln 2}{1600}         \\
              R               & = R_0e^{-\frac{\ln 2}{1600}t} \\
          \end{flalign*}

          \newpage
    \item The Newton's law of cooling stated that the rate of change of the temperature
          of a body is proportional to the difference between its temperature and the
          temperature of the surrounding medium. If the temperature of a body decreases
          from $80^\circ$C to $60^\circ$C in $20$ minutes, Find the temperature of the
          body after $40$ minutes (assume that the temperature of the surrounding medium
          is $20^\circ$C).

          \textbf{Sol.}

          \begin{flalign*}
              -\dfrac{dT}{dt}       & = k(T - T_s)                                       \\
              \dfrac{dT}{T - T_s}   & = -kdt                                             \\
              \ln\vert T - T_s\vert & = -kt + C                                          \\
              T - T_s               & = e^{-kt + C}                                      \\
              T                     & = T_s + Ce^{-kt}                                   \\
              T(0)                  & = T_s + C = T_0                                    \\
              C                     & = T_0 - T_s                                        \\
              T                     & = T_s + (T_0 - T_s)e^{-kt}                         \\
              T(20)                 & = 20 + (80 - 20)e^{-k(20)}                         \\
              60                    & = 20 + 60e^{-20k}                                  \\
              e^{-20k}              & = \dfrac{2}{3}                                     \\
              -20k                  & = \ln\dfrac{2}{3}                                  \\
              k                     & = -\dfrac{1}{20}\ln\dfrac{2}{3}                    \\
              T                     & = T_s + (T_0 - T_s)e^{\frac{1}{20}\ln\frac{2}{3}t} \\
              T(40)                 & = 20 + (80 - 20)e^{\frac{1}{20}\ln\frac{2}{3}40}   \\
                                    & = 46.7^\circ\text{C}
          \end{flalign*}
          The temperature of the body after $40$ minutes is $46.7^\circ\text{C}$.
    \item It's been proven in the experiments that the pressure $p$ of the gas in the
          cylinder of a diesel engine decreases as its volume increases, while the rate
          of change of $p$ with respect to $V$ is proportional to $p$ and inversely
          proportional to $V$. Find the relationship between the pressure $p$ and the
          volume $V$.

          \textbf{Sol.}
          \begin{flalign*}
              -\dfrac{dp}{dV} & = \dfrac{kp}{V}      \\
              \dfrac{dp}{p}   & = -\dfrac{k}{V}dV    \\
              \ln p           & = -k\ln V + C        \\
              p               & = e^{\ln V^{-k} + C} \\
                              & = e^{\ln V^-{k}}e^C  \\
                              & = \dfrac{C}{V^k}
          \end{flalign*}
          $\therefore$ The relationship between the pressure $p$ and the volume $V$ is $p = \dfrac{C}{V^k}$.

          \newpage
    \item A submarine with its mass $m$ starts descending from still water surface, its
          resistance is directly proportional to its descending speed (the constant of
          proportionality is $k$). Find the relationship between the depth $x$ and the
          time $t$.

          \textbf{Sol.}

          Let the descending speed of the submarine $v_0$ at time $t$ be affected by two
          forces: the gravity $mg$ and the resistance $kv$. Hence the forces on the
          submarine are
          \begin{flalign*}
              F = ma = & = mg - kv
          \end{flalign*}
          From Newton's second law, the equation of motion of the submarine is
          \begin{flalign*}
              m\dfrac{dv}{dt} =                                              & = mg - kv                                                                     \\
              \dfrac{dv}{dt}                                                 & = g - \dfrac{k}{m}v                                                           \\
              \dfrac{dv}{dt} + \dfrac{k}{m}v =                               & g                                                                             \\
              p(t)                                                           & = \dfrac{k}{m}                                                                \\
              \mu(t)                                                         & = e^{\int p(t)dt}                                                             \\
                                                                             & = e^{\int \frac{k}{m}dt}                                                      \\
                                                                             & = e^{\frac{k}{m}t}                                                            \\
              e^{\frac{k}{m}t}\dfrac{dv}{dt} + e^{\frac{k}{m}t}\dfrac{k}{m}v & = e^{\frac{k}{m}t}g                                                           \\
              \dfrac{d}{dt}\left(e^{\frac{k}{m}t}v\right)                    & = e^{\frac{k}{m}t}g                                                           \\
              e^{\frac{k}{m}t}v                                              & = \dfrac{mg}{k}e^{\frac{k}{m}t} + C                                           \\
              v                                                              & = \dfrac{mg}{k} + Ce^{-\frac{k}{m}t}                                          \\\\
              v(0)                                                           & = \dfrac{mg}{k} + C                                                           \\
              0                                                              & = \dfrac{mg}{k} + C                                                           \\
              C                                                              & = -\dfrac{mg}{k}                                                              \\
              v                                                              & = \dfrac{mg}{k}(1 - e^{-\frac{k}{m}t})                                        \\
              x                                                              & = \int vdt                                                                    \\
                                                                             & = \int \dfrac{mg}{k}(1 - e^{-\frac{k}{m}t})dt                                 \\
                                                                             & = \dfrac{mg}{k}\int (1 - e^{-\frac{k}{m}t})dt                                 \\
                                                                             & = \dfrac{mg}{k}\left(t + \dfrac{m}{k}e^{-\frac{k}{m}t} + C\right)             \\
                                                                             & = \dfrac{mg}{k}\left(t - \dfrac{mg}{k} + \dfrac{m}{k}e^{-\frac{k}{m}t}\right) \\
          \end{flalign*}
          $\therefore$ The relationship between the depth $x$ and the time $t$ is $x = \dfrac{mg}{k}\left(t - \dfrac{mg}{k} + \dfrac{m}{k}e^{-\frac{k}{m}t}\right)$.

          \newpage
    \item In an electric circuit, let the resistance be $R$ ohms, the inductance $L$
          henrys, and the electromotive force $E$ volts ($R$, $L$, and $E$ are
          constants). It's been proven in the physics lab that the relationship between
          the current intensity $i$ and the electromotive force $E$ is given by the
          equation $E = Ri + L\dfrac{di}{dt}$. Find the general solution and the
          particular solution when $i(0) = 0$.

          \textbf{Sol.}
          \begin{flalign*}
              E                                                              & = Ri + L\dfrac{di}{dt}                         \\
              \dfrac{di}{dt} + \dfrac{R}{L}i                                 & = \dfrac{E}{L}                                 \\
              p(t)                                                           & = \dfrac{R}{L}                                 \\
              \mu(t)                                                         & = e^{\int p(t)dt}                              \\
                                                                             & = e^{\int \frac{R}{L}dt}                       \\
                                                                             & = e^{\frac{R}{L}t}                             \\
              e^{\frac{R}{L}t}\dfrac{di}{dt} + e^{\frac{R}{L}t}\dfrac{R}{L}i & = e^{\frac{R}{L}t}\dfrac{E}{L}                 \\
              \dfrac{d}{dt}\left(e^{\frac{R}{L}t}i\right)                    & = e^{\frac{R}{L}t}\dfrac{E}{L}                 \\
              e^{\frac{R}{L}t}i                                              & = \int e^{\frac{R}{L}t}\dfrac{E}{L}dt          \\
                                                                             & = \dfrac{E}{L}\int e^{\frac{R}{L}t}dt          \\
                                                                             & = \dfrac{E}{L}\dfrac{L}{R}e^{\frac{R}{L}t} + C \\
                                                                             & = \dfrac{E}{R}e^{\frac{R}{L}t} + C             \\
              i                                                              & = \dfrac{E}{R} + Ce^{-\frac{R}{L}t}            \\
              \\
              i(0)                                                           & = \dfrac{E}{R} + C                             \\
              0                                                              & = \dfrac{E}{R} + C                             \\
              C                                                              & = -\dfrac{E}{R}                                \\
              i                                                              & = \dfrac{E}{R} - \dfrac{E}{R}e^{-\frac{R}{L}t} \\
                                                                             & = \dfrac{E}{R}(1 - e^{-\frac{R}{L}t})
          \end{flalign*}
          $\therefore$ The general solution is $i = \dfrac{E}{R} + Ce^{-\frac{R}{L}t}$, and the particular solution is $i = \dfrac{E}{R}(1 - e^{-\frac{R}{L}t})$ when $i(0) = 0$.
\end{enumerate}
\end{document}