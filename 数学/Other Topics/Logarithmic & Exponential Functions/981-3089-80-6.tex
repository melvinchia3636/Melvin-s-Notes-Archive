\documentclass{report}

\usepackage[total={6.5in,9in},top=1in, left=1in]{geometry}
\usepackage{amsmath}
\usepackage{amssymb}
\usepackage{multicol}
\usepackage{enumitem}

\newcommand{\sol}{\vspace{0.2cm}\textbf{Solution:}\vspace{0.2cm}}

\begin{document}

\setlength{\columnseprule}{0.4pt}
\setlength{\columnsep}{3em}

\begin{multicols*}{2}
    \begin{enumerate}[leftmargin=*]
        \item \begin{enumerate}
                  \item Find $x$, correct to two decimal places, such that $2^{x} \cdot 3^{x}=18$.

                        \sol{}
                        \begin{align*}
                            2^{x} \cdot 3^{x} & = 18                     \\
                            (2 \cdot 3)^{x}   & = 18                     \\
                            6^{x}             & = 18                     \\
                            \log 6^{x}        & = \log 18                \\
                            x \log 6          & = \log 18                \\
                            x                 & = \frac{\log 18}{\log 6} \\
                                              & \approx 1.63
                        \end{align*}

                  \item Given that the curve $y=kx^{n}$ passes through the points $(2,2)$ and
                        $\left(5,31 \dfrac{1}{4}\right)$, find the values of $k$ and $n$.

                        \sol{}

                        When $x=2$, $y=2$, $2=k \cdot 2^{n}\ \cdots (1)$.

                        When $x=5$, $y=31 \dfrac{1}{4}$, $31 \dfrac{1}{4}=k \cdot 5^{n} \ \cdots (2)$.

                        Dividing $(2)$ by $(1)$,
                        \begin{align*}
                            \frac{31 \dfrac{1}{4}}{2}    & = \frac{k \cdot 5^{n}}{k \cdot 2^{n}} \\
                            \frac{125}{8}                & = \frac{5^{n}}{2^{n}}                 \\
                            \left(\frac{5}{2}\right)^{n} & = \frac{125}{8}                       \\
                                                         & = \left(\frac{5}{2}\right)^{3}        \\
                            n                            & = 3
                        \end{align*}
                        Substituting $n=3$ into $(1)$,
                        \begin{align*}
                            2 & = k \cdot 2^{3} \\
                            k & = \frac{1}{4}
                        \end{align*}
              \end{enumerate}

        \item Solve the equations $\log _2 y^2=3+\log _2(y+6)$.

              \sol{}
              \begin{align*}
                  \log _2 y^2        & = 3+\log _2(y+6) \\
                  \log _2 y^2        & = \log _2 8(y+6) \\
                  y^2                & = 8(y+6)         \\
                  y^2 - 8y - 48      & = 0              \\
                  (y-12)(y+4)        & = 0              \\
                  y             = 12 & \text{ or } -4
              \end{align*}

        \item \begin{enumerate}
                  \item Given that $q^p=25$, express $\log _5 q$ in terms of $p$. \sol{}
                        \begin{align*}
                            q^p      & = 25                      \\
                            q        & = 25^{\frac{1}{p}}        \\
                            \log_5 q & = \log_5 25^{\frac{1}{p}} \\
                                     & = \frac{1}{p} \log_5 25   \\
                                     & = \frac{1}{p} \cdot 2     \\
                                     & = \frac{2}{p}
                        \end{align*}

                  \item Given that $2 \lg x^2 y=3+\lg x-\lg y$, where $x$ and $y$ are both positive,
                        express, in its simplest form, $y$ in terms of $x$.

                        \sol{}
                        \begin{align*}
                            2 \lg x^2 y & = 3+\lg x-\lg y               \\
                            \lg x^4 y^2 & = \lg \dfrac{1000x}{y}        \\
                            x^4 y^2     & = \dfrac{1000x}{y}            \\
                            x^4 y^3     & = 1000x                       \\
                            y^3         & = \dfrac{1000x}{x^4}          \\
                                        & = \dfrac{1000}{x^3}           \\
                            y           & = \sqrt[3]{\dfrac{1000}{x^3}} \\
                                        & = \dfrac{10}{x}
                        \end{align*}
              \end{enumerate}

        \item \begin{enumerate}
                  \item Solve the equation $\lg y+\lg (2 y-1)=1$.

                        \sol{}
                        \begin{align*}
                            \lg y + \lg (2y-1) & = 1                     \\
                            \lg y(2y-1)        & = 1                     \\
                            y(2y-1)            & = 10                    \\
                            2y^2 - y           & = 10                    \\
                            2y^2 - y - 10      & = 0                     \\
                            (2y-5)(y+2)        & = 0                     \\
                            y                  & = \dfrac{5}{2}\ (y > 0)
                        \end{align*}

                        \newpage
                  \item Given that $5 \log _p 6-\log _p 96=4$, find the value of $p$.

                        \sol{}
                        \begin{align*}
                            5 \log_p 6 - \log_p 96 & = 4   \\
                            \log_p 6^5 - \log_p 96 & = 4   \\
                            \log_p \dfrac{6^5}{96} & = 4   \\
                            \dfrac{7776}{96}       & = p^4 \\
                            p^4                    & = 81  \\
                            p                      & = 3
                        \end{align*}
              \end{enumerate}

        \item \begin{enumerate}
                  \item Denoting $\log _3$ a by $p$, express in terms of $p$
                        \begin{enumerate}
                            \item $\log _3 a^3$,

                                  \sol{}
                                  \begin{align*}
                                      \log_3 a^3 & = 3 \log_3 a \\
                                                 & = 3p
                                  \end{align*}

                            \item $\log _3\left(\dfrac{1}{a}\right)$,

                                  \sol{}
                                  \begin{align*}
                                      \log_3 \left(\dfrac{1}{a}\right) & = \log_3 a^{-1} \\
                                                                       & = -\log_3 a     \\
                                                                       & = -p
                                  \end{align*}

                            \item $\log _9 a$.

                                  \sol{}
                                  \begin{align*}
                                      \log_9 a & = \dfrac{1}{2} \log_3 a \\
                                               & = \dfrac{1}{2} p
                                  \end{align*}

                        \end{enumerate}
                  \item Given that $y=a x^{n}-20$ and that $y=12$ when $x=2$, and $y=140$ when $x=4$,
                        find $n$ and $a$

                        \sol{}

                        When $x=2$, $y=12$,
                        \begin{align*}
                            12            & = a \cdot 2^{n}-20 \\
                            a \cdot 2^{n} & = 32\ \cdots (1)
                        \end{align*}

                        When $x=4$, $y=140$,
                        \begin{align*}
                            140           & = a \cdot 4^{n}-20 \\
                            a \cdot 4^{n} & = 160\ \cdots (2)
                        \end{align*}

                        Dividing $(2)$ by $(1)$,
                        \begin{align*}
                            \frac{a \cdot 4^{n}}{a \cdot 2^{n}} & = \frac{160}{32} \\
                            2^{n}                               & = 5              \\
                            n                                   & = \log_2 5       \\
                                                                & \approx 2.32
                        \end{align*}
                        Substituting $n=\log_2 5$ into $(1)$,
                        \begin{align*}
                            a \cdot 2^{\log_2 5} & = 32           \\
                            a \cdot 5            & = 32           \\
                            a                    & = \frac{32}{5} \\
                                                 & = 6.4
                        \end{align*}
              \end{enumerate}

        \item \begin{enumerate}
                  \item Without using tables or calculators evaluate
                        \begin{enumerate}
                            \item $2 \log _2 12+3 \log _2 5-\log _2 15-\log _2 150$,

                                  \sol{}
                                  \begin{align*}
                                       & 2 \log_2 12 + 3 \log_2 5 - \log_2 15 - \log_2 150   & \\
                                       & = \log_2 12^2 + \log_2 5^3 - \log_2 15 - \log_2 150 & \\
                                       & = \log_2 \dfrac{12^2 \cdot 5^3}{15 \cdot 150}       & \\
                                       & = \log_2 \dfrac{144 \cdot 125}{2250}                & \\
                                       & = \log_2 \dfrac{18000}{2250}                        & \\
                                       & = \log_2 8                                          & \\
                                       & = 3
                                  \end{align*}

                            \item $\log _8 32$.

                                  \sol{}
                                  \begin{align*}
                                      \log_8 32 & = \dfrac{\log_2 32}{\log_2 8} \\
                                                & = \dfrac{5}{3}
                                  \end{align*}
                        \end{enumerate}

                  \item Show that $5^n+5^{n+1}+5^{n+2}$ is divisible by 31 for all positive integer
                        values of $n$.

                        \sol{}
                        \begin{align*}
                            5^n + 5^{n+1} + 5^{n+2} & = 5^n(1 + 5 + 25) \\
                                                    & = 31 \cdot 5^n
                        \end{align*}
                        $\because$ $\forall n \in \mathbb{Z}_+$, $5^n \in \mathbb{Z}_+$,

                        $\therefore$ $31 | (31 \cdot 5^n)$.

                        $\therefore$ $\forall n \in \mathbb{Z}_+$, $31 | (5^n + 5^{n+1} + 5^{n+2})$. $\hfill\blacksquare$
              \end{enumerate}

        \item \begin{enumerate}
                  \item Solve the equation $\lg 25+\lg x-\lg (x-1)=2$.

                        \sol{}
                        \begin{align*}
                            \lg 25 + \lg x - \lg (x-1) & = 2            \\
                            \lg \dfrac{25x}{x-1}       & = 2            \\
                            \dfrac{25x}{x-1}           & = 100          \\
                            25x                        & = 100x - 100   \\
                            75x                        & = 100          \\
                            x                          & = \dfrac{4}{3}
                        \end{align*}

                  \item Solve the equation $5^{y}=10$.

                        \sol{}
                        \begin{align*}
                            5^y & = 10                      \\
                            y   & = \log_5 10               \\
                                & = \dfrac{\log 10}{\log 5} \\
                                & \approx 1.43
                        \end{align*}
                  \item Given that $\lg z=k$ find, in terms of $k$, an expression for $\log _z 10 z$.

                        \sol{}
                        \begin{align*}
                            \log_z 10z & = \log_z 10 + \log_z z        \\
                                       & = \log_z 10 + 1               \\
                                       & = \dfrac{\log 10}{\log z} + 1 \\
                                       & = \dfrac{1}{k} + 1
                        \end{align*}
              \end{enumerate}

        \item \begin{enumerate}
                  \item Solve the equation $\sqrt{4 x-9}=2 \sqrt{x}-1$.

                        \sol{}
                        \begin{align*}
                            \sqrt{4x-9} & = 2\sqrt{x} - 1      \\
                            4x - 9      & = 4x - 4\sqrt{x} + 1 \\
                            4\sqrt{x}   & = 10                 \\
                            \sqrt{x}    & = \dfrac{5}{2}       \\
                            x           & = \dfrac{25}{4}
                        \end{align*}

                        Upon checking, $x=\dfrac{25}{4}$ is a valid solution.

                  \item Solve the equation $7^{x^2}-49^{6-2 x}=0$.

                        \sol{}
                        \begin{align*}
                            7^{x^2} - 49^{6-2x} & = 0                \\
                            7^{x^2}             & = 49^{6-2x}        \\
                            7^{x^2}             & = 7^{2(6-2x)}      \\
                            x^2                 & = 2(6-2x)          \\
                            x^2 + 4x - 12       & = 0                \\
                            (x+6)(x-2)          & = 0                \\
                            x                   & = -6 \text{ or } 2
                        \end{align*}

                  \item Evaluate $\log _3 7 \cdot \log _7 2 \cdot \log _2 3$.

                        \sol{}
                        \begin{align*}
                            \log_3 7 \cdot \log_7 2 \cdot \log_2 3 & = \dfrac{\log_7 7}{\log_7 3} \cdot \dfrac{\log_7 2}{\log_7 7} \cdot \dfrac{\log_7 3}{\log_7 2} \\
                                                                   & = 1
                        \end{align*}
              \end{enumerate}

        \item \begin{enumerate}
                  \item Solve the equation $\dfrac{6}{\sqrt{x-1}}-2 \sqrt{x-1}=1$,

                        \sol{}
                        \begin{align*}
                            \dfrac{6}{\sqrt{x-1}} - 2\sqrt{x-1} & = 1               \\
                            6 - 2(x-1)                          & = \sqrt{x-1}      \\
                            8 - 2x                              & = \sqrt{x-1}      \\
                            4x^2 - 32x + 64                     & = x - 1           \\
                            4x^2 - 33x + 65                     & = 0               \\
                            (4x - 13)(x - 5)                    & = 0               \\
                            x = \dfrac{13}{4}                   & \text{ or } x = 5
                        \end{align*}
                        Upon checking, $x=\dfrac{13}{4}$ is the only valid solution.

                  \item If $5^{x} 25^{2 y}=1$ and $3^{5 x} 9^{y}=\dfrac{1}{9}$ calculate the value of
                        $x$ and of $y$.

                        \sol{}
                        \begin{align*}
                            5^x 25^{2y} & = 1              \\
                            5^x 5^{4y}  & = 1              \\
                            5^{x+4y}    & = 1              \\
                            x + 4y      & = 0\ \cdots\ (1)
                        \end{align*}

                        \begin{align*}
                            3^{5x} 9^y    & = 3^{-2}          \\
                            3^{5x} 3^{2y} & = 3^{-2}          \\
                            3^{5x+2y}     & = 3^{-2}          \\
                            5x + 2y       & = -2\ \cdots\ (2)
                        \end{align*}
                        Multiplying $(2)$ by $2$,
                        \begin{align*}
                            10x + 4y & = -4\ \cdots\ (3)
                        \end{align*}
                        Subtracting $(1)$ from $(3)$,
                        \begin{align*}
                            9x & = -4            \\
                            x  & = -\dfrac{4}{9}
                        \end{align*}
                        Substituting $x=-\dfrac{4}{9}$ into $(1)$,
                        \begin{align*}
                            -\dfrac{4}{9} + 4y & = 0            \\
                            4y                 & = \dfrac{4}{9} \\
                            y                  & = \dfrac{1}{9}
                        \end{align*}
                        Therefore, $x=-\dfrac{4}{9}$ and $y=\dfrac{1}{9}$.

              \end{enumerate}

        \item \begin{enumerate}
                  \item Given that $\log _{p} 7+\log _{p} k=0$, find $k$.

                        \sol{}
                        \begin{align*}
                            \log_p 7 + \log_p k & = 0            \\
                            \log_p 7k           & = 0            \\
                            7k                  & = p^0          \\
                            7k                  & = 1            \\
                            k                   & = \dfrac{1}{7}
                        \end{align*}

                  \item Given that $4 \log _q 3+2 \log _q 2-\log _q 144=2$ find $q$.

                        \sol{}
                        \begin{align*}
                            4 \log_q 3 + 2 \log_q 2 - \log_q 144 & = 2                     \\
                            \log_q 3^4 + \log_q 2^2 - \log_q 144 & = 2                     \\
                            \log_q \dfrac{3^4 \cdot 2^2}{144}    & = 2                     \\
                            \dfrac{3^4 \cdot 2^2}{144}           & = q^2                   \\
                            q^2                                  & = \dfrac{9}{4}          \\
                            q                                    & = \dfrac{3}{2}\ (q > 0)
                        \end{align*}

                  \item Given that $\log _3 2=0.631$ and tha $\log _3 5=1.465$, evaluate $\log _3 1.2$,
                        without using tables or a calculator. \sol{}
                        \begin{align*}
                            \log_3 1.2 & = \log_3 \dfrac{12}{10}                       \\
                                       & = \log_3 12 - \log_3 10                       \\
                                       & = \log_3 2^2 + \log_3 3 - \log_3 2 - \log_3 5 \\
                                       & = \log_3 2 + 1 - \log_3 5                     \\
                                       & = 0.631 + 1 - 1.465                           \\
                                       & = 0.166
                        \end{align*}
              \end{enumerate}

        \item \begin{enumerate}
                  \item Solve the equation $\log _5 x=16 \log _x 5$.

                        \sol{}
                        \begin{align*}
                            \log_5 x   & = 16 \log_x 5                    \\
                            \log_5 x   & = \dfrac{16}{\log_5 x}           \\
                            \log_5^2 x & = 16                             \\
                            \log_5 x   & = \pm 4                          \\
                            x          & = 5^4 \text{ or } 5^{-4}         \\
                                       & = 625 \text{ or } \dfrac{1}{625}
                        \end{align*}

                  \item Find the values of $y$ which satisfy the equation $\left(8^{y}\right)^{y} \cdot
                            \dfrac{1}{32^{y}}=4$

                        \sol{}
                        \begin{align*}
                            \left(8^y\right)^y \cdot \dfrac{1}{32^y} & = 4                           \\
                            8^{y^2} \cdot \dfrac{1}{2^{5y}}          & = 4                           \\
                            8^{y^2} \cdot 2^{-5y}                    & = 4                           \\
                            2^{3y^2} \cdot 2^{-5y}                   & = 4                           \\
                            2^{3y^2 - 5y}                            & = 4                           \\
                            3y^2 - 5y                                & = 2                           \\
                            3y^2 - 5y - 2                            & = 0                           \\
                            (3y + 1)(y - 2)                          & = 0                           \\
                            y                                        & = -\dfrac{1}{3} \text{ or } 2
                        \end{align*}

                        \newpage
                  \item Express $\frac{4+\sqrt{2}}{2 - \sqrt{2}}$ in the form $p+\sqrt{q}$, where $E$
                        and $q$ are integers.

                        \sol{}
                        \begin{align*}
                            \dfrac{4 + \sqrt{2}}{2 - \sqrt{2}} & = \dfrac{(4 + \sqrt{2})(2 + \sqrt{2})}{(2 - \sqrt{2})(2 + \sqrt{2})} \\
                                                               & = \dfrac{8 + 6\sqrt{2} + 2}{4 - 2}                                   \\
                                                               & = \dfrac{10 + 6\sqrt{2}}{2}                                          \\
                                                               & = 5 + 3\sqrt{2}                                                      \\
                                                               & = 5 + \sqrt{18}
                        \end{align*}
              \end{enumerate}

        \item Given that $y=x^{-\frac{1}{3}}$ use the calculus to determine an approximate
              value for $\dfrac{1}{\sqrt[3]{0.9}}$.

              \sol{}
              \begin{align*}
                  y                          & = x^{-\frac{1}{3}}               \\
                  \dfrac{dy}{dx}             & = -\dfrac{1}{3} x^{-\frac{4}{3}} \\
                  \dfrac{\Delta y}{\Delta x} & \approx \dfrac{dy}{dx}           \\
                  \Delta y                   & \approx \dfrac{dy}{dx} \Delta x
              \end{align*}
              When $x = 1$, $y = 1$, $\dfrac{dy}{dx}$, $\Delta x = -0.1$,
              \begin{align*}
                  \Delta y & \approx \dfrac{dy}{dx} \Delta x      \\
                           & = -\dfrac{1}{3} \cdot 1 \cdot (-0.1) \\
                           & = \dfrac{1}{30}
              \end{align*}
              $\therefore$ $\dfrac{1}{\sqrt[3]{0.9}} \approx 1 + \dfrac{1}{30} = \dfrac{31}{30} \approx 1.033$.

              \columnbreak
        \item \begin{enumerate}
                  \item Solve the equation $2 \lg 3+\lg 2 x-\lg (3 x+1)=0$

                        \sol{}
                        \begin{align*}
                            2 \lg 3 + \lg 2x - \lg (3x + 1) & = 0             \\
                            \lg 3^2 + \lg 2x - \lg (3x + 1) & = 0             \\
                            \lg \dfrac{9 \cdot 2x}{3x + 1}  & = 0             \\
                            \dfrac{9 \cdot 2x}{3x + 1}      & = 1             \\
                            18x                             & = 3x + 1        \\
                            15x                             & = 1             \\
                            x                               & = \dfrac{1}{15}
                        \end{align*}
                  \item Given that $\dfrac{5^{3 x}}{25^y}=3125$ and $2^x 4^{(y-1)}=32$, finc the value
                        of $x$ and of $y$.

                        \sol{}
                        \begin{align*}
                            \dfrac{5^{3x}}{25^y} & = 3125           \\
                            5^{3x}               & = 5^{5 + 2y}     \\
                            3x                   & = 5 + 2y         \\
                            3x - 2y              & = 5\ \cdots\ (1)
                        \end{align*}
                        \begin{align*}
                            2^x 4^{(y-1)}  & = 32             \\
                            2^x 2^{2(y-1)} & = 2^5            \\
                            2^{x+2y-2}     & = 2^5            \\
                            x + 2y - 2     & = 5              \\
                            x + 2y         & = 7\ \cdots\ (2)
                        \end{align*}
                        Adding $(1)$ and $(2)$,
                        \begin{align*}
                            4x & = 12 \\
                            x  & = 3
                        \end{align*}
                        Substituting $x=3$ into $(1)$,
                        \begin{align*}
                            9 - 2y & = 5 \\
                            y      & = 2
                        \end{align*}

                  \item Solve the equation $3 x-\sqrt{9 x^2-20}=4$.

                        \sol{}
                        \begin{align*}
                            3x - \sqrt{9x^2 - 20} & = 4               \\
                            \sqrt{9x^2 - 20}      & = 3x - 4          \\
                            9x^2 - 20             & = 9x^2 - 24x + 16 \\
                            24x                   & = 36              \\
                            x                     & = \dfrac{3}{2}
                        \end{align*}
              \end{enumerate}

        \item \begin{enumerate}
                  \item Solve the equation $2 \lg 15+\lg (5-x)-\lg 4 x=2$.

                        \sol{}
                        \begin{align*}
                            2 \lg 15 + \lg (5-x) - \lg 4x & = 2              \\
                            \lg \dfrac{15^2(5-x)}{4x}     & = 2              \\
                            \dfrac{15^2(5-x)}{4x}         & = 100            \\
                            225(5-x)                      & = 400x           \\
                            5 - x                         & = \dfrac{16x}{9} \\
                            45 - 9x                       & = 16x            \\
                            25x                           & = 45             \\
                            x                             & = \dfrac{9}{5}
                        \end{align*}

                  \item Solve the simultaneous equations $\dfrac{125^{x}}{25^{y}}=625$, $2 \times
                            4^{x}=32^{y}$.

                        \sol{}
                        \begin{align*}
                            \dfrac{125^x}{25^y}    & = 625            \\
                            \dfrac{5^{3x}}{5^{2y}} & = 5^4            \\
                            5^{3x-2y}              & = 5^4            \\
                            3x - 2y                & = 4\ \cdots\ (1)
                        \end{align*}
                        \begin{align*}
                            2 \cdot 4^x    & = 32^y            \\
                            2 \cdot 2^{2x} & = 2^{5y}          \\
                            2x + 1         & = 5y              \\
                            2x - 5y        & = -1\ \cdots\ (2)
                        \end{align*}
                        Multiplying $(1)$ by $2$,
                        \begin{align*}
                            6x - 4y & = 8\ \cdots\ (3)
                        \end{align*}
                        Multiplying $(2)$ by $3$,
                        \begin{align*}
                            6x - 15y & = -3\ \cdots\ (4)
                        \end{align*}
                        Subtracting $(4)$ from $(3)$,
                        \begin{align*}
                            11y & = 11 \\
                            y   & = 1
                        \end{align*}
                        Substituting $y=1$ into $(1)$,
                        \begin{align*}
                            3x - 2 & = 4 \\
                            x      & = 2
                        \end{align*}

                  \item Without using tables or a calculator, find the value of
                        $\dfrac{3}{\sqrt{2}-1}-\dfrac{6}{\sqrt{2}}$

                        \sol{}
                        \begin{align*}
                            \dfrac{3}{\sqrt{2}-1} - \dfrac{6}{\sqrt{2}} & = 3(\sqrt{2} - 1) - 3\sqrt{2} \\
                                                                        & = 3
                        \end{align*}
              \end{enumerate}

        \item \begin{enumerate}
                  \item Given that $\log _a N=\dfrac{1}{2}\left(\log _a 24-\log _a 0.375-\right.$ $6
                            \log _a 3$ ), find the value of $N$.

                            Find also the value of $\log _a N$ when $a=\dfrac{2}{3}$.

                        \sol{}
                        \begin{align*}
                            \log_a N & = \dfrac{1}{2}(\log_a 24 - \log_a 0.375 - 6 \log_a 3)        \\
                                     & = \dfrac{1}{2}(\log_a 24 - \log_a \dfrac{3}{8} - \log_a 3^6) \\
                                     & = \dfrac{1}{2}\log_a \left(\dfrac{2^6}{3^6}\right)           \\
                                     & = \log_a \dfrac{2^3}{3^3}                                    \\
                                     & = \log_a \dfrac{8}{27}                                       \\
                            N        & = \dfrac{8}{27}
                        \end{align*}
                        \begin{align*}
                            \log_\frac{2}{3} \dfrac{8}{27} & = \log_\frac{2}{3} \left(\dfrac{2}{3}\right)^{3} \\
                                                           & = 3
                        \end{align*}

                  \item Find the value of $x$ which satisfies the equation $\sqrt{3
                                x-5}-\sqrt{x+2}=\sqrt{x-6}$

                        \sol{}
                        \begin{align*}
                            \sqrt{3x-5} - \sqrt{x+2}     & = \sqrt{x-6}                  \\
                            4x - 3 - 2\sqrt{(3x-5)(x+2)} & = x - 6                       \\
                            2\sqrt{3x^2+x-10}            & = 3x + 3                      \\
                            4(3x^2+x-10)                 & = 9x^2 + 18x + 9              \\
                            12x^2 + 4x - 40              & = 9x^2 + 18x + 9              \\
                            3x^2 - 14x - 49              & = 0                           \\
                            (3x + 7)(x - 7)              & = 0                           \\
                            x                            & = -\dfrac{7}{3} \text{ or } 7
                        \end{align*}
                        Upon checking, $x=7$ is the only valid solution.
              \end{enumerate}

        \item Without using tables or a calculator, solve the following equations.
              \begin{enumerate}[label=(\roman*)]
                  \item $\lg x-\lg \left(\dfrac{10}{x^2}\right)=2$

                        \sol{}
                        \begin{align*}
                            \lg x - \lg \left(\dfrac{10}{x^2}\right) & = 2    \\
                            \lg \dfrac{x^3}{10}                      & = 2    \\
                            \dfrac{x^3}{10}                          & = 10^2 \\
                            x^3                                      & = 1000 \\
                            x                                        & = 10
                        \end{align*}

                  \item  $3^{y^2+3}=9^{2 y}$

                        \sol{}
                        \begin{align*}
                            3^{y^2 + 3}    & = 9^{2y}          \\
                            3^{y^2 + 3}    & = 3^{4y}          \\
                            y^2 + 3        & = 4y              \\
                            y^2 - 4y + 3   & = 0               \\
                            (y - 1)(y - 3) & = 0               \\
                            y              & = 1 \text{ or } 3
                        \end{align*}

                  \item $\log _{z} 16=8$.

                        \sol{}
                        \begin{align*}
                            \log_z 16                   & = 8               \\
                            \dfrac{\log_2 16}{\log_2 z} & = 8               \\
                            \dfrac{4}{\log_2 z}         & = 8               \\
                            \log_2 z                    & = \dfrac{1}{2}    \\
                            z                           & = 2^{\frac{1}{2}} \\
                                                        & = \sqrt{2}
                        \end{align*}
              \end{enumerate}

        \item Solve the equations
              \begin{enumerate}[label=(\roman*)]
                  \item $2^{x-1}=10$,

                        \sol{}
                        \begin{align*}
                            2^{x-1} & = 10         \\
                            x - 1   & = \log_2 10  \\
                                    & \approx 3.32 \\
                            x       & \approx 4.32
                        \end{align*}

                  \item $\log _{y} 8=\frac{1}{3}$

                        \sol{}
                        \begin{align*}
                            \log_y 8        & = \dfrac{1}{3} \\
                            y^{\frac{1}{3}} & = 8            \\
                            y               & = 8^3          \\
                                            & = 512
                        \end{align*}

                  \item $z + \sqrt{32 - z} = 2$.

                        \sol{}
                        \begin{align*}
                            z + \sqrt{32 - z} & = 2                \\
                            \sqrt{32 - z}     & = 2 - z            \\
                            32 - z            & = 4 - 4z + z^2     \\
                            z^2 - 3z - 28     & = 0                \\
                            (z - 7)(z + 4)    & = 0                \\
                            z                 & = 7 \text{ or } -4
                        \end{align*}

              \end{enumerate}

        \item Solve the equations
              \begin{enumerate}[label=(\roman*)]
                  \item $3^{x+1}=7$,

                        \sol{}
                        \begin{align*}
                            3^{x+1} & = 7           \\
                            x + 1   & = \log_3 7    \\
                                    & \approx 1.771 \\
                            x       & \approx 0.771
                        \end{align*}

                  \item $y=\sqrt{y+9}+3$,

                        \sol{}
                        \begin{align*}
                            y            & = \sqrt{y + 9} + 3 \\
                            y - 3        & = \sqrt{y + 9}     \\
                            y^2 - 6y + 9 & = y + 9            \\
                            y(y - 7)     & = 0                \\
                            y            & = 0 \text{ or } 7
                        \end{align*}
                  \item $2 \lg z=\lg (3 z+4)$.

                        \sol{}
                        \begin{align*}
                            2 \lg z        & = \lg (3z + 4)     \\
                            \lg z^2        & = \lg (3z + 4)     \\
                            z^2 - 3z       & = 4                \\
                            (z - 4)(z + 1) & = 0                \\
                            z              & = 4 \text{ or } -1
                        \end{align*}
                        Upon checking, $z=4$ is the only valid solution.
              \end{enumerate}

        \item \begin{enumerate}
                  \item  Solve the equations
                        \begin{enumerate}
                            \item $2 \times 4^{x+1}=16^{2 x}$,

                                  \sol{}
                                  \begin{align*}
                                      2 \cdot 4^{x+1}  & = 16^{2x}      \\
                                      2 \cdot 2^{2x+2} & = 2^{8x}       \\
                                      2x + 3           & = 8x           \\
                                      3                & = 6x           \\
                                      x                & = \dfrac{1}{2}
                                  \end{align*}

                            \item $\log _2 y^2=4+\log _2(y+5)$.

                                  \sol{}
                                  \begin{align*}
                                      \log_2 y^2      & = 4 + \log_2 (y + 5) \\
                                      \log_2 y^2      & = \log_2 16(y + 5)   \\
                                      y^2             & = 16(y + 5)          \\
                                      y^2 - 16y       & = 80                 \\
                                      (y - 20)(y + 4) & = 0                  \\
                                      y               & = 20 \text{ or } -4
                                  \end{align*}

                        \end{enumerate}

                  \item  Given that $y=a x^n+3$, that $y=4.4$ when $x=10$ and $y=12.8$ when $x=100$,
                        find the value of $n$ and of $a$.

                        \sol{}

                        When $x=10$, $y=4.4$,
                        \begin{align*}
                            4.4 & = a \cdot 10^n + 3  \\
                            a   & = \dfrac{1.4}{10^n}
                        \end{align*}

                        When $x=100$, $y=12.8$,
                        \begin{align*}
                            12.8 & = a \cdot 100^n + 3  \\
                            a    & = \dfrac{9.8}{100^n}
                        \end{align*}
                        \begin{align*}
                            \dfrac{1.4}{10^n} & = \dfrac{9.8}{100^n} \\
                            9.8 \cdot 10^n    & = 1.4 \cdot 10^{2n}  \\
                            10^n              & = 7                  \\
                            n                 & = \log_{10} 7        \\
                                              & \approx 0.8451
                        \end{align*}
                        Substituting $n=\log_{10} 7$ into $a = \dfrac{1.4}{10^n}$,
                        \begin{align*}
                            a & = \dfrac{1.4}{10^{\log_{10} 7}} \\
                              & = \dfrac{1.4}{7}                \\
                              & = 0.2
                        \end{align*}
              \end{enumerate}

              \item\begin{enumerate}
                  \item By using the substitution $y=e^x$, find the value of $x$ such that $e^{2
                                    x}=e^{x}+12$.

                        \sol{}
                        \begin{align*}
                            e^{2x}         & = e^x + 12         \\
                            y^2            & = y + 12           \\
                            y^2 - y - 12   & = 0                \\
                            (y - 4)(y + 3) & = 0                \\
                            y              & = 4 \text{ or } -3
                        \end{align*}
                        When $y = 4$, $x = \ln 4 \approx 1.39$.

                        When $y = -3$, $x = \ln (-3)$, which is not a real number.

                  \item Given that $y=a x^b+2$, and that $y=7$ when $x=3$ and $y=52$ when $x=9$, find
                        the value of $a$ and of $b$.

                        \sol{}
                        When $x = 3$, $y = 7$,
                        \begin{align*}
                            7 & = a \cdot 3^b + 2 \\
                            a & = \dfrac{5}{3^b}
                        \end{align*}
                        When $x = 9$, $y = 52$,
                        \begin{align*}
                            52 & = a \cdot 9^b + 2 \\
                            a  & = \dfrac{50}{9^b}
                        \end{align*}
                        \begin{align*}
                            \dfrac{5}{3^b} & = \dfrac{50}{9^b} \\
                            9^b \cdot 5    & = 3^b \cdot 50    \\
                            3^{2b} \cdot 5 & = 3^b \cdot 50    \\
                            3^{2b - b}     & = 10              \\
                            3^b            & = 10              \\
                            b              & = \log_3 10       \\
                                           & \approx 2.1
                        \end{align*}
                        Substituting $b = \log_3 10$ into $a = \dfrac{5}{3^b}$,
                        \begin{align*}
                            a & = \dfrac{5}{3^{\log_3 10}} \\
                              & = \dfrac{5}{10}            \\
                              & = \dfrac{1}{2}
                        \end{align*}

                        \newpage
                  \item Given that $\log _b\left(x^3 y\right)=p$ and $\log
                            _b\left(\dfrac{y}{x^2}\right)=q$, express $\log _b(x y)$ in terms of $p$ and
                        $q$.

                        \sol{}
                        \begin{align*}
                            \log_b (x^3 y) & = p              \\
                            \log_b (y)     & = p - 3 \log_b x
                        \end{align*}
                        \begin{align*}
                            \log_b \left(\dfrac{y}{x^2}\right) & = q                \\
                            \log_b (y) - 2 \log_b x            & = q                \\
                            \log_b (y)                         & = q + 2 \log_b x   \\
                            p - 3 \log_b x                     & = q + 2 \log_b x   \\
                            p                                  & = 5 \log_b x + q   \\
                            \log_b x                           & = \dfrac{p - q}{5}
                        \end{align*}
                        \begin{align*}
                            \log_b (xy) & = \log_b x + \log_b y                             \\
                                        & = \dfrac{p - q}{5} + p - 3 \cdot \dfrac{p - q}{5} \\
                                        & = \dfrac{-2p+2q}{5} + p                           \\
                                        & = \dfrac{3p + 2q}{5}
                        \end{align*}
              \end{enumerate}

        \item \begin{enumerate}
                  \item Sketch the graph of $y=\ln x$ for $x>0$. Express $xe^{x}=7.39$ in the form $\ln
                            x=a x+b$ and state the value of $a$ and of $b$. Insert on your sketch the
                        additional graph required to illustrate how a graphical solution of the
                        equation $xe^{x}=7.39$ may be obtained.

                        \sol{}

                        Lazy to draw the graph. :P

                  \item Given that $\log _3 x=r$ and $\log _9 y=s$ express $x y^2$ and $\dfrac{x^2}{y}$
                        as powers of 3 . Hence, given that $xy^2=81$ and $\dfrac{x^2}{y}=\dfrac{1}{3}$,
                        determine the value of $r$ and of $s$.

                        \sol{}
                        \begin{align*}
                            \log_3 x & = r   \\
                            x        & = 3^r
                        \end{align*}
                        \begin{align*}
                            \log_9 y & = s      \\
                            y        & = 9^s    \\
                                     & = 3^{2s}
                        \end{align*}
                        \begin{align*}
                            x y^2 & = 3^r \cdot 3^{4s} \\
                                  & = 3^{r + 4s}
                        \end{align*}
                        \begin{align*}
                            \dfrac{x^2}{y} & = \dfrac{3^{2r}}{3^{2s}} \\
                                           & = 3^{2r - 2s}
                        \end{align*}
                        \begin{align*}
                            xy^2     & = 81     \\
                            3^{r+4s} & = 81     \\
                            r + 4s   & = 4      \\
                            r        & = 4 - 4s
                        \end{align*}
                        \begin{align*}
                            \dfrac{x^2}{y} & = \dfrac{1}{3}  \\
                            3^{2r - 2s}    & = \dfrac{1}{3}  \\
                            2r - 2s        & = -1            \\
                            2(4 - 4s) - 2s & = -1            \\
                            8 - 8s - 2s    & = -1            \\
                            10s            & = 9             \\
                            s              & = \dfrac{9}{10}
                        \end{align*}
                        Substituting $s = \dfrac{9}{10}$ into $r = 4 - 4s$,
                        \begin{align*}
                            r & = 4 - 4 \cdot \dfrac{9}{10} \\
                              & = 4 - \dfrac{18}{5}         \\
                              & = \dfrac{2}{5}
                        \end{align*}
              \end{enumerate}

        \item \begin{enumerate}
                  \item Solve the equation $5^{x+1}=6$.

                        \sol{}
                        \begin{align*}
                            5^{x+1} & = 6           \\
                            x + 1   & = \log_5 6    \\
                                    & \approx 1.113 \\
                            x       & \approx 0.113
                        \end{align*}

                        \newpage
                  \item Solve the equation $\log _2 x+\log _2(6 x+1)=1$.

                        \sol{}
                        \begin{align*}
                            \log_2 x + \log_2 (6x + 1) & = 1                                      \\
                            \log_2 x(6x + 1)           & = 1                                      \\
                            x(6x + 1)                  & = 2                                      \\
                            6x^2 + x                   & = 2                                      \\
                            6x^2 + x - 2               & = 0                                      \\
                            (3x + 2)(2x - 1)           & = 0                                      \\
                            x                          & = -\dfrac{2}{3} \text{ or } \dfrac{1}{2}
                        \end{align*}
                        Upon checking, $x = \dfrac{1}{2}$ is the only valid solution.

                  \item Given that $\lg x=a$ and $\lg y=b$, express $\lg$ $\sqrt{\dfrac{1000 x^3}{y}}$
                        in terms of $a$ and $b$.

                        \sol{}
                        \begin{align*}
                            \lg \sqrt{\dfrac{1000 x^3}{y}} & = \dfrac{1}{2} \lg \dfrac{1000 x^3}{y}        \\
                                                           & = \dfrac{1}{2} (\lg 1000 + \lg x^3 - \lg y)   \\
                                                           & = \dfrac{1}{2} (3 + 3a - b)                   \\
                                                           & = \dfrac{3}{2} + \dfrac{3a}{2} - \dfrac{b}{2}
                        \end{align*}

                  \item Sketch the graph of $y=e^{2 x}$, for $-1 \leq x \leq 2$, and state the
                        coordinates of the point where the graph crosses the $y$-axis.

                        \sol{}

                        Lazy to draw the graph. :P

                  \item Sketch the graph of $y=\ln 3 x$, for $0 \leq x \leq 2$, and state the
                        coordinates of the point where the graph crosses the $x$ - axis.

                        \sol{}

                        Lazy to draw the graph. :P
              \end{enumerate}

        \item \begin{enumerate}
                  \item Draw the graph of $y=e^x$ for $0 \leq x \leq 1$, taking intervals of 0.25 .

                        By drawing a straight line on your diagram, obtain an approximate solution to
                        the equation $e^{x}=5-5 x$

                        \sol{}

                        Lazy to draw the graph. :P

                  \item Solve the equation $\lg \left(x^2+12 x-3\right)$ $=1+2 \lg x$.
                  \item By means of the substitution $y=2^x$, find the value of $x$ such that
                        $2^{x+2}-3=7 \times 2^{x-1}$

                  \item Solve the equations
                        \begin{enumerate}[label=(\roman*)]
                            \item $\lg \left(x^2-2 x+8\right)=2 \lg x$,

                                  \sol{}
                                  \begin{align*}
                                      \lg \left(x^2 - 2x + 8\right) & = 2 \lg x \\
                                      \lg \left(x^2 - 2x + 8\right) & = \lg x^2 \\
                                      x^2 - 2x + 8                  & = x^2     \\
                                      -2x + 8                       & = 0       \\
                                      x                             & = 4
                                  \end{align*}

                            \item $3^y=7$

                                  \sol{}
                                  \begin{align*}
                                      3^y & = 7          \\
                                      y   & = \log_3 7   \\
                                          & \approx 1.77
                                  \end{align*}

                            \item $\lg 5 z-\lg (3-2 z)=1$

                                  \sol{}
                                  \begin{align*}
                                      \lg 5z - \lg (3 - 2z)  & = 1            \\
                                      \lg \dfrac{5z}{3 - 2z} & = 1            \\
                                      \dfrac{5z}{3 - 2z}     & = 10           \\
                                      5z                     & = 30 - 20z     \\
                                      25z                    & = 30           \\
                                      z                      & = \dfrac{6}{5}
                                  \end{align*}
                        \end{enumerate}

                  \item \begin{enumerate}
                            \item Solve the equation $2^x=5$.

                                  \sol{}
                                  \begin{align*}
                                      2^x & = 5          \\
                                      x   & = \log_2 5   \\
                                          & \approx 2.32
                                  \end{align*}

                                  \newpage
                            \item Solve the equation $\lg x+\lg (3 x+1)=1$.

                                  \sol{}
                                  \begin{align*}
                                      \lg x + \lg (3x + 1) & = 1                           \\
                                      \lg x(3x + 1)        & = 1                           \\
                                      x(3x + 1)            & = 10                          \\
                                      3x^2 + x             & = 10                          \\
                                      3x^2 + x - 10        & = 0                           \\
                                      (3x - 5)(x + 2)      & = 0                           \\
                                      x                    & = \dfrac{5}{3} \text{ or } -2
                                  \end{align*}
                                  Upon checking, $x = \dfrac{5}{3}$ is the only valid solution.

                            \item By using the substitution $y=e^x$, find the value if $x$ such that $8
                                      e^{-x}-e^x=2$.

                                  \sol{}
                                  \begin{align*}
                                      8 e^{-x} - e^x   & = 2                \\
                                      \dfrac{8}{y} - y & = 2                \\
                                      8 - y^2          & = 2y               \\
                                      y^2 + 2y - 8     & = 0                \\
                                      (y + 4)(y - 2)   & = 0                \\
                                      y                & = -4 \text{ or } 2 \\
                                  \end{align*}
                                  When $y = 2$, $x = \ln 2$.

                                  When $y = -4$, $x = \ln (-4)$, which is not a real number.
                            \item Given that $y=a x^b$, that $y=2$ when $x=3$ and that $y=\dfrac{2}{9}$ when
                                  $x=9$, find the value of $a$ and of $b$.

                                  \sol{}
                                  When $x = 3$, $y = 2$,
                                  \begin{align*}
                                      2 & = a \cdot 3^b    \\
                                      a & = \dfrac{2}{3^b}
                                  \end{align*}
                                  When $x = 9$, $y = \dfrac{2}{9}$,
                                  \begin{align*}
                                      \dfrac{2}{9} & = a \cdot 9^b        \\
                                      a            & = \dfrac{2}{9^{b+1}}
                                  \end{align*}
                                  \begin{align*}
                                      \dfrac{2}{3^b} & = \dfrac{2}{9^{b+1}}      \\
                                      9^{b+1}        & = 3^b                     \\
                                      3^{2b + 2}     & = 3^b                     \\
                                      2b + 2         & = b                       \\
                                      b              & = -2                      \\
                                      a              & = \dfrac{2}{3^{-2}}       \\
                                                     & = \dfrac{2}{\dfrac{1}{9}} \\
                                                     & = 18
                                  \end{align*}
                        \end{enumerate}
              \end{enumerate}

        \item \begin{enumerate}
                  \item The population of a village at the beginning of the year 1800 was 240 . The
                        population increased so that, after a period of $n$ years, the new population
                        was $240(1.06)^n$, Find
                        \begin{enumerate}[label=(\roman*)]
                            \item  the population at the beginning of 1820,

                                  \sol{}
                                  \begin{align*}
                                      240(1.06)^n & = 240(1.06)^{20} \\
                                                  & \approx 770
                                  \end{align*}

                            \item the year in which the population first reached 2500.

                                  \sol{}
                                  \begin{align*}
                                      240(1.06)^n & = 2500                                    \\
                                      (1.06)^n    & = \dfrac{2500}{240}                       \\
                                                  & = \dfrac{25}{24}                          \\
                                      n           & = \log_{1.06} \dfrac{250}{24}             \\
                                                  & = \dfrac{\log \dfrac{250}{24}}{\log 1.06} \\
                                                  & \approx 40
                                  \end{align*}
                                  $\therefore$ The year is 1840.
                        \end{enumerate}

                        \newpage
                  \item Find the value of $q$ for which $\lg q=1+\lg 2-2 \lg 5$.

                        \sol{}
                        \begin{align*}
                            \lg q & = 1 + \lg 2 - 2 \lg 5          \\
                                  & = \lg (\dfrac{2 \cdot 10}{25}) \\
                                  & = \lg \dfrac{4}{5}             \\
                            q     & = \dfrac{4}{5}
                        \end{align*}

                  \item By using the substitution $u=2^x$, solve the equation
                        $4^x-9\left(2^x\right)+8=0$.

                        \sol{}
                        \begin{align*}
                            4^x - 9 \cdot 2^x + 8 & = 0               \\
                            u^2 - 9u + 8          & = 0               \\
                            (u - 8)(u - 1)        & = 0               \\
                            u                     & = 8 \text{ or } 1 \\
                            2^x                   & = 8 \text{ or } 1 \\
                            x                     & = 3 \text{ or } 0
                        \end{align*}

                  \item Sketch the curve $y=e^{2 x-1}$ and calculate, correct to two decimal places,
                        the gradient of the curve at the point where it meets the $y$-axis.

                        \sol{}

                        Lazy to draw the graph. :P
              \end{enumerate}

        \item \begin{enumerate}
                  \item The curve $y=a b^x$ passes through the points $(1,96),(2,1152)$ and $(3, p)$.
                        Find the exact values of $a, b$ and $p$.

                        \sol{}
                        When $x = 1$, $y = 96$,
                        \begin{align*}
                            96 & = a b^1         \\
                            a  & = \dfrac{96}{b}
                        \end{align*}
                        When $x = 2$, $y = 1152$,
                        \begin{align*}
                            1152 & = a b^2             \\
                            a    & = \dfrac{1152}{b^2}
                        \end{align*}
                        \begin{align*}
                            \dfrac{96}{b} & = \dfrac{1152}{b^2} \\
                            96b           & = 1152              \\
                            b             & = 12
                        \end{align*}
                        Substituting $b = 12$ into $a = \dfrac{96}{b}$,
                        \begin{align*}
                            a & = \dfrac{96}{12} \\
                              & = 8
                        \end{align*}
                        When $x = 3$, $y = p$,
                        \begin{align*}
                            p & = 8 \cdot 12^3 \\
                              & = 13824
                        \end{align*}

                  \item Solve the equation $\lg (4 x+5)=1+\lg (x-1)$.

                        \sol{}
                        \begin{align*}
                            \lg (4x + 5) & = 1 + \lg (x - 1) \\
                            \lg (4x + 5) & = \lg 10(x - 1)   \\
                            4x + 5       & = 10(x - 1)       \\
                            4x + 5       & = 10x - 10        \\
                            6x           & = 15              \\
                            x            & = \dfrac{5}{2}
                        \end{align*}

                  \item Find the coordinates of the stationary point of the curve $y=x e^{-x}$.

                        Draw the curve $y=x e^{-x}$ for $-1 \leq x \leq 2$ and use your graph to
                        estimale the solution of the equation $x+e^{x}=0$.

                        \sol{}

                        Lazy to draw the graph. :P
              \end{enumerate}

        \item \begin{enumerate}
                  \item Solve the equation
                        \begin{enumerate}[label=(\roman*)]
                            \item $3 \lg (x-1)=\lg 8$,

                                  \sol{}
                                  \begin{align*}
                                      3 \lg (x - 1) & = \lg 8 \\
                                      \lg (x - 1)^3 & = \lg 8 \\
                                      (x - 1)^3     & = 8     \\
                                      x - 1         & = 2     \\
                                      x             & = 3
                                  \end{align*}

                            \item $\lg (20 y)-\lg (y-8)=2$.

                                  \sol{}
                                  \begin{align*}
                                      \lg (20y) - \lg (y - 8) & = 2          \\
                                      \lg \dfrac{20y}{y - 8}  & = 2          \\
                                      \dfrac{20y}{y - 8}      & = 100        \\
                                      20y                     & = 100y - 800 \\
                                      80y                     & = 800        \\
                                      y                       & = 10
                                  \end{align*}
                        \end{enumerate}

                  \item By using the substitution $y=e^{2 x}$, solve the equation $$ e^{2 x}+4 e^{-2
                                    x}=4 $$

                        \sol{}
                        \begin{align*}
                            e^{2x} + 4e^{-2x} & = 4                \\
                            y + \dfrac{4}{y}  & = 4                \\
                            y^2 + 4           & = 4y               \\
                            y^2 - 4y + 4      & = 0                \\
                            (y - 2)^2         & = 0                \\
                            y                 & = 2                \\
                            e^{2x}            & = 2                \\
                            2x                & = \ln 2            \\
                            x                 & = \dfrac{\ln 2}{2}
                        \end{align*}
              \end{enumerate}

        \item Solve
              \begin{enumerate}
                  \item $3^x=2$,

                        \sol{}
                        \begin{align*}
                            3^x & = 2           \\
                            x   & = \log_3 2    \\
                                & \approx 0.631
                        \end{align*}

                  \item $\log _3(4 x)+\log _3(x-1)=1$.

                        \sol{}
                        \begin{align*}
                            \log_3 (4x) + \log_3 (x - 1) & = 1                                      \\
                            \log_3 [4x(x - 1)]           & = 1                                      \\
                            4x(x - 1)                    & = 3                                      \\
                            4x^2 - 4x                    & = 3                                      \\
                            4x^2 - 4x - 3                & = 0                                      \\
                            (2x - 3)(2x + 1)             & = 0                                      \\
                            x                            & = -\dfrac{1}{2} \text{ or } \dfrac{3}{2}
                        \end{align*}
                        Upon checking, $x = \dfrac{3}{2}$ is the only valid solution.
              \end{enumerate}
    \end{enumerate}

\end{multicols*}
\end{document}