\documentclass{report}

\usepackage[fleqn]{amsmath}
\usepackage{amssymb}
\usepackage{setspace}
\usepackage{enumitem}
\usepackage{fontspec}
\usepackage{titlesec}
\usepackage{nicematrix}
\usepackage[total={6.6in,9.2in}]{geometry}

\newcounter{example}
\setcounter{example}{1}

\setcounter{chapter}{10}
\titleformat{\chapter}[display]
{\normalfont\huge\bfseries}{\chaptertitlename\ \thechapter}{20pt}{\Huge}
\titlespacing*{\chapter}{0pt}{0pt}{40pt}
\setmainfont{Times New Roman}

\newlist{example}{enumerate}{2}
\setlist[example]{label=\textbf{Example \arabic*}, leftmargin=*, ref=\theexample(\arabic*), resume}

\newlist{solution}{enumerate}{2}
\setlist[solution]{label=\textbf{Solution}, leftmargin=*}

\begin{document}

\onehalfspacing

\chapter{Logical Reasoning}

\section{Logic}

\textbf{Logic} is a branch of science that studies the way we reason and its patterns. When we are reasoning, We need to use concepts, make judgments, and make inferences. Concept, judgment, and inference are the three basic elements of the thinking process. They are interconnected and follow a certain pattern.

The science of logic emerged over 2000 years ago. Ancient philosophers already
started to study the formation of thinking and its patterns from a long time
ago. Aristotle from ancient Greek was the first to systematically study logic,
thus he is known as the father of classical logic. In the 17th century, Leibniz
from Germany was the first to put forward the idea of using symbolic operations
to study logic problems. This has led to the emergence of mathematical logic, a
branch of logic that uses mathematical methods to study logic problems. In the
19th century, Boole, a British mathematician, created a fringe science that is
in between algebra and logic, known as \textbf{Logical algebra} (also known as
\text{Boolean algebra}). Since the 20th century, mathematical logic has
received greater and deeper development. It describes and studies logic more
accurately and mathematically. It has provided a meaningful tool and method for
the study of the foundation of mathematics. Its research on computerizing,
programming, and mechanizing the thinking process has also become the
theoretical basis of computer science.

Human society is evolving into an age of information. The popularization of
computers has led to the digitalization of science and technology, the
mechanization of human thinking, and computerization is increasing day by day.
Mathematics as a logically rigorous basic science, is playing an increasingly
important role, and logic as one of the foundations of mathematics is also
increasingly valued by people.

\section{Proposition}

\textbf{Proposition} is a declarative sentence that is used to express a certain judgment. For example,
\begin{enumerate}[label=(\alph*)]
    \item The sum of the interior angles of a triangle is $180^\circ$.
    \item $\sqrt{2}$ is not a rational number.
    \item $(a+b)^2$ is equal to $a^2 + 2ab + b^2$. (That is, the equation $a^2 + 2ab + b^2 = (a+b)^2$)
    \item Two lines that are perpendicular to the same plane are parallel to each other.
    \item The equation $x^2 + 4x + 5 = 0$ has two real roots.
    \item $\sin^2 x - \cos^2 x = 2$.
\end{enumerate}
Some of these sentences are true, some are false. The sentences (a), (b), (c), (d) above are true, while (e) and (f) are false.

The sentences that can be judged as true or false are called
\textbf{propositions}. All the six sentences above are propositions.

Some sentences cannot be judged as true or false, these kinds of uncertain
sentences are not propositions. For example,
\begin{enumerate}[label=(\alph*), start=7]
    \item The two base angles of $\triangle ABC$ are equal.
    \item $a$ is the smallest among the three numbers $a$, $b$, and $c$.
\end{enumerate}
Since $\triangle ABC$ and the number $a$, $b$, and $c$ are not specified, the sentence (g) and (h) cannot be judged as true or false.

The proposition that is true is called a \textbf{true proposition}, and the
proposition that is false is called a \textbf{false proposition}. Generally, we
use small letters $p$, $q$, $r$, $s$, $\cdots$ to represent propositions. For
example, below are 4 propositions represented by $p$, $q$, $r$, and $s$
respectively.
\begin{flalign*}
    p: & \ \sin^2 x + \cos^2 x = 1                    \\
    q: & \ \text{When } x \in \mathbb{R},\ x^2 \geq 0 \\
    r: & \ 3\sin x = 4                                \\
    s: & \ \emptyset \in \{0\}
\end{flalign*}
Among the propositions above, $p$ and $q$ are true propositions, while $r$ and $s$ are false propositions. The true or false of a proposition is called the \textbf{truth value} of the proposition. We stipulate that the truth value of a true proposition is 1, and the truth value of a false proposition is 0. For example, the propositions $p$ and $q$ above are true, denoted as $p = 1$ and $q = 1$, while the propositions $r$ and $s$ are false, denoted as $r = 0$ and $s = 0$.

\vspace{0.5cm}
\begin{example}
    \item State whether the following sentences are propositions. If it is a proposition,
    state whether it is true or false and give the reason to your answer.
    \begin{flalign*}
        p: & \ \text{The square of any number is not less than zero.}                            & \\
        q: & \ \text{The parabola $y = x^2 + 1$ has no point of intersection with the $x$-axis.}   \\
        r: & \ x-y = 0.                                                                            \\
        s: & \ \text{At least two interior angles of a triangle are acute angles.}                 \\
        t: & \ \text{For any real number $x$, $2x + 1 > x$.}
    \end{flalign*}
\end{example}
\begin{solution}
    \item \begin{enumerate}[label=]
        \item $p$ is a true proposition. The square of any number is non-negative.
        \item $q$ is a true proposition. The parabola $y = x^2 + 1$ is on top of the $x$-axis and its vertex is $(0, 1)$.
        \item $r$ is not a proposition. For any $x$ and $y$, we cannot tell whether $x-y$ is equal to 0.
        \item $s$ is a true proposition. The sum of the interior angles of a triangle is $180^\circ$, there cannot be more two obtuse angle or straight angle at the same time.
        \item $t$ is a false proposition. For example, $2(-1) + 1 < -2$.
    \end{enumerate}
\end{solution}
\newpage
\begin{example}
    \item Write down the truth value of the following propositions.
    \begin{flalign*}
        p: & \ \text{For any real number $x$, $x < x + 1$.}                 & \\
        q: & \ \text{For any real number $a$, if $a^3 > 0$, then $a > 0$.}    \\
        r: & \ \text{The period of the function $y = \sin x$ is $\pi$.}       \\
        s: & \ \text{The equation $2\sin x - \cos x = 4$ has no solution.}    \\
        t: & \ \text{The line $3x - 4y + 1 = 0$ passes through the origin.}
    \end{flalign*}
\end{example}
\begin{solution}
    \item \begin{enumerate}[label=]
        \item $p = 1 \qquad q = 1 \qquad r = 0 \qquad s = 1 \qquad t = 0$
    \end{enumerate}
\end{solution}

\subsection*{Exercise 11a}
\begin{enumerate}[leftmargin=*]
    \item State whether the following sentences are propositions.
          \begin{enumerate}[label=, leftmargin=*]
              \item $p$: The equation $x^2 - 5x + 6 = 0$ has two positive real roots.
              \item $q$: $x + 5 = y + 3$.
              \item $r$: THe line $y = 3x + b$ and the line $y = 3x - b$ ($b \neq 0$) are parallel to each other.
              \item $s$: $\triangle ABC \cong \triangle A'B'C'$.
              \item $t$: 5 is the greatest common factor of 25 and 30.
              \item $u$: The probability of a sure event is 1, and the probability of an impossible event is 0.
          \end{enumerate}
    \item Write down the truth value of the following propositions, and give a
          counterexample for the false propositions.
          \begin{enumerate}[label=, leftmargin=*]
              \item $p$: All the even numbers are not prime numbers.
              \item $q$: When $x \in \mathbb{R}$, $x^2 + x + 1$ is always greater than 0.
              \item $r$: The maximum value of the function $y = ax^2 + bx + c$ ($a \neq 0$) is $\dfrac{4ac - b^2}{4a}$.
              \item $s$: The equation $\sin x < \sin 2x$ is true for any real number $x$.
              \item $t$: The solution set of the equation $\sin x = \cos x$ is $\left\{x | x = \dfrac{\pi}{4} + 2k\pi, k \in \mathbb{Z}\right\}$.
              \item $u$: The value of the function $y = 2^x$ is always greater than 0, where $x$ is any real number.
          \end{enumerate}
\end{enumerate}

\section{Compound Propositions}

Lets consider the following propositions:
\begin{enumerate}[label=]
    \item $p$: 4 is a factor of 8.
    \item $q$: 4 is a factor of 12.
    \item $r$: 4 is not a factor of 8.
    \item $s$: 4 is a factor of 8 and is a factor of 12.
    \item $t$: 4 is a factor of 8 or is a factor of 12.
\end{enumerate}

Among these propositions, some of them are true propositions, while some of
them are false propositions. In the proposition $r$, $s$, and $t$, they contain
the words "not", "and", and "or" respectively. These words are called
\textbf{logical connectives}.

Propositions that do not contain any logical connectives are called
\textbf{simple propositions}. For example, the propositions $p$ and $q$ above
are simple propositions.

Propositions that are formed by connecting simple propositions using logical
connectives are called \textbf{compound propositions}. For example, the
propositions $r$, $s$, and $t$ above are compound propositions.

\subsection*{Inverse Proposition and its Truth Table}

Let $p$ be a proposition. Adding a logical connective "not" to $p$ gives us a
new proposition, that is, the inverse proposition of $p$, denoted by $\neg p$,
read as "not $p$".

The meaning of the inverse proposition $\neg p$ is the \textbf{negation} of the
original proposition $p$.

\vspace{0.5cm}
\begin{example}
    \item Write down the inverse proposition of the following propositions:
    \begin{enumerate}[label=, leftmargin=*]
        \item $p$: $2 + 2 = 4$.
        \item $q$: 25 is a multiple of 5.
        \item $r$: The equation $y = 3x^3 - x$ is an odd function.
        \item $s$: The square of 15 is 235.
        \item $t$: The base number $a$ of $\log_a x$ can be a negative number.
    \end{enumerate}
\end{example}
\begin{solution}
    \item \begin{enumerate}[label=, leftmargin=*]
        \item $\neg p$: $2 + 2 \neq 4$.
        \item $\neg q$: 25 is not a multiple of 5.
        \item $\neg r$: The equation $y = 3x^3 - x$ is not an odd function.
        \item $\neg s$: The square of 15 is not 235.
        \item $\neg t$: The base number $a$ of $\log_a x$ cannot be a negative number.
    \end{enumerate}
\end{solution}

Apparently, if $p$ is a true proposition, then $\neg p$ is a false proposition.
If $p$ is a false proposition, then $\neg p$ is a true proposition. The truth
value of $p$ and $\neg p$ are opposite to each other, and their truth table is
as follows:
\begin{center}
    \begin{NiceTabular}{|c|c|}[code-before = \rowcolor{lightgray}{1}, hvlines]
        $p$ & $\neg p$ \\
        1   & 0        \\
        0   & 1        \\
    \end{NiceTabular}
\end{center}
This kind of table consisting of true and false that is used to analyse a compound proposition is called a \textbf{truth table}.

\vspace{0.5cm}
\begin{example}
    \item Write down the inverse proposition of the following propositions:
    \begin{enumerate}[label=, leftmargin=*]
        \item $s$: All integers are positive numbers.
        \item $t$: THe square root of 4 must be 2.
        \item $u$: All equilateral triangles are isosceles triangles.
        \item $v$: All even numbers are divisible by 3.
    \end{enumerate}
\end{example}
\begin{solution}
    \item \begin{enumerate}[label=, leftmargin=*]
        \item $\neg s$: There exists an integer that is not a positive number.
        \item $\neg t$: The square root of 4 is not necessary 2.
        \item $\neg u$: There exists an equilateral triangle that is not an isosceles triangle.
        \item $\neg v$: All even numbers are not divisible by 3.
    \end{enumerate}
\end{solution}

\begin{enumerate}[label=\textbf{NOTE: }, leftmargin=*]
    \item In the example above, proposition $s$ is a false proposition, expressing $\neg
              s$ as ``There exists an integer that is not a positive number'' is not
          accurate. Therefore caution must be taken that either the proposition or its
          inverse proposition must be true, but not both, and they cannot be both false
          either.
\end{enumerate}

\subsection*{Exercise 11b}
\begin{enumerate}[leftmargin=*]
    \item Write down the inverse proposition of the following propositions:
          \begin{enumerate}[label=, leftmargin=*]
              \item $p$: The sum of two sides of a triangle is greater than the third side.
              \item $q$: The smallest natural number is 1.
              \item $r$: $\pi$ belongs to the set of irrational numbers.
              \item $s$: The equation $x^2 + 2x + 2 = 0$ has real roots.
              \item $t$: Two lines with equal gradient are not necessary parallel to each other.
              \item $u$: All the numbers with the last digit 3 are divisible by 3.
              \item $v$: Regular polygons have equal sides.
          \end{enumerate}
    \item Write down the truth value of the propositions and their inverse propositions
          in Question 1. If the proposition is true, give a reason. If the proposition
\end{enumerate}

\subsection*{Conjunctive Proposition and its Truth Table}

Let $p$ and $q$ be two propositions. Adding a logical connective "and" to $p$
and $q$ gives us a new proposition, that is, the conjunctive proposition of $p$
and $q$, denoted by $p \land q$, read as "$p$ and $q$". In the field of
Mathematics, $p \land q$ is also read as the \textbf{conjunction} of $p$ and
$q$.

\begin{enumerate}[label=\indent For example, leftmargin=*]
    \item $p$: The weather is clear today.

          $q$: Today is a warm day.

          $p \land q$: The weather is clear and warm today.
\end{enumerate}

The definition of the conjunctive proposition $p \land q$ is that when $p$ and
$q$ are both true, then $p \land q$ is true, otherwise $p \land q$ is false.

The truth table of $p \land q$ is as follows:
\begin{center}
    \begin{NiceTabular}{|c|c|c|}[code-before = \rowcolor{lightgray}{1}, hvlines]
        $p$ & $q$ & $p \land q$ \\
        1   & 1   & 1           \\
        1   & 0   & 0           \\
        0   & 1   & 0           \\
        0   & 0   & 0           \\
    \end{NiceTabular}
\end{center}

\vspace{0.5cm}
\begin{example}
    \item Write down the conjunctive proposition of the following pairs of propositions:
    \begin{enumerate}[label=(\alph*), leftmargin=*]
        \item $p$: 9 is an odd number.

              $q$: 9 is a composite number.

        \item $p$: A trapezium has two parallel sides.

              $q$: A trapezium has at least two equal sides.

        \item $p$: The equation $x^2 - 2x + 1 = 0$ has two distinct roots.

              $q$: The two roots of the equation $x^2 - 2x + 1 = 0$ are both negative numbers.

        \item $p$: The probability of a sure event is 1.

              $q$: The probability of an impossible event is 0.
    \end{enumerate}
\end{example}
\begin{solution}
    \item \begin{enumerate}[label=, leftmargin=*]
        \item $p \land q$: 9 is an odd number and is a composite number.
        \item $p \land q$: A trapezium has two parallel sides and has at least two equal sides.
        \item $p \land q$: The equation $x^2 - 2x + 1 = 0$ has two distinct roots and the two roots are both negative numbers.
        \item $p \land q$: The probability of a sure event is 1 and the probability of an impossible event is 0.
    \end{enumerate}
\end{solution}
\vspace{0.1cm}
\begin{example}
    \item Write down the truth value of the pairs of propositions and their conjunctive
    propositions in Question 5.
\end{example}
\begin{solution}
    \item \begin{enumerate}[label=, leftmargin=*]
        \item $p = 1 \qquad q = 1 \qquad p \land q = 1$
        \item $p = 1 \qquad q = 0 \qquad p \land q = 0$
        \item $p = 0 \qquad q = 0 \qquad p \land q = 0$
        \item $p = 0 \qquad q = 0 \qquad p \land q = 0$
    \end{enumerate}
\end{solution}

\subsection*{Exercise 11c}
\begin{enumerate}[leftmargin=*]
    \item Write down the conjunctive proposition of the following pairs of propositions:
          \begin{enumerate}[label=(\alph*), leftmargin=*]
              \item $p$: Four sides of a square are equal.

                    $q$: Four angles of a square are equal.

              \item $p$: There exists the smallest element in the set of natural numbers.

                    $q$: There exists the largest element in the set of natural numbers.

              \item $p$: The sum of two sides of a triangle is greater than the third side.

                    $q$: The difference between two sides of a triangle is less than the third side.

              \item $p$: The sign of the two roots of the equation $x^2 - 9 = 0$ are different.

                    $q$: The absolute values of the two roots of the equation $x^2 - 9 = 0$ are different.

              \item $p$: The solution set of the inequality $x^2 - 4x + 5 > 0$ is a set of positive real numbers.

                    $q$: The solution set of the inequality $x^2 - 4x + 5 < 0$ is $\emptyset$.
          \end{enumerate}

    \item Write down the truth value of the pairs of propositions and their conjunctive
          propositions in Question 1.

    \item Given that $p = 1$, $q = 0$, $r = 0$, and $s = 1$, write down the truth value
          of $(\neg p) \land (\neg q)$, $(\neg r) \land s$, and $(\neg q) \land (\neg
              r)$, $(\neg p) \land (\neg s)$.
\end{enumerate}

\subsection*{Disjunctive Proposition and its Truth Table}

Let $p$ and $q$ be two propositions. Adding a logical connective "or" to $p$
and $q$ gives us a new proposition, that is, the disjunctive proposition of $p$
and $q$, denoted by $p \lor q$, read as "$p$ or $q$". In the field of
Mathematics, $p \lor q$ is also read as the \textbf{disjunction} of $p$ and
$q$.

The definition of the disjunctive proposition $p \lor q$ is that when either
$p$ or $q$ is true, then $p \lor q$ is true. The only case when $p \lor q$ is
false is when $p$ and $q$ are both false.

The truth table of $p \lor q$ is as follows:
\begin{center}
    \begin{NiceTabular}{|c|c|c|}[code-before = \rowcolor{lightgray}{1}, hvlines]
        $p$ & $q$ & $p \lor q$ \\
        1   & 1   & 1          \\
        1   & 0   & 1          \\
        0   & 1   & 1          \\
        0   & 0   & 0          \\
    \end{NiceTabular}
\end{center}
\begin{enumerate}[label=\indent For example, leftmargin=*]
    \item $p$: I will play basketball tomorrow.

          $q$: I will go swimming tomorrow.

          $p \lor q$: I will play basketball or go swimming tomorrow.
\end{enumerate}

If I said to you that I will play basketball or go swimming tomorrow, and I
actually were to do one of them, then $p \lor q$ is undoubtedly true. If I were
to play basketball but not go swimming, then you can't say I am lying to you,
hence $p \lor q$ is still true. Only if I were to do neither of them, then you
can say I am lying to you, hence $p \lor q$ is false. This matches what are
being shown in the truth table.
\begin{enumerate}[label=\textbf{NOTE: }, leftmargin=*]
    \item In our daily life, sometimes the word ``or'' might have different meaning than
          the one in the truth table. Take the word ``or'' in the restaurant menu as an
          example, it means that only one of the two choices can be chosen, ``coffee or
          tea'' means that if you choose coffee, you cannot choose tea, and vice versa.
          All the propositions that will be mentioned in the later part of this chapter
          are using the meaning of ``or'' in the truth table.
\end{enumerate}

\vspace{0.5cm}
\begin{example}
    \item Write down the disjunctive proposition of the following pairs of propositions:
    \begin{enumerate}[label=(\alph*), leftmargin=*]
        \item $p$: 2 is a natural number.

              $q$: 2 is an even number.

        \item $p$: The equation $x^2 + 4x + 3 = 0$ has two distinct real roots.

              $q$: The equation $x^2 + 4x + 3 = 0$ has two equal real roots.

        \item $p$: The function $y = x^2 + 4x + 3$ has a maximum value.

              $q$: The function $y = x^2 + 4x + 3$ has a minimum value.

        \item $p$: The sum of interior angles of a quadrilateral is $180^\circ$.

              $q$: The sum of exterior angles of a quadrilateral is $540^\circ$.
    \end{enumerate}
\end{example}
\begin{solution}
    \item \begin{enumerate}[label=, leftmargin=*]
        \item $p \lor q$: 2 is either a natural number, or an even number.
        \item $p \lor q$: The equation $x^2 + 4x + 3 = 0$ has either two distinct real roots, or two equal real roots.
        \item $p \lor q$: The function $y = x^2 + 4x + 3$ has either a maximum value or a minimum value.
        \item $p \lor q$: The sum of interior angles of a quadrilateral is either $180^\circ$, or $540^\circ$.
    \end{enumerate}
\end{solution}
\vspace{0.1cm}
\begin{example}
    \item Write down the truth value of the pairs of propositions and their disjunctive
    propositions in Question 7.
\end{example}
\begin{solution}
    \item \begin{enumerate}[label=, leftmargin=*]
        \item $p = 1 \qquad q = 1 \qquad p \lor q = 1$
        \item $p = 1 \qquad q = 0 \qquad p \lor q = 1$
        \item $p = 0 \qquad q = 1 \qquad p \lor q = 1$
        \item $p = 0 \qquad q = 0 \qquad p \lor q = 0$
    \end{enumerate}
\end{solution}

\subsection*{Exercise 11d}
\begin{enumerate}[leftmargin=*]
    \item Write down the disjunctive proposition of the following pairs of propositions:
          \begin{enumerate}[label=(\alph*), leftmargin=*]
              \item $p$: The diagonals of a rhombus are perpendicular to each other.

                    $q$: The diagonals of a rhombus bisects each other.

              \item $p$: The diagonals of a rectangle are equal.

                    $q$: The diagonals of a rectangle are perpendicular to each other.

              \item $p$: The square of an even number is an odd number.

                    $q$: The square of an even number is an even number.
          \end{enumerate}

    \item Write down the truth value of the pairs of propositions and their conjunctive
          propositions in Question 1.

    \item When $p \lor q = 0$, what is the truth value of $p \land q$? Explain your
          answer.

    \item When $p \land q = 0$, can the truth value of $p \lor q$ be determined? Explain
          your answer.
\end{enumerate}

\section{Truth Table and Logical Equivalence}

\subsection*{Truth Table}

Regardless of the complexity of a proposition, its truth value in any case can
always be calculated using a truth table. \vspace{0.5cm}
\begin{example}
    \item Construct a truth table for $\neg(p\land\neg q)$.
\end{example}
\begin{solution}
    \item
    \begin{NiceTabular}{|c|c|c|c|c|}[code-before = \rowcolor{lightgray}{1}, hvlines, t]
        $p$ & $q$ & $\neg q$ & $p \land \neg q$ & $\neg(p \land \neg q)$ \\
        1   & 1   & 0        & 0                & 1                      \\
        1   & 0   & 1        & 1                & 0                      \\
        0   & 1   & 0        & 0                & 1                      \\
        0   & 0   & 1        & 0                & 1                      \\
    \end{NiceTabular}
\end{solution}

The steps of constructing the table are as follows:
\begin{enumerate}[leftmargin=*, label=\indent(\arabic*)]
    \item The propositions $p$ and $q$ has two possibilities each, so when combining
          them, there are $2 \times 2 = 4$ possibilities.
    \item Find the truth value of $\neg q$, it is the opposite of $q$.
    \item Find the conjunction of $p$ and $\neg q$, i.e. $p \land \neg q$. According to
          the truth table of $p \land q$, when $p$ and $q$ are both true, $p \land q$ is
          true, otherwise $p \land q$ is false.
    \item Find the truth value of $\neg(p \land \neg q)$, it is the opposite of $p \land
              \neg q$.
\end{enumerate}

\vspace{0.5cm}
\begin{example}
    \item Construct a truth table for $(p \lor \neg q) \land q$.
\end{example}
\begin{solution}
    \item
    \begin{NiceTabular}{|c|c|c|c|c|c|c|}[code-before = \rowcolor{lightgray}{1}, hvlines, t]
        $p$ & $q$ & $\neg q$ & $p \lor \neg q$ & $(p \lor \neg q) \land q$ \\
        1   & 1   & 0        & 1               & 1                         \\
        1   & 0   & 1        & 1               & 0                         \\
        0   & 1   & 0        & 0               & 0                         \\
        0   & 0   & 1        & 1               & 0                         \\
    \end{NiceTabular}
\end{solution}

Steps:
\begin{enumerate}[leftmargin=*, label=\indent(\arabic*)]
    \item Write down the combinations of the truth values of $p$ and $q$.
    \item Find the truth value of $\neg q$.
    \item Find the truth value of $p \lor \neg q$. According to the truth table of $p
              \lor q$, When both $p$ and $q$ are false, $p \lor q$ is false, otherwise true.
    \item Find the truth value of $(p \lor \neg q) \land q$.
\end{enumerate}

If a compound proposition is true in all cases, then this proposition is called
a \textbf{tautology}. On the other hand, if a compound proposition is false in
all cases, then this proposition is called a \textbf{contradiction}. For
example,

$p \lor \neg p$ is a tautology, its truth table is as follows:
\begin{center}
    \begin{NiceTabular}{|c|c|c|}[code-before = \rowcolor{lightgray}{1}, hvlines, t]
        $p$ & $\neg p$ & $p \lor \neg p$ \\
        1   & 0        & 1               \\
        0   & 1        & 1               \\
    \end{NiceTabular}
\end{center}

$p \land \neg p$ is a contradiction, its truth table is as follows:
\begin{center}
    \begin{NiceTabular}{|c|c|c|}[code-before = \rowcolor{lightgray}{1}, hvlines, t]
        $p$ & $\neg p$ & $p \land \neg p$ \\
        1   & 0        & 0                \\
        0   & 1        & 0                \\
    \end{NiceTabular}
\end{center}

\subsection*{Exercise 11e}
\begin{enumerate}[leftmargin=*]
    \item Construct a truth table for each of the following propositions:
          \begin{enumerate}[leftmargin=*]
              \item $\neg p \wedge q$
              \item $\neg(p \vee q)$
              \item $\neg(p \vee \neg q)$
              \item $(\neg p \vee q) \wedge p$
          \end{enumerate}
    \item Determine whether the following propositions are tautologies or contradictions:
          \begin{enumerate}[leftmargin=*]
              \item $p \wedge \neg q$
              \item $p \vee \neg(p \wedge q)$
              \item $(p \wedge q) \wedge \neg(p \vee q)$
              \item $(p \wedge \neg p) \wedge q$
          \end{enumerate}
    \item Construct a truth table for each of the following propositions:
          \begin{enumerate}[leftmargin=*]
              \item $p \wedge(q \vee r)$
              \item $(\mathrm{p} \wedge q) \vee(p \wedge r)$
          \end{enumerate}
\end{enumerate}

\subsection*{Logical Equivalence}

Let $P$ and $Q$ be two compound propositions, if $P$ and $Q$ have the same
truth table, that is, when the same truth values are being chosen for the
simple propositions in $P$ and $Q$, $P$ and $Q$ have the same truth value, then
$P$ and $Q$ are called \textbf{logically equivalent}, denoted by $P \equiv Q$.

For example, the truth table of $\neg (p \land q)$ and $\neg p \lor \neg q$ is
as follows:
\begin{center}
    \begin{NiceTabular}{|c|c|c|c|c|c|c|}[code-before = \rowcolor{lightgray}{1}, hvlines, t]
        $p$ & $q$ & $p \land q$ & $\neg(p \land q)$ & $\neg p$ & $\neg q$ & $\neg p \lor \neg q$ \\
        1   & 1   & 1           & 0                 & 0        & 0        & 0                    \\
        1   & 0   & 0           & 1                 & 0        & 1        & 1                    \\
        0   & 1   & 0           & 1                 & 1        & 0        & 1                    \\
        0   & 0   & 0           & 1                 & 1        & 1        & 1                    \\
    \end{NiceTabular}
\end{center}

From the truth table above, we can see that in all cases, $\neg (p \land q)$
and $\neg p \lor \neg q$ have the same truth value, hence $\neg (p \land q)$
and $\neg p \lor \neg q$ are logically equivalent, i.e. \noindent

\begin{center}
    \fbox{%
        \parbox{20em}{\begin{center}
                $\neg (p \land q) \equiv \neg p \lor \neg q$
            \end{center}}}%
\end{center}

The meaning of this expression is that the negation of the conjunction of $p$
and $q$ is logically equivalent to the disjunction of the negation of $\neg p$
and $\neg q$.

Similarly, we can list down the following truth table:
\begin{center}
    \begin{NiceTabular}{|c|c|c|c|c|c|c|}[code-before = \rowcolor{lightgray}{1}, hvlines, t]
        $p$ & $q$ & $p \lor q$ & $\neg(p \lor q)$ & $\neg p$ & $\neg q$ & $\neg p \land \neg q$ \\
        1   & 1   & 1          & 0                & 0        & 0        & 0                     \\
        1   & 0   & 1          & 0                & 0        & 1        & 0                     \\
        0   & 1   & 1          & 0                & 1        & 0        & 0                     \\
        0   & 0   & 0          & 1                & 1        & 1        & 1                     \\
    \end{NiceTabular}
\end{center}

From the truth table above, we can see that $\neg (p \lor q)$ and $\neg p \land
    \neg q$ are logically equivalent, i.e.

\begin{center}
    \fbox{%
        \parbox{20em}{\begin{center}
                $\neg (p \lor q) \equiv \neg p \land \neg q$
            \end{center}}}%
\end{center}

The meaning of this expression is that the negation of the disjunction of $p$
and $q$ is logically equivalent to the conjunction of the negation of $\neg p$
and $\neg q$.

The two formulas above are known as the \textbf{De Morgan's Laws}.
\vspace{0.5cm}
\begin{example}
    \item Use the truth table to prove that $\neg (\neg p) \equiv p$.
\end{example}
\begin{solution}
    \item
    \begin{NiceTabular}{|c|c|}[code-before = \rowcolor{lightgray}{1}, hvlines, t]
        $p$ & $\neg p$ \\
        1   & 0        \\
        0   & 1        \\
    \end{NiceTabular}\\

    From the truth table above, we know that $\neg (\neg p) \equiv p$.

    That means, negating the original proposition two times, we get the original
    proposition itself.

    $\neg (\neg p) \equiv p$ is known as the \textbf{double negation law}.
\end{solution}

\vspace{0.5cm}
\begin{example}
    \item Find the inverse proposition of the following propositions:
    \begin{enumerate}[label=(\alph*)]
        \item The opposite sides of a parallelogram are parallel and equal to each other.
        \item All the odd numbers are divisible by 3 or 5.
    \end{enumerate}
\end{example}
\begin{solution}
    \item \begin{enumerate}[label=(\alph*)]
        \item $p$: The opposite sides of a parallelogram are parallel to each other.

              $q$: The opposite sides of a parallelogram are equal to each other.

              $p \land q$: The opposite sides of a parallelogram are parallel and equal to each other.

              The target inverse proposition is $\neg(p \land q)$.

              And $\neg(p \land q) \equiv \neg p \lor \neg q$.

              $\neg p$: The opposite sides of a parallelogram are not parallel to each other.

              $\neg q$: The opposite sides of a parallelogram are not equal to each other.

              $\neg p \lor \neg q \equiv \neg(p \land q)$: The opposite sides of a parallelogram are not parallel to each other or are not equal to each other.

        \item $p$: All the odd numbers are divisible by 3.

              $q$: All the odd numbers are divisible by 5.

              $p \lor q$: All the odd numbers are divisible by 3 or 5.

              And $\neg(p \lor q) \equiv \neg p \land \neg q$.

              $\neg p$: Not all the odd numbers are divisible by 3.

              $\neg q$: Not all the odd numbers are divisible by 5.

              $\therefore \neg p \land \neg q$: Not all the odd numbers are divisible by 3 and 5.
    \end{enumerate}
\end{solution}
\vspace{0.1cm}
\begin{example}
    \item Given that $p = 0$, $q = 1$, find the truth value of $\neg (p \land q)$ and
    $\neg (p \lor q)$.
\end{example}

\begin{enumerate}[label=\textbf{Sol. \arabic*}, leftmargin=*]
    \item $\begin{aligned}[t]
                   & \because p = 0, q = 1                                \\
                   & \therefore p \land q = 0, p \lor q = 1               \\
                   & \therefore \neg (p \land q) = 1, \neg (p \lor q) = 0
              \end{aligned}$

    \item $\begin{aligned}[t]
                  \because   & \ p = 0, q = 1                                   \\
                  \therefore & \ \neg p = 1, \neg q = 0                         \\
                             & \ \neg (p \land q) \equiv \neg p \lor \neg q = 1 \\
                             & \ \neg (p \lor q) \equiv \neg p \land \neg q = 0
              \end{aligned}$
\end{enumerate}

\subsection*{Exercise 11f}
\begin{enumerate}
    \item Prove the following equivalences using truth tables:
          \begin{enumerate}
              \item (a) $p \vee q \equiv \sim(\sim p \wedge \sim q)$
              \item $(p \wedge q) \wedge r \equiv p \wedge(q \wedge r)$
              \item $p \vee(q \wedge r) \equiv(p \vee q) \wedge(p \vee r)$
              \item $\sim(p \vee \sim q) \equiv \sim p \wedge q$
          \end{enumerate}
    \item  Find the inverse proposition of the following propositions and their truth
          values:
          \begin{enumerate}
              \item The three center lines of a triangle intersect at one point, and the three
                    altitude lines of the triangle also intersects at this point.
              \item The odd number $2n$ is divisible by 2 or 4.
          \end{enumerate}
    \item \begin{enumerate}
              \item Given $p=0, q=1$, find the truth values of $\sim(p \vee q)$ and $\sim(p \wedge
                        q)$.
              \item Given $p=1, q=0$, find the truth values of $\sim(p \vee q)$ and $\sim(p \wedge
                        q)$.
              \item Given $p=1, q=1$, find the truth values of $\sim(p \vee q)$ and $\sim(p \wedge
                        q)$.
          \end{enumerate}
    \item Simplify the following propositions using De Morgan's Laws and double negation
          law:
          \begin{enumerate}
              \item $\sim(\sim p \wedge q)$
              \item $\sim(\sim p \vee \sim q)$
              \item $\sim(\sim p \vee q)$
          \end{enumerate}
\end{enumerate}

\section{Implication}

Let $p$ and $q$ be two propositions, another proposition can be formed using
the form ``if $p$ then $q$'', known as the \textbf{implication} of $p$ and $q$,
denoted by $p \Rightarrow q$, read as ``$p$ implies $q$''. For example,

$p$: $x = 3$

$q$: $x^2 = 9$

$p \Rightarrow q$: If $x = 3$, then $x^2 = 9$.

\noindent In the implication $p \Rightarrow q$, $p$ is called the \textbf{hypothesis} or
\textbf{antecedent}, and $q$ is called the \textbf{conclusion} or
\textbf{consequent}. The truth value of $p \Rightarrow q$ is as follows:
\begin{center}
    \begin{NiceTabular}{|c|c|c|}[code-before = \rowcolor{lightgray}{1}, hvlines]
        $p$ & $q$ & $p \Rightarrow q$ \\
        1   & 1   & 1                 \\
        1   & 0   & 0                 \\
        0   & 1   & 1                 \\
        0   & 0   & 1                 \\
    \end{NiceTabular}
\end{center}

In the field of Mathematics, deriving $p \Rightarrow q = 1$ from $p = 1$ and $q
    = 0$ is the most common way to prove the truthfulness of a proposition.

When $p = 1$, $q = 0$, $p \Rightarrow q = 0$. This case can be easily
understood.

When $p = 0$, $q = 1$, $p \Rightarrow q = 1$. This case can be understood as
follows: Although the hypothesis $p$ is not satisfied, the conclusion $q$ still
stands, hence, $p \Rightarrow q = 1$. Take a look at this example: If it rains
tonight, I will stay at home. Whe $p$ is false but $q$ is true, that means: It
doesn't rain tonight, but I still stay at home. Since I only told that I will
stay at home if it rains tonight, but I didn't mention anything about the case
when it doesn't rain tonight, therefore this sentence is logically correct.
Hence, when $p$ is false and $q$ is true, $p \Rightarrow q$ is true.

When $p = 0$, $q = 0$, $p \Rightarrow q = 1$. This case can be understood using
the following example.
\begin{align*}
    p & : \text{You can jump 100m.} \\
    q & : \text{I can jump 200m.}
\end{align*}
Obviously, the proposition $p$ and $q$ are both false propositions. However, the proposition formed by $p$ and $q$ seems to be a logical joke, i.e.
\[p \Rightarrow q : \text{If you can jump 100m, then I can jump 200m.}\]

The reality is though, since you will never be able to jump 100m, I need not
have to fulfil the matter of jumping 200m. Hence, $p \Rightarrow q$ is true.

\end{document}