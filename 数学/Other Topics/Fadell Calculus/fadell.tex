\documentclass{report}

\usepackage{amsmath,amssymb,amsthm}
\usepackage[total={6.5in,9in},centering]{geometry}
\usepackage{multicol}
\setcounter{secnumdepth}{0}

\begin{document}
\chapter{Limits}

\section{Exercise 1.1}

\textit{In Problems 1 to 4, find $\Delta y$ for $x$ and $\Delta x$ as given for each
      function.}
\begin{enumerate}
      \item $y(x) = 2x - 3$; $x = 2$, $\Delta x = 1$; $x = 5$, $\Delta x = 1$
      \item $y(x) = 3x + 1$; $x = -1$, $\Delta x = 2$; $x = 7$. $\Delta x = 2$
      \item $y(x) = x^2 - x + 1$; $x = -2$, $\Delta x = 1$; $x = 3$, $\Delta x = 1$
      \item $y(x) = x^{2} + 3x - 5$; $x = -5$, $\Delta x = 2$; $x = 6$, $\Delta x = 2$
      \item If $x$ is near $3$, what number is the value of the function $3x^2 - 2$ near?
            Is it true that the values of the function are arbitrarily near this number for
            all values of $s$ that are sufficiently near 3? Is this number the limit of the
            function as $x \to 3$?
      \item What number is the value of the function $x^2 - x - 1$ near if $x$ is near $5$?
            Are the values of the function arbitrarily near this number 14 for all
            replacements for $x$ that are sufficiently near 5? Is this number the limit of
            the function as $x \to 5$?
      \item What number is the value of the function $\dfrac{x^2 + 6}{x - 4}$ near if $x$
            is near 6? Is the value of the function arbitrarily near this number for all
            replacements for $x$ that are sufficiently near 6? Is this number the limit of
            the function as $x \to 6$?
      \item What number is the value of the function $\dfrac{x^3 - 5}{x + 2}$ near if $x$
            is near 2? Is it true that the value of this function is arbitrarily near this
            number if $x$ is sufficiently near 2? Is this number the limit of the function
            as $x \to 2$?
      \item Is there a number $L$ such that the values of the function $\dfrac{x^2 + 1}{x -
                        2}$ are near $L$ for all value of $x$ near 2? What statement can you make about
            the limit of this function as $x \to 2$?
      \item Is there a number $L$ such that the values of the function $\dfrac{x^2 + 7x +
                        10}{x^2 - 9}$ are near $L$ for all values of $x$ near 3? What statement can you
            make about the limit of this function as $x \to 3$?
      \item Find the value of the function $\dfrac{x^2 - 9}{x - 3}$ for $x = 2.9$, $2.99$,
            $3.1$, $3.01$. These results indicate that a number $L$ may be the limit of
            this function as $x \to 3$. WHat is this number? Prove that this number is
            actually the limit.
      \item Find the values of the function $\dfrac{x^2 - 3x - 10}{x - 5}$ for $x = 4.9$,
            $4.99$, $5.1$, $5.01$. These results indicate that what number may be the limit
            of this function as $x \to 5$? Prove that this number is actually the limit.
\end{enumerate}
\textit{In each of Problems 13 to 16, find the limit of the given functions as $h \to 0$.}
\begin{enumerate}
      \setcounter{enumi}{12}
      \begin{multicols}{2}
            \item $\dfrac{(3+h)^{2}-9}{h}$
            \item $\dfrac{(2+h)^{3}-8}{h}$
            \item $\dfrac{\dfrac{1}{2+h}-\dfrac{1}{2}}{h}$
            \item $\dfrac{\sqrt{4+h}-2}{h}$
      \end{multicols}
      \item Evaluate $\lim\limits_{x \to 2}\dfrac{4x}{3x - 5}$. Find the value of this
            function corresponding to $x = 2$/ Is the function continuous for $x = 2$? Is
            there a value of $x$ for which the function is not continuous? Sketch the
            graph.
      \item Evaluate $\lim\limits_{x \to 1}\dfrac{3x}{x^2 + 1}$. Find the value of this
            function corresponding to $x = 1$. Is the function not continuous? Sketch the
            graph.
      \item Evaluate $\lim\limits_{x \to -3}\dfrac{4x + 1}{(2 + x^2)}$. Find the value of
            the function for $x = -4$. Is the function continuous for $x = -3$? Is there
            any value of $x$ for which it is not continuous? Sketch the graph.
\end{enumerate}
\textit{Find each of the following limits if it exists.}
\begin{enumerate}
      \setcounter{enumi}{20}
      \begin{multicols}{2}
            \item $\lim\limits_{x\to2}{\dfrac{x^{3}-8}{x=2}}$
            \item $\lim\limits_{x\to-1}{\dfrac{x^{3}+1}{x+1}}$
      \end{multicols}
      \begin{multicols}{2}
            \item $\lim\limits_{x\to\frac{1}{2}}{\dfrac{2x^{2}+7x-4}{4x-2}}$
            \item $\lim\limits_{x\to-5}{\dfrac{3x+15}{x^{2}-25}}$
      \end{multicols}
      \begin{multicols}{2}
            \item $\lim\limits_{x\to2}{\dfrac{2-x}{x+2}}$
            \item $\lim\limits_{h\to0}{\dfrac{h^{2}+3h}{h - 1}}$
      \end{multicols}
      \begin{multicols}{2}
            \item $\lim\limits_{h\to0}\dfrac{1}{h}[(6+h)^{2}-36]$
            \item $\lim\limits_{x\to-a}{\dfrac{x^{2}-a^{2}}{x+a}}$
      \end{multicols}
      \begin{multicols}{2}
            \item $\lim\limits_{x\to2}{\dfrac{x^{3}-4x}{x^{3}-2x^{2}}}$
            \item $\lim\limits_{h\to0}\dfrac{h^{3}-4h}{h^{3}-2h^{2}}$
      \end{multicols}
      \begin{multicols}{2}
            \item $\lim\limits_{x\to x_{1}}{\dfrac{4x^{2}\ -\ {4x_{1}}^{2}}{x - x_{1}}}$
            \item $\lim\limits_{x \to x_1}\dfrac{x^3 - {x_1}^3}{x - x_1}$
      \end{multicols}
      \begin{multicols}{2}
            \item $\lim\limits_{h\to0}{\dfrac{(x+h)^{2}-x^{2}}{h}}$
            \item $\lim\limits_{h\to0}{\dfrac{(x+h)^{3}-x^{3}}{h}}$
      \end{multicols}
      \begin{multicols}{2}
            \item $\lim\limits_{h\to0}{\dfrac{1}{h}}\left[{\dfrac{1}{\sqrt{x+h}}}-{\dfrac{1}{\sqrt{x}}}\right]$
            \item $\lim\limits_{h\to0}\dfrac{\sqrt{x+h}-\sqrt x}{h}$
      \end{multicols}
\end{enumerate}

\section{Exercise 1.2}
\begin{enumerate}
      \item If $\lim\limits_{x \to \alpha}f(x) = 0$ and $\lim\limits_{x \to \alpha}g(x) = k
                  \neq 0$, what can be said about $\lim\limits_{x \to \alpha}\dfrac{f(x)}{g(x)}$?
      \item If $\lim\limits_{x \to \alpha}f(x) \neq 0$ and $\lim\limits_{x \to \alpha}g(x)
                  = 0$, what can be said about $\lim\limits_{x \to \alpha}\dfrac{f(x)}{g(x)}$?
      \item If $\lim\limits_{x \to \alpha}f(h) = 0$ and $\lim\limits_{x \to \alpha}g(x) =
                  k$, what can be said about $\lim\limits_{x \to \alpha}{f(x)}{g(x)}$?
      \item If $\lim\limits_{x \to \alpha}f(x) = A$ and $\lim\limits_{x \to \alpha}g(x) =
                  A$, what can be said about $\lim\limits_{x \to \alpha}\left[f(x) -
                        g(x)\right]$?
      \item Sketch the graph of the function $4^{-\frac{1}{x}}$. Discuss the continuity of
            the function.
      \item Sketch the graph of the function $\dfrac{1}{1 + 2^{\frac{1}{x}}}$. Discuss the
            continuity of the function.
      \item Sketch the graph of the function $\dfrac{2^{\frac{1}{x}}}{1 +
                        2^{\frac{1}{x}}}$. Discuss the continuity of the function.
      \item If $\lim\limits_{x \to \alpha}f(x) = L$ and $\lim\limits_{x \to \alpha}g(x) = M
                  \neq 0$, prove that $\lim\limits_{x \to \alpha}\dfrac{f(x)}{g(x)} =
                  \dfrac{L}{M}$.
\end{enumerate}
\textit{In Problems 9 to 16 prove ny use of the $\epsilon$, $\delta$ definition that each function is continuous for the given value of $x$.}
\begin{enumerate}
      \setcounter{enumi}{8}
      \begin{multicols}{2}
            \item $f(x) = x^2 + 2$, $x = 3$
            \item $f(x) = 3x = 7$, $x = a$
      \end{multicols}
      \begin{multicols}{2}
            \item $f(x) = x^3$, $x = 2$
            \item $f(x) = x^3 - x$, $x = 4$
      \end{multicols}
      \begin{multicols}{2}
            \item $f(x) = \sqrt{2x - 1}$, $x = 5$
            \item $f(x) = \sqrt{3x + 1}$, $x = 5$
      \end{multicols}
      \begin{multicols}{2}
            \item $f(x) = \dfrac{x + 2}{x + 1}$, $x = a \neq -1$
            \item $f(x) = \dfrac{3x - 1}{2x + 3}$, $x = a \neq -1.5$
      \end{multicols}
\end{enumerate}
\textit{In Problems 17 to 20, assume that $f(x)$ and $g(x)$ are continuous for $x = a$.}
\begin{enumerate}
      \setcounter{enumi}{16}
      \item Prove that the sum of two continuous functions is continuous.
      \item Prove that the product of a constant and a continuous function is continuous.
      \item Prove that the product of two continuous functions is continuous.
      \item Prove that $\dfrac{f(x)}{g(x)}$ is continuous at $x = a$ unless $g(a) = 0$.
\end{enumerate}

\section{Exercise 2.1}
\begin{enumerate}
      \item In Fig. 2.1 assume that $x = 4$ in. so that $A = 16$ sq in. Go through the
            steps of Example 1 using 4 and $x_1$ instead of $x$ and $x_1$, and thus find
            the rate of change of the area relative to $x$ for the instant that $x = 4$ in.
            Compute the average rate of change of the area with respect to $x$ over the
            interval from $x = 4$ to $4.5$ in.
      \item If the edge of a cube is 2 in., its surface area is $6(2)^2$ or $24$ sq in. If
            the edge is increased to $x$ in., the new surface area will be $6x^2$ sq in.

            Find the value of the fraction \[\dfrac{6x^2 - 24}{x - 2}\] if $x = 2.5$ and interpret the result. Find the limit of the fraction as $x \to
                  2$ and interpret this result.
      \item Find the average rate of change of the volume of a cube relative to its edge
            $x$ over the interval from $x = 2.4$ to $2.6$ in. Find the instantaneous rate
            of $x = 2.4$ in.
      \item Find the average rate of change of the surface area of a cube relative to its
            edge $x$ over the interval from $x = 4$ to $4.5$ in. Find the instantaneous
            rate if $x = 4$ in.
      \item Find the rate of change of the area of a circle relative to its radius.
      \item Find the rate of change of the surface area fo a sphere relative to its radius.
      \item Find the rate of change of the volume of a cube relative to its edge.
      \item Find the rate of change of the area of an equilateral triangle relative to a
            side.
      \item A rectangular box has a square base of side $x$ ft and its height is $5x$ ft.
            Find the rate of change of its volume relative to $x$.
      \item A right pyramid has a square base of side $x$ in. and its height is $6x$ in.
            Find the rate of change of its volume relative to $x$.
      \item Find the rate of change of the volume of a cube relative to its surface area.
            Hint: If the edge is $x_1$ then the volume is ${x_1}^3$ and the surface area is
            $6{x_1}^2$. If the edge changes to a new value $x$, the volume becomes $x^3$
            and the surface area becomes $6x^2$. Consequently, \[\dfrac{V - V_1}{S - S_1} = \dfrac{x^3 - {x_1}^3}{6x^2 - 6{x_1}^2}\]
            \vspace{-1.2em}
      \item Find the rate of change of the volume of a sphere relative to the area $\pi
                  r^2$ of a great circle.
      \item Find the rate of chang eof the volume of a right circular cylinder of radius
            $r$ and height $h$ relative to the cross-sectional area.
      \item Find the rate of change of th volume of a sphere relative to its surface area.
      \item The radius $r$ of a right circular cylinder is increasing and its height $h$ is
            a constant. Find the rate at which its total surface area increases relative to
            $r$.
      \item Find the rate of change of the volume of a right circular cone relative to its
            radius $r$ if its height $h$ is a constant.
      \item Show that when $x = 25$, the square root of $x$ is increasing one-tenth as fast
            as $x$.
      \item Find, for any positive value of $x$, the rate of change of the square root of
            $x$ relative to $x$.
      \item Find the average rate of change of the value of the function $x^2 - 4x$
            relative to $x$ over the interval from $x = 7$ to $9$. Find the instantaneous
            rate when $x= 7$.
      \item Find the average rate of change of the value of the function $x^3 + 6$ relative
            to $x$ over the interval from $x = 5$ to $7$. Find the instantaneous rate when
            $x = 6$.
      \item Find the rate of chang eof the value of the function $\dfrac{1}{x}$ relative to
            $x$. What is the significance of the negative sign?
      \item Find the rate of change of the value of the function $\dfrac{x+1}{x-1}$
            relative to $x$. IN particular, show that if $x$ has the value $3$ and is
            increasing, the value of this function is decreasing one-half as fast as $x$ is
            increasing.
      \item Show that for $x = 1$ the value of the function $x^2 + \sqrt{x}$ increases
            $2\dfrac{1}{2}$ times as fast as $x$.
      \item Show that for $x = 4$, the value of the function $\dfrac{1}{\sqrt{x}}$
            decreases one-sixteenth as fast as $x$ increases.
\end{enumerate}

\newpage
\section{Exercise 2.2}

\textit{In each of Problems 1 to 20 find the derivative with respect to $x$ of the given function using form (2) of the definition.}
\begin{enumerate}
      \begin{multicols}{2}
            \item $3x^{2}+2$
            \item $4x^{2}+3x$
      \end{multicols}
      \begin{multicols}{2}
            \item $x^{2}-5x$
            \item $\dfrac{x^{3}}{2}$
      \end{multicols}
      \begin{multicols}{2}
            \item $x^{3}-2x$
            \item $2x^{3}+3x$
      \end{multicols}
      \begin{multicols}{2}
            \item $2x^{2}-7x$
            \item $x^{4}-3x^{2}$
      \end{multicols}
      \begin{multicols}{2}
            \item $\dfrac{3}{x}$
            \item $\dfrac{4}{x+2}$
      \end{multicols}
      \begin{multicols}{2}
            \item $\dfrac{x}{2x+1}$
            \item $\dfrac{x+2}{x+3}$
      \end{multicols}
      \begin{multicols}{2}
            \item $\dfrac{8}{x^{2}+4}$
            \item $\dfrac{1}{{\sqrt{x}}}$
      \end{multicols}
      \begin{multicols}{2}
            \item $\dfrac{3}{x^{2}-2}$
            \item $\dfrac{x}{x^{2}+2}$
      \end{multicols}
      \begin{multicols}{2}
            \item $\dfrac{8x}{x^{2}-16}$
            \item $\dfrac{x^{2}}{x^{2}+4}$
      \end{multicols}
      \begin{multicols}{2}
            \item $\dfrac{x^{2}-1}{x^{2}+1}$
            \item $\dfrac{{x}^{2}+4}{{x}^{2}-4}$
      \end{multicols}
      \begin{multicols}{2}
            \item Given $y = 4t^3 + 3$, find $D_t y$.
            \item Given $w = u^2 - 3u$, find $D_u w$.
      \end{multicols}
      \begin{multicols}{2}
            \item Given $S = 4\sqrt{w}$, find $\dfrac{dS}{dw}$.
            \item Given $A = 4\pi r^2$, find $\dfrac{dA}{dr}$.
      \end{multicols}
      \begin{multicols}{2}
            \item Given $y = \sqrt{x + 2}$, find $D_x y$.
            \item Given $Q = t(2t + 1)$, find $\dfrac{dQ}{dt}$.
      \end{multicols}
      \begin{multicols}{2}
            \item Given $S = \dfrac{t - 1}{t^2 + 4}$, find $D_t S$.
            \item Given $T = \dfrac{u - 3}{u^2 + 2}$, find $D_u T$.
      \end{multicols}
\end{enumerate}
\textit{In each of Problems 29 to 36 sketch the graph of the given equation, and find the slope of the tangent line to the graph at the points indicated. Draw teh corresponding tangent line.}
\begin{enumerate}
      \setcounter{enumi}{28}
      \begin{multicols}{2}
            \item $y = \dfrac{x^2}{2 + x}$; $(-2, 0)$, $(3, 2.5)$
            \item $y = \dfrac{2x}{3 + 1}$; $(-1, \dfrac{1}{3})$, $(3, 3)$
      \end{multicols}
      \begin{multicols}{2}
            \item $y = \dfrac{2}{\sqrt{x}}$; $(1, 2)$, $(4, 4)$
            \item $y = \dfrac{x^3}{2}$; $\left(1, \dfrac{1}{2}\right)$, $(2, 4)$
      \end{multicols}
      \begin{multicols}{2}
            \item $y = \dfrac{4}{x}$; $(2, 2)$, $\left(8, \dfrac{1}{2}\right)$
            \item $y = \dfrac{x+2}{x-1}$; $(0, -2)$, $(-2, 0)$
      \end{multicols}
      \begin{multicols}{2}
            \item $y = \dfrac{3x - 6}{x+2}$; $(2, 0)$, $(-5, 7)$
            \item $y = \dfrac{4x}{x+4}$; $(-2, -4)$, $(0, 0)$
      \end{multicols}
\end{enumerate}

\newpage
\section{Exercise 2.3}
\begin{enumerate}
      \item Given an opinion as the truth of each of the following assertions:
            \begin{enumerate}
                  \item If $f(x)$ is a constant for $a \leq x \leq b$, then the average rate of change
                        of $f(x)$ is over this interval is zero.
                  \item If the average rate of change of $f(x)$ over the interval $a \leq x \leq b$ is
                        zero, then $f(x)$ is a constant over this interval.
            \end{enumerate}
      \item Show that the average rate of change of $f(x)$ over the interval $a \leq x \leq
                  b$ is zero, then $f(x)$ is constant over this interval.
      \item Show that the function $\phi(x) = \sqrt{x}$ does not have a right-hand
            derivative at $x = 0$. \textit{Hint:} Consider the fraction \[\dfrac{\phi(x) - \phi(0)}{x - 0} = \dfrac{\sqrt{x} - 0}{x - 0} \quad x > 0\]
            Show that it does not have a limit as $x \to 0^{+}$.
      \item Show that the function $f(x) = \sqrt{1 - x^2}$ does not have a left-hand
            derivative at the point for which $x = 1$. \textit{Hint:} Consider the fraction \[\dfrac{f(x) - f(1)}{x - 1} = \dfrac{\sqrt{1 - x^2} - 0}{x - 1} \quad 0 < x < 1\]
            Show that it does not have a limit as $x \to 1^{-}$.
\end{enumerate}
\textit{In each of Problem 5 to 10 determine the $x$ interval or intervals over which the value of $y$ is increasing.}
\begin{enumerate}
      \setcounter{enumi}{4}
      \begin{multicols}{2}
            \item $y = x^2 - 8x$
            \item $y = 4x - x^2$
      \end{multicols}
      \begin{multicols}{2}
            \item $y = \dfrac{2}{x}$
            \item $y = -\dfrac{5}{x+1}$
      \end{multicols}
      \begin{multicols}{2}
            \item $y = x^3 - 6x^2$
            \item $y = 3x - x^3$
      \end{multicols}
      \item Show that the function $\dfrac{x+1}{x-1}$ is a decreasing function over any
            interval that does not include the point $x = 1$.
      \item Let $y = \dfrac{1}{x^2 + 1}$. Show that $y$ decreases as $x$ increases if $x >
                  0$ and that $y$ increases as $x$ increases if $x < 0$.
      \item If $y = \dfrac{4x}{x^2 + 4}$, show that $y$ increases as $x$ increases over the
            interval $-2 < x < 2$. Draw the corresponding graph. What slope does the curve
            have at the origin? At what points is the slope equal to zero?
      \item Sketch the graph of the equation $y = \dfrac{8}{x^2 - 4}$. Use $D_x y$ to show
            that $y$ increases as $x$ increases if $x < -2$ and if $-2 < x < 0$. What is
            the slope of the tangent line to the graph at the point $\left(4,
                  \dfrac{2}{3}\right)$?
      \item If $y = x^3$, find the value of $D_x y$ for $x = 2$ by evaluating \[\lim\limits_{x \to 2}\dfrac{x^3 - 2^3}{x-2} \qquad \text{or} \qquad \lim\limits_{\Delta x \to 0}\dfrac{(2 + \Delta x)^3 - 2^3}{\Delta x}\]
      \item If $y = \dfrac{4}{x}$, find the value of $D_x y$ for $x = 2$ by evaluating \[\lim\limits_{x \to 2}\dfrac{\dfrac{4}{x} - \dfrac{4}{2}}{x-2} \qquad \text{or} \qquad \lim\limits_{\Delta x \to 0}\dfrac{\dfrac{4}{2 + \Delta x} - \dfrac{4}{2}}{\Delta x}\]
      \item If $y = \sqrt{x + 2}$, find $D_x y$ in terms of $x$ by evaluating \[\lim\limits_{x \to x_1}\dfrac{\sqrt{x + 2} - \sqrt{x_1 + 2}}{x - x_1}\]
            Check by writing down the corresponding fraction involving $\Delta x$ and
            finding its limit as $\Delta x \to 0$.
      \item A right pyramid has a rectangular base that is $x$ ft wide and $2x$ ft long.
            The height of the pyramid is $3x$ ft. Express its volume $V$ as a function of
            $x$. Find the value of $D_x V$ for $x = 2\dfrac{1}{2}$ ft. In what units is the
            rate expressed?
      \item The number $N$ of grams of a substance in solution varies with the time $t$ in
            minutes that the substance has been in contact with the solvent in accordance
            with the formula $N = \dfrac{16}{t+5}$. Find the value of $D_t N$ if $t = 3$.
            In what units is this rate expressed?
      \item If an object si dropped from the top of a cliff and falls under the action of
            gravity alone, its distance $S$ in feet below the top of the cliff at the end
            of $t$ sec is given approximately by the formula $S = 16t^2$. Find the value of
            $D_t S$ when $t = 4$, and explain the meaning of the result. In what units is
            this rate expressed?
      \item A cylinder contains 200 cu in. of air at a pressure of $15$ lb per sq in. If
            the air is now compressed by moving a piston in the cylinder and if this is
            done under a condition of constant temperature, the pressure will increase as
            the volume decreases, the relation between them being $pv = 3,000$. Find the
            value of $D_v p$ for $v = 50$ cu in. In what units is this rate expressed?
      \item  A certain quantity $Q$ varies with the time $t$ in accordance with the formula
            $Q = 8t^2 - t^3$, $0 \leq t \leq 8$. Over what part of this time interval is
            $Q$ increasing?
      \item If $r$ in. is the radius of a sphere, then the number of cubic inches in its
            volume and the number of square inches in its surface area are given,
            respectively, by the functions $\dfrac{4\pi r^3}{3}$ and $4\pi r^2$. Evaluate \[\lim\limits_{r\rightarrow r_{1}}\frac{4\pi \dfrac{r^{3}}{3}-\dfrac{4\pi {r_{1}}^{3}}{3}}{4\pi r^{2}-4\pi{r_{1}}^{2}}\]
            What physical meaning can be attached to the result?
\end{enumerate}

\newpage

\end{document}