\documentclass{report}

\usepackage{amsmath}
\usepackage{amssymb}
\usepackage{setspace}
\usepackage{enumitem}
\usepackage{fontspec}
\usepackage[total={6.6in,9.2in}]{geometry}

\setmainfont{Times New Roman}

\title{\Huge{\textbf{Mathematical Induction}}}
\author{Melvin Chia}
\date{10 June 2023}

\begin{document}
\maketitle

\onehalfspacing

\chapter{Mathematical Induction}

\section{Mathematical Induction}

We inspect the following example:
\begin{flalign*}
    1^3 & = 1 = 1^2                                             \\
    1^3 & + 2^3                   = 9 = (1 + 2)^2               \\
    1^3 & + 2^3 + 3^3             = 36 = (1 + 2 + 3)^2          \\
    1^3 & + 2^3 + 3^3 + 4^3       = 100 = (1 + 2 + 3 + 4)^2     \\
    1^3 & + 2^3 + 3^3 + 4^3 + 5^3 = 225 = (1 + 2 + 3 + 4 + 5)^2 \\
        & \ \vdots
\end{flalign*}
From the above example, we can conclude that
\begin{flalign*}
    1^3 + 2^3 + 3^3 + \cdots + n^3 & = (1 + 2 + 3 + \cdots + n)^2         \\
                                   & = \left[\dfrac{n(n + 1)}{2}\right]^2
\end{flalign*}
Reasoning in the way of obtaining a general formula from a few examples is called \textbf{induction}. We can use induction to help us derive a general formula from a few examples. However, the general formula obtained from only a few examples may not be correct. For example:
\begin{flalign*}
    a^n = (n^2 - 5n + 5)^2
\end{flalign*}
can easily be proven
\begin{flalign*}
    a_1 & = (1^2 - 5 \times 1 + 5)^2 = 1 \\
    a_2 & = (2^2 - 5 \times 2 + 5)^2 = 1 \\
    a_3 & = (3^2 - 5 \times 3 + 5)^2 = 1 \\
    a_4 & = (4^2 - 5 \times 4 + 5)^2 = 1
\end{flalign*}
If we make a conclusion based on the above examples: for all natural number $n$,
\begin{flalign*}
    a^n = (n^2 - 5n + 5)^2 = 1
\end{flalign*}
is true, then we are wrong. In fact,
\begin{flalign*}
    a_5 & = (5^2 - 5 \times 5 + 5)^2 = 25 \neq 1
\end{flalign*}

That is to say, the general formula of propositions related to natural numbers
obtained from induction is not necessarily true. In order to prove its truth,
we usually adopt the following method.

First, we prove that the proposition is true for the first value $n_1$ (for
example $n_1 = 1$). Next, we assume that the proposition is true for $n = k$
($k \in \mathbb{N}, k \geq n_1$), and then prove that the proposition is true
for $n = k + 1$. In this way, we can prove that the proposition is true for all
natural numbers $n$ after $n_1$. This method is called \textbf{mathematical
    induction}.

For example, we use mathematical induction to prove that the following equation
is true for all natural numbers:
\begin{flalign*}
    1^3 + 2^3 + 3^3 + \cdots + n^3 = \left[\dfrac{n(n + 1)}{2}\right]^2
\end{flalign*}
\begin{enumerate}[label = (\arabic*)]
    \item When $n = 1$, LHS $= 1^3 = 1$, RHS $ = \left[\dfrac{1(1 + 1)}{2}\right]^2 = 1$,
          so the equation is true for $n = 1$.
    \item Assume that the equation is true for $n = k$, that is,
          \begin{flalign*}
              1^3 + 2^3 + 3^3 + \cdots + k^3 = \left[\dfrac{k(k + 1)}{2}\right]^2
          \end{flalign*}
          Hence, when $n = k + 1$,
          \begin{flalign*}
              1^3 + 2^3 + 3^3 + \cdots + k^3 + (k + 1)^3 & = \left[\dfrac{k(k + 1)}{2}\right]^2 + (k + 1)^3 \\
                                                         & = \frac{k^2(k + 1)^2}{4} + \dfrac{4(k + 1)^3}{4} \\
                                                         & = \frac{(k + 1)^2[k^2 + 4(k + 1)]}{4}            \\
                                                         & = \frac{(k + 1)^2(k+2)^2}{4}                     \\
                                                         & = \left[\dfrac{(k + 1)(k + 2)}{2}\right]^2
          \end{flalign*}
\end{enumerate}
Therefore, the equation is true when $n = k + 1$.

From (1), when $n = 1$, the equation is true. From (2), when $n = 1 + 1 = 2$,
the equation is also true. Since the equation is true when $n = 2$, from (2),
when $n = 2 + 1 = 3$, the equation is also true. Recursively, we can prove that
the equation is true for $n = 4, 5, 6, \cdots$. Therefore, from (1) and (2), we
can conclude that the equation is true for all $n \in \mathbb{N}$.

From the example above, we can see that the process of mathematical induction
is as follows:
\begin{enumerate}[label = (\arabic*)]
    \item Prove that the proposition is true for the first value $n_1$ (for example $n_1
              = 1$ or $2$).
    \item Assume that the proposition is true for $n = k$ ($k \in \mathbb{N}, k \geq
              n_1$), and then prove that the proposition is true for $n = k + 1$.
\end{enumerate}

After the above two steps, we can conclude that the proposition is true for all
natural numbers $n$ after $n_1$.

It is worth noting that the two steps above are indispensable. From the
calculation of the value of each term of
\begin{flalign*}
    a_n = (n^2 - 5n + 5)^2
\end{flalign*}
earlier, we can see that completing step (1) but not completing step (2) will result in a wrong conclusion. It's because we can't prove recursively that the proposition is true for $n = 2, 3, 4, 5, \cdots$. Similarly, if we complete step (2) but not step (1), we will also get a wrong conclusion.

For example, assume that when $n = k$, the equation $2 + 6 + 10 + \cdots + 2(2n
    - 1) = 2n^2 + 2$ is true, that is,
\begin{flalign*}
    2 + 6 + 10 + \cdots + 2(2k - 1) = 2k^2 + 2
\end{flalign*}
So, when $n = k + 1$,
\begin{flalign*}
    2 + 6 + 10 + \cdots + 2(2k - 1) + 2[2(k + 1) - 1] & = 2k^2 + 2 + 4(k + 1) - 2 \\
                                                      & = 2k^2 + 2 + 4k + 2       \\
                                                      & = 2(k + 1)^2 + 2
\end{flalign*}
That is, if the equation is true for $n = k$, then it is also true for $n = k +
    1$. However, if we make a conclusion that the equation is true for all $n \in \mathbb{N}$ based on this, we will be wrong. In fact,

When $n = 1$, LHS $= 2$, RHS $= 2 \times 1^2 + 2 = 4$,
\begin{flalign*}
    \text{LHS} \neq \text{RHS}.
\end{flalign*}
This indicates that step (2) is meaningless if step (1) is not completed.
\vspace{0.8em}
\begin{enumerate}[label = \textbf{Example \arabic*}, leftmargin=*]
    \item Prove that $1 + 3 + 5 + \cdots + (2n - 1) = n^2$ using mathematical induction.
\end{enumerate}
\begin{enumerate}[label = \textbf{Solution}, leftmargin=*]
    \item \begin{enumerate}[label = (\arabic*)]
              \item When $n = 1$, LHS $= 1$, RHS $= 1^2 = 1$, so the equation is true for $n = 1$.
              \item Assume that the equation is true for $n = k$, that is,
                    \begin{flalign*}
                        1 + 3 + 5 + \cdots + (2k - 1) = k^2
                    \end{flalign*}
                    Hence, when $n = k + 1$,
                    \begin{flalign*}
                         & 1 + 3 + 5 + \cdots + (2k - 1) + [2(k + 1) - 1] \\
                         & = k^2 + [2(k + 1) - 1]                         \\
                         & = k^2 + 2k + 1                                 \\
                         & = (k + 1)^2
                    \end{flalign*}
                    Therefore, the equation is true when $n = k + 1$.

          \end{enumerate}

          From (1) and (2), we can conclude that the equation is true for all $n \in
              \mathbb{N}$.
\end{enumerate}
\vspace{0.8em}
\begin{enumerate}[label=\textbf{Example \arabic*}, leftmargin=*]
    \setcounter{enumi}{2}
    \item Prove that $1^2 + 2^2 + 3^2 + \cdots + n^2 = \dfrac{n(n + 1)(2n + 1)}{6}$ using
          mathematical induction.
\end{enumerate}
\begin{enumerate}[label = \textbf{Solution}, leftmargin=*]
    \item \begin{enumerate}[label = (\arabic*)]
              \item When $n = 1$, LHS $= 1^2 = 1$, RHS $= \dfrac{1(1 + 1)(2 \times 1 + 1)}{6} = 1$,
                    so the equation is true for $n = 1$.
              \item Assume that the equation is true for $n = k$, that is,
                    \begin{flalign*}
                        1^2 + 2^2 + 3^2 + \cdots + k^2 = \dfrac{k(k + 1)(2k + 1)}{6}
                    \end{flalign*}
                    Hence, when $n = k + 1$,
                    \begin{flalign*}
                         & 1^2 + 2^2 + 3^2 + \cdots + k^2 + (k + 1)^2      \\
                         & = \dfrac{k(k + 1)(2k + 1)}{6} + (k + 1)^2       \\
                         & = \dfrac{k(k + 1)(2k + 1) + 6(k + 1)^2}{6}      \\
                         & = \dfrac{(k + 1)(2k^2 + 7k + 6)}{6}             \\
                         & = \dfrac{(k + 1)(k + 2)(2k + 3)}{6}             \\
                         & = \dfrac{(k + 1)[(k + 1) + 1][2(k + 1) + 1]}{6}
                    \end{flalign*}
                    Therefore, the equation is true when $n = k + 1$.
          \end{enumerate}

          From (1) and (2), we can conclude that the equation is true for all $n \in
              \mathbb{N}$.
\end{enumerate}

\subsection*{Exercise 1a}
Use mathematical induction to prove the following statements (1 - 7).
\begin{enumerate}
    \item $1 + 2 + 3 + \cdots + n = \dfrac{n(n + 1)}{2}$
    \item $1 \cdot 2 + 2 \cdot 3 + 3 \cdot 4 + \cdots + n(n + 1) = \dfrac{1}{3}n(n + 1)(n + 2)$
    \item $1^2 + 3^2 + 5^2 + \cdots + (2n - 1)^2 = \dfrac{n(4n^2 - 1)}{3}$
    \item $1 \cdot 4 + 2 \cdot 7 + 3 \cdot 10 + \cdots + n(3n + 1) = n(n + 1)^2$
    \item $2 \cdot 2 + 3 \cdot 2^2 + 4 \cdot 2^3 + \cdots + (n + 1) \cdot 2^n = n \cdot 2^{n + 1}$
    \item $\dfrac{1}{1 \cdot 3} + \dfrac{1}{3 \cdot 5} + \dfrac{1}{5 \cdot 7} + \cdots + \dfrac{1}{(2n - 1)(2n + 1)} = \dfrac{n}{2n + 1}$
\end{enumerate}

\newpage
\section{Application of Mathematical Induction}

Mathematical induction can be used to prove proposition that contains any
natural number. In the previous section, by the introduction of mathematical
induction, we have shown the application of it in proving equations through
\textbf{Example 1} and \textbf{Example 2}. \vspace{0.8em}
\begin{enumerate}[label = \textbf{Example \arabic*}, leftmargin=*]
    \setcounter{enumi}{2}
    \item By using mathematical induction, prove that
          \begin{flalign*}
              \left(1 - \dfrac{1}{4}\right)\left(1 - \dfrac{1}{9}\right)\left(1 - \dfrac{1}{16}\right) \cdots \left(1 - \dfrac{1}{n^2}\right) = \dfrac{n + 1}{2n} \quad (n > 1,\ n \in \mathbb{N})
          \end{flalign*}
\end{enumerate}
\begin{enumerate}[label = \textbf{Solution}, leftmargin=*]
    \item \begin{enumerate}[label = (\arabic*)]
              \item \makebox[2cm]{When $n = 2$, }LHS $= \left(1 - \dfrac{1}{4}\right) = \dfrac{3}{4}$

                    \makebox[2cm]{}RHS $= \dfrac{2 + 1}{2 \times 2} = \dfrac{3}{4}$

                    \makebox[2cm]{}so the equation is true for $n = 2$.

              \item Assume that the equation is true for $n = k$, that is,
                    \begin{flalign*}
                        \left(1 - \dfrac{1}{4}\right)\left(1 - \dfrac{1}{9}\right)\left(1 - \dfrac{1}{16}\right) \cdots \left(1 - \dfrac{1}{k^2}\right) = \dfrac{k + 1}{2k}
                    \end{flalign*}
                    Hence, when $n = k + 1$,
                    \begin{flalign*}
                         & \left(1-{\frac{1}{4}}\right)\left(1-{\frac{1}{9}}\right)\left(1-{\frac{1}{16}}\right)\cdots\left(1-{\frac{1}{k^{2}}}\right)\left[1-{\frac{1}{(k+1)^{\frac{1}{2}}}}\right] \\
                         & = {\frac{k+1}{2k}}{\left[1-{\frac{1}{(k+1)^{2}}}\right]}                                                                                                                  \\
                         & = {\frac{k+1}{2k}} \cdot{\frac{k^{2}+2k}{(k+1)^{2}}}                                                                                                                      \\
                         & = \frac{k + 2}{2(k + 1)}                                                                                                                                                  \\
                         & ={\frac{(k+1)+1}{2(k+1)}}
                    \end{flalign*}
                    Therefore, the equation is true when $n = k + 1$.
          \end{enumerate}
          From (1) and (2), we can conclude that the equation is true for all $n \in
              \mathbb{N}$.
\end{enumerate}
\newpage
\begin{enumerate}[label = \textbf{Example \arabic*}, leftmargin=*]
    \setcounter{enumi}{3}
    \item Prove that $x^{2n} - y^{2n}$ ($n \in \mathbb{N}$) is divisible by $x + y$
\end{enumerate}
\begin{enumerate}[label = \textbf{Solution}, leftmargin=*]
    \item \begin{enumerate}[label = (\arabic*)]
              \item When $n = 1$, $x^2 - y^2 = (x + y)(x - y)$ can be divided by $x + y$, so the
                    statement is true for $n = 1$.
              \item Assume that the statement is true for $n = k$, that is, $x^{2k} - y^{2k}$ is
                    divisible by $x + y$.

                    Hence, when $n = k + 1$,
                    \begin{flalign*}
                        x^{2(k+1)} - y^{2(k+1)} & = x^2 \cdot x^{2k} - y^2 \cdot y^{2k}                                       \\
                                                & = x^2 \cdot x^{2k} - x^2 \cdot y^{2k} + x^2 \cdot y^{2k} - y^2 \cdot y^{2k} \\
                                                & = x^2(x^{2k} - y^{2k}) + y^{2k}(x^2 - y^2)
                    \end{flalign*}
                    Since $x^{2k} - y^{2k}$ and $x^2 - y^2$ are both divisible by $x + y$,

                    The above sum $x^2(x^{2k} - y^{2k}) + y^{2k}(x^2 - y^2)$ is also divisible by
                    $x + y$.

                    Therefore, when $n = k + 1$, $x^{2(k+1)} - y^{2(k+1)}$ is divisible by $x + y$.
          \end{enumerate}

          From (1) and (2), we can conclude that the statement is true for all $n \in
              \mathbb{N}$.
\end{enumerate}
\vspace{0.8em}
\begin{enumerate}[label = \textbf{Example \arabic*}, leftmargin=*]
    \setcounter{enumi}{4}
    \item $p > -1 \implies \forall\ n \in \mathbb{N}, (1 + p)^n \geq 1 + np$
\end{enumerate}
\begin{enumerate}[label = \textbf{Solution}, leftmargin=*]
    \item \begin{enumerate}[label = (\arabic*)]
              \item \makebox[2cm]{When $n = 1$, }LHS $= (1 + p)^1 = 1 + p$

                    \makebox[2cm]{}RHS $= 1 + 1 \times p = 1 + p$

                    \makebox[2cm]{}so the inequality is true for $n = 1$.

              \item Assume that the inequality is true for $n = k$, that is,
                    \begin{flalign*}
                        (1 + p)^k \geq 1 + kp
                    \end{flalign*}
                    Hence, when $n = k + 1$,
                    \begin{flalign*}
                        (1 + p)^{k+1} & = (1 + p)^k (1 + p)   \\
                                      & \geq (1 + kp)(1 + p)  \\
                                      & = 1 + p + kp + kp^2   \\
                                      & = 1 + (k + 1)p + kp^2 \\
                                      & \geq 1 + (k + 1)p
                    \end{flalign*}
                    Therefore, when $n = k + 1$, the inequality is true.
          \end{enumerate}

          From (1) and (2), we can conclude that the inequality is true for all $n \in
              \mathbb{N}$.
\end{enumerate}
\newpage
\begin{enumerate}[label = \textbf{Example \arabic*}, leftmargin=*]
    \setcounter{enumi}{5}
    \item A plane has $n$ lines such that no two of them are parallel and no three of
          them are concurrent. Prove that the amount of point of intersection is $V_n =
              \dfrac{1}{2}n(n - 1)$, $n \geq 2$.
\end{enumerate}
\begin{enumerate}[label = \textbf{Solution}, leftmargin=*]
    \item \begin{enumerate}[label = (\arabic*)]
              \item When $n = 2$, there is only one point of intersection, that is, $V_2 = 1$.

                    Also, when $n = 2$, $\dfrac{1}{2} \times 2 \times (2-1) = 1$.

                    Therefore, the statement is true for $n = 2$.

              \item Assume that the statement is true for $n = k (k \geq 2)$, that is, the amount
                    of point of intersection that satisfies the given $k$ lines on the plane is
                    $V_k = \dfrac{1}{2}k(k - 1)$.

                    Now consider the case when adding one more line to the plane, that is, the case
                    when there are $k + 1$ lines. The new line will intersect with the other $k$
                    lines, that is, there will be $k$ new points of intersection. Therefore, the
                    amount of point of intersection on the plane is
                    \begin{flalign*}
                        V_{k+1} = V_k + k & = \dfrac{1}{2}k(k-1) + k              \\
                                          & = \dfrac{1}{2}k\left[(k-1) + 2\right] \\
                                          & = \dfrac{1}{2}(k+1)[(k+1)-1]
                    \end{flalign*}
                    Therefore, when $n = k + 1$, the statement is true.
          \end{enumerate}

          From (1) and (2), we can conclude that the statement is true for all $n \geq 2$
          and $n \in \mathbb{N}$.
\end{enumerate}
\vspace{0.8em}
\begin{enumerate}[label = \textbf{Example \arabic*}, leftmargin=*]
    \setcounter{enumi}{6}
    \item Prove that $3^{4n + 2} + 5^{2n + 1}$ is divisible by $14$ for all $n \geq 0$,
          $n \in \mathbb{Z}$.
\end{enumerate}
\begin{enumerate}[label = \textbf{Solution}, leftmargin=*]
    \item \begin{enumerate}[label = (\arabic*)]
              \item When $n = 0$, $3^2 + 5^1 = 14$ can be divided by $14$,

                    that is, the statement is true for $n = 0$.

              \item Assume that the statement is true for $n = k$, that is, $3^{4k + 2} + 5^{2k +
                                1}$ is divisible by $14$.

                    Hence, when $n = k + 1$,
                    \begin{flalign*}
                        3^{4(k+1)+2} + 5^{2(k+1)+1} & = 3^{4k+2+4} + 5^{2k+1+2}                     \\
                                                    & = 81 \cdot 3^{4k+2} + 25 \cdot 5^{2k+1}       \\
                                                    & = 81(3^{4k+2} + 5^{2k+1}) - 56 \cdot 5^{2k+1}
                    \end{flalign*}
                    Since $3^{4k+2} + 5^{2k+1}$ and $56 \cdot 5^{2k+1}$ are both divisible by $14$,

                    $3^{4(k+1)+2} + 5^{2(k+1)+1}$ is divisible by $14$,

                    that is to say, when $n = k + 1$, the statement is true.
          \end{enumerate}
          From (1) and (2), we can conclude that the statement is true for all $n \geq 0$, $n \in
              \mathbb{Z}$.
\end{enumerate}
\newpage
\begin{enumerate}[label = \textbf{Example \arabic*}, leftmargin=*]
    \setcounter{enumi}{7}
    \item Prove that $\sum 2^{n-1} = 2^n - 1$ for all $n \in \mathbb{N}$ using the method
          of mathematical induction.
\end{enumerate}
\begin{enumerate}[label = \textbf{Solution}, leftmargin=*]
    \item \begin{enumerate}[label = (\arabic*)]
              \item \makebox[2cm]{When $n = 1$,} LHS $= 2^{1-1} = 1$

                    \makebox[2cm]{} RHS $= 2^1 - 1 = 1$

                    \makebox[2cm]{} $\therefore\ $ the formula is true for $n = 1$.

              \item Assume that the formula is true for $n = k$, that is, $2_0 + 2_1 + \cdots +
                        2_{k-1} = 2^k - 1$.
                    \begin{flalign*}
                        \text{Hence, when $n = k + 1$, } 2^0 + 2^1 + \cdots + 2^{k-1} + 2^k & = 2^k - 1 + 2^k   & \\
                                                                                            & = 2 \cdot 2^k - 1   \\
                                                                                            & = 2^{k+1} - 1
                    \end{flalign*}
                    Therefore, when $n = k + 1$, the formula is true.
          \end{enumerate}
          From (1) and (2), we can conclude that the formula is true for all $n \in \mathbb{N}$.
\end{enumerate}

\section*{Exercise 1b}
Prove the following statements using the method of mathematical induction:
\begin{enumerate}
    \item $-1 + 3 - 5 + \cdots + (-1)^n(2n - 1) = (-1)^n \cdot n$
    \item $\sum (5n - 1) = \dfrac{n(5n + 3)}{2}$, $n \in \mathbb{N}$
    \item $\sum 3^{n-1} = \dfrac{3^n - 1}{2}$, $n \in \mathbb{N}$
    \item $2^n > n^2$, $n > 4$ and $n \in \mathbb{N}$
    \item $2^n + 2 > n^2$, $n \in \mathbb{N}$
    \item The sum of the interior angles of a polygon with $n$ sides is $(n-2)\pi$, $n
              \geq 3$.
    \item $(a^n - b^n)$ is divisible by $(a - b)$.
    \item $x^{n+2} + (x+1)^{2n + 1}$ is divisible by $x^2 + x + 1$, $n \geq 0$ and $n \in
              \mathbb{Z}$.
    \item $x^n + 5n$ ($n \in \mathbb{N}$) is divisible by $6$.
    \item The sum of the cube of three consecutive integers is divisible by $9$.
    \item For all natural number $n$, $9^n - 8n - 1$ is a multiple of $64$, $n \geq 2$.
    \item Determine the general formula for the following, and prove it using the method
          of mathematical induction.
          \begin{flalign*}
              1\ \ \ \ \ \ \ \ \ \ \ \ \ \ \  & = 1   \\
              3 + 5\ \ \ \ \ \ \ \ \ \ \ \    & = 8   \\
              7 + 9 + 11\ \ \ \ \ \ \ \       & = 27  \\
              13 + 15 + 17 + 19\ \ \ \        & = 64  \\
              21 + 23 + 25 + 27 + 29          & = 125 \\
          \end{flalign*}
\end{enumerate}
\end{document}