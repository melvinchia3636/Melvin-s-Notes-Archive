\documentclass{report}

\usepackage[total={7in,9in}]{geometry}
\usepackage{amsmath}
\usepackage{amssymb}
\usepackage{multicol}
\usepackage{enumitem}
\usepackage[light,math]{iwona}
\usepackage[T1]{fontenc}

\newcommand{\sol}{\vspace{1em}\\\textbf{Solution.}\vspace{0.5em}}
\newcommand{\qed}{\ \\\strut\hfill$\blacksquare$\vspace{1em}}
\allowdisplaybreaks

\begin{document}
\begin{center}
      \Large{How to Prove It: A Structured Approach, Second Edition}\\
      \vspace{1em}
      \Large{\textbf{Exercises for Section 1.1}}\vspace{1em}
\end{center}
\begin{enumerate}[leftmargin=*]
      \item  Analyze the logical forms of the following statements:
            \begin{enumerate}
                  \item We'll have either a reading assignment or homework problems, but we won't have
                        both homework problems and a test. \sol{}

                        Let $P$ be the statement ``We'll have a reading assignment'', $Q$ be the
                        statement ``We'll have homework problems'', then the statement can be written
                        as $P \vee Q \wedge \neg(P \wedge Q)$. \qed

                  \item You won't go skiing, or you will and there won't be any snow. \sol{}

                        Let $P$ be the statement ``You will go skiing'', $Q$ be the statement ``There
                        will be snow'', then the statement can be written as $\neg P \vee (P \wedge
                              \neg Q)$. \qed

                  \item $\sqrt{7} \nleq 2$.
                        \sol{}

                        The statement can be written as $\neg\big[(\sqrt{7} > 2) \vee (\sqrt{7} =
                                    2)\big]$. \qed
            \end{enumerate}
      \item Analyze the logical forms of the following statements:
            \begin{enumerate}
                  \item  Either John and Bill are both telling the truth, or neither of them is. \sol{}

                        Let $P$ be the statement ``John is telling the truth'', $Q$ be the statement
                        ``Bill is telling the truth'', then the statement can be written as $(P \wedge
                              Q) \vee \neg(P \vee Q)$. \qed

                  \item I'll have either fish or chicken, but I won't have both fish and mashed
                        potatoes. \sol{}

                        Let $P$ be the statement ``I'll have fish'', $Q$ be the statement ``I'll have
                        chicken'', $R$ be the statement ``I'll have mashed potatoes'', then the
                        statement can be written as $(P \vee Q) \wedge \neg(P \wedge R)$. \qed

                  \item 3 is a common divisor of 6,9, and 15.
                        \sol{}

                        Let $P(n)$ be the statement ``3 is a divisor of $n$'', then the statement can
                        be written as $P(6) \wedge P(9) \wedge P(15)$. \qed
            \end{enumerate}

      \item  Analyze the logical forms of the following statements:

            Let $P$ be the statement ``Alice are in the room'', $Q$ be the statement ``Bob
            are in the room'', then
            \begin{enumerate}
                  \item Alice and Bob are not both in the room. \sol{}

                        The statement can be written as $\neg(P \wedge Q)$. \qed

                  \item Alice and Bob are both not in the room. \sol{}

                        The statement can be written as $\neg P \wedge \neg Q$, or $\neg(P \vee Q)$.
                        \qed

                  \item Either Alice or Bob is not in the room. \sol{}

                        The statement can be written as $\neg P \vee \neg Q$, or $\neg(P \wedge Q)$.
                        \qed

                  \item Neither Alice nor Bob is in the room. \sol{}

                        The statement can be written as $\neg P \wedge \neg Q$, or $\neg(P \vee Q)$.
                        \qed
            \end{enumerate}

      \item  Which of the following expressions are well-formed formulas?
            \begin{enumerate}
                  \setlength{\columnsep}{1cm}
                  \begin{multicols}{2}
                        \item $\neg(\neg P \vee \neg \neg R)$. \sol{}

                        Yes.\\

                        \item $\neg(P, Q, \wedge R)$. \sol{}

                        No, since the comma is not a logical connective.\\

                        \vfill\null
                        \columnbreak
                        \item $P \wedge \neg P$. \sol{}

                        Yes.\\

                        \item $(P \wedge Q)(P \vee R)$. \sol{}

                        No, since there are no logical connective between each pair of statements.\\
                  \end{multicols}
            \end{enumerate}\vspace{-2em}‎\qed

      \item Let $P$ stand for the statement "I will buy the pants" and $S$ for the
            statement "I will buy the shirt." What English sentences are represented by the
            following expressions?
            \begin{enumerate}
                  \item $\neg(P \wedge \neg S)$.
                        \sol{}

                        I will not buy the pants without buying the shirt. \qed

                  \item $\neg P \wedge \neg S$.
                        \sol{}

                        I will not buy the pants and I will not buy the shirt. \qed

                  \item $\neg P \vee \neg S$.
                        \sol{}

                        Either I will not buy the pants or I will not buy the shirt. \qed
            \end{enumerate}

      \item Let $S$ stand for the statement "Steve is happy" and $G$ for "George is happy."
            What English sentences are represented by the following expressions?
            \begin{enumerate}
                  \item $(S \vee G) \wedge(\neg S \vee \neg G)$. \sol{}
                        \begin{align*}
                              (S \vee G) \wedge(\neg S \vee \neg G) & \equiv (S \wedge \neg S) \vee (S \wedge \neg G) \vee (G \wedge \neg S) \vee (G \wedge \neg G) \\
                                                                    & \equiv \bot \vee (S \wedge \neg G) \vee (G \wedge \neg S) \vee \bot                           \\
                                                                    & \equiv (S \wedge \neg G) \vee (G \wedge \neg S)                                               \\
                                                                    & \equiv \text{Either Steve is happy or George is happy, but not both.}
                        \end{align*} \qed

                  \item $[S \vee(G \wedge \neg S)] \vee \neg G$. \sol{}
                        \begin{align*}
                              [S \vee(G \wedge \neg S)] \vee \neg G & \equiv (S \vee \neg G) \vee [(G \wedge \neg S) \vee \neg G] \\
                                                                    & \equiv (S \vee \neg G) \vee [\neg S \vee (G \vee \neg G)]   \\
                                                                    & \equiv (S \vee \neg G) \vee [\neg S \vee \top]              \\
                                                                    & \equiv (S \vee \neg G) \vee \top                            \\
                                                                    & \equiv \top                                                 \\
                                                                    & \equiv \text{George is either happy or unhappy.}
                        \end{align*}
                        Without simplification, the statement can be written as ``Either Steve is happy, or George is happy but Steve is not, or George is unhappy.'' \qed

                  \item $S \vee[G \wedge(\neg S \vee \neg G)]$.
                        \sol{}
                        \begin{align*}
                              S \vee[G \wedge(\neg S \vee \neg G)] & \equiv (S \vee G) \wedge [S \vee (\neg S \vee \neg G)]           \\
                                                                   & \equiv (S \vee G) \wedge [(S \vee \neg S) \vee \neg G]           \\
                                                                   & \equiv (S \vee G) \wedge [\top \vee \neg G]                      \\
                                                                   & \equiv (S \vee G) \wedge \top                                    \\
                                                                   & \equiv S \vee G                                                  \\
                                                                   & \equiv \text{Either Steve is happy or George is happy, or both.}
                        \end{align*}
                        Without simplification, the statement can be written as ``Either Steve is happy, or George is happy while either Steve is not or George is not.'' \qed
            \end{enumerate}

      \item Identify the premises and conclusions of the following deductive arguments and
            analyze their logical forms. Do you think the reasoning is valid? (Although you
            will have only your intuition to guide you in answering this last question, in
            the next section we will develop some techniques for determining the validity
            of arguments.)
            \begin{enumerate}
                  \item Jane and Pete won't both win the math prize. Pete will win either the math
                        prize or the chemistry prize. Jane will win the math prize. Therefore, Pete
                        will win the chemistry prize.\sol{}

                        Premises:
                        \begin{enumerate}
                              \item Jane and Pete won't both win the math prize.
                              \item Pete will win either the math prize or the chemistry prize.
                              \item Jane will win the math prize.
                        \end{enumerate}
                        \vspace{1em}
                        Conclusion: Pete will win the chemistry prize.\\

                        Let $P$ be the statement ``Pete will win the math prize'',

                        $Q$ be the statement ``Pete will win the chemistry prize'',

                        $R$ be the statement ``Jane will win the math prize'',\\

                        then the argument can be written as $\neg(P \wedge R) \wedge (P \vee Q) \wedge
                              R \Rightarrow Q$.\\

                        Reasoning steps:
                        \begin{enumerate}[label=(\arabic*)]
                              \item Premise (iii) stated that Jane will win the math prize, but premise (i) stated
                                    that Jane and Pete won't both win the math prize. Therefore, \textbf{Pete won't
                                          win the math prize} since Jane will win it.
                              \item From premise (ii), Pete will win either the math prize or the chemistry prize,
                                    therefore \textbf{he will win the chemistry prize} since he won't win the math
                                    prize.\\
                              \item Hence, the conclusion is valid.
                        \end{enumerate}\qed

                  \item The main course will be either beef or fish. The vegetable will be either peas
                        or corn. We will not have both fish as a main course and corn as a vegetable.
                        Therefore, we will not have both beef as a main course and peas as a vegetable.
                        \sol{}

                        Premises:
                        \begin{enumerate}
                              \item The main course will be either beef or fish.
                              \item The vegetable will be either peas or corn.
                              \item We will not have both fish as a main course and corn as a vegetable.
                        \end{enumerate}
                        \vspace{1em}
                        Conclusion: We will not have both beef as a main course and peas as a vegetable.\\

                        Let $P$ be the statement ``The main course will be beef'',

                        $Q$ be the statement ``The vegetable will be peas'',\\

                        then the argument can be written as $(P \vee \neg P) \wedge (Q \vee \neg Q)
                              \wedge \neg(\neg P \wedge \neg Q) \Rightarrow \neg(P \wedge Q)$.\\

                        Reasoning steps:
                        \begin{enumerate}
                              \item Since the main course will be either beef or fish, not having beef as the main
                                    course means that the main course will be fish.

                              \item Since the vegetable will be either peas or corn, not having peas as the
                                    vegetable means that the vegetable will be corn, which contradicts with premise
                                    (iii).

                              \item Hence, the conclusion is not valid.
                        \end{enumerate} \qed

                  \item Either John or Bill is telling the truth. Either Sam or Bill is lying.
                        Therefore, either John is telling the truth or Sam is lying. \sol{}

                        Premises:
                        \begin{enumerate}
                              \item Either John or Bill is telling the truth.
                              \item Either Sam or Bill is lying.
                        \end{enumerate}
                        \vspace{1em}
                        Conclusion: Either John is telling the truth or Sam is lying.\\

                        Let $P$ be the statement ``John is telling the truth'',

                        $Q$ be the statement ``Bill is telling the truth'',

                        $R$ be the statement ``Sam is telling the truth'',\\

                        then the argument can be written as $(P \vee Q) \wedge (\neg R \vee Q) \wedge
                              (P \vee \neg R) \Rightarrow (P \vee \neg R)$.\\

                        Reasoning steps:
                        \begin{enumerate}
                              \item If John is telling the truth, then from premise (i), Bill is lying.

                              \item If Bill is lying, then from premise (ii), Sam is telling the truth.

                              \item If John is lying, then from premise (i), Bill is telling the truth.

                              \item If Bill is telling the truth, then from premise (ii), Sam is lying.

                              \item Hence, the conclusion is valid.
                        \end{enumerate} \qed

                  \item Either sales will go up and the boss will be happy, or expenses will go up and
                        the boss won't be happy. Therefore, sales and expenses will not both go up.
                        \sol{}

                        Premises:

                        Either sales will go up and the boss will be happy, or expenses will go up and
                        the boss won't be happy.\\

                        Conclusion: Sales and expenses will not both go up.\\

                        Let $P$ be the statement ``Sales will go up'',

                        $Q$ be the statement ``Expenses will go up'',

                        $R$ be the statement ``The boss will be happy'',\\

                        then the argument can be written as $(P \wedge R) \vee (Q \wedge \neg R)
                              \Rightarrow \neg(P \wedge Q)$.\\

                        Reasoning steps:

                        There is one case where sales will go up and the boss will be happy, and
                        expenses will go up but the boss will not be happy as well. Therefore, the
                        reasoning is not valid. \qed
            \end{enumerate}
\end{enumerate}
\end{document}