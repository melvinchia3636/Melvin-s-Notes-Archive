\documentclass{report}

\usepackage[fleqn]{amsmath}
\usepackage{amssymb}
\usepackage[a4paper, total={7in,10in}]{geometry}
\usepackage{multicol}

\setlength{\columnseprule}{1pt}
\setlength{\columnsep}{24pt}
\hfuzz=100pt

\begin{document}
\begin{multicols*}{2}
    \begin{enumerate}
        \small
        \item $\sqrt{\sqrt{3} + \sqrt{\sqrt{3}+x}} = x$
              \\\\\textbf{Sol.}

              $\because \forall n \in \mathbb{R}, n >= 0,\ \sqrt{n} >= 0 \quad \therefore x > 0$
              \begin{flalign*}
                  \sqrt{\sqrt{3} + \sqrt{\sqrt{3}+x}}   & = x                               \\
                  \sqrt{3} + \sqrt{\sqrt{3}+x}          & = x^2                             \\
                  \sqrt{\sqrt{3} + x}                   & = x^2 - \sqrt{3}                  \\
                  x + \sqrt{3}                          & = {\left(x^2 - \sqrt{3}\right)}^2 \\
                                                        & = x^4 - 2\sqrt{3}x^2 + 3          \\
                  x^4 - 2\sqrt{3}x^2 + 3 - x - \sqrt{3} & = 0
              \end{flalign*}
              Let $a = \sqrt{3}$,
              \begin{flalign*}
                  x^4 - 2ax^2 + a^2 - x - a                                                                   & = 0 \\
                  a^2 - (2x^2 + 1)a + x^4 - x                                                                 & = 0 \\
                  a^2 - (2x^2 + 1)a + x(x^3 - 1)                                                              & = 0 \\
                  a^2 - (2x^2 + 1)a + x(x-1)(x^2 + x + 1)                                                     & = 0 \\
                  a^2 - (2x^2 + 1)a + (x^2 - x)(x^2 + x + 1)                                                  & = 0 \\
                  \left[a - (x^2 - x)\right]\left[a - (x^2 + x + 1)\right]                                    & = 0 \\
                  a                                                        = x^2 - x \text{ or } a  = x^2 + x & + 1
              \end{flalign*}
              When $a = x^2 - x$,
              \begin{flalign*}
                  x^2 - x            & = \sqrt{3}                             \\
                  x^2 - x - \sqrt{3} & = 0                                    \\
                  x                  & = \frac{1 \pm \sqrt{1 + 4\sqrt{3}}}{2} \\
                  \because\ x        & > 0                                    \\
                  \therefore x       & = \frac{1 + \sqrt{1 + 4\sqrt{3}}}{2}
              \end{flalign*}
              When $a = x^2 + x + 1$,
              \begin{flalign*}
                  x^2 + x + 1            & = \sqrt{3}                                    \\
                  x^2 + x + 1 - \sqrt{3} & = 0                                           \\
                  x                      & = \frac{-1 \pm \sqrt{1 - 4(1 - \sqrt{3})}}{2} \\
                                         & = \frac{-1 \pm \sqrt{4\sqrt{3} - 3}}{2}       \\
                  \because\ x            & > 0                                           \\
                  \therefore\ x          & = \frac{-1 + \sqrt{4\sqrt{3} - 3}}{2}
              \end{flalign*}
              $\therefore\ x = \dfrac{1 + \sqrt{1 + 4\sqrt{3}}}{2}$ or $x = \dfrac{-1 + \sqrt{4\sqrt{3} - 3}}{2}$
    \end{enumerate}
\end{multicols*}
\end{document}