\documentclass{report}
\usepackage[a4paper, total={7.5in, 10in}]{geometry}
\usepackage[fleqn]{amsmath}
\usepackage{amssymb}
\usepackage{amsthm}
\usepackage{enumitem}
\usepackage[]{mdframed}
\usepackage{multicol}
\usepackage{thmtools}
\usepackage{graphicx}
\usepackage{tikz}
\usepackage{tipa}
\usepackage{array, makecell, cellspace}
\usepackage{bigints}
\usepackage[export]{adjustbox}
\setlength{\cellspacetoplimit}{13.2ex}
\setlength{\cellspacebottomlimit}{13.2ex}

\usepackage{ifxetex}

\ifxetex
      \usepackage{substitutefont}
      \substitutefont{T3}{\rmdefault}{cmr}
\fi

\usepackage{fontspec}
\setmainfont[Mapping=tex-text]{Georgia}

\title{Praktis 4\\Permutation and Combination}
\author{Melvin Chia}

\newcommand\perm[2][^n]{\prescript{#1\mkern-2.5mu}{}P_{#2}}
\newcommand\permtwo[2][^n]{{}_{#1}P_{#2}}
\newcommand\comb[2][^n]{{}_{#1}C_{#2}}
\newcommand\combtwo[2][^n]{\prescript{#1\mkern-2.5mu}{}C_{#2}}

\newcommand{\sol}[1]{

      \noindent \textbf{Sol.}
}
\newcommand{\prooff}[1]{

      \noindent \textbf{Proof.}
}

\newcommand{\arc}[1]{{%
                  \setbox9=\hbox{#1}%
                  \ooalign{\resizebox{\wd9}{\height}{\texttoptiebar{\phantom{A}}}\cr#1}}}

\def\eos{\quad\hbox{\rlap{\hbox{\vrule depth 1.5pt height 2.6mm width 0.2mm \hskip 1mm \vrule height 2.6mm width 0.2mm}}{\vbox{\hrule height 0.2mm width 1.4mm \vskip 2.8mm \hrule depth 1.5pt height -0.35mm width 1.2mm}}}}

\counterwithout{equation}{chapter}
\setlength{\columnseprule}{1pt}
\setlength{\columnsep}{24pt}
\hfuzz=100pt
\setcounter{chapter}{4}

\begin{document}
\maketitle

\begin{multicols*}{2}
      \noindent\Large{\underline{\textbf{Praktis Formatif}}}
      \normalsize
      \section{Permutation}
      \begin{enumerate}
            \item There are 3 types of fruits and 5 types of cupcakes in the kitchen. Determine
                  the number of selections made by Aiman if
                  \begin{enumerate}
                        \item he can only choose one of the item, \sol{}

                              Using the addition rule, the number of selections made by Aiman is $3 + 5 = 8$.
                              $\eos$

                        \item he can choose one type of fruits and one type of cupcakes. \sol{}

                              Using the multiplication rule, the number of selections made by Aiman is $3
                                    \times 5 = 15$. $\eos$
                  \end{enumerate}

            \item During the school mid-year holidays, Zoe wants to do revision for her 3 Science
                  subjects, 2 Mathematics subjects, 3 language subjects and 2 other core
                  subjects. Find the number of ways the revision can be done if
                  \begin{enumerate}
                        \item Zoe revises one subject only, \sol{}

                              Using the addition rule, the number of ways the revision can be done is $3 + 2
                                    + 3 + 2 = 10$. $\eos$

                        \item Zoe wants to revise one science subject, one mathematics subject and one core
                              subject in a day. \sol{}

                              Using the multiplication rule, there are $3 \times 2 \times 2 = 12$ ways to do
                              the revision. $\eos$
                  \end{enumerate}

            \item Ali plans to visit Zaleha who stays in Sarawak. Ali can choose to ride in
                  either 3 of his friends's car or purchase a ticket from any of the 2 bus
                  companies to Kuala Lumpur International Airport. From there, Ali can choose
                  from any 3 flight companies to Sarawak. In how many ways can Ali go to Sarawak?
                  \sol{}

                  To go to Kuala Lumpur International Airport, Ali can choose to ride in 4 of his
                  friends's car or purchase a ticket from any of the 3 bus companies. Using the
                  addition rule, there ara $3 + 2 = 5$ ways to go to Kuala Lumpur International
                  Airport.

                  To go to Sarawak, Ali can choose from any 3 flight companies. Using the
                  multiplication rule, there are $5 \times 3 = 15$ ways to go to Sarawak. $\eos$

            \item During School's Entrepreneurship Carnival, Sandy is interested to participate
                  in the Treasure Hunt game. The participants need to get two clues from Station
                  A and Station B. There are 3 paths to reach station A. From Station A, there
                  are 4 paths to Station B. All paths are of different distances. After obtaining
                  two clues, every participant needs to return to the starting point to get his
                  last clue. Determine the number of ways to travel to and from if Sandy
                  \begin{enumerate}
                        \item choose the same path, \sol{}

                              Since Sandy choose the same path to and from Station A and Station B, using the
                              multiplication rule, the number of ways to travel to and from is $3 \times 4 =
                                    12$. $\eos$

                        \item does not use the same path. \sol{}

                              Since Sandy does not use the same path to and from Station A and Station B,

                              From starting point to Station A, there are 3 paths.

                              From Station A to Station B, there are 4 paths.

                              From Station B back to Station A, there are 3 paths.

                              From Station A back to starting point, there are 2 paths.

                              Using the multiplication rule, there are $3 \times 4 \times 3 \times 2 = 72$
                              ways to travel to and from. $\eos$
                  \end{enumerate}

            \item Chong wants to set a passcode to her safe deposit box. The passcode consists of
                  a 4-digit number. Find the number of passcodes that can be formed by Chong.
                  \sol{}

                  There are 10 digits. Since each digit can be used more than once, there are
                  $10^4 = 10\,000$ passcodes that can be formed by Chong. $\eos$

            \item Solve each of the following situations.
                  \begin{enumerate}
                        \item Find the number of ways to rearrange all the letters in the word CUTE. \sol{}

                              There are 4 letters. There are $4! = 24$ ways to rearrange all the letters.
                              $\eos$

                        \item How many 4-digit numbers that are different that can be formed by using the
                              digits 1, 2, 3, and 4 without repetition? \sol{}

                              There are 4 digits. Since each digit can be used only once, there are $4! = 24$
                              4-digit numbers that are different that can be formed by using these digits.
                              $\eos$

                        \item How many ways can 5 people queue up to receive vaccinations in hospital? \sol{}

                              There are $5! = 120$ ways to queue up to receive vaccinations. $\eos$
                  \end{enumerate}

            \item Find the number of arrangements in each of the following situations.
                  \begin{enumerate}
                        \item Determine the number of ways to arrange 10 VIP guests to sit at a round table.
                              \sol{}

                              There are $(10 - 1)! = 9! = 362\,880$ ways to arrange 10 VIP guests to sit at a
                              round table. $\eos$

                        \item Find the number of ways fro 8 scouts to stand around a campfire. \sol{}

                              There are $(8 - 1)! = 7! = 5\,040$ ways to arrange 8 scouts to stand around a
                              campfire. $\eos$

                        \item Find the number of ways to distribute 10 types of fruits to 10 students sitting
                              at a round table. \sol{}

                              There are $(10 - 1)! = 9! = 362\,880$ ways to distribute 10 types of fruits to
                              10 students sitting at a round table. $\eos$
                  \end{enumerate}

            \item Determine the number of arrangements for each of the following situations.
                  \begin{enumerate}
                        \item How many ways to form a bracelet with 12 different pearls? \sol{}

                              Since the bracelet can be flipped, the clockwise and anticlockwise arrangements
                              are the same.

                              Hence, there are $\dfrac{(12 - 1)!}{2} = \dfrac{11!}{2} = 19\,958\,400$ ways to
                              form a bracelet. $\eos$

                        \item Find the number of ways to arrange 10 types of flowers to form a floral hoop.
                              \sol{}

                              Since the floral hoop can be flipped, the clockwise and anticlockwise
                              arrangements are the same.

                              Hence, there are $\dfrac{(10 - 1)!}{2} = \dfrac{9!}{2} = 181\,440$ ways to
                              arrange 10 types of flowers to form a floral hoop. $\eos$
                  \end{enumerate}

            \item Find the value of $n$ for each of the following.
                  \begin{enumerate}
                        \item $\permtwo[n]{2} = 6$
                              \sol{}
                              \begin{flalign*}
                                    \permtwo[n]{2} & = 6                    \\
                                    n(n - 1)       & = 6                    \\
                                    n^2 - n        & = 6                    \\
                                    n^2 - n - 6    & = 0                    \\
                                    (n - 3)(n + 2) & = 0                    \\
                                    n              & = 3 \quad (n > 0) \eos
                              \end{flalign*}

                        \item $\permtwo[n+2]{3} = 42n$
                              \sol{}
                              \begin{flalign*}
                                    \permtwo[n+2]{3} & = 42n                  \\
                                    (n + 2)(n + 1)n  & = 42n                  \\
                                    n^2 + 3n + 2     & = 42                   \\
                                    n^2 - 3n - 40    & = 0                    \\
                                    (n - 8)(n + 5)   & = 0                    \\
                                    n                & = 8 \quad (n > 0) \eos
                              \end{flalign*}

                        \item $7(\permtwo[n+1]{2}) = 5(\permtwo[n+2]{2})$
                              \sol{}
                              \begin{flalign*}
                                    7(\permtwo[n+1]{2})                       & = 5(\permtwo[n+2]{2}) \\
                                    \frac{\permtwo[n+1]{2}}{\permtwo[n+2]{2}} & = \frac{5}{7}         \\
                                    \frac{(n + 1)(n)}{(n + 2)(n + 1)}         & = \frac{5}{7}         \\
                                    \frac{n}{n+2}                             & = \frac{5}{7}         \\
                                    7n                                        & = 5n + 10             \\
                                    2n                                        & = 10                  \\
                                    n                                         & = 5
                              \end{flalign*}

                        \item $\permtwo[2n]{2} = 3(\permtwo[n+1]{2})$
                              \sol{}
                              \begin{flalign*}
                                    \permtwo[2n]{2}                          & = 3(\permtwo[n+1]{2}) \\
                                    \frac{\permtwo[2n]{2}}{\permtwo[n+1]{2}} & = 3                   \\
                                    \frac{(2n)(2n - 1)}{(n + 1)n}            & = 3                   \\
                                    \frac{4n^2 - 2n}{n^2 + n}                & = 3                   \\
                                    4n^2 - 2n                                & = 3n^2 + 3n           \\
                                    n^2 - 5n                                 & = 0                   \\
                                    n(n - 5)                                 & = 0                   \\
                                    n                                        & = 5 \quad (n > 0)
                              \end{flalign*}
                  \end{enumerate}

            \item Determine the number of arrangements of the following situations.
                  \begin{enumerate}
                        \item How many 3-digit numbers that are different that can be formed from digits 1,
                              2, 3, 4, and 5 without repetition? \sol{}

                              There are $\permtwo[5]{3} = 60$ ways to arrange 4 digits from 5 digits. $\eos$

                        \item Find the number of ways to arrange 7 students in 4 chairs. \sol{}

                              There are $\permtwo[7]{4} = 840$ ways to arrange 7 students in 4 chairs. $\eos$

                        \item How many ways can 4 cars park in 8 empty parking lots along the street as shown
                              in the following diagram? \sol{}

                              There are $\permtwo[8]{4} = 1\,680$ ways to arrange 4 cars in 8 empty parking
                              lots. $\eos$

                        \item There are 12 participants in a 100-m run. Find the number of possible results
                              obtained if presents are only given to the champion, $1^{st}$ runner-up and
                              $2^{nd}$ runner-up only. \sol{}

                              There are $\permtwo[12]{3} = 1\,320$ results of having 3 winners in 12
                              participants. $\eos$
                  \end{enumerate}

            \item Determine the number of arrangements of $r$ out of $n$ different objects in a
                  circle.
                  \begin{enumerate}
                        \item Find the number of ways to arrange 10 students to sit in 6 chairs at a round
                              table. \sol{}

                              There are $\dfrac{\permtwo[10]{6}}{6} = 25\,200$ ways to arrange 10 students to
                              sit in 6 chairs at a round table. $\eos$

                        \item If each of the children is given only a new school ag, find the number of ways
                              to distribute 9 school bags to 5 children sitting in a circle. \sol{}

                              There are $\dfrac{\permtwo[9]{5}}{5} = 3\,024$ ways to distribute 9 school bags
                              to 5 children sitting in a circle. $\eos$

                        \item Determine the number of ways 5 people can sit in 7 empty chairs at a round
                              table. \sol{}

                              There are $\dfrac{\permtwo[7]{5}}{7} = 360$ ways to arrange 5 people to sit
                  \end{enumerate}

            \item Solve each of the following questions.
                  \begin{enumerate}
                        \item The following diagram shows a hula hoop that can be dismantled into 6 parts.The
                              user can fix the hoop again by choosing the combination of colours desired.

                              How many ways can the hula hoop be fixed if Siti has 8 parts of different
                              colours? \sol{}

                              Since the hoop can be flipped, the clockwise and anticlockwise arrangements are
                              the same.

                              Hence, there are $\dfrac{\permtwo[8]{6}}{2\cdot 6} = 1\,680$ ways to fix the
                              hula hoop. $\eos$

                        \item Find the number of ways to make a bracelet which contains 8 pearls chosen from
                              16 different pearls. \sol{}

                              Since the bracelet can be flipped, the clockwise and anticlockwise arrangements
                              are the same.

                              Hence, there are $\dfrac{\permtwo[16]{8}}{2\cdot 8} = 32\,432\,400$ ways to
                              make a bracelet. $\eos$
                  \end{enumerate}

            \item Solve the following permutation questions involving identical objects.
                  \begin{enumerate}
                        \item Find the number of ways to rearrange the letters from the word
                              \begin{enumerate}
                                    \item LOOKOUT \sol{}

                                          There are 3 Os.

                                          Hence, there are $\dfrac{7!}{3!} = 840$ ways to rearrange the word. $\eos$

                                    \item MATHEMATICS \sol{}

                                          There are 2Ms, 2T, and 2As.

                                          Hence, there are $\dfrac{12!}{2!2!2!} = 4\,989\,600$ ways to rearrange the
                                          word. $\eos$

                                    \item MISSISSIPPI \sol{}

                                          There are 4Is, 4Ss, and 2Ps.

                                          Hence, there are $\dfrac{11!}{4!4!2!} = 34\,650$ ways to rearrange the word.
                                          $\eos$
                              \end{enumerate}

                        \item Find the number of ways to form a 5-digit number from cards that are labelled
                              with 1, 2, 2, 2, 3, 4, 5, 6, 7, 8, and 8 if all the digit 2 must be used.
                              \sol{}

                              Since all the digit 2 must be used, the 3 digit 2 cards can be treated as 1
                              card.

                              There are 2 card of 8.

                              If the number has no 8, there are $\comb[6]{2} = 15$ ways to choose the 2 other
                              digits.

                              Since 3 digit 2 cards and the 2 other digits can be arranged in any order,
                              there are $\frac{5!}{3!} = 20$ ways to arrange them.

                              Hence, there are $15 \times 20 = 300$ ways to form a 5-digit number with no 8.

                              If the number has 1 8, there are $\comb[6]{1} = 6$ ways to choose the 1 other
                              digits.

                              Since 3 digit 2 cards and the 2 other digits can be arranged in any order,
                              there are $\frac{5!}{3!} = 20$ ways to arrange them.

                              Hence, there are $6 \cdot 20 = 120$ ways to form a 5-digit number with 1 8.

                              If the number has 2 8s, there are only 1 way to choose the 2 other digits.

                              Since 3 digit 2 cards and the 2 digit 8 cards can be arranged in any order,
                              there are $\frac{5!}{3!2!} = 10$ ways to arrange them.

                              Hence, there are $1 \cdot 10 = 10$ ways to form a 5-digit number with 2 8s.

                              Therefore, there are $300 + 120 + 10 = 430$ ways to form a 5-digit number from
                              the 11 cards. $\eos$
                  \end{enumerate}

            \item Find the number of ways to rearrange all the letters in the word ENGLISH if
                  \begin{enumerate}
                        \item vowels must be placed at both ends, \sol{}

                              There are 2 vowels.

                              There are $2! = 2$ ways to arrange the 2 vowels at both ends.

                              There are $5! = 120$ ways to arrange the 5 consonants in the middle.

                              Hence, there are $2 \times 120 = 240$ ways to rearrange all the letters. $\eos$

                        \item it must start with a consonant, \sol{}

                              There are 5 ways to choose the first letter.

                              There are $6! = 720$ ways to arrange the remaining letters.

                              Hence, there are $5 \times 720 = 3\,600$ ways to arrange them. $\eos$

                        \item vowels must be side by side, \sol{}

                              Since the vowels must be side by side, the vowels can be treated as 1 letter.

                              There are $6! = 720$ ways to arrange the 6 letters.

                              Since the vowels can be arranged in any order, there are $2! = 2$ ways to
                              arrange them.

                              Hence, there are $720 \times 2 = 1\,440$ ways to arrange them. $\eos$

                        \item only 5 letters are arranged with the vowels placed side by side. \sol{}

                              Since the vowels must be side by side, the vowels can be treated as 1 letter.

                              Since the vowels can be arranged in any order, there are $2! = 2$ ways to
                              arrange them.

                              There are $\comb[5]{3} = 10$ ways to choose the other 3 letters.

                              There are $4! = 24$ ways to arrange the 4 letters (3 consonants and 2 vowel as
                              1 letter).

                              Hence, there are $2 \times 10 \times 24 = 480$ ways to arrange them. $\eos$

                        \item only 5 letters are arranged with N and G must be included. \sol{}

                              There are $\comb[5]{3} = 10$ ways to choose the other 3 letters.

                              There are $5! = 120$ ways to arrange the 5 letters.

                              Hence, there are $10 \times 120 = 1\,200$ ways to arrange them. $\eos$
                  \end{enumerate}

            \item Find the number of ways to arrange 7 family members at a round table if
                  \begin{enumerate}
                        \item there are 5 vacant chairs and both parents must sit next to each other, \sol{}

                              Since both parents must sit next to each other, the parents can be treated as 1
                              person.

                              Since the parent can be arranged in any order, there are $2! = 2$ ways to
                              arrange them.

                              There are $\permtwo[5]{3} = 60$ ways to arrange the 3 other family members
                              relative to the parent.

                              Hence, there are $2 \times 60 = 120$ ways to arrange them. $\eos$

                        \item there are 7 vacant chairs and both parents must sit next to each other, \sol{}

                              Since both parents must sit next to each other, the parents can be treated as 1
                              person.

                              Since the parent can be arranged in any order, there are $2! = 2$ ways to
                              arrange them.

                              There are $(6-1)! = 120$ ways to arrange the 6 family members.

                              Hence, there are $2 \times 120 = 240$ ways to arrange them. $\eos$

                        \item there are 10 vacant chairs and both parents must sit next to each other. \sol{}

                              Since both parents must sit next to each other, the parents can be treated as 1
                              person.

                              Since the parent can be arranged in any order, there are $2! = 2$ ways to
                              arrange them.

                              There are $\permtwo[8]{5} = 6\,720$ ways to arrange the 5 other family members
                              in the remaining 8 chairs relative to the parent.

                              Hence, there are $2 \times 6\,720 = 13\,440$ ways to arrange them. $\eos$
                  \end{enumerate}

            \item How many ways can 4 male students and 2 female students be seated in a row if
                  \begin{enumerate}
                        \item they can sit anywhere, \sol{}

                              There are $6! = 720$ ways to arrange the 6 students. $\eos$

                        \item 2 female students must sit together,
                              \sol{}

                              Since two female students must sit together, they can be treated as one
                              student.

                              Since the two female students can be arranged in any order, there are $2! = 2$
                              ways to arrange them.

                              There are $5! = 120$ ways to arrange 5 students.

                              Hence, there are $2 \times 120 = 240$ ways to arrange them. $\eos$

                        \item 2 female students must be separated.
                              \sol{}

                              There are $720 - 240 = 480$ ways to arrange them. $\eos$
                  \end{enumerate}

            \item Find the number of ways to arrange 7 different story books and 3 different
                  magazines on a bookshelf if
                  \begin{enumerate}
                        \item no condition is imposed, \sol{}

                              There are $10! = 3\,628\,800$ ways to arrange them. $\eos$

                        \item the magazines must be put together, \sol{}

                              Since the magazines must be put together, they can be treated as one book.

                              Since the magazines can be arranged in any order, there are $3! = 6$ ways to
                              arrange them.

                              There are $8! = 40\,320$ ways to arrange the 8 books.

                              Hence, there are $6 \times 40\,320 = 241\,920$ ways to arrange them. $\eos$

                        \item the magazines cannot be put together. \sol{}

                              First, arrange the 7 story books in $7! = 5\,040$ ways.

                              Then, arrange the 3 magazines in 8 slots in between the story books, the
                              beginning and the end of the bookshelf in $\permtwo[8]{3} = 336$ ways.

                              Hence, there are $5\,040 \times 336 = 1\,693\,440$ ways to arrange them. $\eos$
                  \end{enumerate}

            \item GIven cards that are labelled with the digits 0, 3, 4, 5, 6, and 7. Find the
                  number of arrangements of those digits without repetition to form
                  \begin{enumerate}
                        \item 4-digit odd numbers,
                              \sol{}

                              Since the first digit cannot be 0, there are 5 digits to choose from.

                              Since the number is odd, there last digit can be either 3, 5, or 7. There are 3
                              digits to choose from.

                              If the first digit is odd, there are 2 digits to choose from.

                              If the first digit is even, there are 3 digits to choose from.

                              Hence, there are $3 \times 2 + 2 \times 3 = 12$ ways to arrange the first digit
                              and the last digit.

                              There are $\permtwo[4]{2} = 12$ ways to arrange the remaining 2 digits.

                              Hence, there are $12 \times 12 = 144$ ways to arrange them. $\eos$

                        \item 4-digit numbers that begin with an even digit,
                              \sol{}

                              The first digit can be either 4 or 6. There are 2 digits to choose from.

                              There are $\permtwo[5]{3} = 60$ ways to arrange the remaining 3 digits.

                              Hence, there are $2 \times 60 = 120$ ways to arrange them. $\eos$

                        \item 4-digit numbers with all the odd digits together,
                              \sol{}

                              There are 3 odd digits. There are $3! = 6$ ways to arrange them. If the odd
                              digits are at the beginning, the last digit can be chosen in 3 ways. If the odd
                              digits are at the end, the first digit can be chosen in 2 ways.

                              Hence, there are $6 \times 3 + 6 \times 2 = 30$ ways to arrange them. $\eos$

                        \item 4-digit numbers with odd and even digits at the alternate positions,
                              \sol{}

                              If the digits are arranged in the order `odd, even, odd, even', there are
                              $\permtwo[3]{2} \times \permtwo[3]{2} = 36$ ways to arrange them.

                              If the digits are arranged in the order `even, odd, even, odd', there are
                              $\permtwo[2]{1} \times \permtwo[2]{1} \times \permtwo[3]{2} = 24$ ways to
                              arrange them.

                              Hence, there are $36 + 24 = 60$ ways to arrange them. $\eos$

                        \item even numbers that are greater than $50\ 000$. \sol{}

                              If the number is a 5-digit number:

                              Since the number is greater than $50\ 000$, the first digit must be 5, 6, or 7.
                              There are 3 digits to choose from.

                              Since the number is even, the last digit can be either 0, 4, or 6.

                              If the first digit is 6, there are 2 digits to choose from.

                              If the first digit is 5 or 7, there are 3 digits to choose from.

                              Hence, there are $2 \times 1 + 2 \times 3 = 8$ ways to arrange the first digit
                              and the last digit.

                              There are $\permtwo[4]{3} = 24$ ways to arrange the remaining 3 digits.

                              Hence, there are $8 \times 24 = 192$ ways to arrange them. \\ \\ If the number
                              is a 6-digit number:

                              Since the beginning of the number cannot be 0, there are 5 digits to choose
                              from.

                              Since the number is even, the last digit can be either 0, 4, or 6.

                              If the first digit is 4, or 6, there are 2 digits to choose from.

                              If the first digit is 3, 5 or 7, there are 3 digits to choose from.

                              Hence, there are $2 \times 2 + 3 \times 3 = 13$ ways to arrange the first digit
                              and the last digit.

                              There are $4! = 24$ ways to arrange the remaining 4 digits.

                              Hence, there are $13 \times 24 = 312$ ways to arrange them.\\\\ Therefore,
                              there are $192 + 312 = 504$ ways to arrange them. $\eos$
                  \end{enumerate}

            \item In every football match, the result may be win, lose or draw. Determine the
                  number of possible outcomes obtained in a round of match that involves 12 teams
                  (6 matches). \sol{}

                  There are 6 matches, each matches can yield 3 outcomes. Hence, there are $3^6 =
                        729$ possible outcomes. $\eos$

            \item The following diagram shows the seating arrangement in the meeting room of
                  company $X$.

                  Determine the number of ways 9 workers can be seated during the meeting if
                  \begin{enumerate}
                        \item no condition is imposed, \sol{}

                              There are 9 seats and 9 workers. Hence, there are $9! = 362\,880$ ways to
                              arrange them. $\eos$

                        \item 3 particular workers must sit in the same row.
                              \sol{}

                              There are $6! = 720$ ways to arrange the remaining 6 workers.

                              If the 3 workers seat at the top row, there are $\permtwo[5]{3} = 60$ ways to
                              arrange them.

                              If the 3 workers seat at the bottom row, there are $\permtwo[4]{3} = 24$ ways
                              to arrange them.

                              Hence there are $60 + 24 = 84$ ways to arrange them.

                              Therefore, there are $720 \times 84 = 60\,480$ ways to arrange them. $\eos$
                  \end{enumerate}

            \item 8 teachers travel in 2 cars to attend a course. Only 5 teachers have driving licenses. Calculate the number of seating arrangements of the teachers in the 2 cars if each car can accommodate only 4 people.
                  \sol{}

                  There are $\permtwo[5]{2} = 20$ ways to arrange the drivers for the 2 cars.

                  There are $6! = 720$ ways to arrange the remaining 6 teachers in the 2 cars.

                  Hence, there are $20 \times 720 = 14\,400$ ways to arrange them. $\eos$

            \item Find the numbers of different arrangements using all the letters in the word
                  COMMITMENT. Hence, determine the number of arrangements which \sol{}

                  There are 2Ts and 3Ms.

                  Hence, there are $\dfrac{12!}{2!3!} = 302\,400$ ways to arrange them. $\eos$

                  \begin{enumerate}
                        \item begin and end with the letter T, \sol{}

                              2 Ts are placed at the beginning and end of the word. There are only 1 way to arrange the Ts.

                              There are 3Ms.

                              There are $\dfrac{8!}{3!} = 6\,720$ ways to arrange the 8 letters except the 2
                              Ts.

                              Hence, there are $6\,720 \times 1 = 6\,720$ ways to arrange them. $\eos$

                        \item contain MMM, \sol{}

                              Since the Ms are placed side by side, we can treat them as a single letter.

                              There are 2 Ts.

                              Hence, there are $\dfrac{9!}{3!} = 60\,480$ ways to arrange them. $\eos$

                        \item do not contain TT. \sol{}

                              If the word contains TT, we treat it as a single letter.

                              There are 3Ms.

                              Hence, there are $302\,400 - 60\,480 = 241\,920$ ways to arrange them. $\eos$
                  \end{enumerate}

            \item Determine the number of ways 4 doctors and 4 nurses can be seated at a round
                  table if
                  \begin{enumerate}
                        \item no condition is imposed, \sol{}

                              There are 8 seats and 8 people. Hence, there are $(8-1)! = 5\,040$ ways to
                              arrange them. $\eos$

                        \item 2 nurses should not sit side by side,
                              \sol{}

                              If the two nurses sit side by side, we can treat them as a single person.

                              Since the two nurses can switch their seats, there are 2 ways to arrange them.

                              There are 7 seats and 7 people. Hence, there are $(7-1)! = 720$ ways to arrange
                              them.

                              Hence, there are $720 \times 2 = 1440$ ways to arrange them.

                              Therefore, there are $5\,040 - 1\,440 = 3\,600$ ways to arrange them such that
                              the 2 nurses do not sit side by side. $\eos$

                        \item the doctors and nurses sit in alternate positions. \sol{}

                              First, arrange the doctors. There are $(4-1)! = 6$ ways to arrange them.

                              Then, arrange the nurses in between the doctors. There are $4! = 24$ ways to
                              arrange them.

                              Hence, there are $6 \times 24 = 144$ ways to arrange them. $\eos$
                  \end{enumerate}

      \end{enumerate}
      \section{Combination}
      \begin{enumerate}
            \setcounter{enumi}{23}
            \item \begin{enumerate}
                        \item Find the number of ways to choose 2 out of 5 story books on the bookshelf.
                              \sol{}

                              There are $\comb[5]{2} = 10$ ways to choose 2 books. $\eos$

                        \item Determine the number of ways to select 4 representatives from 10 students to
                              participate in the national debate competition. \sol{}

                              There are $\comb[10]{4} = 210$ ways to choose them. $\eos$
                  \end{enumerate}

            \item Calculate the number of ways to form groups in which 9 students are divided
                  into
                  \begin{enumerate}
                        \item 2 groups of 4 and 5 students respectively,
                              \sol{}

                              Choose 4 students from 9 students. There are $\comb[9]{4} = 126$ ways to choose
                              them.

                              Choose 5 students from 5 students. There are $\comb[5]{5} = 1$ way to choose

                              Hence, there are $126 \times 1 = 126$ ways to arrange them. $\eos$

                        \item 3 groups of 2, 3, and 4 students respectively,
                              \sol{}

                              Choose 2 students from 9 students. There are $\comb[9]{2} = 36$ ways to choose
                              them.

                              Choose 3 students from 7 students. There are $\comb[7]{3} = 35$ ways to choose
                              them.

                              Choose 4 students from 4 students. There are $\comb[4]{4} = 1$ way to choose
                              them.

                              Hence, there are $36 \times 35 \times 1 = 1\,260$ ways to arrange them. $\eos$

                        \item 2 groups in which the difference between the group members must be at least 3 people.
                              \sol{}

                              Two group of 1 and 8 students respectively, there are $\comb[9]{1} \cdot
                                    \comb[8]{8} = 9$ ways to arrange them.

                              Two group of 2 and 7 students respectively, there are $\comb[9]{2} \cdot
                                    \comb[7]{7} = 36$ ways to arrange them.

                              Two group of 3 and 6 students respectively, there are $\comb[9]{3} \cdot
                                    \comb[6]{6} = 84$ ways to arrange them.

                              Hence, there are $9 + 36 + 84 = 129$ ways to arrange them. $\eos$

                  \end{enumerate}

            \item Four letters are selected from the word HITUNG. How many different selections
                  that are possible? From the selections, determine the number of selections that
                  \sol{}

                  There are $\comb[6]{4} = 15$ ways to choose 4 letters from 6 letters.
                  \begin{enumerate}
                        \item do not contain the vowel U, \sol{}

                              There are 5 letters without U. Hence, there are $\comb[5]{4} = 5$ ways to
                              choose them. $\eos$

                        \item contains the vowel U. \sol{}

                              There are $\comb[5]{3} = 10$ ways to choose the other 3 letters. $\eos$
                  \end{enumerate}

            \item The following diagram shows 8 cards of 1-digit number.

                  Find the number of selections of one card if

                  \begin{enumerate}
                        \item the chosen number is a multiple of 2, \sol{}

                              There are 4 cards of even digit. Hence, there are $\comb[4]{1} = 4$ ways to
                              choose them. $\eos$

                        \item a prime number is chosen, \sol{}

                              There are 4 cards of prime digit. Hence, there are $\comb[4]{1} = 4$ ways to
                              choose them. $\eos$

                        \item the number that is less than 6 is chosen. \sol{}

                              There are 4 cards of digit less than 6. Hence, there are $\comb[4]{1} = 4$ ways
                              to choose them. $\eos$
                  \end{enumerate}

            \item A team of 4 members is selected from 4 men and 6 women. Find the number of ways
                  the team can be formed if
                  \begin{enumerate}
                        \item no condition is imposed, \sol{}

                              There are $\comb[10]{4} = 210$ ways to form the team. $\eos$

                        \item the team consists of 1 man and 3 women, \sol{}

                              Choose 1 man from 4 men, there are $\comb[4]{1} = 4$ ways to do so.

                              Choose 3 women from 6 women, there are $\comb[6]{3} = 20$ ways to do so.

                              Hence, there are $4 \times 20 = 80$ ways to form the team. $\eos$

                        \item the number of men in the team is at least 2 people. \sol{}

                              According to the condition imposed, the team with all women or only one man is
                              not allowed.

                              The team with all women, there are $\comb[4]{0} \times \comb[6]{4} = 15$ way to
                              form the team.

                              The team with only one man, there are $\comb[4]{1} \times \comb[6]{3} = 80$ way
                              to form the team.

                              Hence, there are $15 + 80 = 95$ ways to form the team with less than two men.

                              Therefore, there are $210 - 95 = 115$ ways to form the team in which at least 2
                              men are chosen. $\eos$
                  \end{enumerate}

            \item The following diagram shows points that can be connected to form a geometrical
                  shape.

                  Find the possible number of ways to form
                  \begin{enumerate}
                        \item a triangle, \sol{}

                              The triangle can be either upright or upside down.

                              For upright triangle, choose one point from the top row and two points from the
                              bottom row. There are $\comb[4]{1} \times \comb[5]{2} = 40$ ways to do so.

                              For upside down triangle, choose one point from the bottom row and two points
                              from the top row. There are $\comb[5]{1} \times \comb[4]{2} = 30$ ways to do
                              so.

                              Hence, there are $40 + 30 = 70$ ways to form a triangle. $\eos$

                        \item a quadrilateral, \sol{}

                              Choose 2 points from the top row and 2 points from the bottom row to form a
                              quadrilateral. There are $\comb[4]{2} \times \comb[5]{2} = 60$ ways to do so.
                              $\eos$

                        \item a triangle in which point A or point C but not both and point F must be used.
                              \sol{}

                              The first vertex of the triangle is point F.

                              The second vertex can be either point A or point C. There are 2 ways to choose
                              it.

                              The third vertex can be any other point. There are $\comb[6]{1} = 6$ ways to
                              choose it.

                              Hence, there are $2 \times 6 = 12$ ways to form a triangle in which point A or
                              point C but not both and point F must be used. $\eos$
                  \end{enumerate}

            \item A tennis team of 4 men and 4 women is to be selected from 6 men and 7 women.
                  \begin{enumerate}
                        \item Find the number of selections to form the team. \sol{}

                              Choose 4 men from 6 men. There are $\comb[6]{4} = 15$ ways to choose them.

                              Choose 4 women from 7 women. There are $\comb[7]{4} = 35$ ways to choose them.

                              Hence, there are $15 \times 35 = 525$ ways to form the team. $\eos$

                        \item Determine the number of formations of the team if 2 out of 7 women must be
                              selected together or not selected at all.

                              If the two women are selected together, there are $\comb[5]{2} = 10$ ways to
                              select the other two female members.

                              If the two women are not selected at all, there are $\comb[5]{4} = 5$ ways to
                              choose the 4 female members.

                              Hence, there are $10 + 5 = 15$ ways to choose the 4 female members.

                              Therefore, there are $15 \times 15 = 225$ ways to form the team. $\eos$
                  \end{enumerate}

            \item During a meeting, 3 executive officers, 3 managers and 4 workers are seated at
                  a round table. Determine the number of ways they are seated if
                  \begin{enumerate}
                        \item no condition is imposed, \sol{}

                              There are $(10 - 1)! = 362,880$ ways to seat them. $\eos$

                        \item a particular executive officer must sit between a manager and a worker, \sol{}

                              Choose one manager from 3 managers. There are $\comb[3]{1} = 3$ ways to do so.

                              Choose one worker from 4 workers. There are $\comb[4]{1} = 4$ ways to do so.

                              Treat the executive officer, the manager and the worker as one person.

                              Since the manager and the worker can be seated in any order, there are $2! = 2$
                              ways to do so.

                              There are $(8-1)! = 40,320$ ways to seat the 8 people.

                              Hence, there are $3 \times 4 \times 2 \times 5,040 = 129\,960$ ways to seat
                              them.

                        \item 3 executive officers sit separately.
                              \sol{}

                              First, arrange 3 managers and 4 workers to sit. There are $(7-1)! = 720$ ways
                              to do so.

                              Then, arrange the 3 executive officers to sit in between the managers and the
                              workers. There are $\permtwo[7]{3} = 210$ ways to do so.

                              Hence, there are $720 \times 210 = 151,200$ ways to seat them. $\eos$
                  \end{enumerate}

            \item A quiz team of 10 players is to be chosen from a class of 8 boys and 12 girls.
                  Find
                  \begin{enumerate}
                        \item the number of different teams that can be formed if the number of boys is equal
                              to the number of girls, \sol{}

                              Choose 5 boys from 8 boys. There are $\comb[8]{5} = 56$ ways to choose them.

                              Choose 5 girls from 12 girls. There are $\comb[12]{5} = 792$ ways to choose
                              them.

                              Hence, there are $56 \times 792 = 44,352$ ways to form the team. $\eos$

                        \item the number of different teams that can be formed if the number of girls is more
                              than the number of boys, \sol{}

                              Choosing 1 boy and 9 girls, there are $\comb[8]{1} \times \comb[12]{9} =
                                    1\,760$ ways to form the team.

                              Choosing 2 boys and 8 girls, there are $\comb[8]{2} \times \comb[12]{8} =
                                    13\,860$

                              Choosing 3 boys and 7 girls, there are $\comb[8]{3} \times \comb[12]{7} =
                                    44\,352$

                              Choosing 4 boys and 6 girls, there are $\comb[8]{4} \times \comb[12]{6} =
                                    64\,680$

                              Hence, there are $1\,760 + 13\,860 + 44\,352 + 64\,680 = 124\,652$ ways to form
                              the team. $\eos$

                        \item the number of different teams that can be formed if Kamal and Ali as well as
                              Fatimah and Mei Mei must be chosen. \sol{}

                              Since 4 people must be chosen, there are 6 positions remaining.

                              Choosing 6 people from the rest of 14 people, there are $\comb[16]{6} = 8\,008$
                              ways to do so. $\eos$
                  \end{enumerate}

            \item During the National Mathematics Olympiad Competition 2021, all the 100
                  participants gather in a hall for a briefing. After the briefing all
                  participants are divided equally into 5 groups. All the participants in each
                  group are instructed to take their seats in their respective classrooms. Before
                  the competition begins, the participants in Group A shake hands with each
                  other. Find
                  \begin{enumerate}
                        \item the number of handshakes made between the participants in Group A, \sol{}

                              There are 20 participants in Group A.

                              Choose 2 participants from 20 participants to shake hands. There are
                              $\comb[20]{2} = 190$ hand shakes made. $\eos$

                        \item the number of handshakes made if Afiq, Ben and Cathy do not shake hands. \sol{}

                              Since these 3 participants do not shake hands, there are $\comb[3]{2} = 3$ hand
                              shakes that do not take place.

                              Hence, there are $190 - 3 = 187$ hand shakes made. $\eos$
                  \end{enumerate}
      \end{enumerate}
\end{multicols*}
\begin{multicols*}{2}
      \noindent\Large{\underline{\textbf{Praktis Summatif}}}
      \normalsize
      \setcounter{section}{0}
      \section{Kertas 1}
      \begin{enumerate}
            \item \begin{enumerate}
                        \item State the values of r if $\comb[7]{r} = 1$. \sol{}
                              \begin{flalign*}
                                    \comb[7]{r} & = 1                             \\
                                    \comb[7]{r} & = \comb[7]{0} = \comb[7]{7} = 1 \\
                                    r           & = 0,\ 7 \eos
                              \end{flalign*}

                        \item Express $s$ in terms of $t$ and $u$ if $\comb[s]{t} = \comb[s]{u}$ \sol{}
                              \begin{flalign*}
                                    \comb[s]{t} & = \comb[s]{u} \\
                                    s           & = t + u \eos
                              \end{flalign*}
                  \end{enumerate}
            \item \begin{enumerate}
                        \item Show that $\comb[n]{r} = \comb[n]{n-r}$ where $n$ and $r$ are positive integers
                              and $n > r$. \sol{}
                              \begin{flalign*}
                                    \comb[n]{n-r} & = \frac{n!}{(n-r)![n-(n-r)]!}          \\
                                                  & = \frac{n!}{(n-r)!r!}                  \\
                                                  & = \comb[n]{r} \quad (\text{shown})\eos
                              \end{flalign*}
                        \item Find the value of $r$ if $\dfrac{\permtwo[n]{r}}{\comb[n]{r}} = 120$. \sol{}
                              \begin{flalign*}
                                    \dfrac{\permtwo[n]{r}}{\comb[n]{r}}         & = 120    \\
                                    \frac{n!}{(n-r)!} \cdot \frac{(n-r)!r!}{n!} & = 120    \\
                                    r!                                          & = 120    \\
                                    r                                           & = 5 \eos
                              \end{flalign*}
                  \end{enumerate}

            \item A committee of 7 students is selected from 7 male students and 10 female
                  students find the number of ways the committee can be formed if
                  \begin{enumerate}
                        \item no condition is imposed, \sol{}

                              Choose 7 students from 17 students. There are $\comb[17]{7} = 19\,448$ ways to
                              do so. $\eos$

                        \item there are 3 male students and 4 female students. \sol{}

                              Choose 3 male students from 7 male students, there are $\comb[7]{3} = 35$ ways
                              to do so.

                              Choose 4 female students from 10 female students, there are $\comb[10]{4} =
                                    210$ ways to do so.

                              Hence, there are $35 \times 210 = 7\,350$ ways to form the committee. $\eos$

                        \item  The number of female students must be more than the number of male students.
                              \sol{}

                              According to the condition imposed, the team in which the number of female
                              students is less than or equal to the number of male students is not allowed.

                              Choose 5 male students and 5 female students, there are $\comb[7]{5} \times
                                    \comb[10]{5} = 5\,292$ ways to do so.

                              Choose 6 male students and 4 female students, there are $\comb[7]{6} \times
                                    \comb[10]{4} = 1\,470$ ways to do so.

                              Choose 7 male students and 3 female students, there are $\comb[7]{7} \times
                                    \comb[10]{3} = 120$ ways to do so.

                  \end{enumerate}
            \item During the school year end dinner, every round table must be seated with 10
                  people. The VIP table has only 6 seats. 6 teachers are selected from 5
                  mathematics teachers and 5 science teachers to fill in the vacant seats at the
                  VIP table. Find a number of ways to seat the teachers if
                  \begin{enumerate}
                        \item no condition is imposed, \sol{}

                              Choose 6 teachers from 10 teachers. There are $\comb[10]{6} = 210$ ways to do
                              so.

                              Arranging these 6 teachers in the VIP table, there are $(6 - 1)! = 120$ ways to
                              do so.

                              Hence, there are $210 \times 120 = 25\,200$ ways to seat the teachers. $\eos$

                        \item the mathematics teachers cannot sit next to each other.
                  \end{enumerate}
            \item Diagram below shows 8 cards that are labelled with letters.
                  \begin{enumerate}
                        \item Find the number of arrangements if
                              \begin{enumerate}
                                    \item all the letters are used without repetition,
                                    \item all the letters are used and the vowels must be side by side.
                              \end{enumerate}
                        \item Find the number of different ways to select 5 cards in which S and T needs to
                              be chosen.
                        \item Find the number of different 5-letter codes that begin with a vowel and end
                              with a consonant could be formed.
                  \end{enumerate}

            \item Agnes decorates her hat with 18 artificial flowers. She uses the same number
                  and the same type of artificial flowers, but of a smaller size to form a
                  bracelet as shown in the diagram below.

                  Given that the ratio of the number of roses to the number of morning glories to
                  the number of sunflowers on the hat is $3:2:1$.
                  \begin{enumerate}
                        \item Find the number of roses used.
                        \item If the morning glories have to be side by side on both the hat and bracelet
                              find the total number of ways to arrange the artificial flowers.
                  \end{enumerate}

            \item Diagram below shows 8 cards where 3 cards are labelled with letters and 4 cards
                  are labelled with digits.

                  Find the number of different arrangements that can be done if
                  \begin{enumerate}
                        \item no condition is imposed,
                        \item the arrangements begin with R and ends with an even digit.
                  \end{enumerate}

            \item 10 participants successfully enter the final round of a competition. The score is used to determine the champion, $1^{st}$ runner up, $2^{nd}$ runner up, and $3^{rd}$ runner up.
                  \begin{enumerate}
                        \item Find the number of different results that are possible to be obtained.
                        \item If Ben Hong and Haikal are two of the 10 participants, find the number of
                              different results obtained if
                              \begin{enumerate}
                                    \item neither Ben Hong nor Haikal wins the competition,
                                    \item Ben Hong and Haikal win the competition.
                              \end{enumerate}
                  \end{enumerate}

            \item Diagram below shows cards that are labelled with digits.

                  Find the number of 4-digit numbers that can be formed if the digit are used
                  without repetition. From the numbers formed, find the number of 4-digit numbers
                  that are
                  \begin{enumerate}
                        \item greater than $6\,000$,
                        \item odd numbers and greater than $6\,000$.
                  \end{enumerate}

            \item Diagram below shows the arrangement of tables in an exhibition room. A few
                  panels of partition board are arranged in the middle of the room to create a
                  one-way path.

                  After the visiting hour to the exhibition, the worker uses pieces of cloth to
                  cover the exhibition objects on each table. It is given that the worker brings
                  three pieces of red cloth, three pieces of green cloth, two pieces of blue
                  cloth and a piece of yellow cloth. Find the number of ways to cover the tables
                  with cloth if
                  \begin{enumerate}
                        \item the worker chooses the cloth at random,
                        \item the pieces of green cloth are used side by side,
                        \item the yellow cloth must be used to cover table A and the blue cloth cannot be
                              used to cover its adjacent table.
                  \end{enumerate}
      \end{enumerate}
\end{multicols*}

\end{document}