\documentclass{report}
\usepackage[a4paper, total={7.5in, 10in}]{geometry}
\usepackage[fleqn]{amsmath}
\usepackage{amssymb}
\usepackage{amsthm}
\usepackage{enumitem}
\usepackage[]{mdframed}
\usepackage{multicol}
\usepackage{thmtools}
\usepackage{graphicx}
\usepackage{tikz}
\usepackage{tipa}
\usepackage{array, makecell, cellspace}
\usepackage{bigints}
\usepackage[export]{adjustbox}
\setlength{\cellspacetoplimit}{13.2ex}
\setlength{\cellspacebottomlimit}{13.2ex}

\usepackage{ifxetex}

\ifxetex
      \usepackage{substitutefont}
      \substitutefont{T3}{\rmdefault}{cmr}
\fi

\usepackage{fontspec}
\setmainfont[Mapping=tex-text]{Georgia}

\title{Praktis 3\\Integration}
\author{Melvin Chia}

\newcommand{\sol}[1]{

      \noindent \textbf{Sol.}
}
\newcommand{\prooff}[1]{

      \noindent \textbf{Proof.}
}

\newcommand{\arc}[1]{{%
                  \setbox9=\hbox{#1}%
                  \ooalign{\resizebox{\wd9}{\height}{\texttoptiebar{\phantom{A}}}\cr#1}}}

\def\eos{\quad\hbox{\rlap{\hbox{\vrule depth 1.5pt height 2.6mm width 0.2mm \hskip 1mm \vrule height 2.6mm width 0.2mm}}{\vbox{\hrule height 0.2mm width 1.4mm \vskip 2.8mm \hrule depth 1.5pt height -0.35mm width 1.2mm}}}}

\counterwithout{equation}{chapter}
\setlength{\columnseprule}{1pt}
\setlength{\columnsep}{24pt}
\hfuzz=100pt
\setcounter{chapter}{3}

\begin{document}
\maketitle

\begin{multicols*}{2}
      \noindent\Large{\underline{\textbf{Praktis Formatif}}}
      \normalsize
      \section{Integration as the Inverse of Differentiation}
      \begin{enumerate}
            \item \begin{enumerate}
                        \item Given ${\dfrac{d}{d x}}(2x^{3}+5x^{2}-7x)=6x^{2}+ 10x - 7$, find $\displaystyle
                                    \int6x^{2}+10x-7\ d x$. \sol{}
                              \begin{flalign*}
                                    \int6x^{2}+10x-7\ dx & = 2x^{3}+5x^{2}-7x \eos
                              \end{flalign*}

                        \item Given ${\dfrac{d}{d x}}(5x^{4}+3x^{2}+x)=20x^{3}+6x+1$, find $\displaystyle
                                    \int20x^{3}+6x+1\ dx$. \sol{}
                              \begin{flalign*}
                                    \int20x^{3}+6x+1\ dx & = 5x^{4}+3x^{2}+x \eos
                              \end{flalign*}
                  \end{enumerate}

            \item \begin{enumerate}
                        \item Given ${\dfrac{d}{d x}}(4x-5x^{2}+2x^{3})=4-10x+6x^{2}$, find $\displaystyle
                                    \int2-5x+3x^{2}\ dx$. \sol{}
                              \begin{flalign*}
                                    \int2-5x+3x^{2}\ dx & = \frac{2}{2}\int2-5x+3x^{2}\ dx    \\
                                                        & = \frac{1}{2} \int 4-10x+6x^{2}\ dx \\
                                                        & = \frac{1}{2}(4x-5x^{2}+2x^{3})     \\
                                                        & = 2x-\frac{5}{2}x^{2}+x^{3} \eos
                              \end{flalign*}

                        \item Given $\dfrac{d}{d x}\Big(2x-{\dfrac{3}{x^{4}}}\Big)=2+{\dfrac{12}{x^{5}}}$,
                              find $\displaystyle \int 6+{\dfrac{36}{x^{5}}}\ dx$ \sol{}
                              \begin{flalign*}
                                    \int 6+{\dfrac{36}{x^{5}}}\ dx & = 6\int 1+{\dfrac{6}{x^{5}}}\ dx       \\
                                                                   & = 3\int 2+{\dfrac{12}{x^{5}}}\ dx      \\
                                                                   & 3\left(2x-{\dfrac{3}{x^{4}}}   \right) \\
                                                                   & = 6x-{\dfrac{9}{x^{4}}} \eos
                              \end{flalign*}

                        \item Given $f(x)={\dfrac{d}{d x}}[g(x)]$, find $\displaystyle \int 2f(x)\,d x$.
                              \sol{}
                              \begin{flalign*}
                                    \int 2f(x)\,dx & = 2\int f(x)\,dx \\
                                                   & = 2g(x) \eos
                              \end{flalign*}

                        \item Differentiate $\dfrac{2x^{2}}{3x-1}$ with respect to $x$ and hence, find
                              $\displaystyle \int{\dfrac{6x(3x-2)}{{(3x-1)}^{2}}}\,d x$. \sol{}
                              \begin{flalign*}
                                    \frac{d}{dx}\left(\dfrac{2x^{2}}{3x-1}\right) & = \frac{4x(3x-1) - 3(2x^2)}{{(3x-1)}^2}  \\
                                                                                  & = \frac{12x^2 - 4x - 6x^2}{{(3x-1)}^2}   \\
                                                                                  & = \frac{6x^2 - 4x}{{(3x-1)}^2}           \\
                                                                                  & = \frac{2x(3x-2)}{{(3x-1)}^2}            \\
                                    \int{\dfrac{6x(3x-2)}{{(3x-1)}^{2}}}\,d x     & = 3\int \frac{2x(3x-2)}{{(3x-1)}^2}\,d x \\
                                                                                  & = 3\left(\dfrac{2x^2}{3x-1}\right)       \\
                                                                                  & = \dfrac{6x^2}{3x-1} \eos
                              \end{flalign*}
                  \end{enumerate}

            \item The daily production of bread of a bakery shop is given b y the function $R(x)
                        = -50(x^2 - 12x)$, where $x$ represents the number of bakers who work in the
                  shop with condition $x$ is not more than 6.
                  \begin{enumerate}
                        \item Find the rate of daily production of bread in terms of $x$. \sol{}
                              \begin{flalign*}
                                    R'(x) & = -100x + 600 \eos
                              \end{flalign*}

                        \item If the rate of daily production of bread becomes $300 - 50x$ on a particular
                              day, calculate the revenue of the bakery shop if all the loaves of bread baked
                              by three bakers on that day are sold out at a price of RM$5.50$ for each loaf.
                              \sol{}
                              \begin{flalign*}
                                    \int 300 - 50x\ dx & = \frac{1}{2}\int (600 - 100x)\ dx \\
                                                       & = \frac{1}{2}(-50x^2 + 600x)       \\
                                                       & = -25x^2 + 300x                    \\
                                    R(3)               & = -25{(3)}^2 + 300(3)              \\
                                                       & = -225 + 900                       \\
                                                       & = 675
                              \end{flalign*}
                              \begin{flalign*}
                                    \text{Revenue} & = 675 \times 5.50       \\
                                                   & = \text{RM}3712.50 \eos
                              \end{flalign*}
                  \end{enumerate}

            \item Given $f(x) = x^4 - 2x^3$ and $f'(x) = 4x^3 - 6x^2$. Express
                  $f'(x)\displaystyle \int f'(x)\,dx$ in factored form. \sol{}
                  \begin{flalign*}
                        f'(x)\displaystyle \int f'(x)\,dx & = (4x^3 - 6x^2)(x^4 - 2x^3) \\
                                                          & = 2x^5(2x - 3)(x - 2) \eos
                  \end{flalign*}

            \item Given $y=\dfrac{2x-6}{x}$.
                  \begin{enumerate}
                        \item Find $\dfrac{dy}{dx}$. \sol{}
                              \begin{flalign*}
                                    \frac{dy}{dx} & = \frac{2x - 2x-6}{x^2} \\
                                                  & = -\frac{6}{x^2} \eos
                              \end{flalign*}

                        \item Solve $4+\displaystyle \int\left(\dfrac{dy}{dx}\right)\,dx=0$. \sol{}
                              \begin{flalign*}
                                    4+\displaystyle \int\left(\dfrac{dy}{dx}\right)\,dx=0 \\
                                    4+\displaystyle \int\left(-\frac{6}{x^2}\right)\,dx=0 \\
                                    4 + \frac{2x-6}{x} & = 0                              \\
                                    4x + 2x - 6        & = 0                              \\
                                    6x                 & = 6                              \\
                                    x                  & = 1 \eos
                              \end{flalign*}
                  \end{enumerate}

            \item Given $f'(x) = g(x)$. Find ${\dfrac{3f(x)}{\displaystyle \int g(x)dx}}$. \sol{}
                  \begin{flalign*}
                        f'(x)                                      & = g(x)                      \\
                        f(x)                                       & = \displaystyle \int g(x)dx \\
                        {\dfrac{3f(x)}{\displaystyle \int g(x)dx}} & = \frac{3f(x)}{f(x)}        \\
                                                                   & = 3 \eos
                  \end{flalign*}

            \item The population of town $A$ is given by a function $P(t) =
                        \dfrac{5}{6}(2.72^{1.2t}) - t^2 + 1495$ and the population continues to
                  increase at the rate of $2.72^{1.2t} - 2t$ people per year where $t$ is the
                  number of years. Given that the population of town $b$ increases at twice the
                  rate of the population of town $A$ based on the same model, find, to the
                  nearest integer,
                  \begin{enumerate}
                        \item the rate of increase of the population of town $B$ at $t=5$ years. \sol{}
                              \begin{flalign*}
                                    P_B'(5) & = 2[2.72^{1.2(5)} - 2(5)]          \\
                                            & = 2[404.96 - 10]                   \\
                                            & = 2(394.96)                        \\
                                            & = 789.92                           \\
                                            & = 790 \text{ people per year} \eos
                              \end{flalign*}

                        \item the population of town $B$ after 5 years. \sol{}
                              \begin{flalign*}
                                    P_B(5) & = 2\left[\dfrac{5}{6}(2.72^{1.2\cdot 5}) - {(5)}^2 + 1495\right] \\
                                           & = \frac{5}{3}(2.72^{6}) - 50 + 2990                              \\
                                           & = 3614.93                                                        \\
                                           & = 3615 \text{ people} \eos
                              \end{flalign*}
                  \end{enumerate}
      \end{enumerate}

      \section{Indefinite Integral}
      \begin{enumerate}
            \setcounter{enumi}{7}

            \item By using the indefinite integral formula, find the integral of each of the
                  following constants or algebraic functions.
                  \begin{enumerate}
                        \item $\displaystyle \int 3\,dx$
                              \sol{}
                              \begin{flalign*}
                                    \int 3\,dx & = 3x + C \eos
                              \end{flalign*}

                        \item $\displaystyle \int 24x\,dx$
                              \sol{}
                              \begin{flalign*}
                                    \int 24x\,dx & = 12x^2 + C \eos
                              \end{flalign*}

                        \item $\displaystyle \int 6x^{2}\,dx$
                              \sol{}
                              \begin{flalign*}
                                    \int 6x^{2}\,dx & = 2x^3 + C \eos
                              \end{flalign*}

                        \item $\displaystyle \int 3x^2 + 4x\,dx$
                              \sol{}
                              \begin{flalign*}
                                    \int 3x^2 + 4x\,dx & = x^3 + 2x^2 + C \eos
                              \end{flalign*}

                        \item $\displaystyle \int \dfrac{2}{x^4}\,dx$
                              \sol{}
                              \begin{flalign*}
                                    \int \dfrac{2}{x^4}\,dx & = -\dfrac{2}{x^3} + C \eos
                              \end{flalign*}

                        \item $\displaystyle \int x^2(x-3)\,dx$
                              \sol{}
                              \begin{flalign*}
                                    \int x^2(x-3)\,dx & = \int x^3-3x^2\,dx             \\
                                                      & = \frac{1}{4}x^4 - x^3 + C \eos
                              \end{flalign*}

                        \item $\displaystyle \int (x+2)(2x^4 - 1)\,dx$
                              \sol{}
                              \begin{flalign*}
                                     & \int (x+2)(2x^4 - 1)\,dx                                           \\
                                     & = \int 2x^5 - x + 4x^4 - 2                                       & \\
                                     & = \frac{1}{3}x^6 + \frac{4}{5}x^5 - \frac{1}{2}x^2 - 2x + C \eos
                              \end{flalign*}

                        \item $\displaystyle \int \dfrac{x^2 + 3x + 2}{x+2}\,dx$
                              \sol{}
                              \begin{flalign*}
                                    \displaystyle \int \dfrac{x^2 + 3x + 2}{x+2}\,dx & = \displaystyle \int \dfrac{(x+2)(x+1)}{x+2}\,dx \\
                                                                                     & = \displaystyle \int x+1\,dx                     \\
                                                                                     & = \frac{1}{2}x^2 + x + C \eos
                              \end{flalign*}
                  \end{enumerate}

            \item Find the indefinite integral for each of the following by using
                  \begin{enumerate}
                        \item the substitution method.
                        \item the indefinite integral formula.
                  \end{enumerate}
                  \begin{enumerate}[label=\roman*.]
                        \item $\displaystyle \int{\dfrac{2}{{(x+2)}^{5}}}\,d x$
                              \sol{}
                              \begin{enumerate}[label=(\alph*)]
                                    \item Let $v = {(x+2)}$.
                                          \begin{flalign*}
                                                \int{\dfrac{2}{{(x+2)}^{5}}}\,d x & = \int{\dfrac{2}{v^5}}\,dv       \\
                                                                                  & = \int 2v^{-5}\,dv               \\
                                                                                  & = -\frac{1}{2}v^{-4} + C         \\
                                                                                  & = -\frac{1}{2v^4} + C            \\
                                                                                  & = -\frac{1}{2{(x+2)}^4} + C \eos
                                          \end{flalign*}

                                    \item \begin{flalign*}
                                                \int{\dfrac{2}{{(x+2)}^{5}}}\,d x & = \int2{(x+2)}^{-5}\,dx                     \\
                                                                                  & = 2\int{(x+2)}^{-5}\,dx                     \\
                                                                                  & = 2\left[\frac{{(x+2)}^{-4}}{-4}\right] + C \\
                                                                                  & = -\frac{1}{2{(x+2)}^4} + C \eos
                                          \end{flalign*}
                              \end{enumerate}

                        \item $\displaystyle \int{\dfrac{3}{5}{(3x+2)}^8}\,d x$
                              \sol{}
                              \begin{enumerate}[label=(\alph*)]
                                    \item Let $v = 3x+2$, $\dfrac{dv}{dx} = 3$.
                                          \begin{flalign*}
                                                \int{\dfrac{3}{5}{(3x+2)}^8}\,d x & = \int{\dfrac{3}{5}v^8}\,dv         \\
                                                                                  & = \int{\dfrac{3}{5}v^8}\frac{dv}{3} \\
                                                                                  & = \int{\dfrac{1}{5}v^8}\,dv         \\
                                                                                  & = \dfrac{1}{45}v^9 + C              \\
                                                                                  & = \dfrac{(3x+2)^9}{45} + C \eos
                                          \end{flalign*}

                                    \item \begin{flalign*}
                                                \int{\dfrac{3}{5}{(3x+2)}^8}\,d x & = \dfrac{3}{5}\int{(3x+2)^8}\,dx                     \\
                                                                                  & = \dfrac{3}{5}\left[\frac{{(3x+2)}^9}{27}\right] + C \\
                                                                                  & = \dfrac{{(3x+2)}^9}{45} + C \eos
                                          \end{flalign*}
                              \end{enumerate}
                  \end{enumerate}

            \item Determine the equation of a curve based on the following information.
                  \begin{enumerate}
                        \item The gradient function of the curve is $\dfrac{dy}{dx} = 3x^2 + x - 2$ and it
                              passes through the point $p(2, 15)$. \sol{}
                              \begin{flalign*}
                                    \dfrac{dy}{dx} & = 3x^2 + x - 2                 \\
                                    y              & = \int 3x^2 + x - 2\,dx        \\
                                                   & = x^3 + \frac{x^2}{2} - 2x + C \\
                              \end{flalign*}
                              When $x = 2$, $y = 15$,
                              \begin{flalign*}
                                    15 & = 2^3 + \frac{2^2}{2} - 2(2) + C \\
                                    15 & = 8 + 2 - 4 + C                  \\
                                    15 & = 6 + C                          \\
                                    C  & = 9
                              \end{flalign*}
                              Hence, the equation of the curve is $y = x^3 + \dfrac{x^2}{2} - 2x + 9$. $\eos$

                        \item The gradient function of the curve is $f'(x) = 2x+9$ and $f(3) = 21$. \sol{}
                              \begin{flalign*}
                                    f'(x) & = 2x+9           \\
                                    f(x)  & = \int 2x+9\,dx  \\
                                          & = x^2 + 9x + C   \\
                                    f(3)  & = 3^2 + 9(3) + C \\
                                    21    & = 9 + 27 + C     \\
                                    C     & = -15
                              \end{flalign*}
                              Hence, the equation of the curve is $f(x) = x^2 + 9x - 15$. $\eos$

                        \item The gradient function of the curve is given by $g(t) = \dfrac{5t^2 - 6t +
                                          1}{t^3(t-1)}$ and it passes through the point $(1, 3)$. \sol{}
                              \begin{flalign*}
                                    g(t) & = \dfrac{5t^2 - 6t + 1}{t^3(t-1)}   \\
                                         & = \dfrac{(5t-1)(t-1)}{t^3(t-1)}     \\
                                         & = \dfrac{5t-1}{t^3}                 \\
                                         & = \dfrac{5}{t^2} - \dfrac{1}{t^3}   \\
                                         & = 5t^{-2} - t^{-3}                  \\
                                    f(t) & = \int 5t^{-2} - t^{-3}\,dt         \\
                                         & = -\frac{5}{t} + \frac{1}{2t^2} + C \\
                              \end{flalign*}
                              When $t = 1$, $f(1) = 3$,
                              \begin{flalign*}
                                    3 & = -5 + \frac{1}{2} + C \\
                                    3 & = -\frac{9}{2} + C     \\
                                    C & = \frac{15}{2}
                              \end{flalign*}
                              Hence, the equation of the curve is $f(t) = -\dfrac{5}{t} + \dfrac{1}{2t^2} + \dfrac{15}{2}$. $\eos$
                  \end{enumerate}

            \item Tommy moves in his roller skates at the rate of change in displacement,
                  $\dfrac{ds}{dt} = t^2 + 9$ metres per second, where $t$ is the time in seconds.
                  At $t = 3$ seconds, Tommy is 4 metres away from his starting place. Find the
                  displacement, $s$ metres, when $t = 10$ seconds. \sol{}
                  \begin{flalign*}
                        \frac{ds}{dt} & = t^2 + 9                \\
                        s             & = \int t^2 + 9\,dt       \\
                                      & = \frac{t^3}{3} + 9t + C
                  \end{flalign*}
                  When $t = 3$, $s = 4$,
                  \begin{flalign*}
                        4 & = \frac{3^3}{3} + 9(3) + C \\
                        4 & = 9 + 27 + C               \\
                        4 & = 54 + C                   \\
                        C & = -32                      \\
                        s & = \frac{t^3}{3} + 9t - 32
                  \end{flalign*}
                  When $t = 10$,
                  \begin{flalign*}
                        s & = \frac{10^3}{3} + 9(10) - 32   \\
                          & = 333 + 90 - 32                 \\
                          & = 391\frac{1}{3}\textit{m} \eos
                  \end{flalign*}

            \item Given the gradient function of a curve is $\dfrac{dy}{dx} = kx^2 + 2x$ where
                  $k$ is a constant. The curve passes through point $A(1, 6)$ and point $B(-2,
                        0)$. Determine the equation of the curve. \sol{}
                  \begin{flalign*}
                        \dfrac{dy}{dx} & = kx^2 + 2x                \\
                        y              & = \int kx^2 + 2x\,dx       \\
                                       & = \frac{kx^3}{3} + x^2 + C
                  \end{flalign*}
                  When $x = 1$, $y = 6$,
                  \begin{flalign*}
                        6      & = \frac{k}{3} + 1 + C           \\
                        k + 3C & = 15                  \quad (1)
                  \end{flalign*}
                  When $x = -2$, $y = 0$, \begin{flalign*}
                        0       & = -\frac{8k}{3} + 4 + C          \\
                        8k - 3C & = 12                   \quad (2)
                  \end{flalign*}
                  \begin{flalign*}
                        (2) + (1) \cdot 8:\ 9k & = 27 \\
                        k                      & = 3  \\
                        C                      & = 4
                  \end{flalign*}
                  Hence, the equation of the curve is $y = x^3 + x^2 + 4$ $\eos$
      \end{enumerate}

      \section{Definite Integral}
      \begin{enumerate}
            \setcounter{enumi}{12}
            \item Calculate each of the following.
                  \begin{enumerate}
                        \item $\displaystyle \int_{2}^{1}{\left(\sqrt{x} + \dfrac{1}{x}\right)}$
                              \sol{}
                              \begin{flalign*}
                                    \int_{2}^{1}{\left(\sqrt{x} + \dfrac{1}{\sqrt{x}}\right)} & = \int_{1}^{2}{\left(x^{\frac{1}{2}} + x^{-\frac{1}{2}}\right)}               & \\
                                                                                              & = {\left[\frac{2\sqrt{x^3}}{3} + 2\sqrt{x}\right]}_1^2                          \\
                                                                                              & = \left[\frac{4\sqrt{2}}{3} + 2\sqrt{2}\right] - \left[\frac{2}{3} + 2\right]   \\
                                                                                              & = \frac{10\sqrt{2}}{3} - \frac{8}{3}                                            \\
                                                                                              & = \frac{10\sqrt{2} - 8}{3}                                                      \\
                                                                                              & \approx 2.0474 \eos
                              \end{flalign*}

                        \item $\displaystyle \int_{0}^{3}\left(\dfrac{x^4 + 3x}{x}\right)\,dx$
                              \sol{}
                              \begin{flalign*}
                                    \int_{0}^{3}\left(\dfrac{x^4 + 3x}{x}\right)\,dx & = \int_{0}^{3}\left(x^3 + 3\right)\,dx        \\
                                                                                     & = {\left[\frac{1}{4}x^4 + 3x\right]}_0^3      \\
                                                                                     & = \left[\frac{1}{4}(3^4) + 3\cdot3\right] - 0 \\
                                                                                     & = \frac{81}{4} + 9                            \\
                                                                                     & = \frac{117}{4}                               \\
                                                                                     & =  29.25 \eos
                              \end{flalign*}

                        \item $\displaystyle \int_{-2}^{-1}\left(\dfrac{(4-x)(3-x)}{x^5}\right)\,dx$
                              \sol{}
                              \begin{flalign*}
                                     & \int_{-2}^{-1}\left(\dfrac{(4-x)(3-x)}{x^5}\right)\,dx                                                  \\
                                     & = \int_{-2}^{-1}\left(\dfrac{x^2 - 7x + 12}{x^5}\right)\,dx                                             \\
                                     & = \int_{-2}^{-1}\left(\frac{1}{x^3} - \frac{7}{x^4} + \frac{12}{x^5}\right)\,dx                         \\
                                     & = {\left[-\frac{1}{2x^2} + \frac{7}{3x^3} -\frac{3}{x^4}\right]}_{-2}^{-1}                              \\
                                     & = \left[-\frac{1}{2} - \frac{7}{3} - 3\right] - \left[-\frac{1}{8} - \frac{7}{24} - \frac{3}{16}\right] \\
                                     & = -\frac{35}{6} + \frac{29}{48}                                                                         \\
                                     & = -5\frac{11}{48} \eos
                              \end{flalign*}
                  \end{enumerate}

            \item Given $\displaystyle \int_{a}^{b}f(x)\,dx = 5$, $\displaystyle
                        \int_{b}^{c}f(x)\,dx = 8$ and $\displaystyle \int_{a}^{b}g(x)\,dx = 2$. Find
                  each of the following.

                        [answer can be in terms of $a$ and/or $b$.]
                  \begin{enumerate}
                        \item $\displaystyle \int_{a}^{b}3f(x)\,dx$
                              \sol{}
                              \begin{flalign*}
                                    \int_{a}^{b}3f(x)\,dx & = 3\int_a^b f(x)\,dx \\
                                                          & = 3\cdot 5           \\
                                                          & = 15 \eos
                              \end{flalign*}

                        \item $\displaystyle \int_{a}^{c}f(x)\,dx$
                              \sol{}
                              \begin{flalign*}
                                    \int_{a}^{c}f(x)\,dx & = \int_{a}^{b}f(x)\,dx + \int_{b}^{c}f(x)\,dx \\
                                                         & = 5 + 8                                       \\
                                                         & = 13 \eos
                              \end{flalign*}

                        \item $\displaystyle \int_{a}^{b}[f(x) + g(x)]\,dx$
                              \sol{}
                              \begin{flalign*}
                                    \int_{a}^{b}[f(x) + g(x)]\,dx & = \int_{a}^{b}f(x)\,dx - \int_{a}^{b}g(x)\,dx & \\
                                                                  & = 5 - 2                                         \\
                                                                  & = 3 \eos
                              \end{flalign*}

                        \item $\displaystyle \int_{c}^{a}f(x)\,dx$
                              \sol{}
                              \begin{flalign*}
                                    \int_{c}^{a}f(x)\,dx & = -\int_{a}^{c}f(x)\,dx \\
                                                         & = -13 \eos
                              \end{flalign*}

                        \item $\displaystyle \int_{a}^{b}[g(x) + 3]\,dx$
                              \sol{}
                              \begin{flalign*}
                                    \int_{a}^{b}[g(x) + 3]\,dx & = \int_{a}^{b}g(x)\,dx + \int_{a}^{b}3\,dx \\
                                                               & = 2 + 3(b - a)                             \\
                                                               & = 3b - 3a + 2 \eos
                              \end{flalign*}

                        \item $\displaystyle \int_{a}^{a}f(x)\,dx$
                              \sol{}
                              \begin{flalign*}
                                    \int_{a}^{a}f(x)\,dx & = 0 \eos
                              \end{flalign*}

                        \item The value of $k$ such that $\displaystyle \int_{b}^{a}[f(x) + kx]\,dx = 25$ if
                              $a = 1$ and $b = 4$. \sol{}
                              \begin{flalign*}
                                    \int_{b}^{a}[f(x) + kx]\,dx               & = \int_{b}^{a}f(x)\,dx + \int_{b}^{a}kx\,dx \\
                                                                              & = -5 + \int_{b}^{a}kx\,dx                   \\
                                    -5 + \int_{1}^4 kx\,dx                    & = 25                                        \\
                                    \int_{4}^1 kx\,dx                         & = 30                                        \\
                                    k{\left[\frac{x^2}{2}\right]}_{4}^1       & = 30                                        \\
                                    k\left(\frac{1}{2} - \frac{16}{2} \right) & = 30                                        \\
                                    -\frac{15k}{2}                            & = 30                                        \\
                                    -15k                                      & = 60                                        \\
                                    k                                         & = -4 \eos
                              \end{flalign*}
                  \end{enumerate}

            \item Find the area of the shaded region for each of the following diagrams.
                  \begin{enumerate}
                        \item \includegraphics[width=0.3\textwidth,valign=t]{./images/1.png}
                              \sol{}
                              \begin{flalign*}
                                    A & = \int_2^4 (x^3 - 4x^2 + x + 10)\,dx                                                  \\
                                      & = \left[\frac{1}{4}x^4 - \frac{4}{3}x^3 + \frac{1}{2}x^2 + 10x\right]_2^4           & \\
                                      & = \left(64 - \frac{256}{3} + 8 + 40\right) - \left(4 - \frac{32}{3} + 2 + 20\right)   \\
                                      & = \frac{80}{3} - \frac{46}{3}                                                         \\
                                      & = 11\frac{1}{3}\text{ units}^2 \eos
                              \end{flalign*}

                        \item \includegraphics[width=0.3\textwidth,valign=t]{./images/2.png}
                              \sol{}
                              \begin{flalign*}
                                    A & = \int_{-2}^{5}\sqrt{3x+ 10}\,dx                                   \\
                                      & = \int_{-2}^{5}{(3x+10)}^{\frac{1}{2}}\,dx                         \\
                                      & = {\left[\frac{{2(3x + 10)}^{\frac{3}{2}}}{9}\right]}_{-2}^{5}     \\
                                      & = \frac{2{(25)}^{\frac{3}{2}}}{9} - \frac{2{(4)}^{\frac{3}{2}}}{9} \\
                                      & = \frac{250}{9} - \frac{16}{9}                                     \\
                                      & = \frac{234}{9}                                                    \\
                                      & = 26\text{ units}^2 \eos
                              \end{flalign*}

                        \item \includegraphics[width=0.3\textwidth,valign=t]{./images/3.png}
                              \sol{}
                              \begin{flalign*}
                                    A & = \left|\int_{3}^{6} (x-3)(x-6)\,dx                                        \right|   & \\
                                      & = \left|\int_{3}^{6} (x^2 - 9x + 18)\,dx                                     \right|   \\
                                      & = \left|{\left[\frac{1}{3}x^3 - \frac{9}{2}x^2 + 18x\right]}_3^6         \right|       \\
                                      & = \left|\left(72 - 162 + 108\right) - \left(9 - \frac{81}{2} + 54\right) \right|       \\
                                      & = \left|18 - \frac{45}{2}                                                \right|       \\
                                      & = 4.5\text{ units}^2 \eos
                              \end{flalign*}

                        \item \includegraphics[width=0.3\textwidth,valign=t]{./images/4.png}
                              \sol{}

                              When $y = 5$,
                              \begin{flalign*}
                                    x(x-4)       & = 5               \\
                                    x^2 - 4x - 5 & = 0               \\
                                    (x-5)(x+1)   & = 0               \\
                                    x = -1       & \text{ or } x = 5
                              \end{flalign*}
                              When $y = 0$,
                              \begin{flalign*}
                                    x(x-4) & = 0               \\
                                    x = 0  & \text{ or } x = 4
                              \end{flalign*}
                              \begin{flalign*}
                                    A & = \int_{-1}^{0} x(x-4)\,dx + \left|\int_{0}^{3} x(x-4)\,dx\right|                                           \\
                                      & \ \ \ \ + \int_{4}^{5} x(x-4)\,dx                                                                         & \\
                                      & = \int_{-1}^{0} (x^2 - 4x)\,dx + \left|\int_{0}^{3} (x^2 - 4x)\,dx\right|                                   \\
                                      & \ \ \ \ + \int_{4}^{5} (x^2 - 4x)\,dx                                                                       \\
                                      & = {\left[\frac{1}{3}x^3 - 2x^2\right]}_{-1}^0 + \left|{\left[\frac{1}{3}x^3 - 2x^2\right]}_{0}^{3}\right|   \\
                                      & \ \ \ \ + {\left[\frac{1}{3}x^3 - 2x^2\right]}_{4}^5                                                        \\
                                      & = 0 - \left(-\frac{1}{3} - 2\right) + \left|\left(9 - 18\right) - 0\right|                                  \\
                                      & \ \ \ \ + \left(\frac{125}{3} - 50\right) - \left(\frac{64}{3} - 32\right)                                  \\
                                      & = 13\frac{2}{3}\text{ units}^2 \eos
                              \end{flalign*}
                  \end{enumerate}

            \item Determine the area bounded by the curve, the horizontal line(s) and the y-axis.
                  \begin{enumerate}
                        \item \includegraphics[width=0.3\textwidth,valign=t]{./images/5.png}
                              \sol{}
                              When $x = 0$,
                              \begin{flalign*}
                                    y  & = 5-2x          \\
                                    y  & = 5-2(0)        \\
                                       & = 5             \\
                                    \\
                                    y  & = 5-2x          \\
                                    2x & = 5 - y         \\
                                    x  & = \frac{5-y}{2}
                              \end{flalign*}
                              \begin{flalign*}
                                    A & = \int_1^5 \frac{5-y}{2}\,dy                                                        \\
                                      & =\int_1^5 \left(\frac{5}{2} - \frac{1}{2}y\right)\,dy                               \\
                                      & = {\left[\frac{5}{2}y - \frac{1}{4}y^2\right]}_1^5                                  \\
                                      & = \left(\frac{25}{2} - \frac{25}{4}\right) - \left(\frac{5}{2} - \frac{1}{4}\right) \\
                                      & = \frac{25}{4} - \frac{9}{4}                                                        \\
                                      & = 4\text{ units}^2 \eos
                              \end{flalign*}

                        \item \includegraphics[width=0.3\textwidth,valign=t]{./images/6.png}
                              \sol{}
                              \begin{flalign*}
                                    {(y+2)}^2 & = x + 4            \\
                                    x         & = {(y+2)}^2 - 4    \\
                                              & = y^2 + 4y + 4 - 4 \\
                                              & = y^2 + 4y
                              \end{flalign*}
                              \begin{flalign*}
                                    A & = \int_{-2}^{0}(y^2 + 4y)\,dy                 \\
                                      & = {\left[\frac{1}{3}y^3 + 2y^2\right]}_{-2}^0 \\
                                      & = 0 - \left(-\frac{8}{3} + 8\right)           \\
                                      & = 5\frac{1}{3}\text{ units}^2 \eos
                              \end{flalign*}

                        \item \includegraphics[width=0.3\textwidth,valign=t]{./images/7.png}
                              \sol{}
                              \begin{flalign*}
                                    A & = \int_{-1}^{1} (y^2 + 3)\,dy                                  \\
                                      & = {\left[\frac{1}{3}y^3 + 3y\right]}_{-1}^1                    \\
                                      & = \left(\frac{1}{3} + 3\right) - \left(-\frac{1}{3} - 3\right) \\
                                      & = 6\frac{2}{3}\text{ units}^2 \eos
                              \end{flalign*}

                        \item \includegraphics[width=0.3\textwidth,valign=t]{./images/8.png}
                              \sol{}
                              \begin{flalign*}
                                    A & = \int_{-1}^{0}(2y^3 - y^2 - 6y)\,dy + \left|\int_{0}^{1}(2y^3 - y^2 - 6y)\,dy\right|                                                     & \\
                                      & = {\left[\frac{1}{2}y^4 - \frac{1}{3}y^3 - 3y^2\right]}_{-1}^0 + \left|\left[\frac{1}{2}y^4 - \frac{1}{3}y^3 - 3y^2\right]_{0}^{1}\right|   \\
                                      & = 0 - \left(\frac{1}{2} + \frac{1}{3} - 3\right) + \left|\left(\frac{1}{2} - \frac{1}{3} - 3\right) - 0\right|                              \\
                                      & = \frac{13}{6} + \frac{17}{6}                                                                                                               \\
                                      & = 5\text{ units}^2 \eos
                              \end{flalign*}
                  \end{enumerate}

            \item Find the area of the shaded region for each of the following.
                  \begin{enumerate}
                        \item \includegraphics[width=0.3\textwidth,valign=t]{./images/9.png}
                              \sol{}
                              \begin{flalign*}
                                    x        & = x(4-x)          \\
                                    x        & = 4x - x^2        \\
                                    x^2 - 3x & = 0               \\
                                    x(x-3)   & = 0               \\
                                    x = 0    & \text{ or } x = 3
                              \end{flalign*}
                              \begin{flalign*}
                                    A & = \int_{0}^{3} \left[x(4-x) - x\right]\,dx           \\
                                      & = \int_{0}^{3} \left[4x - x^2 - x\right]\,dx         \\
                                      & = \int_{0}^{3} \left[3x - x^2\right]\,dx             \\
                                      & = {\left[\frac{3}{2}x^2 - \frac{1}{3}x^3\right]}_0^3 \\
                                      & = \left(\frac{27}{2} - 9\right) - 0                  \\
                                      & = 4.5\text{ units}^2 \eos
                              \end{flalign*}

                        \item \includegraphics[width=0.3\textwidth,valign=t]{./images/10.png}
                              \sol{}
                              \begin{flalign*}
                                    2x + 1   & = x^2 - x + 1     \\
                                    x^2 - 3x & = 0               \\
                                    x(x-3)   & = 0               \\
                                    x = 0    & \text{ or } x = 3
                              \end{flalign*}
                              \begin{flalign*}
                                    A & = \int_0^3 \left[2x + 1 - x^2 + x - 1\right]\,dx      \\
                                      & = \int_0^3 \left[-x^2 + 3x\right]\,dx                 \\
                                      & = {\left[-\frac{1}{3}x^3 + \frac{3}{2}x^2\right]}_0^3 \\
                                      & = \left(-9 + \frac{27}{2}\right) - 0                  \\
                                      & = 4.5\text{ units}^2 \eos
                              \end{flalign*}

                        \item \includegraphics[width=0.3\textwidth,valign=t]{./images/11.png}
                              \sol{}
                              \begin{flalign*}
                                    2y              & = x                 \\
                                    y               & = \frac{1}{2}x      \\
                                    \frac{1}{2}x    & = x^2 - 6x + 9      \\
                                    x               & = 2x^2 - 12x + 18   \\
                                    2x^2 - 13x + 18 & = 0                 \\
                                    (2x - 9)(x - 2) & = 0                 \\
                                    x = 2           & \text{ or } x = 4.5 \\
                                    x^2 - 6x + 9    & = 0                 \\
                                    {(x-3)}^2       & = 0                 \\
                                    x               & = 3
                              \end{flalign*}
                              \begin{flalign*}
                                    A & = \int_{0}^{2} \frac{x}{2}\,dx + \int_{2}^{3} (x^2 - 6x + 9)\,dx                    \\
                                      & = {\left[\frac{1}{4}x^2\right]}_0^2 + {\left[\frac{1}{3}x^3 - 3x^2 + 9x\right]}_2^3 \\
                                      & = 1 - 0 + \left(9 - 27 + 27\right) - \left(\frac{8}{3} - 12 + 18\right)             \\
                                      & = 10 - \frac{26}{3}                                                                 \\
                                      & = 1\frac{1}{3}\text{ units}^2 \eos
                              \end{flalign*}

                        \item \includegraphics[width=0.3\textwidth,valign=t]{./images/12.png}
                              \sol{}
                              \begin{flalign*}
                                    \frac{3}{x^3} + 1 & =4  \\
                                    \frac{3}{x^3}     & = 3 \\
                                    x^3               & = 1 \\
                                    x                 & = 1
                              \end{flalign*}
                              \begin{flalign*}
                                    A & = 1 \cdot 4 + \int_1^6 \left(\frac{3}{x^3} + 1\right)\,dx            \\
                                      & = 4 + {\left[-\frac{3}{2x^2} + x\right]}_1^6                         \\
                                      & = 4 + \left(-\frac{1}{24} + 6\right) - \left(-\frac{3}{2} + 6\right) \\
                                      & = 4 + \frac{143}{24} + \frac{9}{2}                                   \\
                                      & = 14\frac{11}{24}\text{ units}^2 \eos
                              \end{flalign*}
                  \end{enumerate}

            \item The following diagram shows a part of the curve $x = y(y-6)$ and the straight
                  line $y = x+6$.
                  \begin{center}
                        \includegraphics[width=0.3\textwidth,valign=t]{./images/13.png}
                  \end{center}
                  Find the area of the shaded region.
                  \sol{}
                  \begin{flalign*}
                        y                  & = x + 6           \\
                        x                  & = y - 6           \\
                        y - 6              & = y(y - 6)        \\
                                           & = y^2 - 6y        \\
                        y^2 - 7y + 6       & = 0               \\
                        (y - 6)(y - 1)     & = 0               \\
                        y              = 6 & \text{ or } y = 1 \\
                        1                  & = x + 6           \\
                        x                  & = -5
                  \end{flalign*}
                  \begin{flalign*}
                        A & = \left|\int_0^1 y(y-6)\,dy\right| + \frac{1}{2}(5)(5)                 \\
                          & = \left|\int_0^1 (y^2 - 6y)\,dy\right| + \frac{25}{2}                  \\
                          & = \left|{\left[\frac{1}{3}y^3 - 3y^2\right]}_0^1\right| + \frac{25}{2} \\
                          & = \left|\frac{1}{3} - 3 - 0\right| + \frac{25}{2}                      \\
                          & = \frac{8}{3} + \frac{25}{2}                                           \\
                          & = 15\frac{1}{6}\textit{ units}^2 \eos
                  \end{flalign*}

            \item The following diagram shows a part of the curve $y = x(6-x)$ and a straight
                  line $y = 2x$. The straight line $PQ$ is perpendicular to the x-axis.
                  \begin{center}
                        \includegraphics[width=0.3\textwidth,valign=t]{./images/14.png}
                  \end{center}
                  Find the area of the shaded region.
                  \sol{}
                  \begin{flalign*}
                        x(6 - x) & = 0               \\
                        x = 0    & \text{ or } x = 6 \\
                        2x       & = x(6-x)          \\
                        2x       & = 6x - x^2        \\
                        x^2 - 4x & = 0               \\
                        x(x - 4) & = 0               \\
                        x = 0    & \text{ or } x = 4 \\
                  \end{flalign*}
                  \begin{flalign*}
                        A & = \int_4^6 x(6-x)\,dx                                    \\
                          & = \int_4^6 (6x-x^2)\,dx                                  \\
                          & = {\left[3x^2 - \frac{1}{3}x^3\right]}_4^6               \\
                          & = \left(108 - 72\right) - \left(48 - \frac{64}{3}\right) \\
                          & = 36 - \frac{80}{3}                                      \\
                          & = 9\frac{1}{3}\textit{ units}^2 \eos
                  \end{flalign*}

            \item Find the generated volume, in terms of $\pi$, when the shaded region is rotated
                  through $360^{\circ}$ about the $x$-axis.
                  \begin{enumerate}
                        \item \includegraphics[width=0.3\textwidth,valign=t]{./images/15.png}
                              \sol{}
                              \begin{flalign*}
                                    V_x & = \int_0^3 \pi y^2\,dx                                       \\
                                        & = \pi\int_0^3 {(x^2 + 2)}^2\,dx                              \\
                                        & = \pi\int_0^3 (x^4 + 4x^2 + 4)\,dx                           \\
                                        & = \pi{\left[\frac{1}{5}x^5 + \frac{4}{3}x^3 + 4x\right]}_0^3 \\
                                        & = \left[\left(\frac{243}{5} + 36 + 12\right) - 0\right]\pi   \\
                                        & = 96.6\pi\textit{ units}^3 \eos
                              \end{flalign*}

                        \item \includegraphics[width=0.3\textwidth,valign=t]{./images/16.png}
                              \sol{}
                              \begin{flalign*}
                                    V_x & = \int_4^9 \pi{(4+\sqrt{x})}^2\,dx                                                            & \\
                                        & = \pi\int_4^9 (16 + 8\sqrt{x} + x)\,dx                                                          \\
                                        & = \pi{\left[16x + \frac{16x^\frac{3}{2}}{3} + \frac{1}{2}x^2\right]}_4^9                        \\
                                        & = \pi\left[\left(144 + 144 + \frac{81}{2}\right) - \left(64 + \frac{128}{3} + 8\right)\right]   \\
                                        & = 213.83\pi\textit{ units}^3 \eos
                              \end{flalign*}

                        \item \includegraphics[width=0.3\textwidth,valign=t]{./images/17.png}
                              \sol{}
                              \begin{flalign*}
                                    x^2 + 6x + 9        & = 3-x              \\
                                    x^2 + 7x + 6        & = 0                \\
                                    (x + 6)(x + 1)      & = 0                \\
                                    x              = -6 & \text{ or } x = -1 \\
                                    x^2 + 6x + 9        & = 0                \\
                                    {(x + 3)}^2         & = 0                \\
                                    x                   & = -3               \\
                                    0                   & = 3-x              \\
                                    x                   & = 3
                              \end{flalign*}
                              \begin{flalign*}
                                    V_x & = \int_{-3}^{-1} \pi{(x+3)}^4\,dx + \int_{-1}^{3} \pi {(3-x)}^2\,dx                                             \\
                                        & =  \pi\int_{-3}^{-1} {(x+3)}^4\,dx + \pi\int_{-1}^{3} {(3-x)}^2\,dx                                             \\
                                        & = \pi\left\{{\left[\frac{{(x+3)}^5}{5}\right]}_{-3}^{-1} + {\left[-\frac{{(3-x)}^3}{3}\right]}_{-1}^{3}\right\} \\
                                        & = \pi\left[\left(\frac{32}{5} - 0\right) + \left(0 + \frac{64}{3}\right)\right]                                 \\
                                        & = 27\frac{11}{15}\pi \textit{ units}^3 \eos
                              \end{flalign*}

                        \item \includegraphics[width=0.3\textwidth,valign=t]{./images/18.png}
                              \sol{}

                              Let the line be $l$. Since $l$ passes through $(-9, 0)$ and $(0, 9)$,
                              \begin{flalign*}
                                    m_l          & = \dfrac{9}{9} = 1 \\
                                    y - 9        & = 1(x - 0)         \\
                                    y            & = x + 9            \\
                                    x + 9        & = {(x+3)}^2        \\
                                                 & = x^2 + 6x + 9     \\
                                    x^2 + 5x     & = 0                \\
                                    x(x+5)       & = 0                \\
                                    x        = 0 & \text{ or } x = -5
                              \end{flalign*}
                              \begin{flalign*}
                                    V_x & = \int_{-5}^{0} \pi{(x+9)}^2\,dx - \int_{-5}^{0} \pi{(x+3)}^4\,dx                                             \\
                                        & = \pi\int_{-5}^{0} {(x+9)}^2\,dx - \pi\int_{-5}^{0} {(x+3)}^4\,dx                                             \\
                                        & = \pi\left\{{\left[\frac{{(x+9)}^3}{3}\right]}_{-5}^{0} - {\left[\frac{{(x+3)}^5}{5}\right]}_{-5}^{0}\right\} \\
                                        & = \pi\left[\left(243 - \frac{64}{3}\right) - \left(\frac{243}{5} + \frac{32}{5}\right)\right]                 \\
                                        & = 116\frac{2}{3}\pi \textit{ units}^3 \eos
                              \end{flalign*}
                  \end{enumerate}

            \item Find the generated volume, in terms of $\pi$, when the shaded region is rotated
                  through $360^\circ$ about the y-axis.
                  \begin{enumerate}
                        \item \includegraphics[width=0.3\textwidth,valign=t]{./images/19.png}
                              \sol{}
                              \begin{flalign*}
                                    y   & = x^2 + 1        \\
                                    x^2 & = y - 1          \\
                                    x   & - \sqrt{y-1} = 0
                              \end{flalign*}
                              \begin{flalign*}
                                    V_y & = \int_{2}^{10} \pi{(\sqrt{y-1})}^2\,dy                     \\
                                        & = \pi\int_{2}^{10} y-1\,dy                                  \\
                                        & = \pi{\left[\frac{y^2}{2} - y\right]}_2^{10}                \\
                                        & = \pi\left[\left(50 - 10\right) - \left(2 - 2\right)\right] \\
                                        & = 40\pi \textit{ units}^3 \eos
                              \end{flalign*}

                        \item \includegraphics[width=0.3\textwidth,valign=t]{./images/20.png}
                              \sol{}
                              \begin{flalign*}
                                    y & = x^2      \\
                                    x & = \sqrt{y}
                              \end{flalign*}
                              \begin{flalign*}
                                    V_y & = \int_{0}^{4} \pi {\left(\sqrt{y}\right)}^2\,dy + \frac{1}{3} \pi \cdot 4 \cdot 6 \\
                                        & = \pi\int_{0}^{4} y\,dy + 8\pi                                                     \\
                                        & = \pi{\left[\frac{y^2}{2}\right]}_0^4 + 8\pi                                       \\
                                        & = 8\pi + 8\pi                                                                      \\
                                        & = 16\pi \textit{ units}^3 \eos
                              \end{flalign*}

                        \item \includegraphics[width=0.3\textwidth,valign=t]{./images/21.png}
                              \sol{}
                              \begin{flalign*}
                                    y         & = 3x           \\
                                    x         & = \dfrac{y}{3} \\
                                    y         & = 2 + x^2      \\
                                    x^2       & = y - 2        \\
                                    x         & = \sqrt{y - 2} \\
                                    x = 1,\ y & = 3(1) = 3     \\
                                    x = 2,\ y & = 3(2) = 6
                              \end{flalign*}
                              \begin{flalign*}
                                    V_y & = \int_3^6 \pi {\left(\sqrt{y - 2}\right)}^2\,dy - \int_3^6 \pi {\left(\frac{y}{3}\right)}^2\,dy & \\
                                        & = \pi\int_3^6  {(y - 2)}\,dy - \pi\int_3^6 {\frac{y^2}{9}}\,dy                                     \\
                                        & = \pi{\left[\frac{y^2}{2} - 2y\right]}_3^6 - \pi{\left[\frac{y^3}{27}\right]}_3^6                  \\
                                        & = \pi{\left\{\left[(18 - 12) - \left(\frac{9}{2} - 6\right)\right] - \left(8 - 1\right)\right\}}   \\
                                        & = \pi\left(\frac{15}{2} - 7\right)                                                                 \\
                                        & = \frac{1}{2} \pi\textit{ units}^3 \eos
                              \end{flalign*}
                  \end{enumerate}

            \item The region bounded by the curve $y = \dfrac{8}{x}$, the x-axis, and the
                  straight line $x = 2$ and $x = k$ is rotated through $360^\circ$ about the
                  x-axis as shown in the following diagram.
                  \begin{center}
                        \includegraphics[width=0.3\textwidth,valign=t]{./images/22.png}
                  \end{center}
                  Express the volume generated by the region in terms of $k$. If the value of $k$ becomes extremely large, deduce the nearest value of volume.
                  \begin{flalign*}
                        V_x & = \int_2^k \pi{\left(\frac{8}{x}\right)}^2\,dx \\
                            & = \pi\int_2^k \frac{64}{x^2}\,dx               \\
                            & = \pi{\left[-\frac{64}{x}\right]}_2^k          \\
                            & = -\frac{64\pi}{k} + 32\pi \eos
                  \end{flalign*}
                  \begin{flalign*}
                        k            & \to \infty \Rightarrow \frac{1}{k} \approx 0 \\
                        \therefore\  & V_x \approx 32 \textit{ units}^3 \eos
                  \end{flalign*}

            \item The following diagram shows a part of the curve $y =4 +3x - x^2$ and the
                  straight line $y = x+1$.
                  \begin{center}
                        \includegraphics[width=0.33 \textwidth,valign=t]{./images/23.png}
                  \end{center}
                  Find the ratio of the area of the shaded region $A$ to the area of the shaded region $B$.
                  \sol{}
                  \begin{flalign*}
                        x + 1         & = 4 + 3x - x^2    \\
                        -x^2 + 2x + 3 & = 0               \\
                        x^2 - 2x - 3  & = 0               \\
                        (x-3)(x+1)    & = 0               \\
                        x = 3         & \text{ or }x = -1 \\
                        x = 3,\ y     & = 3 + 1 = 4
                  \end{flalign*}
                  \begin{flalign*}
                        A_A & = \int_{-1}^3 (4+3x-x^2)\,dx - \frac{1}{2} \cdot 4 \cdot 4                             \\
                            & = {\left[4x + \frac{3}{2}x^2 - \frac{1}{3}x^3\right]}_{-1}^3 - 8                       \\
                            & = \left(12 + \frac{27}{2} - 9\right) - \left(-4 + \frac{3}{2} + \frac{1}{3}\right) - 8 \\
                            & = \frac{33}{2} + \frac{13}{6} - 8                                                      \\
                            & = \frac{32}{3} \textit{ units}^2
                  \end{flalign*}
                  \begin{flalign*}
                        4 + 3x - x^2 & = 0               \\
                        x^2 - 3x - 4 & = 0               \\
                        (x-4)(x+1)   & = 0               \\
                        x = 4        & \text{ or }x = -1 \\
                  \end{flalign*}
                  \begin{flalign*}
                        A_B & = \int_{3}^4 (4+3x-x^2)\,dx + \frac{1}{2} \cdot 4 \cdot 4                      \\
                            & = {\left[4x + \frac{3}{2}x^2 - \frac{1}{3}x^3\right]}_{3}^4 + 8                \\
                            & = \left(16 + 24 - \frac{64}{3}\right) - \left(12 + \frac{27}{2} - 9\right) + 8 \\
                            & = \frac{56}{3} - \frac{33}{2} + 8                                              \\
                            & = \frac{61}{6} \textit{ units}^2
                  \end{flalign*}
                  \begin{flalign*}
                        \therefore\ A_A : A_B & = \frac{32}{3} : \frac{61}{6} \\
                                              & = \frac{64}{61}               \\
                                              & = 64 : 61 \eos
                  \end{flalign*}

            \item The following diagram shows two curves $y = x^2 - 1$ and $y = 3+2x-x^2$.
                  \begin{center}
                        \includegraphics[width=0.3\textwidth,valign=t]{./images/24.png}
                  \end{center}
                  Find the coordinate of the points $P$ and $Q$. Hence, calculate the area of the shaded region.
      \end{enumerate}
\end{multicols*}

\end{document}