\section{Revision Exercise 28}

\begin{enumerate}
    \item $\displaystyle\int_0^a\left(2 x^2-3 x+2\right) d x$
    \item $\displaystyle\int_1^3\left(x^2+\dfrac{1}{x^3}\right) d x$
    \item $\displaystyle\int_{-\dfrac{\pi}{6}}^{\frac{\pi}{2}}(3 \sin \theta-2 \cos 2 \theta) d \theta$
    \item $\displaystyle\int_{-\dfrac{\pi}{4}}^{\frac{\pi}{4}}\left(3 \sec ^2 \theta+\tan ^2 \theta\right) d \theta$
    \item $\displaystyle\int_0^{\ln 2} e^{3 x} d x$
    \item $\displaystyle\int_1^3 \dfrac{2}{3 x-1} d x$
    \item $\displaystyle\int_1^{16} \dfrac{2 x+3}{\sqrt{x}} d x$
    \item $\displaystyle\int_1^4 \dfrac{(\sqrt{x}-1)^2}{x} d x$
    \item $\displaystyle\int_1^2\left(x+\dfrac{4}{x^2}\right)^2 d x$
    \item $\displaystyle\int_0^1 \dfrac{x+1}{x^2+2 x+3} d x$
    \item $\displaystyle\int_{-1}^2 \dfrac{5 x}{\left(1+x^2\right)^4} d x$
    \item $\displaystyle\int_0^2 \dfrac{x}{\sqrt{25-4 x^2}} d x$
    \item $\displaystyle\int_2^4 \dfrac{3 x-2}{(2 x-3)^2} d x$
    \item $\displaystyle\int_2^4 \dfrac{2}{x^3-x} d x$
    \item $\displaystyle\int_1^3 \dfrac{1}{x^3+2 x^2+x} d x$
    \item $\displaystyle\int_0^\pi(\sin \theta+\cos \theta)^2 d \theta$
    \item $\displaystyle\int_0^{\frac{\pi}{3}} \sec ^2 \theta \tan \theta d \theta$
    \item $\displaystyle\int_{-\dfrac{\pi}{2}}^{\frac{\pi}{2}} \sin ^2 \theta \cos \theta d \theta$
    \item $\displaystyle\int_0^1 \dfrac{e^x}{e^x+1} d x$
    \item $\displaystyle\int_{\dfrac{\pi}{6}}^{\frac{\pi}{3}} \dfrac{\sec ^2 \theta}{\tan \theta} d \theta$
    \item Given that $\displaystyle\int_0^4 f(x) d x=2, \displaystyle\int_0^3 g(x) d
              x=4$, and $\displaystyle\int_3^8 g(x) d x=12$. Find the value of
          $\displaystyle\int_0^8\left[f\left(\dfrac{x}{2}\right)-2 g(x)\right] d x$.
    \item Given the function $y=(x+3) \sqrt{2 x-3}$, find $\dfrac{d y}{d x}$. Hence, find
          $\int_2^6 \dfrac{x}{\sqrt{2 x-3}} d x$.
    \item Given the function $y=x \ln x$, find $\dfrac{d y}{d x}$. Hence, find the
          following definite integrals:
          \begin{enumerate}
              \item $\displaystyle\int_1^4 \ln x d x$
              \item $\displaystyle\int_1^4 \ln (2 x) d x$
          \end{enumerate}
    \item Find the area of the region bounded by the curve $y=\dfrac{1}{x+1}$, the lines
          $x=1, x=7$, and the $x$-axis.
    \item Find the area of the region bounded by the curve $y=\dfrac{3}{x}$ and the line
          $y=4-x$.
    \item Find the area of the region bounded by the curve $x=y^2-5 y$ and the line $x+7
              y=24$.
    \item Find the area of the region bounded by the curves $y=x^2$ 及 $y^3=x$.
    \item Shown in the diagram below is the shaded region bounded by the curves $y=\ln x,
              y=\ln (2 x-1)$, and the line $y=3$. Find the area of this region.
    \item Find the area of the region bounded by the curves $x=y^3-y$ 及 $x=y-y^2$.
    \item Shown in the diagram below is the shaded region bounded by the curves $y=\sin
              x$ 及 $y=\sin 2 x$ in the interval $0 \leq x \leq \pi$. Find the area of this
          region.
    \item Find the volume of the solid of revolution formed by rotating the region
          bounded by the curve $y=\frac{1}{x+2}$, the line $x=2$, and two axes about the
          $x$-axis.
    \item Find the volume of the solid of revolution formed by rotating the region
          bounded by the curve $y=e^x-3$ and the two axes about the $x$-axis.
    \item Find the volume of the solid of revolution formed by rotating the region
          bounded by the curve $x=y^2-3 y$ and the $y$-axis about the $y$-axis.
    \item Find the volume of the solid of revolution formed by rotating the region
          bounded by the curve $y=x^2$ and the line $y=x+2$ about the $x$-axis.
    \item Find the volume of the solid of revolution formed by rotating the region
          bounded by the curve $y^2=x+9$ and the line $y=x+3$ about the $y$-axis.
    \item Given that a region is bounded by the curve $y^2=8 x$ and $y=x^2$. Find the
          volume of the solid of revolution formed by rotating this region about the
          $x$-axis and the $y$-axis respectively.
    \item SHown in the diagram below is the shaded region bounded by the curve $y =
              2\cos\pi x$, the line $y = 3x$, and the $y$-axis.
          \begin{enumerate}
              \item Prove that the $x$-coordinate of point $A$ is $\dfrac{1}{3}$.
              \item Find the volume of the solid of revolution formed by rotating this region about
                    the $x$-axis.
          \end{enumerate}
    \item Given that a region is bounded by the curve $xy = 12$, the line $x = 4$, and $y
              = 6$. Find the volume of the solid of revolution formed by rotating this region
          about the $x$-axis and the $y$-axis respectively.
\end{enumerate}