\subsection{Exercise 26.2}

\noindent \hspace{1.2em}\textit{Determine which intervals the following functions is an increasing function or a decreasing function.}
\begin{enumerate}
    \begin{multicols}{2}
        \item $f(x)=x^2-2 x+4$
        \sol{}
        \begin{flalign*}
            f'(x)  & = 2x - 2 & \\
            2x - 2 & = 0      & \\
            x      & = 1
        \end{flalign*}
        In the interval $(-\infty,1)$, $f'(x)<0$, so $f(x)$ is a decreasing function in the interval $(-\infty,1]$.

        In the interval $(1,\infty)$, $f'(x)>0$, so $f(x)$ is an increasing function in
        the interval $[1,\infty)$. \vfill\null

                        \item $f(x)=2 x^3-6 x^2+7$
                        \sol{}
                        \begin{flalign*}
                            f'(x)      & = 6x^2 - 12x          & \\
                            6x^2 - 12x & = 0                   & \\
                            x(x - 2)   & = 0                   & \\
                            x          & = 0 \text{ or } x = 2
                        \end{flalign*}
                        In the interval $(-\infty,0)$, $f'(x)>0$, so $f(x)$ is an increasing function in the interval $(-\infty,0]$.

        In the interval $(0,2)$, $f'(x)<0$, so $f(x)$ is a decreasing function in the
        interval $[0,2]$.

        In the interval $(2,\infty)$, $f'(x)>0$, so $f(x)$ is an increasing function in
        the interval $[2,\infty)$.
    \end{multicols}
    \vfill\null

    \begin{multicols}{2}
        \item $f(x)=x^3+x$
        \sol{}
        \begin{flalign*}
            f'(x)    & = 3x^2 + 1        & \\
            3x^2 + 1 & = 0               & \\
            x^2      & = -\frac{1}{3}      \\
            x        & \notin \mathbb{R}
        \end{flalign*}
        Since $f'(x)>0$ for all $x \in \mathbb{R}$, $f(x)$ is an increasing function.
        \vfill\null

        \item $f(x)=2+3 x-x^3$
        \sol{}
        \begin{flalign*}
            f'(x)    & = 3 - 3x^2 & \\
            3 - 3x^2 & = 0        & \\
            x^2      & = 1          \\
            x        & = \pm 1
        \end{flalign*}
        In the interval $(-\infty,-1)$, $f'(x)<0$, so $f(x)$ is a decreasing function in the interval $(-\infty,-1]$.

        In the interval $(-1,1)$, $f'(x)>0$, so $f(x)$ is an increasing function in the
        interval $[-1,1]$.

        In the interval $(1,\infty)$, $f'(x)<0$, so $f(x)$ is a decreasing function in
        the interval $[1,\infty)$.
    \end{multicols}
    \vfill\null
    \newpage

    \begin{multicols}{2}
        \item $f(x)=x^2(x-3)$
        \sol{}
        \begin{flalign*}
            f(x)      & = x^3 - 3x^2          & \\
            f'(x)     & = 3x^2 - 6x           & \\
            3x^2 - 6x & = 0                   & \\
            x(x - 2)  & = 0                   & \\
            x         & = 0 \text{ or } x = 2
        \end{flalign*}
        In the interval $(-\infty,0)$, $f'(x)>0$, so $f(x)$ is an increasing function in the interval $(-\infty,0]$.

        In the interval $(0,2)$, $f'(x)<0$, so $f(x)$ is a decreasing function in the
        interval $[0,2]$.

        In the interval $(2,\infty)$, $f'(x)>0$, so $f(x)$ is an increasing function in
        the interval $[2,\infty)$. \vfil\null

                        \item $f(x)=3 x^4+2 x^3-3 x^2-2$
                        \sol{}
                        \begin{flalign*}
                            f'(x)                    & = 12x^3 + 6x^2 - 6x         & \\
                            12x^3 + 6x^2             & = 6x                        & \\
                            x(2x^2 + x - 1)          & = 0                         & \\
                            x(x + 1)(2x - 1)         & = 0                         & \\
                            x = 0 \text{ or } x = -1 & \text{ or } x = \frac{1}{2}
                        \end{flalign*}
                        In the interval $(-\infty,-1)$, $f'(x)<0$, so $f(x)$ is a decreasing function in the interval $(-\infty,-1]$.

        In the interval $(-1,0)$, $f'(x)>0$, so $f(x)$ is an increasing function in the
        interval $[-1,0]$.

        In the interval $(0,\frac{1}{2})$, $f'(x)<0$, so $f(x)$ is a decreasing
        function in the interval $[0,\frac{1}{2}]$.

        In the interval $(\frac{1}{2},\infty)$, $f'(x)>0$, so $f(x)$ is an increasing
        function in the interval $[\frac{1}{2},\infty)$.
    \end{multicols}
    \vfill\null

    \begin{multicols}{2}
        \item $f(x)=\dfrac{x}{x^2+1}$
        \sol{}
        \begin{flalign*}
            f'(x)   & = \frac{(x^2+1) - x(2x)}{(x^2+1)^2} & \\
                    & = \frac{1 - x^2}{(x^2+1)^2}         & \\
            1 - x^2 & = 0                                 & \\
            x^2     & = 1                                 & \\
            x       & = \pm 1
        \end{flalign*}
        In the interval $(-\infty,-1)$, $f'(x) < 0$, so $f(x)$ is a decreasing function in the interval $(-\infty,-1]$.

        In the interval $(-1,1)$, $f'(x) > 0$, so $f(x)$ is an increasing function in
        the interval $[-1,1]$.

        In the interval $(1,\infty)$, $f'(x) < 0$, so $f(x)$ is a decreasing function
        in the interval $[1,\infty)$.\columnbreak

        \item $f(x)=\cos 2 x, 0 \leq x \leq \pi$
        \sol{}
        \begin{flalign*}
            f'(x)      & = -2 \sin 2x                       & \\
            -2 \sin 2x & = 0                                & \\
            \sin 2x    & = 0                                & \\
            2x         & = \sin^{-1} 0                      & \\
            x          & = 0 \text{ or } x = \dfrac{\pi}{2}
        \end{flalign*}
        In the interval $\left[0,\dfrac{\pi}{2}\right]$, $f'(x) < 0$, so $f(x)$ is a decreasing function in the interval $\left[0,\dfrac{\pi}{2}\right]$.

        In the interval $\left[\dfrac{\pi}{2},\pi\right]$, $f'(x) > 0$, so $f(x)$ is an
        increasing function in the interval $\left[\dfrac{\pi}{2},\pi\right]$.
    \end{multicols}
    \vfill\null
\end{enumerate}