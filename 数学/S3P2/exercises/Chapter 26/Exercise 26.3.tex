\newpage
\subsection{Exercise 26.3}

\noindent \hspace{1.2em}\textit{Find the extreme values of the following functions (Question 1 to 6):}
\begin{enumerate}
    \begin{multicols}{2}
        \item $f(x)=\dfrac{1}{2} x^2-3 x$
        \sol{}
        \begin{flalign*}
            f'(x)  & = x - 3                    & \\
            0      & = x - 3                    & \\
            x      & = 3                        & \\
            f(3)   & = \dfrac{1}{2}(3)^2 - 3(3) & \\
                   & = -\dfrac{9}{2}            & \\
            f''(x) & = 1                        & \\
            f''(3) & = 1 > 0                    &
        \end{flalign*}
        $\therefore$ $f(3) = -\dfrac{9}{2}$ is a relative minimum value.

        \item $f(x)=4+2 x-x^2$
        \sol{}
        \begin{flalign*}
            f'(x)  & = 2 - 2x           & \\
            0      & = 2 - 2x           & \\
            x      & = 1                & \\
            f(1)   & = 4 + 2(1) - (1)^2 & \\
                   & = 5                & \\
            f''(x) & = -2               & \\
            f''(1) & = -2 < 0           &
        \end{flalign*}
        $\therefore$ $f(1) = 5$ is a relative maximum value.
    \end{multicols}
    \vfill\null

    \begin{multicols}{2}
        \item $f(x)=-2 x^2+4 x+7$
        \sol{}
        \begin{flalign*}
            f'(x)  & = -4x + 4            & \\
            0      & = -4x + 4            & \\
            x      & = 1                  & \\
            f(1)   & = -2(1)^2 + 4(1) + 7 & \\
                   & = 9                  & \\
            f''(x) & = -4                 & \\
            f''(1) & = -4 < 0             &
        \end{flalign*}
        $\therefore$ $f(1) = 9$ is a relative maximum value.
        \vfill\null
        \item $f(x)=3 x^2-2 x+1$
        \sol{}
        \begin{flalign*}
            f'(x)  & = 6x - 2                                                        & \\
            0      & = 6x - 2                                                        & \\
            x      & = \dfrac{1}{3}                                                  & \\
            f(1)   & = 3\left(\dfrac{1}{3}\right)^2 - 2\left(\dfrac{1}{3}\right) + 1 & \\
                   & = \dfrac{2}{3}                                                  & \\
            f''(x) & = 6                                                             & \\
            f''(1) & = 6 > 0                                                         &
        \end{flalign*}
        $\therefore$ $f(1) = \dfrac{2}{3}$ is a relative minimum value.
    \end{multicols}
    \vfill\null

    \newpage
    \begin{multicols}{2}
        \item $f(x)=2 x^3-9 x^2-24 x-12$
        \sol{}
        \begin{flalign*}
            f'(x)          & = 6x^2 - 18x - 24                 & \\
            0              & = 6x^2 - 18x - 24                 & \\
            x^2 - 3x - 4   & = 0                               & \\
            (x - 4)(x + 1) & = 0                               & \\
            x = 4          & \text{ or } x = -1                & \\
            f(4)           & = 2(4)^3 - 9(4)^2 - 24(4) - 12    & \\
                           & = -124                            & \\
            f(-1)          & = 2(-1)^3 - 9(-1)^2 - 24(-1) - 12 & \\
                           & = 1                               & \\
            f''(x)         & = 12x - 18                        & \\
            f''(4)         & = 12(4) - 18 = 30 > 0             & \\
            f''(-1)        & = 12(-1) - 18 = -30 < 0           &
        \end{flalign*}
        $\therefore$ $f(4) = -124$ is a relative minimum value and $f(-1) = 1$ is a relative maximum value.

        \item $f(x)=15+9 x-3 x^2-x^3$
        \sol{}
        \begin{flalign*}
            f'(x)          & = 9 - 6x - 3x^2                 & \\
            0              & = 9 - 6x - 3x^2                 & \\
            x^2 + 2x - 3   & = 0                             & \\
            (x + 3)(x - 1) & = 0                             & \\
            x = -3         & \text{ or } x = 1               & \\
            f(-3)          & = 15 + 9(-3) - 3(-3)^2 - (-3)^3 & \\
                           & = -12                           & \\
            f(1)           & = 15 + 9(1) - 3(1)^2 - (1)^3    & \\
                           & = 20                            & \\
            f''(x)         & = -6x - 6                       & \\
            f''(-3)        & = -6(-3) - 6 = 12 > 0           & \\
            f''(1)         & = -6(1) - 6 = -12 < 0           &
        \end{flalign*}
        $\therefore$ $f(-3) = -12$ is a relative minimum value and $f(1) = 20$ is a relative maximum value.
    \end{multicols}
\end{enumerate}
\vfill\null

\noindent \hspace{1.2em}\textit{Find the coordinates of the extreme points of the following functions (Question 7 to 11):}
\begin{enumerate}[resume]
    \begin{multicols}{2}
        \item $f(x)=x\left(x^2-12\right)$
        \sol{}
        \begin{flalign*}
            f(x)    & = x^3 - 12x            & \\
            f'(x)   & = 3x^2 - 12            & \\
            0       & = 3x^2 - 12            & \\
            x       & = \pm 2                & \\
            f(2)    & = 2^3 - 12(2) = -16    & \\
            f(-2)   & = (-2)^3 - 12(-2) = 16 & \\
            f''(x)  & = 6x                   & \\
            f''(2)  & = 12 > 0               & \\
            f''(-2) & = -12 < 0              &
        \end{flalign*}
        $\therefore$ $(2, -16)$ is a relative minimum point and $(-2, 16)$ is a relative maximum point.
        \vfill\null
        \item $f(x)=4 x^3-3 x^2-6 x+2$
        \sol{}
        \begin{flalign*}
            f(x)                          & = 4x^3 - 3x^2 - 6x + 2                                                                            & \\
            f'(x)                         & = 12x^2 - 6x - 6                                                                                  & \\
            0                             & = 12x^2 - 6x - 6                                                                                  & \\
            2x^2 - x - 1                  & = 0                                                                                               & \\
            (2x + 1)(x - 1)               & = 0                                                                                               & \\
            x = -\dfrac{1}{2}             & \text{ or } x = 1                                                                                 & \\
            f\left(-\dfrac{1}{2}\right)   & = 4\left(-\dfrac{1}{2}\right)^3 - 3\left(-\dfrac{1}{2}\right)^2 - 6\left(-\dfrac{1}{2}\right) + 2 & \\
                                          & = \dfrac{15}{4}                                                                                   & \\
            f(1)                          & = 4(1)^3 - 3(1)^2 - 6(1) + 2                                                                      & \\
                                          & = -3                                                                                              & \\
            f''(x)                        & = 24x - 6                                                                                         & \\
            f''\left(-\dfrac{1}{2}\right) & = 24\left(-\dfrac{1}{2}\right) - 6 = -18 < 0                                                      & \\
            f''(1)                        & = 24(1) - 6 = 18 > 0                                                                              &
        \end{flalign*}
        $\therefore$ $\left(-\dfrac{1}{2}, \dfrac{15}{4}\right)$ is a relative maximum point and $(1, -3)$ is a relative minimum point.
    \end{multicols}
    \vfill\null

    \begin{multicols}{2}
        \item $f(x)=x(x-8)(x-3)$
        \sol{}
        \begin{flalign*}
            f(x)                         & = x(x^2 - 11x + 24)                                                                         & \\
                                         & = x^3 - 11x^2 + 24x                                                                         & \\
            f'(x)                        & = 3x^2 - 22x + 24                                                                           & \\
            0                            & = 3x^2 - 22x + 24                                                                           & \\
            (3x - 4)(x - 6)              & = 0                                                                                         & \\
            x = \dfrac{4}{3}             & \text{ or } x = 6                                                                           & \\
            f\left(\dfrac{4}{3}\right)   & = \left(\dfrac{4}{3}\right)^3 - 11\left(\dfrac{4}{3}\right)^2 + 24\left(\dfrac{4}{3}\right) & \\
                                         & = \dfrac{400}{27}                                                                           & \\
            f(6)                         & = (6)^3 - 11(6)^2 + 24(6)                                                                   & \\
                                         & = -36                                                                                       & \\
            f''(x)                       & = 6x - 22                                                                                   & \\
            f''\left(\dfrac{4}{3}\right) & = 6\left(\dfrac{4}{3}\right) - 22 = -14 < 0                                                 & \\
            f''(6)                       & = 6(6) - 22 = 14 > 0                                                                        &
        \end{flalign*}
        $\therefore$ $\left(\dfrac{4}{3}, \dfrac{400}{27}\right)$ is a relative maximum point and $(6, -36)$ is a relative minimum point.
        \columnbreak
        \item $f(x)=4 x^2+\dfrac{1}{x}$
        \sol{}
        \begin{flalign*}
            f'(x)                        & = 8x - \dfrac{1}{x^2}                                                      & \\
            0                            & = 8x - \dfrac{1}{x^2}                                                      & \\
            8x^3                         & = 1                                                                        & \\
            x^3                          & = \dfrac{1}{8}                                                             & \\
            x                            & = \dfrac{1}{2}                                                             & \\
            f\left(\dfrac{1}{2}\right)   & = 4\left(\dfrac{1}{2}\right)^2 + \dfrac{1}{\left(\dfrac{1}{2}\right)}  = 3 & \\
            f''(x)                       & = 8 + \dfrac{2}{x^3}                                                       & \\
            f''\left(\dfrac{1}{2}\right) & = 8 + \dfrac{2}{\left(\dfrac{1}{2}\right)^3} = 24 > 0
        \end{flalign*}
        $\therefore$ $\left(\dfrac{1}{2}, 3\right)$ is a relative minimum point.
    \end{multicols}

    \begin{multicols}{2}
        \item $f(x)=x-2 \sin x, \quad-\pi<x<\pi$
        \sol{}
        \begin{flalign*}
            f'(x)                           & = 1 - 2\cos x                                                                      & \\
            0                               & = 1 - 2\cos x                                                                      & \\
            \cos x                          & = \dfrac{1}{2}                                                                     & \\
            x                               & = \pm \dfrac{\pi}{3}                                                               & \\
            f\left(\dfrac{\pi}{3}\right)    & = \dfrac{\pi}{3} - 2\sin\left(\dfrac{\pi}{3}\right) = \dfrac{\pi}{3} - \sqrt{3}    & \\
            f\left(-\dfrac{\pi}{3}\right)   & = -\dfrac{\pi}{3} - 2\sin\left(-\dfrac{\pi}{3}\right) = -\dfrac{\pi}{3} + \sqrt{3} & \\
            f''(x)                          & = 2\sin x                                                                          & \\
            f''\left(\dfrac{\pi}{3}\right)  & = 2\sin\left(\dfrac{\pi}{3}\right) = \sqrt{3} > 0                                  & \\
            f''\left(-\dfrac{\pi}{3}\right) & = 2\sin\left(-\dfrac{\pi}{3}\right) = -\sqrt{3} < 0                                &
        \end{flalign*}
        $\therefore$ $\left(\dfrac{\pi}{3}, \dfrac{\pi}{3} - \sqrt{3}\right)$ is a relative minimum point and $\left(-\dfrac{\pi}{3}, -\dfrac{\pi}{3} + \sqrt{3}\right)$ is a relative maximum point.

        \item Find the stationery points of the function $f(x)=x^2(3-x)$, and determine
        whether the stationery points are relative maximum point or relative minimum
        point. \sol{}
        \begin{flalign*}
            f(x)   & = 3x^2 - x^3        & \\
            f'(x)  & = 6x - 3x^2         & \\
            0      & = 6x - 3x^2         & \\
            x      & = 0 \text{ or } 2   & \\
            f(0)   & = (0)^2(3 - 0) = 0  & \\
            f(2)   & = (2)^2(3 - 2) = 4  & \\
            f''(x) & = 6 - 6x            & \\
            f''(0) & = 6 - 6(0) = 6 > 0  & \\
            f''(2) & = 6 - 6(2) = -6 < 0
        \end{flalign*}
        $\therefore$ $(0, 0)$ is a relative minimum point and $(2, 4)$ is a relative maximum point.
    \end{multicols}
\end{enumerate}