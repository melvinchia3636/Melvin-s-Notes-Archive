\chapter{Indefinite Integrals}

\section{Indefinite Integrals as the Inverse of Differentiation}

Let function $F(x)$ and $f(x)$ be defined at the interval $(a, b)$. If any
point $x$ in this interval satisfies $F'(x) = f(x)$, then $F(x)$ is is the
preimage of $f(x)$ at the interval $(a, b)$.

According to the definition above, to find the preimage of a function $f(x)$,
we need to find the function $F(x)$ that satisfies $F'(x) = f(x)$. For example,
\begin{flalign*}
    (x^2)'     & = 2x   & \\
    (x^2 + 1)' & = 2x   & \\
    (x^2 - 2)' & = 2x   & \\
               & \vdots
\end{flalign*}

For any constant $C$, the derivative of $x^2 + C$ is $2x$. Hence, the preimage
of $2x$ is $x^2 + C$, where $C$ is an arbitary constant.

Since $\left[F(x) + C\right]' = F'(x) + 0 = F'(x)$, if the function $F(x)$ is a
preimage of $f(x)$, then $F(x) + C$ (C is a constant) is also a preimage of
$f(x)$. That is to say, there are infinite number of preimages of a function
$f(x)$.

In the other hand, if $F(x)$ and $G(x)$ are both preimages of $f(x)$, then
$F'(x) = f(x)$, $G'(x) = f(x)$.
\begin{flalign*}
    \because\ \left[G(x) - F(x)\right]' & = G'(x) - F'(x) & \\
                                        & = f(x) - f(x)   & \\
                                        & = 0
\end{flalign*}
\vspace{-3em}
\begin{flalign*}
    \therefore\ G(x) - F(x) & = C        & \\
    \text{i.e.}\ G(x)       & = F(x) + C
\end{flalign*}
This shows that for any two preimages of $f(x)$, the difference between them is a constant. If $F(x)$ is a preimage of $f(x)$, then all the preimages of $f(x)$ can be expressed as $F(x) + C$, where $C$ is a constant.
\subsection*{The Concept of Indefinite Integral}
Let function $F(x)$ be a preimage of $f(x)$. All the preimages $F(x) + C$ ($C$
is a constant) of a function $f(x)$ is called the indefinite integral of
$f(x)$, denoted by $\displaystyle\int f(x)dx$, i.e. $\displaystyle\int f(x)dx =
    F(x) + C$. $\displaystyle\int$ is called the integral sign, $f(x)$ is called
the integrand, $C$ is called the constant of integration.

Finding the indefinite integral of a function $f(x)$ is equivalent to finding
all the preimages of $f(x)$. From the explanation above, we just need to find
one preimage of $f(x)$, then add a constant $C$ to it to get the indefinite
integral of $f(x)$.

\section{Arithmetic Properties of Indefinite Integrals}

\subsection*{Basic Formulas of Indefinite Integrals}

In order to learn the methods and skills of finding indefinite integrals, we
must first learn some basic formulas of indefinite integrals. We know that
finding the indefinite integral is equivalent to finding the anti-derivative.
Therefore, we can get the formulas of indefinite integrals from the
corresponding formulas of derivatives.

For example, when $n \neq -1$, $\dfrac{d}{dx}\left(\dfrac{x^{n+1}}{n+1}\right)
    = x^n$.

Hence, $\displaystyle\int x^{n}dx = \dfrac{x^{n+1}}{n+1} + C$.

Similarly, we can also get the other basic formulas of indefinite integrals.
The basic formulas of indefinite integrals are listed below:
\begin{center}
    \framebox{
        \parbox[t][6cm]{12cm}{ \addvspace{0.2cm} \centering \begin{multicols}{2}
                \begin{enumerate}[label = ]
                    \item $\displaystyle\int x^{n}dx = \dfrac{x^{n+1}}{n+1} + C$, $n \neq -1$
                    \item $\displaystyle\int e^{x}dx = e^{x} + C$
                    \item $\displaystyle\int \sin xdx = -\cos x + C$
                    \item $\displaystyle\int \sec^2xdx = \tan x + C$
                    \item $\displaystyle\int \csc x\cot xdx = -\csc x + C$
                    \item $\displaystyle\int \dfrac{1}{x}dx = \ln|x| + C$
                    \item $\displaystyle\int a^{x}dx = \dfrac{a^x}{\ln a} + C$,
                    \item $\displaystyle\int \cos xdx = \sin x + C$
                    \item $\displaystyle\int \csc^2xdx = -\cot x + C$
                    \item $\displaystyle\int \sec x\tan xdx = \sec x + C$
                \end{enumerate}
            \end{multicols} }}
\end{center}
\vspace{0.9em}

\subsection{Practice 1}
\begin{enumerate}
    \item Find the equations of tangent and normal to the curve $y = x^3$ where $x = 2$.
    \item Given that the gradient of tangent to the curve $y = x^2 - 2x + 3$ at point $Q$
          is $4$, find the coordinates of the point $Q$.
    \item Find the equations of the tangent and normal to the curve $x^3 - 2xy + y^2 = 1$
          at point $(1, 2)$.
\end{enumerate}
\subsection{Exercise 27.2a}
Find the following indefinite integrals:
\begin{enumerate}
    \begin{multicols}{2}
        \item $\displaystyle\int3xdx$
        \sol{}
        \begin{flalign*}
            I & = \dfrac{3}{2}x^2 + C &
        \end{flalign*}
        \item $\displaystyle\int5x^{4}dx$
        \sol{}
        \begin{flalign*}
            I & = x^5 + C &
        \end{flalign*}
    \end{multicols}

    \begin{multicols}{2}
        \item $\displaystyle\int5dx$
        \sol{}
        \begin{flalign*}
            I & = 5x + C &
        \end{flalign*}
        \item $\displaystyle\int x^{-9}dx$
        \sol{}
        \begin{flalign*}
            I & = -\dfrac{1}{8}x^{-8} + C &
        \end{flalign*}
    \end{multicols}

    \begin{multicols}{2}
        \item $\displaystyle\int x^{\frac{1}{2}}dx$
        \sol{}
        \begin{flalign*}
            I & = \dfrac{2}{3}x^{\frac{3}{2}} + C &
        \end{flalign*}
        \vfill{}\null{}
        \item $\displaystyle\int{2x^{-{\frac{1}{2}}}dx}$
        \sol{}
        \begin{flalign*}
            I & = 4x^{\frac{1}{2}} + C &
        \end{flalign*}
        \vfill{}\null{}
    \end{multicols}
    \vspace{-3em}

    \begin{multicols}{2}
        \item $\displaystyle\int{\dfrac{1}{x^5}}dx$
        \sol{}
        \begin{flalign*}
            I & = -\dfrac{1}{4}x^{-4} + C &
        \end{flalign*}
        \item $\displaystyle\int{\left(\dfrac{1}{x}\right)}^{4}dx$
        \sol{}
        \begin{flalign*}
            I & = -\dfrac{1}{3}x^{-3} + C &
        \end{flalign*}
    \end{multicols}

    \begin{multicols}{2}
        \item $\displaystyle\int{\sqrt{3x}}dx$
        \sol{}
        \begin{flalign*}
            I & = \int{\sqrt{3}\sqrt{x}}dx                & \\
              & = \sqrt{3}\int{\sqrt{x}}dx                & \\
              & = \dfrac{2\sqrt{3}}{3}x^{\frac{3}{2}} + C & \\
              & = \dfrac{2}{\sqrt{3}}x^{\frac{3}{2}} + C
        \end{flalign*}

        \item $\displaystyle\int{x^3\sqrt[3]{x^2}}dx$
        \sol{}
        \begin{flalign*}
            I & = \int{x^3x^{\frac{2}{3}}}dx        & \\
              & = \int{x^{\frac{11}{3}}}dx          & \\
              & = \dfrac{3}{14}x^{\frac{14}{3}} + C
        \end{flalign*}
    \end{multicols}

    \begin{multicols}{2}
        \item $\displaystyle\int{\cos(-x)}dx$
        \sol{}
        \begin{flalign*}
            I & = \int{\cos x}dx & \\
              & = \sin x + C
        \end{flalign*}

        \item $\displaystyle\int{\dfrac{2}{\csc x}}dx$
        \sol{}
        \begin{flalign*}
            I & = \int{2\sin x}dx & \\
              & = -2\cos x + C
        \end{flalign*}
    \end{multicols}

    \newpage

    \begin{multicols}{2}
        \item $\displaystyle\int{\dfrac{1}{\sin^2x}}dx$
        \sol{}
        \begin{flalign*}
            I & = \int{\csc^2x}dx & \\
              & = -\cot x + C
        \end{flalign*}
        \vfill{}\null{}

        \item $\displaystyle\int{\dfrac{1}{1 - \sin^2x}}dx$
        \sol{}
        \begin{flalign*}
            I & = \int{\dfrac{1}{\cos^2x}}dx & \\
              & = \int{\sec^2x}dx            & \\
              & = \tan x + C
        \end{flalign*}
    \end{multicols}

    \begin{multicols}{2}
        \item $\displaystyle\int{\dfrac{\sin x}{\cos^2x}}dx$
        \sol{}
        \begin{flalign*}
            I & = \int{\dfrac{\sin x}{\cos x}\cdot\dfrac{1}{\cos x}}dx & \\
              & = \int{\tan x\sec x}dx                                 & \\
              & = \sec x + C
        \end{flalign*}

        \item $\displaystyle\int{\dfrac{\cos x}{\sin^2x}}dx$
        \sol{}
        \begin{flalign*}
            I & = \int{\dfrac{\cos x}{\sin x}\cdot\dfrac{1}{\sin x}}dx & \\
              & = \int{\cot x\csc x}dx                                 & \\
              & = -\csc x + C
        \end{flalign*}
    \end{multicols}
\end{enumerate}

Using the arithmetic properties of derivatives, we can also get the following
arithmetic properties of indefinite integrals:
\begin{center}
    \framebox{
        \parbox[t][2.6cm]{12cm}{ \addvspace{0.2cm} \centering \begin{enumerate}
                \item Constant Multiple Rule: $\displaystyle\int kf(x)dx = k\int f(x)dx$, where $k$
                      is a constant.
                \item Sum and Difference Rule: $\displaystyle\int \left[f(x) \pm g(x)\right]dx = \int
                          f(x)dx \pm \int g(x)dx$
            \end{enumerate} }}
\end{center}
\vspace{0.9em}

Note that when integrating by parts, the result of each part of the integral
contains a constant of integration. However, the sum (or difference) of
multiple constants of integration is still a constant of integration.
Therefore, when integrating by parts, we only need to add one constant of
integration to the final result.

\subsection{Practice 2}

\begin{enumerate}
    \begin{multicols}{2}
        \item $\displaystyle\int_2^8 x d x$
        \sol{}
        \begin{flalign*}
            I & = \left[\dfrac{1}{2}x^2\right]_2^8 & \\
              & = \dfrac{1}{2}(64 - 4)             & \\
              & = 30
        \end{flalign*}

        \item $\displaystyle\int_{-2}^4 x^3 d x$
        \sol{}
        \begin{flalign*}
            I & = \left[\dfrac{1}{4}x^4\right]_{-2}^4 & \\
              & = \dfrac{1}{4}(256 - 16)              & \\
              & = 60
        \end{flalign*}
    \end{multicols}
    \begin{multicols}{2}
        \item $\displaystyle\int_{-\pi}^\pi \cos x d x$
        \sol{}
        \begin{flalign*}
            I & = \bigg[\sin x\bigg]_{-\pi}^\pi & \\
              & = \sin\pi - \sin(-\pi)          & \\
              & = 0 - 0                         & \\
              & = 0
        \end{flalign*}

        \item $\displaystyle\int_0^{\frac{\pi}{4}} \sec ^2 x d x$
        \sol{}
        \begin{flalign*}
            I & = \bigg[\tan x\bigg]_0^{\frac{\pi}{4}} & \\
              & = \tan\dfrac{\pi}{4} - \tan 0          & \\
              & = 1 - 0                                & \\
              & = 1
        \end{flalign*}
    \end{multicols}
\end{enumerate}

\newpage
\subsection{Practice 3}

\begin{enumerate}
    \item Find the following definite integrals (Question 1 to 4):
          \begin{enumerate}
              \begin{multicols}{2}
                  \item $\displaystyle\int_1^4 \dfrac{2 x^2+3 x+2}{x} d x$
                  \sol{}
                  \begin{flalign*}
                      I & = \int_1^4 \left(2x + 3 + \dfrac{2}{x}\right) d x & \\
                        & = \bigg[x^2 + 3x + 2\ln|x|\bigg]_1^4              & \\
                        & = 16 + 12 + 2\ln 4 - 1 - 3 - 2\ln 1               & \\
                        & = 24 + 2\ln 4
                  \end{flalign*}

                  \item $\displaystyle\int_{-1}^1\left(3 e^{2 x}-5 x\right) d x$
                  \sol{}
                  \begin{flalign*}
                      I & = \int_{-1}^1\left(3 e^{2 x}-5 x\right) d x                          & \\
                        & = \bigg[\dfrac{3}{2}e^{2x} - \dfrac{5}{2}x^2\bigg]_{-1}^1            & \\
                        & = \dfrac{3}{2}e^2 - \dfrac{5}{2} - \dfrac{3}{2}e^{-2} + \dfrac{5}{2} & \\
                        & = \dfrac{3}{2}(e^2 - e^{-2})
                  \end{flalign*}
              \end{multicols}

              \begin{multicols}{2}
                  \item $\displaystyle\int_{-\frac{\pi}{3}}^{\frac{\pi}{3}}\left(2 \cos x-3 \sec ^2 x\right) d x$
                  \sol{}
                  \begin{flalign*}
                      I & = \bigg[2\sin x - 3\tan x\bigg]_{-\frac{\pi}{3}}^{\frac{\pi}{3}}                                                    & \\
                        & = 2\sin\dfrac{\pi}{3} - 3\tan\dfrac{\pi}{3} - 2\sin\left(-\dfrac{\pi}{3}\right) + 3\tan\left(-\dfrac{\pi}{3}\right) & \\
                        & = \sqrt{3} - 3\sqrt{3} + \sqrt{3} - 3\sqrt{3}                                                                       & \\
                        & = -4\sqrt{3}
                  \end{flalign*}

                  \item $\displaystyle\int_{-2}^4 f(x) d x, f(x)=\left\{\begin{array}{cc}x^2-2, & -2 \leq x<2 \\ 4-x, & 2 \leq x \leq 4\end{array}\right.$
                  \sol{}
                  \begin{flalign*}
                      I & = \int_{-2}^{2}(x^2 - 2)dx + \int_{2}^{4}(4 - x)dx                                     & \\
                        & = \bigg[\dfrac{1}{3}x^3 - 2x\bigg]_{-2}^{2} + \bigg[4x - \dfrac{1}{2}x^2\bigg]_{2}^{4} & \\
                        & = \dfrac{8}{3} - 4 + \dfrac{8}{3} - 4 + 16 - 8 - 8 + 2                                 & \\
                        & = -\dfrac{2}{3}
                  \end{flalign*}
              \end{multicols}
          \end{enumerate}
          \newpage

    \item Given that $\displaystyle\int_{-2}^5 f(x) d x=2, \displaystyle\int_{-2}^3 f(x)
              d x=-1, \displaystyle\int_3^4 g(x) d x=3 \text { and } \displaystyle\int_4^5
              g(x) d x=2$, find:
          \begin{enumerate}
              \item $\displaystyle\int_3^5 f(x) d x$;
                    \sol{}
                    \begin{flalign*}
                        \int_3^5 f(x) d x & = \int_{-2}^5 f(x) d x - \int_{-2}^3 f(x) d x & \\
                                          & = 2 - (-1)                                    & \\
                                          & = 3
                    \end{flalign*}

              \item $\displaystyle\int_3^5\left(\dfrac{1}{3} g(x)+\dfrac{1}{2} f(x)\right) d x$.
                    \sol{}
                    \begin{flalign*}
                        \int_3^5\left(\dfrac{1}{3} g(x)+\dfrac{1}{2} f(x)\right) d x & = \dfrac{1}{3}\int_3^5 g(x) d x + \dfrac{1}{2}\int_3^5 f(x) d x                                  & \\
                                                                                     & = \dfrac{1}{3}\left(\int_3^4 g(x) d x + \int_4^5 g(x) d x\right) + \dfrac{1}{2}\int_3^5 f(x) d x & \\
                                                                                     & = \dfrac{1}{3}(3 + 2) + \dfrac{1}{2}(3)                                                          & \\
                                                                                     & = \dfrac{19}{6}
                    \end{flalign*}
          \end{enumerate}

    \item Given that $f(x)=\sqrt{x^2+1}$, find $f^{\prime}(x)$. Hence, find
          $\displaystyle\int_0^1 \dfrac{x}{\sqrt{x^2+1}} d x$. \sol{}
          \begin{flalign*}
              f^{\prime}(x)                        & = \dfrac{1}{2}(x^2 + 1)^{-\frac{1}{2}}(2x) & \\
                                                   & = \dfrac{x}{\sqrt{x^2 + 1}}                & \\
              \\
              \int_0^1 \dfrac{x}{\sqrt{x^2+1}} d x & = \bigg[\sqrt{x^2 + 1}\bigg]_0^1           & \\
                                                   & = \sqrt{2} - 1
          \end{flalign*}
\end{enumerate}

\subsection{Exercise 27.2b}
Find the following indefinite integrals (Question 1 to 20):
\begin{enumerate}
    \begin{multicols}{2}
        \item $\displaystyle\int(x^3 - 3x + 1)dx$
        \sol{}
        \begin{flalign*}
            I & = \int x^3dx - \int 3xdx + \int dx          & \\
              & = \dfrac{1}{4}x^4 - \dfrac{3}{2}x^2 + x + C
        \end{flalign*}

        \item $\displaystyle\int\left(5x^4 + 2\sqrt{x}\right)dx$
        \sol{}
        \begin{flalign*}
            I & = \int 5x^4dx + \int 2\sqrt{x}dx        & \\
              & = x^5 + \dfrac{4}{3}x^{\frac{3}{2}} + C
        \end{flalign*}
    \end{multicols}

    \begin{multicols}{2}
        \item $\displaystyle\int\left({\dfrac{x^{2}}{2}}-{\dfrac{2}{x^{2}}}\right)dx$
        \sol{}
        \begin{flalign*}
            I & = \int{\dfrac{x^{2}}{2}}dx - \int{\dfrac{2}{x^{2}}}dx & \\
              & = \dfrac{1}{6}x^{3} + \dfrac{2}{x} + C
        \end{flalign*}
        \vfill{}\null{}

        \item $\displaystyle\int(\sin x-3\cos x)dx$
        \sol{}
        \begin{flalign*}
            I & = \int\sin xdx - \int3\cos xdx & \\
              & = -\cos x - 3\sin x + C
        \end{flalign*}
        \vfill{}\null{}
    \end{multicols}
    \vspace{-4em}
    \begin{multicols}{2}
        \item $\displaystyle\int(x-5)^2dx$
        \sol{}
        \begin{flalign*}
            I & = \int(x^2 - 10x + 25)dx           & \\
              & = \dfrac{1}{3}x^3 - 5x^2 + 25x + C
        \end{flalign*}

        \item $\displaystyle\int(x-1)(x-2)dx$
        \sol{}
        \begin{flalign*}
            I & = \int(x^2 - 3x + 2)dx                       & \\
              & = \dfrac{1}{3}x^3 - \dfrac{3}{2}x^2 + 2x + C
        \end{flalign*}
    \end{multicols}

    \begin{multicols}{2}
        \item $\displaystyle\int(x^2 + 2)\sqrt{x}dx$
        \sol{}
        \begin{flalign*}
            I & = \int x^{\frac{5}{2}}dx + \int 2x^{\frac{1}{2}}dx              & \\
              & = \dfrac{2}{7}x^{\frac{7}{2}} + \dfrac{4}{3}x^{\frac{3}{2}} + C
        \end{flalign*}

        \item $\displaystyle\int\dfrac{x^4 - 5}{x^2}dx$
        \sol{}
        \begin{flalign*}
            I & = \int x^2dx - \int\dfrac{5}{x^2}dx  & \\
              & = \dfrac{1}{3}x^3 + \dfrac{5}{x} + C
        \end{flalign*}
    \end{multicols}

    \begin{multicols}{2}
        \item $\displaystyle\int\dfrac{x+5}{\sqrt{x}}dx$
        \sol{}
        \begin{flalign*}
            I & = \int x^{\frac{1}{2}}dx + \int 5x^{-\frac{1}{2}}dx   & \\
              & = \dfrac{2}{3}x^{\frac{3}{2}} + 10x^{\frac{1}{2}} + C
        \end{flalign*}
        \vfill{}\null{}

        \item $\displaystyle\int\dfrac{\sqrt[3]{x^2} - \sqrt[4]{x}}{\sqrt{x}}dx$
        \sol{}
        \begin{flalign*}
            I & = \int \dfrac{x^{\frac{2}{3}} - x^{\frac{1}{4}}}{x^{\frac{1}{2}}}dx & \\
              & = \int \left(x^{\frac{1}{6}} - x^{-\frac{1}{4}}\right)dx            & \\
              & = \dfrac{6}{7}x^{\frac{7}{6}} - \dfrac{4}{3}x^{\frac{3}{4}} + C
        \end{flalign*}
    \end{multicols}

    \begin{multicols}{2}
        \item $\displaystyle\int\dfrac{x^2 - 9}{x + 3}dx$
        \sol{}
        \begin{flalign*}
            I & = \int \dfrac{(x + 3)(x - 3)}{x + 3}dx & \\
              & = \int (x - 3)dx                       & \\
              & = \dfrac{1}{2}x^2 - 3x + C
        \end{flalign*}

        \item $\displaystyle\int\dfrac{x^3 - 8}{x - 2}dx$
        \sol{}
        \begin{flalign*}
            I & = \int \dfrac{(x - 2)(x^2 + 2x + 4)}{x - 2}dx & \\
              & = \int (x^2 + 2x + 4)dx                       & \\
              & = \dfrac{1}{3}x^3 + x^2 + 4x + C
        \end{flalign*}
    \end{multicols}

    \begin{multicols}{2}
        \item $\displaystyle\int\sqrt[3]{x^2}\left(\sqrt{x} - \dfrac{1}{x}\right)dx$
        \sol{}
        \begin{flalign*}
            I & = \int x^{\frac{2}{3}}\left(x^{\frac{1}{2}} - x^{-1}\right)dx     & \\
              & = \int \left(x^{\frac{7}{6}} - x^{-\frac{1}{3}}\right)dx          & \\
              & = \dfrac{6}{13}x^{\frac{13}{6}} - \dfrac{3}{2}x^{\frac{2}{3}} + C
        \end{flalign*}

        \item $\displaystyle\int\dfrac{(2x + 1)^2}{x}dx$
        \sol{}
        \begin{flalign*}
            I & = \int\dfrac{4x^2 + 4x + 1}{x}dx            & \\
              & = \int \left(4x + 4 + \dfrac{1}{x}\right)dx & \\
              & = 2x^2 + 4x + \ln|x| + C
        \end{flalign*}
    \end{multicols}

    \begin{multicols}{2}
        \item $\displaystyle\int\dfrac{(x + 1)(3x^2 - 4)}{2x^3}dx$
        \sol{}
        \begin{flalign*}
            I & = \int\dfrac{3x^3 + 3x^2 - 4x - 4}{2x^3}dx                                          & \\
              & = \int\left(\dfrac{3}{2} + \dfrac{3}{2x} - \dfrac{2}{x^2} - \dfrac{2}{x^3}\right)dx & \\
              & = \dfrac{3}{2}x - \dfrac{3}{2}\ln|x| + \dfrac{2}{x} + \dfrac{1}{x^2} + C
        \end{flalign*}
        \vfill{}\null{}

        \item $\displaystyle\int\left(\dfrac{x - 1}{x^2}\right)^2dx$
        \sol{}
        \begin{flalign*}
            I & = \int\left(\dfrac{x^2 - 2x + 1}{x^4}\right)dx         & \\
              & = \int\left(x^{-2} - 2x^{-3} + x^{-4}\right)dx         & \\
              & = -x^{-1} + x^{-2} - \dfrac{1}{3}x^{-3} + C            & \\
              & = -\dfrac{1}{x} + \dfrac{1}{x^2} - \dfrac{1}{3x^3} + C
        \end{flalign*}
    \end{multicols}

    \begin{multicols}{2}
        \item $\displaystyle\int\left(\sin\dfrac{x}{2} - \cos\dfrac{x}{2}\right)^2dx$
        \sol{}
        \begin{flalign*}
            I & = \int\left(\sin^2\dfrac{x}{2} - 2\sin\dfrac{x}{2}\cos\dfrac{x}{2} + \cos^2\dfrac{x}{2}\right)dx & \\
              & = \int\left(1 - \sin x\right)dx = x + \cos x + C
        \end{flalign*}
        \vfill{}\null{}

        \item $\displaystyle\int\tan^2xdx$
        \sol{}
        \begin{flalign*}
            I & = \int(\sec^2x - 1)dx & \\
              & = \tan x - x + C
        \end{flalign*}
        \vfill{}\null{}
    \end{multicols}
    \vspace{-3em}

    \begin{multicols}{2}
        \item $\displaystyle\int\left(e^2 + \dfrac{1}{4x}\right)dx$
        \sol{}
        \begin{flalign*}
            I & = \int e^2dx + \dfrac{1}{4}\int\dfrac{1}{x}dx & \\
              & = e^2x + \dfrac{1}{4}\ln|x| + C
        \end{flalign*}

        \item $\displaystyle\int(2e)^xdx$
        \sol{}
        \begin{flalign*}
            I & = \dfrac{(2e)^x}{\ln(2e)} + C   & \\
              & = \dfrac{(2e)^x}{\ln 2 + 1} + C
        \end{flalign*}
    \end{multicols}

    \begin{multicols}{2}
        \item Given the function $y = \dfrac{5x}{3 - x}$, find $\dfrac{dy}{dx}$. Hence,
        \\find $\displaystyle\int\dfrac{1}{(3 - x)^2}dx$. \sol{}
        \begin{flalign*}
            \dfrac{dy}{dx}             & = \dfrac{5(3 - x) - 5x(-1)}{(3 - x)^2}          & \\
                                       & = \dfrac{15}{(3 - x)^2}                         & \\
                                       &                                                   \\
            \int\dfrac{1}{(3 - x)^2}dx & = \int\dfrac{1}{15}\cdot\dfrac{15}{(3 - x)^2}dx & \\
                                       & = \dfrac{1}{15}\int\dfrac{15}{(3 - x)^2}dx      & \\
                                       & = \dfrac{1}{15}\int\dfrac{dy}{dx}dx             & \\
                                       & = \dfrac{1}{15}y + C                            & \\
                                       & = \dfrac{1}{15}\cdot\dfrac{5x}{3 - x} + C       & \\
                                       & = \dfrac{x}{3(3 - x)} + C
        \end{flalign*}

        \item If the function $y = \dfrac{2x^2}{3x - 1}$, find $\dfrac{dy}{dx}$. Hence,
        \\find $\displaystyle\int\dfrac{2x - 3x^2}{(3x - 1)^2}dx$. \sol{}
        \begin{flalign*}
            \dfrac{dy}{dx}                      & = \dfrac{2(3x - 1)(2x) - 2x^2(3)}{(3x - 1)^2}     & \\
                                                & = \dfrac{12x^2 - 4x - 6x^2}{(3x - 1)^2}           & \\
                                                & = \dfrac{6x^2 - 4x}{(3x - 1)^2}                   & \\
                                                &                                                     \\
            \int\dfrac{2x - 3x^2}{(3x - 1)^2}dx & = -dfrac{1}{2}\int\dfrac{6x^2 - 4x}{(3x - 1)^2}dx & \\
                                                & = -\dfrac{1}{2}\int\dfrac{dy}{dx}dx               & \\
                                                & = -\dfrac{1}{2}y + C                              & \\
                                                & = -\dfrac{1}{2}\cdot\dfrac{2x^2}{3x - 1} + C      & \\
                                                & = -\dfrac{x^2}{3x - 1} + C
        \end{flalign*}
    \end{multicols}

    \item Prove that $\dfrac{d}{dx}\left(\dfrac{3x^2}{x^2 + 2}\right) = \dfrac{12x}{(x^2
                  + 2)^2}$. Hence, find $\displaystyle\int\dfrac{4x}{(x^2 + 2)^2}dx$. \sol{}
          \begin{flalign*}
              \dfrac{d}{dx}\left(\dfrac{3x^2}{x^2 + 2}\right) & = \dfrac{(x^2 + 2)(6x) - (3x^2)(2x)}{(x^2 + 2)^2} & \\
                                                              & = \dfrac{6x^3 + 12x - 6x^3}{(x^2 + 2)^2}          & \\
                                                              & = \dfrac{12x}{(x^2 + 2)^2} \qquad \blacksquare
          \end{flalign*}
          \newpage{}

    \item Given that ${\dfrac{d}{dx}}{\left(\dfrac{3x^{2}-1}{5x^{2}+7}\right)}=f(x)$,
          find $\displaystyle\int\left[3x^2 - 1 - 2f(x)\right]dx$. \sol{}
          \begin{flalign*}
              \int\left[3x^2 - 1 - 2f(x)\right]dx & = \int3x^2dx - \int dx - 2\int f(x)dx            & \\
                                                  & = x^3 - x - 2\cdot\dfrac{3x^2 - 1}{5x^2 + 7} + C & \\
                                                  & = x^3 - x - \dfrac{6x^2 - 2}{5x^2 + 7} + C
          \end{flalign*}

    \item Given that ${\dfrac{d}{dx}}\left({\dfrac{2+x^{3}}{2-x^{3}}}\right)=3g(x)$, find
          $\displaystyle\int\left[g(x) - 3x + 2\right]dx$. \sol{}
          \begin{flalign*}
              \int\left[g(x) - 3x + 2\right]dx & = \int g(x)dx - 3\int xdx + 2\int dx                                           & \\
                                               & = \int g(x)dx - \dfrac{3}{2}x^2 + 2x + C                                       & \\
                                               & = \int\dfrac{1}{3}\cdot3g(x)dx - \dfrac{3}{2}x^2 + 2x + C                      & \\
                                               & = \dfrac{1}{3}\left(\dfrac{2 + x^3}{2 - x^3}\right) - \dfrac{3}{2}x^2 + 2x + C & \\
                                               & = \dfrac{2 + x^3}{3(2 - x^3)} - \dfrac{3}{2}x^2 + 2x + C
          \end{flalign*}
\end{enumerate}

\section{Integration by Substitution}

In the last section, we learned to find the indefinite integral of some
functions using some basic formulas of indefinite integrals and two arithmetic
properties of indefinite integrals. However, for the indefinite integral of
some more complicated functions like $\displaystyle\int 3\sqrt{3x + 1}dx$,
$\displaystyle\int 2\sin2xdx$, etc., we cannot find their indefinite integrals
straight away using the basic formulas of indefinite integrals and the
arithmetic properties of indefinite integrals. Hence, we need to learn some
other methods to find the indefinite integral of these functions. Here we will
introduce a method called integration by substitution.

Consider a function $F(u)$, where $u$ is a function of $x$, i.e. $u = g(x)$.

Using the chain rule, we have $\dfrac{d}{dx}F\left(g(x)\right) =
    F'\left(g(x)\right) \cdot g'(x)$. Hence,
\begin{center}
    \framebox{
        \parbox[t][1.2cm]{6cm}{ \addvspace{0.2cm} \centering $\displaystyle\int
                F'\left(g(x)\right) \cdot g'(x)dx = F\left(g(x)\right) + C$ }}
\end{center}
\vspace{0.9em}
Generally speaking, during the calculation process, we let $u = g(x)$.
\begin{flalign*}
    \therefore\ \displaystyle\int F'\left(g(x)\right) \cdot g'(x)dx & = \int F'(u)\dfrac{du}{dx}dx & \\
                                                                    & = \int F'(u)du               & \\
                                                                    & = F(u) + C                   & \\
                                                                    & = F\left(g(x)\right) + C
\end{flalign*}
For the functions that we cannot find their indefinite integrals straight away using the basic formulas of indefinite integrals, if it can be expressed in the form of $\displaystyle\int F'\left(g(x)\right) \cdot g'(x)dx$, we can perform substitution using $u = g(x)$ and express the indefinite integral as $\displaystyle\int F'(u)du$ to find its indefinite integral.

\subsection{Practice 4}

\begin{enumerate}
    \begin{multicols}{2}
        \item $\displaystyle\int_0^3 4 e^{2 x} d x$
        \sol{}

        Let $u = 2x$, $du = 2dx$.

        When $x = 0$, $u = 0$.

        When $x = 3$, $u = 6$.
        \begin{flalign*}
            I & = \int_0^6 2 e^u d u    & \\
              & = \bigg[2 e^u\bigg]_0^6 & \\
              & = 2 e^6 - 2             & \\
              & = 2(e^6 - 1)
        \end{flalign*}

        \item $\displaystyle\int_1^3 \dfrac{x}{3 x^2+5} d x$
        \sol{}

        Let $u = 3x^2 + 5$, $du = 6xdx$.

        When $x = 1$, $u = 8$.

        When $x = 3$, $u = 32$.
        \begin{flalign*}
            I & = \dfrac{1}{6}\int_8^{32} \dfrac{1}{u} d u & \\
              & = \dfrac{1}{6} \bigg[\ln|u|\bigg]_8^{32}   & \\
              & = \dfrac{1}{6} \ln 32 - \dfrac{1}{6} \ln 8 & \\
              & = \dfrac{1}{6} \ln 4                       & \\
              & = \dfrac{1}{3} \ln 2
        \end{flalign*}
    \end{multicols}

    \begin{multicols}{2}
        \item $\displaystyle\int_0^4 \dfrac{x}{25-x^2} d x$
        \sol{}

        Let $u = 25 - x^2$, $du = -2xdx$.

        When $x = 0$, $u = 25$.

        When $x = 4$, $u = 9$.
        \begin{flalign*}
            I & = -\dfrac{1}{2}\int_{25}^9 \dfrac{1}{u} d u & \\
              & = -\dfrac{1}{2} \bigg[\ln|u|\bigg]_{25}^9   & \\
              & = -\dfrac{1}{2} \ln 9 + \dfrac{1}{2} \ln 25 & \\
              & = \dfrac{1}{2} \ln\dfrac{25}{9}             & \\
              & = \ln\dfrac{5}{3}
        \end{flalign*}

        \item $\displaystyle\int_0^\pi 2 \sin x \cos ^2 x d x$
        \sol{}

        Let $u = \cos x$, $du = -\sin xdx$.

        When $x = 0$, $u = 1$.

        When $x = \pi$, $u = -1$.
        \begin{flalign*}
            I & = 2\int_{-1}^1 u^2 d u                & \\
              & = 2\bigg[\dfrac{1}{3}u^3\bigg]_{-1}^1 & \\
              & = \dfrac{4}{3}
        \end{flalign*}
    \end{multicols}
\end{enumerate}
\subsection{Exercise 27.3a}
\noindent \hspace{1.2em}\textit{Find the following indefinite integral:}
\begin{enumerate}
    \begin{multicols}{2}
        \item $\displaystyle\int(2x+1)^{3} dx$
        \sol{}

        Let $u = 2x + 1$, $du = 2dx$.
        \begin{flalign*}
            I & = \dfrac{1}{2}\int u^{3}du              & \\
              & = \dfrac{1}{2}\cdot\dfrac{u^{4}}{4} + C & \\
              & = \dfrac{1}{8}(2x + 1)^{4} + C
        \end{flalign*}

        \item $\displaystyle\int(3x+2)^{5} dx$
        \sol{}

        Let $u = 3x + 2$, $du = 3dx$.
        \begin{flalign*}
            I & = \dfrac{1}{3}\int u^{5}du              & \\
              & = \dfrac{1}{3}\cdot\dfrac{u^{6}}{6} + C & \\
              & = \dfrac{1}{18}(3x + 2)^{6} + C
        \end{flalign*}
    \end{multicols}

    \begin{multicols}{2}
        \item $\displaystyle\int(3-x)^{6} dx$
        \sol{}

        Let $u = 3 - x$, $du = -dx$.
        \begin{flalign*}
            I & = -\int u^{6}du                                       & \\
              & = -\dfrac{u^{7}}{7} + C = -\dfrac{(3 - x)^{7}}{7} + C
        \end{flalign*}

        \item $\displaystyle\int(2x-1)^{-3} dx$
        \sol{}

        Let $u = 2x - 1$, $du = 2dx$.
        \begin{flalign*}
            I & = \dfrac{1}{2}\int u^{-3}du                                               & \\
              & = \dfrac{1}{2}\cdot\dfrac{u^{-2}}{-2} + C = -\dfrac{1}{4(2x - 1)^{2}} + C
        \end{flalign*}
    \end{multicols}

    \begin{multicols}{2}
        \item $\displaystyle\int4{\sqrt{2x-1}} dx$
        \sol{}

        Let $u = 2x - 1$, $du = 2dx$.
        \begin{flalign*}
            I & = 2\int{\sqrt{u}}du                      & \\
              & = 2\cdot\dfrac{2}{3}u^{\frac{3}{2}} + C  & \\
              & = \dfrac{4}{3}(2x - 1)^{\frac{3}{2}} + C
        \end{flalign*}

        \item $\displaystyle\int2(3x+1)^{2} dx$
        \sol{}

        Let $u = 3x + 1$, $du = 3dx$.
        \begin{flalign*}
            I & = \dfrac{2}{3}\int u^{2}du              & \\
              & = \dfrac{2}{3}\cdot\dfrac{u^{3}}{3} + C & \\
              & = \dfrac{2}{9}(3x + 1)^{3} + C
        \end{flalign*}
    \end{multicols}

    \begin{multicols}{2}
        \item $\displaystyle\int\dfrac{dx}{(2x+5)^{8}}$
        \sol{}

        Let $u = 2x + 5$, $du = 2dx$.
        \begin{flalign*}
            I & = \dfrac{1}{2}\int u^{-8}du               & \\
              & = \dfrac{1}{2}\cdot\dfrac{u^{-7}}{-7} + C & \\
              & = -\dfrac{1}{14(2x + 5)^{7}} + C
        \end{flalign*}

        \item $\displaystyle\int\dfrac{2}{(3-2x)^{2}} dx$
        \sol{}

        Let $u = 3 - 2x$, $du = -2dx$.
        \begin{flalign*}
            I & = -\int\dfrac{1}{u^{2}}du & \\
              & = \dfrac{1}{u} + C        & \\
              & = \dfrac{1}{3 - 2x} + C
        \end{flalign*}
    \end{multicols}

    \begin{multicols}{2}
        \item $\displaystyle\int x{\sqrt{x^{2}+1}} dx$
        \sol{}

        Let $u = x^2 + 1$, $du = 2xdx$.
        \begin{flalign*}
            I & = \dfrac{1}{2}\int{\sqrt{u}}du                     & \\
              & = \dfrac{1}{2}\cdot\dfrac{2}{3}u^{\frac{3}{2}} + C & \\
              & = \dfrac{1}{3}(x^2 + 1)^{\frac{3}{2}} + C
        \end{flalign*}

        \item $\displaystyle\int 3x^{2}\left(x^{3}+4\right)^{3} dx$
        \sol{}

        Let $u = x^3 + 4$, $du = 3x^2dx$.
        \begin{flalign*}
            I & = \int u^{3}du                  & \\
              & = \dfrac{u^{4}}{4} + C          & \\
              & = \dfrac{1}{4}(x^3 + 4)^{4} + C
        \end{flalign*}
    \end{multicols}

    \begin{multicols}{2}
        \item $\displaystyle\int15x^{2}\left(x^{3}-1\right)^{4} dx$
        \sol{}

        Let $u = x^3 - 1$, $du = 3x^2dx$.
        \begin{flalign*}
            I \int15x^{2}\le & = \int 5u^{4}du         & \\
                             & = \dfrac{5u^{5}}{5} + C & \\
                             & = u^{5} + C             & \\
                             & = (x^3 - 1)^{5} + C
        \end{flalign*}

        \item $\displaystyle\int\left(2x+1\right)\!\left(x^{2}+x\!+\!2\right)^{5} dx$
        \sol{}

        Let $u = x^2 + x + 2$, $du = (2x + 1)dx$.
        \begin{flalign*}
            I & = \int u^{5}du                      & \\
              & = \dfrac{u^{6}}{6} + C              & \\
              & = \dfrac{1}{6}(x^2 + x + 2)^{6} + C
        \end{flalign*}
    \end{multicols}

    \newpage
    \begin{multicols}{2}
        \item $\displaystyle\int\left(x^{2}-2x\right)\left(x^{3}-3x^{2}+1\right)^{4} dx$
        \sol{}

        Let $u = x^3 - 3x^2 + 1$, $du = (3x^2 - 6x)dx$.
        \begin{flalign*}
            I & = \int u^{4}du                         & \\
              & = \dfrac{u^{5}}{5} + C                 & \\
              & = \dfrac{1}{5}(x^3 - 3x^2 + 1)^{5} + C
        \end{flalign*}

        \item $\displaystyle\int\dfrac{x+1}{{{x}^{2}}+2x+3} dx$
        \sol{}

        Let $u = x^2 + 2x + 3$, $du = (2x + 2)dx = 2(x + 1)dx$.
        \begin{flalign*}
            I & = \dfrac{1}{2}\int\dfrac{1}{u}du    & \\
              & = \dfrac{1}{2}\ln|u| + C            & \\
              & = \dfrac{1}{2}\ln|x^2 + 2x + 3| + C
        \end{flalign*}

    \end{multicols}
    \begin{multicols}{2}
        \item $\displaystyle\int x^{2}\cos\left(x^{3}+2\right) dx$
        \sol{}

        Let $u = x^3 + 2$, $du = 3x^2dx$.
        \begin{flalign*}
            I & = \dfrac{1}{3}\int\cos udu      & \\
              & = \dfrac{1}{3}\sin u + C        & \\
              & = \dfrac{1}{3}\sin(x^3 + 2) + C
        \end{flalign*}

        \item $\displaystyle\int\sin{\dfrac{x}{2}} dx$
        \sol{}

        Let $u = \dfrac{x}{2}$, $du = \dfrac{1}{2}dx$.
        \begin{flalign*}
            I & = 2\int\sin udu            & \\
              & = -2\cos u + C             & \\
              & = -2\cos{\dfrac{x}{2}} + C
        \end{flalign*}

    \end{multicols}
    \begin{multicols}{2}
        \item $\displaystyle\int\dfrac{\ln^2 x}{x} dx$
        \sol{}

        Let $u = \ln x$, $du = \dfrac{1}{x}dx$.
        \begin{flalign*}
            I & = \int u^2du              & \\
              & = \dfrac{u^3}{3} + C      & \\
              & = \dfrac{1}{3}\ln^3 x + C
        \end{flalign*}

        \item $\displaystyle\int e^{1 - 2x} dx$
        \sol{}

        Let $u = 1 - 2x$, $du = -2dx$.
        \begin{flalign*}
            I & = -\dfrac{1}{2}\int e^udu     & \\
              & = -\dfrac{1}{2}e^u + C        & \\
              & = -\dfrac{1}{2}e^{1 - 2x} + C
        \end{flalign*}

    \end{multicols}
    \begin{multicols}{2}
        \item $\displaystyle\int\left(e^{x}+e^{-x}\right) dx$
        \sol{}

        Let $u = -x$, $du = -dx$.
        \begin{flalign*}
            I & = \int e^x dx - \int e^u du & \\
              & = e^x - e^u + C             & \\
              & = e^x - e^{-x} + C
        \end{flalign*}

        \item $\displaystyle\int x e^{x^2} dx$
        \sol{}

        Let $u = x^2$, $du = 2xdx$.
        \begin{flalign*}
            I & = \dfrac{1}{2}\int e^u du & \\
              & = \dfrac{1}{2}e^u + C     & \\
              & = \dfrac{1}{2}e^{x^2} + C
        \end{flalign*}
    \end{multicols}

\end{enumerate}

\newpage
\subsection{Practice 5}
Find the following indefinite integral:
\begin{enumerate}
    \begin{multicols}{2}
        \item $\displaystyle\int\sin2x\cos2x dx$
        \sol{}
        \begin{flalign*}
            I & = \dfrac{1}{2}\int\sin4x dx & \\
              & = -\dfrac{1}{8}\cos4x + C
        \end{flalign*}
        \vfill{}\null{}

        \item $\displaystyle\int\cos^2 2x dx$
        \sol{}
        \begin{flalign*}
            I & = \int\dfrac{1 + \cos 4x}{2} dx                    & \\
              & = \dfrac{1}{2}\int dx + \dfrac{1}{2}\int\cos 4x dx & \\
              & = \dfrac{1}{2}x + \dfrac{1}{8}\sin 4x + C
        \end{flalign*}
    \end{multicols}
    \begin{multicols}{2}
        \item $\displaystyle\int\sin^3 x dx$
        \sol{}
        \begin{flalign*}
            I & = \int\sin^2 x\sin x dx                 & \\
              & = \int(1 - \cos^2 x)\sin x dx           & \\
              & = \int\sin x dx - \int\cos^2 x\sin x dx
        \end{flalign*}
        Let $u = \cos x$, $du = -\sin xdx$.
        \begin{flalign*}
            I & = -\int \cos x dx + \int u^2 du    & \\
              & = -\sin x + \dfrac{1}{3}u^3 + C    & \\
              & = \dfrac{1}{3}\cos^3 x -\sin x + C
        \end{flalign*}

        \item $\displaystyle\int\cos^3 x dx$
        \sol{}
        \begin{flalign*}
            I & = \int\cos^2 x\cos x dx                 & \\
              & = \int(1 - \sin^2 x)\cos x dx           & \\
              & = \int\cos x dx - \int\sin^2 x\cos x dx
        \end{flalign*}
        Let $u = \sin x$, $du = \cos xdx$.
        \begin{flalign*}
            I & = \int \cos x dx - \int u^2 du      & \\
              & = \sin x - \dfrac{1}{3}u^3 + C      & \\
              & = \sin x - \dfrac{1}{3}\sin^3 x + C
        \end{flalign*}
    \end{multicols}
    \begin{multicols}{2}
        \item $\displaystyle\int\tan^4 x\sec^2 x dx$
        \sol{}

        Let $u = \tan x$, $du = \sec^2 xdx$.
        \begin{flalign*}
            I & = \int u^4 du              & \\
              & = \dfrac{u^5}{5} + C       & \\
              & = \dfrac{1}{5}\tan^5 x + C
        \end{flalign*}
        \vfill{}\null{}
        \columnbreak
        \item $\displaystyle\int\tan^4\dfrac{x}{2} dx$
        \sol{}
        \begin{flalign*}
            I & = \int\tan^2\dfrac{x}{2}\tan^2\dfrac{x}{2} dx                             & \\
              & = \int\left(\sec^2\dfrac{x}{2} - 1\right)\tan^2\dfrac{x}{2} dx            & \\
              & = \int\sec^2\dfrac{x}{2}\tan^2\dfrac{x}{2} dx - \int\tan^2\dfrac{x}{2} dx
        \end{flalign*}
        Let $u = \tan\dfrac{x}{2}$, $du = \dfrac{1}{2}\sec^2\dfrac{x}{2}dx$.
        \begin{flalign*}
            I & = 2\int u^2 du - \int \tan^2\dfrac{x}{2} dx                     & \\
              & = \dfrac{2u^3}{3} - \int \left(\sec^2\dfrac{x}{2} - 1\right) dx & \\
              & = \dfrac{2u^3}{3} - \int \sec^2\dfrac{x}{2} dx + \int dx        & \\
              & = \dfrac{2}{3}\tan^3\dfrac{x}{2} - 2\tan\dfrac{x}{2} + x + C
        \end{flalign*}
    \end{multicols}
\end{enumerate}
\subsection{Exercise 27.3b}
\noindent \hspace{1.2em}\textit{Find the following indefinite integral:}
\begin{enumerate}
    \begin{multicols}{2}
        \item $\displaystyle\int\sin^2\dfrac{x}{2} dx$
        \sol{}
        \begin{flalign*}
            I & = \int\left(\dfrac{1 - \cos x}{2}\right) dx       & \\
              & = \dfrac{1}{2}\int dx - \dfrac{1}{2}\int\cos x dx & \\
              & = \dfrac{1}{2}x - \dfrac{1}{2}\sin x + C
        \end{flalign*}
        \vfill{}\null{}

        \item $\displaystyle\int\tan^2 5x dx$
        \sol{}
        \begin{flalign*}
            I & = \int\left(\sec^2 5x - 1\right) dx        & \\
              & = \int\sec^2 5x dx - \int dx               & \\
              & = \dfrac{1}{5}\int\sec^2 5xd(5x) - \int dx & \\
              & = \dfrac{1}{5}\tan 5x - x + C
        \end{flalign*}
    \end{multicols}

    \begin{multicols}{2}
        \item $\displaystyle\int\dfrac{1}{\sec^2 4x} dx$
        \sol{}
        \begin{flalign*}
            I & = \int\cos^2 4x dx                                 & \\
              & = \int\dfrac{1 + \cos 8x}{2} dx                    & \\
              & = \dfrac{1}{2}\int dx + \dfrac{1}{2}\int\cos 8x dx & \\
              & = \dfrac{1}{16}\sin 8x + \dfrac{1}{2}x + C
        \end{flalign*}

        \item $\displaystyle\int\cos^2(3x - 1) dx$
        \sol{}
        \begin{flalign*}
            I & = \int\left(\dfrac{1 + \cos 2(3x - 1)}{2}\right) dx      & \\
              & = \dfrac{1}{2}\int dx + \dfrac{1}{2}\int\cos (6x - 2) dx & \\
              & = \dfrac{1}{12}\sin (6x - 2) + \dfrac{1}{2}x + C
        \end{flalign*}
    \end{multicols}

    \begin{multicols}{2}
        \item $\displaystyle\int\sec5x\tan5x dx$
        \sol{}
        \begin{flalign*}
            I & = \dfrac{1}{5}\int\sec5x\tan5x d(5x) & \\
              & = \dfrac{1}{5}\sec5x + C
        \end{flalign*}

        \item $\displaystyle\int-\csc3x\cot3x dx$
        \sol{}
        \begin{flalign*}
            I & = \dfrac{1}{3}\int-\csc3x\cot3x d(3x) & \\
              & = \dfrac{1}{3}\csc3x + C
        \end{flalign*}
    \end{multicols}

    \begin{multicols}{2}
        \item $\displaystyle\int\left(\sin\dfrac{x}{8} - \sec^2 2x\right) dx$
        \sol{}
        \begin{flalign*}
            I & = \int\sin\dfrac{x}{8} dx - \int\sec^2 2x dx                                           & \\
              & = -8\int\sin \dfrac{x}{8} d\left(\dfrac{x}{8}\right) - \dfrac{1}{2}\int\sec^2 2x d(2x) & \\
              & = -8\cos\dfrac{x}{8} - \dfrac{1}{2}\tan 2x + C
        \end{flalign*}
        \vfill{}\null{}
        \columnbreak{}

        \item $\displaystyle\int\left(\sin{\dfrac{x}{2}}+\cos{\dfrac{x}{2}}\right)^{2} dx$
        \sol{}

        Let $u = \dfrac{x}{2}$, $du = \dfrac{1}{2}dx$.
        \begin{flalign*}
            I & = 2\int\left(\sin u + \cos u\right)^2 du                   & \\
              & = 2\int\left(\sin^2 u + 2\sin u\cos u + \cos^2 u\right) du & \\
              & = 2\int\left(1 + \sin 2u\right) du                         & \\
              & = 2\left(u - \dfrac{1}{2}\cos 2u\right) + C                & \\
              & = x - \cos x + C
        \end{flalign*}
    \end{multicols}

    \begin{multicols}{2}
        \item $\displaystyle\int(\sec x+\tan x)^{2} dx$
        \sol{}
        \begin{flalign*}
            I & = \int(\sec^2 x + 2\sec x\tan x + \tan^2 x) dx                   & \\
              & = \int\sec^2 x + 2\int\sec x\tan x dx + \int\tan^2 x dx          & \\
              & = \int\sec^2 x dx + 2\int\sec x\tan x dx + \int(\sec^2 x - 1) dx & \\
              & = \tan x + 2\sec x + \tan x - x + C                              & \\
              & = 2\tan x + 2\sec x - x + C
        \end{flalign*}

        \item $\displaystyle\int(2-\sin x)^{2} dx$
        \sol{}
        \begin{flalign*}
            I & = \int(4 - 4\sin x + \sin^2 x) dx                                               & \\
              & = \int 4 dx - 4\int\sin x dx + \dfrac{1}{2}\int(1 - \cos 2x) dx                 & \\
              & = \int 4 dx - 4\int\sin x dx + \dfrac{1}{2}\int dx - \dfrac{1}{2}\int\cos 2x dx & \\
              & = 4x + 4\cos x + \dfrac{1}{2}x - \dfrac{1}{4}\sin 2x + C                        & \\
              & = \dfrac{9}{2}x + 4\cos x - \dfrac{1}{4}\sin 2x + C
        \end{flalign*}
    \end{multicols}

    \begin{multicols}{2}
        \item $\displaystyle\int\cos^4 x dx$
        \sol{}
        \begin{flalign*}
            I & = \int\cos^2 x\cos^2 x dx                                                                         & \\
              & = \int(1 - \sin^2 x)\cos^2 x dx                                                                   & \\
              & = \int\cos^2 x dx - \int\sin^2 x\cos^2 x dx                                                       & \\
              & = \int\dfrac{1 + \cos 2x}{2} dx - \int\dfrac{1 - \cos 2x}{2}\cdot\dfrac{1 + \cos 2x}{2} dx        & \\
              & = \dfrac{1}{2}\int dx + \dfrac{1}{2}\int\cos 2x dx - \dfrac{1}{4}\int(1 - \cos^2 2x) dx           & \\
              & = \dfrac{1}{2}\int dx + \dfrac{1}{2}\int\cos 2x dx - \dfrac{1}{4}\int dx                          & \\
              & \ \ \ \ + \dfrac{1}{4}\int\dfrac{1 + \cos 4x}{2} dx                                               & \\
              & = \dfrac{1}{2}\int dx + \dfrac{1}{2}\int\cos 2x dx - \dfrac{1}{4}\int dx + \dfrac{1}{8}\int dx    & \\
              & \ \ \ \ + \dfrac{1}{8}\int\cos 4x dx                                                              & \\
              & = \dfrac{1}{2}x + \dfrac{1}{4}\sin 2x - \dfrac{1}{4}x + \dfrac{1}{16}x + \dfrac{1}{32}\sin 4x + C & \\
              & = \dfrac{3}{8}x + \dfrac{1}{4}\sin 2x + \dfrac{1}{32}\sin 4x + C
        \end{flalign*}

        \item $\displaystyle\int\left(1+\tan^{2}x\right)\left(1-\tan^{2}x\right) dx$
        \sol{}
        \begin{flalign*}
            I & = \int\left(1 - \tan^4 x\right) dx                              & \\
              & = \int dx - \int\tan^4 x dx                                     & \\
              & = \int dx - \int(\sec^2 x - 1)\tan^2 x dx                       & \\
              & = \int dx - \int\sec^2 x\tan^2 x dx + \int\tan^2 x dx           & \\
              & = \int dx - \int\sec^2 x\tan^2 x dx + \int(\sec^2 x - 1)dx      & \\
              & = \int dx - \int\sec^2 x\tan^2 x dx + \int\sec^2 x dx - \int dx & \\
              & = \int\sec^2 x dx - \int\sec^2 x\tan^2 x dx
        \end{flalign*}
        Let $u = \tan x$, $du = \sec^2 xdx$.
        \begin{flalign*}
            I & = \int\sec ^{2}x dx - \int u^{2}du  & \\
              & = \tan x - \dfrac{u^{3}}{3} + C     & \\
              & = \tan x - \dfrac{\tan^{3}x}{3} + C
        \end{flalign*}
    \end{multicols}

    \begin{multicols}{2}
        \item $\displaystyle\int\sin^{2}4x\cos4x dx$
        \sol{}

        Let $u = \sin 4x$, $du = 4\cos 4x dx$.
        \begin{flalign*}
            I & = \dfrac{1}{4}\int u^2 du             & \\
              & = \dfrac{1}{4}\cdot\dfrac{u^3}{3} + C & \\
              & = \dfrac{1}{12}\sin^3 4x + C
        \end{flalign*}

        \item $\displaystyle\int3\cot^{3}3x\csc^{2}3x dx$
        \sol{}

        Let $u = \cot 3x$, $du = -3\csc^2 3x dx$.
        \begin{flalign*}
            I & = -\int u^3 du               & \\
              & = -\dfrac{u^4}{4} + C        & \\
              & = -\dfrac{1}{4}\cot^4 3x + C
        \end{flalign*}
    \end{multicols}

    \begin{multicols}{2}
        \item $\displaystyle\int\tan^{2}x\sec^{4}x dx$
        \sol{}
        \begin{flalign*}
            I & = \int\tan^{2}x(\tan^2 x + 1)\sec^2 x dx            & \\
              & = \int\tan^4 x\sec^2 x dx + \int\tan^2 x\sec^2 x dx
        \end{flalign*}
        Let $u = \tan x$, $du = \sec^2 xdx$.
        \begin{flalign*}
            I & = \int u^4 du + \int u^2 du                       & \\
              & = \dfrac{u^5}{5} + \dfrac{u^3}{3} + C             & \\
              & = \dfrac{1}{5}\tan^5 x + \dfrac{1}{3}\tan^3 x + C
        \end{flalign*}

        \item $\displaystyle\int\sec x\cdot\tan^{3}x dx$
        \sol{}
        \begin{flalign*}
            I & = \int\sec x\cdot(\sec^2 x - 1)\tan x dx      & \\
              & = \int\sec^3 x\tan x dx - \int\sec x\tan x dx
        \end{flalign*}
        Let $u = \sec x$, $du = \sec x\tan xdx$.
        \begin{flalign*}
            I & = \int u^2 du - \int \sec x\tan xdx & \\
              & = \dfrac{u^3}{3} - \sec x + C       & \\
              & = \dfrac{1}{3}\sec^3 x - \sec x + C
        \end{flalign*}
    \end{multicols}
\end{enumerate}

\section{Integration by Partial Fractions}

Generally speaking, in order to find the indefinite integral of a proper
fraction, we first have to decompose the proper fraction into partial
fractions, then perform integration on each term. This method of integration is
called integration by partial fractions.

To find the partial fractions of an improper fraction, we can first decompose
the fraction into the sum of a polynomial and a proper fraction using
polynomial division.

\subsection{Practice 6}

\begin{enumerate}
    \begin{multicols}{2}
        \item Find the area of the region bounded by the curve $y = x^2 - 4x + 5$ and the
        line $y = x + 1$. \sol{}
        \begin{flalign*}
            x^2 - 5x + 4        & = 0 & \\
            (x - 4)(x - 1)      & = 0 & \\
            x = 4 \text{ or } x & = 1
        \end{flalign*}
        In the interval $1 \leq x \leq 4$, $x + 1 \geq x^2 - 4x + 5$
        \begin{flalign*}
            A & = \int_1^4 \left[(x + 1) - (x^2 - 4x + 5)\right] d x         & \\
              & = \int_1^4 (-x^2 + 5x - 4) d x                               & \\
              & = \bigg[-\dfrac{1}{3}x^3 + \dfrac{5}{2}x^2 - 4x\bigg]_1^4    & \\
              & = -\dfrac{64}{3} + 40 - 16 + \dfrac{1}{3} - \dfrac{5}{2} + 4 & \\
              & = \dfrac{9}{2}
        \end{flalign*}

        \item Find the area of the region bounded by the curve $x = 4y - y^2$ and the line $x
            - 2y + 3 = 0$. \sol{}
        \begin{flalign*}
            4y - y^2 - 2y + 3 & = 0 & \\
            -y^2 + 2y + 3     & = 0 & \\
            (y - 3)(y + 1)    & = 0
        \end{flalign*}
        In the interval $-1 \leq y \leq 3$, $4y - y^2 \geq 2y - 3$.
        \begin{flalign*}
            A & = \int_{-1}^3 \left[(4y - y^2) - (2y - 3)\right] d y & \\
              & = \int_{-1}^3 (-y^2 + 2y + 3) d y                    & \\
              & = \bigg[-\dfrac{1}{3}y^3 + y^2 + 3y\bigg]_{-1}^3     & \\
              & = -9 + 9 + 9 - \dfrac{1}{3} - 1 + 3                  & \\
              & = \dfrac{32}{3}
        \end{flalign*}
    \end{multicols}
\end{enumerate}


\newpage

\subsection{Exercise 27.4}

\noindent \hspace{1.2em}\textit{Find the following indefinite integral:}
\begin{enumerate}
    \begin{multicols}{2}
        \item $\displaystyle\int\dfrac{1}{x(x+1)} dx$
        \sol{}

        Let $\dfrac{1}{x(x+1)} = \dfrac{A}{x} + \dfrac{B}{x+1}$.
        \begin{flalign*}
            \dfrac{1}{x(x+1)} & = \dfrac{A}{x} + \dfrac{B}{x+1} & \\
            1                 & = Ax + A + Bx                   & \\
                              & = (A + B)x + A
        \end{flalign*}
        \vspace{-2em}
        \begin{flalign*}
            \begin{cases}
                A + B = 0 \\
                A = 1
            \end{cases}
            \Rightarrow
            \begin{cases}
                A = 1 \\
                B = -1
            \end{cases}
        \end{flalign*}
        \vspace{-1em}
        \begin{flalign*}
            I & = \int\left(\dfrac{1}{x} - \dfrac{1}{x+1}\right) dx & \\
              & = \ln|x| - \ln|x+1| + C
        \end{flalign*}

        \item $\displaystyle\int\dfrac{x}{(x+1)(x-3)} dx$
        \sol{}

        Let $\dfrac{x}{(x+1)(x-3)} = \dfrac{A}{x+1} + \dfrac{B}{x-3}$.
        \begin{flalign*}
            \dfrac{x}{(x+1)(x-3)} & = \dfrac{A}{x+1} + \dfrac{B}{x-3} & \\
            x                     & = Ax - 3A + Bx + B                & \\
                                  & = (A + B)x + (-3A + B)
        \end{flalign*}
        \vspace{-2em}
        \begin{flalign*}
            \begin{cases}
                A + B = 1 \\
                -3A + B = 0
            \end{cases}
            \Rightarrow
            \begin{cases}
                A = \dfrac{1}{4} \\
                B = \dfrac{3}{4}
            \end{cases}
        \end{flalign*}
        \vspace{-1em}
        \begin{flalign*}
            I & = \int\left(\dfrac{1}{4}\cdot\dfrac{1}{x+1} + \dfrac{3}{4}\cdot\dfrac{1}{x-3}\right) dx & \\
              & = \dfrac{1}{4}\ln|x+1| + \dfrac{3}{4}\ln|x-3| + C
        \end{flalign*}
    \end{multicols}

    \begin{multicols}{2}
        \item $\displaystyle\int\dfrac{4x-13}{2x^2+x-6} dx$
        \sol{}

        Let $\dfrac{4x-13}{2x^2+x-6} = \dfrac{A}{2x-3} + \dfrac{B}{x+2}$.
        \begin{flalign*}
            \dfrac{4x-13}{2x^2+x-6} & = \dfrac{A}{2x-3} + \dfrac{B}{x+2} & \\
            4x - 13                 & = Ax + 2A + 2Bx - 3B               & \\
                                    & = (A + 2B)x + (2A - 3B)
        \end{flalign*}
        \vspace{-2em}
        \begin{flalign*}
            \begin{cases}
                A + 2B = 4 \\
                2A - 3B = -13
            \end{cases}
            \Rightarrow
            \begin{cases}
                A = -2 \\
                B = 3
            \end{cases}
        \end{flalign*}
        \vspace{-1em}
        \begin{flalign*}
            I & = \int\left(\dfrac{-2}{2x-3} + \dfrac{3}{x+2}\right) dx & \\
              & = 3\ln|x+2| -\ln|2x-3|+ C
        \end{flalign*}

        \item $\displaystyle\int\dfrac{5x-1}{1-x^{2}} dx$
        \sol{}

        Let $\dfrac{5x-1}{1-x^{2}} = \dfrac{A}{1-x} + \dfrac{B}{1+x}$.
        \begin{flalign*}
            \dfrac{5x-1}{1-x^{2}} & = \dfrac{A}{1-x} + \dfrac{B}{1+x} & \\
            5x - 1                & = A + Ax + B - Bx                 & \\
                                  & = (A - B)x + (A + B)
        \end{flalign*}
        \vspace{-2em}
        \begin{flalign*}
            \begin{cases}
                A - B = 5 \\
                A + B = -1
            \end{cases}
            \Rightarrow
            \begin{cases}
                A = 2 \\
                B = -3
            \end{cases}
        \end{flalign*}
        \vspace{-1em}
        \begin{flalign*}
            I & = \int\left(\dfrac{2}{1-x} - \dfrac{3}{1+x}\right) dx & \\
              & = -2\ln|1-x| - 3\ln|1+x| + C
        \end{flalign*}
    \end{multicols}

    \newpage

    \begin{multicols}{2}
        \item $\displaystyle\int\dfrac{x^{2}+5}{(x+1)(x-1)} dx$
        \sol{}
        \begin{flalign*}
            I & = \int\dfrac{x^{2}+5}{x^2 - 1}               & \\
              & = \int\left(1 + \dfrac{6}{x^2 - 1}\right) dx
        \end{flalign*}
        Let $\dfrac{6}{x^2 - 1} = \dfrac{A}{x-1} + \dfrac{B}{x+1}$.
        \begin{flalign*}
            \dfrac{6}{x^2 - 1} & = \dfrac{A}{x-1} + \dfrac{B}{x+1} & \\
            6                  & = Ax + A + Bx - B                 & \\
                               & = (A + B)x + (A - B)
        \end{flalign*}
        \vspace{-2em}
        \begin{flalign*}
            \begin{cases}
                A + B = 0 \\
                A - B = 6
            \end{cases}
            \Rightarrow
            \begin{cases}
                A = 3 \\
                B = -3
            \end{cases}
        \end{flalign*}
        \vspace{-1em}
        \begin{flalign*}
            I & = \int\left(1 + \dfrac{3}{x-1} - \dfrac{3}{x+1}\right) dx & \\
              & = x + 3\ln|x-1| - 3\ln|x+1| + C
        \end{flalign*}

        \item $\displaystyle\int\dfrac{x^{3}+2}{x^{2}-1} dx$
        \sol{}
        \begin{flalign*}
            I & = \int\dfrac{x^{3}+2}{x^{2}-1} dx                 & \\
              & = \int \left(x + \dfrac{x + 2}{x^{2}-1}\right) dx
        \end{flalign*}
        Let $\dfrac{x + 2}{x^{2}-1} = \dfrac{A}{x-1} + \dfrac{B}{x+1}$.
        \begin{flalign*}
            \dfrac{x + 2}{x^{2}-1} & = \dfrac{A}{x-1} + \dfrac{B}{x+1} & \\
            x + 2                  & = Ax + A + Bx - B                 & \\
                                   & = (A + B)x + (A - B)
        \end{flalign*}
        \vspace{-2em}
        \begin{flalign*}
            \begin{cases}
                A + B = 1 \\
                A - B = 2
            \end{cases}
            \Rightarrow
            \begin{cases}
                A = \dfrac{3}{2} \\
                B = -\dfrac{1}{2}
            \end{cases}
        \end{flalign*}
        \vspace{-1em}
        \begin{flalign*}
            I & = \int\left(x + \dfrac{3}{2}\cdot\dfrac{1}{x-1} - \dfrac{1}{2}\cdot\dfrac{1}{x+1}\right) dx & \\
              & = \dfrac{1}{2}x^2 + \dfrac{3}{2}\ln|x-1| - \dfrac{1}{2}\ln|x+1| + C
        \end{flalign*}
    \end{multicols}

    \begin{multicols}{2}
        \item $\displaystyle\int\dfrac{2x-1}{x^{2}+2x+1} dx$
        \sol{}
        \begin{flalign*}
            I & = \int\dfrac{2x-1}{(x+1)^2} dx
        \end{flalign*}
        Let $u = x + 1$, $du = dx$, $x = u - 1$.
        \begin{flalign*}
            I & = \int\dfrac{2u - 3}{u^2} du     & \\
              & = \int(2u^{-1} - 3u^{-2}) du     & \\
              & = 2\ln|u| + \dfrac{3}{u} + C     & \\
              & = 2\ln|x+1| + \dfrac{3}{x+1} + C
        \end{flalign*}

        \item $\displaystyle\int\dfrac{4x-3}{(2x+1)^{2}} dx$
        \sol{}

        Let $u = 2x + 1$, $du = 2dx$, $x = \dfrac{u - 1}{2}$.
        \begin{flalign*}
            I & = \dfrac{1}{2}\int\dfrac{2u - 5}{u^2}du  & \\
              & = \dfrac{1}{2}\int(2u^{-1} - 5u^{-2})du  & \\
              & = \ln|u| + \dfrac{5}{2u} + C             & \\
              & = \ln|2x + 1| + \dfrac{5}{2(2x + 1)} + C
        \end{flalign*}
    \end{multicols}
\end{enumerate}

\newpage
\section{Applications of Indefinite Integrals}

We can use derivative to find the slope of a tangent line to a curve at any
point on the curve. Conversely, if the slope of a tangent line to a curve at a
point on the curve is known, we can use indefinite integration to find the
equation of the curve.

\subsection{Practice 7}

\begin{enumerate}
    \item Find the volume of a cone with radius $r$ and height $h$ using definite
          integrals. \sol{}

          Let $y$ be the height of any cross section of the cone, and $x$ be the radius
          of the cross section.
          \begin{flalign*}
              \dfrac{y}{h} & = \dfrac{x}{r}   & \\
              y            & = \dfrac{h}{r} x
          \end{flalign*}
          \begin{flalign*}
              V & = \int_0^h \pi y^2 d x                                  & \\
                & = \int_0^h \pi \left(\dfrac{h}{r} x\right)^2 d x        & \\
                & = \pi \dfrac{h^2}{r^2} \int_0^h x^2 d x                 & \\
                & = \pi \dfrac{h^2}{r^2} \left[\dfrac{1}{3}x^3\right]_0^h & \\
                & = \pi \dfrac{h^2}{r^2} \cdot \dfrac{1}{3}h^3            & \\
                & = \dfrac{1}{3} \pi r^2 h
          \end{flalign*}

    \item Shown in the diagram below is the shaded region bounded by the ellipse
          $\dfrac{x^2}{a^2} + \dfrac{y^2}{b^2} = 1$, where $a > 0$ and $b > 0$. If the
          volume of the solid of revolution formed by rotating this region about the
          $x$-axis and the $y$-axis is $V_x$ and $V_y$ respectively,
          \begin{center}
              \includegraphics[width=0.3\textwidth]{assets/28-prac7-2.png}
          \end{center}
          \begin{enumerate}
              \item find $V_x$ and $V_y$. \sol{} \vspace{-0.8cm}
                    \begin{multicols}{2}
                        \begin{flalign*}
                            \dfrac{x^2}{a^2} + \dfrac{y^2}{b^2} & = 1                                    & \\
                            \dfrac{y^2}{b^2}                    & = 1 - \dfrac{x^2}{a^2}                 & \\
                            y^2                                 & = b^2\left(1 - \dfrac{x^2}{a^2}\right)
                        \end{flalign*}
                        \begin{flalign*}
                            V_x & = \int_{-a}^a \pi y^2 d x                                                  & \\
                                & = \pi b^2 \int_{-a}^a \left(1 - \dfrac{x^2}{a^2}\right) d x                & \\
                                & = \pi b^2 \left[x - \dfrac{x^3}{3a^2}\right]_{-a}^a                        & \\
                                & = \pi b^2 \left[a - \dfrac{a^3}{3a^2} - (-a) + \dfrac{(-a)^3}{3a^2}\right] & \\
                                & = \pi b^2 \left[a - \dfrac{a}{3} + a - \dfrac{a}{3}\right]                 & \\
                                & = \dfrac{4}{3} \pi a b^2
                        \end{flalign*}
                        \vfill\null\columnbreak
                        \begin{flalign*}
                            \dfrac{x^2}{a^2} + \dfrac{y^2}{b^2} & = 1                                    & \\
                            \dfrac{x^2}{a^2}                    & = 1 - \dfrac{y^2}{b^2}                 & \\
                            x^2                                 & = a^2\left(1 - \dfrac{y^2}{b^2}\right)
                        \end{flalign*}
                        \begin{flalign*}
                            V_y & = \int_{-b}^b \pi x^2 d y                                                  & \\
                                & = \pi a^2 \int_{-b}^b \left(1 - \dfrac{y^2}{b^2}\right) d y                & \\
                                & = \pi a^2 \left[y - \dfrac{y^3}{3b^2}\right]_{-b}^b                        & \\
                                & = \pi a^2 \left[b - \dfrac{b^3}{3b^2} - (-b) + \dfrac{(-b)^3}{3b^2}\right] & \\
                                & = \pi a^2 \left[b - \dfrac{b}{3} + b - \dfrac{b}{3}\right]                 & \\
                                & = \dfrac{4}{3} \pi a^2 b
                        \end{flalign*}
                    \end{multicols}
                    \vspace{-1cm}

              \item if $V_x = 3V_y$, find the value of $a:b$. \sol{}
                    \begin{flalign*}
                        \dfrac{4}{3} \pi a b^2 & = 3 \cdot \dfrac{4}{3} \pi a^2 b & \\
                        4\pi a b^2             & = 12 \pi a^2 b                   & \\
                        b                      & = 3a                             & \\
                        a:b                    & = 1:3
                    \end{flalign*}
          \end{enumerate}
\end{enumerate}
\subsection{Exercise 27.5}

\begin{enumerate}
    \item Given that $\dfrac{dy}{dx} = 4x^3 - 6x^2 + 3$, and when $x = 2$, $y = 7$.
          Express $y$ in terms of $x$. \sol{}
          \begin{flalign*}
              \dfrac{dy}{dx} & = 4x^3 - 6x^2 + 3         & \\
              dy             & = (4x^3 - 6x^2 + 3)dx     & \\
              \int dy        & = \int(4x^3 - 6x^2 + 3)dx & \\
              y              & = x^4 - 2x^3 + 3x + C
          \end{flalign*}
          When $x = 2$, $y = 7$.
          \begin{flalign*}
              7 & = 16 - 16 + 6 + C & \\
              C & = 1
          \end{flalign*}
          $\therefore$ $y = x^4 - 2x^3 + 3x + 1$.
          \newpage
    \item The gradient of the tangent line of a point on a curve is $\dfrac{dy}{dx} = x^2
              + 2x - 4$, and the curve passes through point $(3, 3)$. Find the equation of
          the curve. \sol{}
          \begin{flalign*}
              \dfrac{dy}{dx} & = x^2 + 2x - 4                   & \\
              dy             & = (x^2 + 2x - 4)dx               & \\
              \int dy        & = \int(x^2 + 2x - 4)dx           & \\
              y              & = \dfrac{1}{3}x^3 + x^2 - 4x + C
          \end{flalign*}
          When $x = 3$, $y = 3$.
          \begin{flalign*}
              3 & = 9 + 9 - 12 + C & \\
              C & = -3
          \end{flalign*}
          $\therefore$ The equation of the curve is $y = \dfrac{1}{3}x^3 + x^2 - 4x - 3$.

    \item The gradient of a curve at the point $(1, -1)$ is $-4$, and $\dfrac{dy}{dx} =
              \sqrt{x} + k$. Find
          \begin{enumerate}
              \item The value of $k$. \sol{}

                    When $x = 1$, $\dfrac{dy}{dx} = -4$
                    \begin{flalign*}
                        -4 & = \sqrt{1} + k & \\
                        k  & = -5
                    \end{flalign*}

              \item The equation of the curve. \sol{}
                    \begin{flalign*}
                        \dfrac{dy}{dx} & = \sqrt{x} - 5                         & \\
                        dy             & = (\sqrt{x} - 5)dx                     & \\
                        \int dy        & = \int(\sqrt{x} - 5)dx                 & \\
                        y              & = \dfrac{2}{3}x^{\frac{3}{2}} - 5x + C
                    \end{flalign*}
                    When $x = 1$, $y = -1$.
                    \begin{flalign*}
                        -1 & = \dfrac{2}{3} - 5 + C & \\
                        C  & = \dfrac{10}{3}
                    \end{flalign*}
                    $\therefore$ The equation of the curve is $y = \dfrac{2}{3}x^{\frac{3}{2}} - 5x + \dfrac{10}{3}$.

          \end{enumerate}
\end{enumerate}

\newpage
\section{Revision Exercise 27}

Find the following indefinite integral (Question 1 to 34):
\begin{enumerate}
    \begin{multicols}{2}
        \item $\displaystyle\int 2x^{\frac{1}{5}}dx$
        \sol{}
        \begin{flalign*}
            I & = 2 \cdot \dfrac{5}{6}x^{\frac{6}{5}} + C & \\
              & = \dfrac{5}{3}x^{\frac{6}{5}} + C
        \end{flalign*}
        \vfill{}\null{}
        \item $\displaystyle\int{(2x-1)}^3dx$
        \sol{}
        \begin{flalign*}
            I & = \dfrac{1}{2}\int{(2x-1)}^3d(2x-1)             & \\
              & = \dfrac{1}{2} \cdot \dfrac{1}{4}{(2x-1)}^4 + C & \\
              & = \dfrac{1}{8}{(2x-1)}^4 + C
        \end{flalign*}
    \end{multicols}

    \begin{multicols}{2}
        \item $\displaystyle\int{(x+4)}^{100}dx$
        \sol{}
        \begin{flalign*}
            I & = \int{(x+4)}^{100}d(x+4)         & \\
              & = \dfrac{1}{101}{(x+4)}^{101} + C
        \end{flalign*}
        \item $\displaystyle\int{\left(\dfrac{5}{x^2}+2x^{\frac{1}{2}}+3\right)}dx$
        \sol{}
        \begin{flalign*}
            I & =5\int{x^{-2}} + 2\int{x^{\frac{1}{2}}} + 3\int dx     & \\
              & = -\dfrac{5}{x} + \dfrac{4}{3}x^{\frac{3}{2}} + 3x + C
        \end{flalign*}
    \end{multicols}

    \begin{multicols}{2}
        \item $\displaystyle\int{\left(3x^2 + \dfrac{1}{x^2} - \sin x\right)}dx$
        \sol{}
        \begin{flalign*}
            I & = 3\int x^2 + \int x^{-2} - \int \sin x dx & \\
              & = x^3 - x^{-1} + \cos x + C                & \\
              & = x^3 + \dfrac{1}{x} + \cos x + C
        \end{flalign*}
        \item $\displaystyle\int{\left(4\cos x + \dfrac{1}{x} + x^3\right)}dx$
        \sol{}
        \begin{flalign*}
            I & = 4\int \cos x + \int x^{-1} + \int x^3 dx          & \\
              & = 4\sin x + \ln \vert x \vert + \dfrac{1}{4}x^4 + C
        \end{flalign*}
    \end{multicols}

    \begin{multicols}{2}
        \item $\displaystyle\int\dfrac{3x^3 - 2x^2 + x^{-1}}{x^2}dx$
        \sol{}
        \begin{flalign*}
            I & = \int(3x - 2 + x^{-3})dx                       & \\
              & = \dfrac{3}{2}x^2 - 2x - \dfrac{1}{2}x^{-2} + C
        \end{flalign*}
        \item $\displaystyle\int(2x-1)(x+2)dx$
        \sol{}
        \begin{flalign*}
            I & = \int(2x^2 + 3x - 2)dx                      & \\
              & = \dfrac{2}{3}x^3 + \dfrac{3}{2}x^2 - 2x + C
        \end{flalign*}
    \end{multicols}

    \begin{multicols}{2}
        \item $\displaystyle\int{\left(x-\dfrac{1}{x^2}\right)}^2dx$
        \sol{}
        \begin{flalign*}
            I & = \int(x^2 - 2x^{-1} + x^{-4})dx                                & \\
              & = \dfrac{1}{3}x^3 - 2\ln \vert x \vert - \dfrac{1}{3}x^{-3} + C
        \end{flalign*}
        \item $\displaystyle\int{\left(x + \dfrac{1}{x}\right)}^3dx$
        \sol{}
        \begin{flalign*}
            I & = \int(x^3 + 3x + 3x^{-1} + x^{-3})dx                                             & \\
              & = \dfrac{1}{4}x^4 + \dfrac{3}{2}x^2 + 3\ln \vert x \vert - \dfrac{1}{2}x^{-2} + C
        \end{flalign*}
    \end{multicols}
    \newpage
    \begin{multicols}{2}
        \item $\displaystyle\int10^{-x}dx$
        \sol{}
        \begin{flalign*}
            I & = -\int10^{-x}d(-x)            & \\
              & = -\dfrac{10^{-x}}{\ln 10} + C & \\
              & = -\dfrac{1}{10^{x}\ln 10} + C
        \end{flalign*}
        \item $\displaystyle\int{\left(e^x - e^{-x}\right)}^2dx$
        \sol{}
        \begin{flalign*}
            I & = \int(e^{2x} - 2 + e^{-2x})dx                      & \\
              & = \dfrac{1}{2}e^{2x} - 2x - \dfrac{1}{2}e^{-2x} + C
        \end{flalign*}
    \end{multicols}

    \begin{multicols}{2}
        \item $\displaystyle\int2x{(x^2 - 1)}^4dx$
        \sol{}
        \begin{flalign*}
            I & = \int{(x^2 - 1)}^4d(x^2 - 1)   & \\
              & = \dfrac{1}{5}{(x^2 - 1)}^5 + C
        \end{flalign*}
        \item $\displaystyle\int3x^2{(x^3 + 1)}^4dx$
        \sol{}
        \begin{flalign*}
            I & = \int{(x^3 + 1)}^4d(x^3 + 1)   & \\
              & = \dfrac{1}{5}{(x^3 + 1)}^5 + C
        \end{flalign*}
    \end{multicols}

    \begin{multicols}{2}
        \item $\displaystyle\int\dfrac{x+1}{{(x^2 + 2x + 5)}^3}dx$
        \sol{}
        \begin{flalign*}
            I & = \dfrac{1}{2}\int\dfrac{1}{(x^2 + 2x + 5)^3}d(x^2 + 2x + 5) & \\
              & = -\dfrac{1}{4(x^2 + 2x + 5)^2} + C
        \end{flalign*}
        \item $\displaystyle\int\dfrac{2x}{\sqrt{x^2-4}}dx$
        \sol{}
        \begin{flalign*}
            I & = \int\dfrac{1}{\sqrt{x^2-4}}d(x^2-4) & \\
              & = 2\sqrt{x^2-4} + C
        \end{flalign*}
    \end{multicols}

    \begin{multicols}{2}
        \item $\displaystyle\int\dfrac{x-2}{\sqrt{(x-1)(x-3)}}dx$
        \sol{}
        \begin{flalign*}
            I & = \int\dfrac{x-2}{\sqrt{x^2-4x+3}}dx                    & \\
              & = \dfrac{1}{2}\int\dfrac{1}{\sqrt{x^2-4x+3}}d(x^2-4x+3) & \\
              & = \dfrac{1}{2}\cdot 2\sqrt{x^2-4x+3} + C                & \\
              & = \sqrt{x^2-4x+3} + C
        \end{flalign*}
        \vfill{}\null{}
        \columnbreak
        \item $\displaystyle\int\dfrac{7}{2x^2 + 5x - 3}dx$
        \sol{}
        \begin{flalign*}
            I = \int\dfrac{7}{(2x - 1)(x + 3)}dx &
        \end{flalign*}
        \vspace{-2em}
        \begin{flalign*}
            \text{Let } \dfrac{7}{(2x - 1)(x + 3)} & = \dfrac{A}{2x - 1} + \dfrac{B}{x + 3} & \\
            A(x + 3) + B(2x - 1)                   & = 7                                    & \\
            (A + 2B)x + (3A - B)                   & = 7
        \end{flalign*}
        Comparing coefficients,
        \begin{flalign*}
            A + 2B    & = 0  & \\
            3A - B    & = 7  & \\
            A = 2,\ B & = -1
        \end{flalign*}
        \vspace{-2em}
        \begin{flalign*}
            I & = \int\left(\dfrac{2}{2x-1} - \dfrac{1}{x + 3}\right)dx & \\
              & = \int\dfrac{2}{2x-1}dx - \int\dfrac{1}{x + 3}dx        & \\
              & = \ln|2x - 1| - \ln|x + 3| + C                          & \\
              & = \ln\left|\dfrac{2x - 1}{x + 3}\right| + C
        \end{flalign*}
    \end{multicols}

    \begin{multicols}{2}
        \item $\displaystyle\int\dfrac{8 - 7x}{2 + x - 3x^2}dx$
        \sol{}
        \begin{flalign*}
            I & = -\int\dfrac{7x + 8}{(3x + 2)(x - 1)}dx
        \end{flalign*}
        \vspace{-2em}
        \begin{flalign*}
            \text{Let } \dfrac{7x + 8}{(3x + 2)(x - 1)} & = \dfrac{A}{3x + 2} + \dfrac{B}{x - 1} & \\
            A(x - 1) + B(3x + 2)                        & = 7x + 8                               & \\
            (A + 3B)x + (-A + 2B)                       & = 7x + 8
        \end{flalign*}
        Comparing coefficients,
        \begin{flalign*}
            A + 3B     & = 7 & \\
            -A + 2B    & = 8 & \\
            A = -2,\ B & = 3
        \end{flalign*}
        \vspace{-2em}
        \begin{flalign*}
            I & = -\int\left(-\dfrac{2}{3x + 2} + \dfrac{3}{x - 1}\right)dx & \\
              & = -\int-\dfrac{2}{3x + 2}dx - \int\dfrac{3}{x - 1}dx        & \\
              & = \dfrac{2}{3}\ln|3x + 2| - 3\ln|x - 1| + C                 & \\
        \end{flalign*}
        \item $\displaystyle\int\dfrac{x+1}{(3x+2)(5x+3)}dx$
        \sol{}
        \vspace{-2em}
        \begin{flalign*}
            \text{Let } \dfrac{x+1}{(3x+2)(5x+3)} & = \dfrac{A}{3x+2} + \dfrac{B}{5x+3} & \\
            A(5x + 3) + B(3x + 2)                 & = x + 1                             & \\
            (5A + 3B)x + (3A + 2B)                & = x + 1
        \end{flalign*}
        Comparing coefficients,
        \begin{flalign*}
            5A + 3B    & = 1 & \\
            3A + 2B    & = 1 & \\
            A = -1,\ B & = 2
        \end{flalign*}
        \begin{flalign*}
            I & = \int\left(\dfrac{-1}{3x+2} + \dfrac{2}{5x+3}\right)dx  & \\
              & = -\int\dfrac{1}{3x+2}dx + \int\dfrac{2}{5x+3}dx         & \\
              & = -\dfrac{1}{3}\ln|3x + 2| + \dfrac{2}{5}\ln|5x + 3| + C & \\
        \end{flalign*}
    \end{multicols}

    \begin{multicols}{2}
        \item $\displaystyle\int\dfrac{2x^2 + 5x - 2}{2x^2 + x - 3}dx$
        \sol{}
        \begin{flalign*}
            I & = \int\left(1 + \dfrac{4x + 1}{2x^2 + x - 3}\right)dx          & \\
              & = \int\left[1 + \dfrac{(2x^2 + x - 3)'}{2x^2 + x - 3}\right]dx & \\
              & = x + \ln\vert2x^2 + x - 3\vert + C
        \end{flalign*}
        \item $\displaystyle\int{\left(\dfrac{x+1}{x-1}\right)}^2dx$
        \sol{}
        \begin{flalign*}
            I & = \int\left(1 + \dfrac{2}{x-1}\right)^2dx                    & \\
              & = \int\left[1 + \dfrac{4}{x-1} + \dfrac{4}{(x-1)^2}\right]dx & \\
              & = x + 4\ln\vert x - 1\vert - \dfrac{4}{x - 1} + C
        \end{flalign*}
    \end{multicols}

    \begin{multicols}{2}
        \item $\displaystyle\int\dfrac{{(x-1)}^3}{{(x-2)}^2}dx$
        \sol{}

        Let $u = x - 2$, $du = dx$.
        \begin{flalign*}
            I & = \int\dfrac{{(u+1)}^3}{u^2}du                                                &   & \\
              & = \int\left(\dfrac{u^3 + 3u^2 + 3u + 1}{u^2}\right)du                         &     \\
              & = \int\left(u + 3 + \dfrac{3}{u} + \dfrac{1}{u^2}\right)du                    &     \\
              & = \dfrac{u^2}{2} + 3u + 3\ln\vert u\vert - \dfrac{1}{u} + C                   &     \\
              & = \dfrac{x^2 - 4x + 4 + 6x - 12}{2} + 3\ln\vert x-2\vert - \dfrac{1}{x-2} + C &     \\
              & = \dfrac{1}{2}x^2 + x + 3\ln\vert x-2\vert - \dfrac{1}{x-2} + C
        \end{flalign*}
        \item $\displaystyle\int\dfrac{x^2}{{(x+2)}^3}dx$
        \sol{}

        Let $u = x + 2$, $du = dx$.
        \begin{flalign*}
            I & = \int\dfrac{{(u-2)}^2}{u^3}du                                      & \\
              & = \int\left(\dfrac{u^2 - 4u + 4}{u^3}\right)du                      & \\
              & = \int\left(\dfrac{1}{u} - \dfrac{4}{u^2} + \dfrac{4}{u^3}\right)du & \\
              & = \ln\vert u\vert + \dfrac{4}{u} - \dfrac{2}{u^2} + C               & \\
              & = \ln\vert x+2\vert + \dfrac{4}{x+2} - \dfrac{2}{(x+2)^2} + C
        \end{flalign*}
    \end{multicols}

    \begin{multicols}{2}
        \item $\displaystyle\int\left(3\sin2x-4e^{3x}\right)dx$
        \sol{}
        \begin{flalign*}
            I & = \dfrac{3}{2}\int\sin2xd(2x) - \int4e^{3x}dx  & \\
              & = -\dfrac{3}{2}\cos2x - \dfrac{4}{3}e^{3x} + C & \\
        \end{flalign*}
        \item $\displaystyle\int\sin(5x-6)dx$
        \sol{}
        \begin{flalign*}
            I & = \dfrac{1}{5}\int\sin(5x-6)d(5x-6) & \\
              & = -\dfrac{1}{5}\cos(5x-6) + C
        \end{flalign*}
    \end{multicols}

    \begin{multicols}{2}
        \item $\displaystyle\int\left(\cos6x+\sec^2 4x\right)dx$
        \sol{}
        \begin{flalign*}
            I & = \dfrac{1}{6}\int\cos6xd(6x) + \dfrac{1}{4}\int\sec^2 4xd(4x) & \\
              & = \dfrac{1}{6}\sin6x + \dfrac{1}{4}\tan4x + C
        \end{flalign*}
        \item $\displaystyle\int\left(\sin\dfrac{x}{2}+\cos2x-\cos\dfrac{x}{7}\right)dx$
        \sol{}
        \begin{flalign*}
            I & = 2\int\sin\dfrac{x}{2}d\left(\dfrac{x}{2}\right) + \int\cos2xd(2x) & \\
              & \ \ \ \ - 7\int\cos\dfrac{x}{7}d\left(\dfrac{x}{7}\right)           & \\
              & = -2\cos\dfrac{x}{2} + \dfrac{1}{2}\sin2x - 7\sin\dfrac{x}{7} + C
        \end{flalign*}
    \end{multicols}

    \begin{multicols}{2}
        \item $\displaystyle\int\tan^2 3xdx$
        \sol{}
        \begin{flalign*}
            I & = \int\left(\sec^2 3x - 1\right)dx         & \\
              & = \dfrac{1}{3}\int\sec^2 3xd(3x) - \int dx & \\
              & = \dfrac{1}{3}\tan3x - x + C
        \end{flalign*}
        \item $\displaystyle\int\tan x\sec^2 xdx$
        \sol{}
        \begin{flalign*}
            I & = \int\tan xd(\tan x)      & \\
              & = \dfrac{1}{2}\tan^2 x + C
        \end{flalign*}
    \end{multicols}

    \begin{multicols}{2}
        \item $\displaystyle\int\dfrac{3\sin x}{\cos2x + 1}dx$
        \sol{}
        \begin{flalign*}
            I & = \int\dfrac{3\sin x}{2\cos^2 x}dx           & \\
              & = \dfrac{3}{2}\int\dfrac{\sin x}{\cos^2 x}dx & \\
              & = \dfrac{3}{2}\int\sec x\tan xdx             & \\
              & = \dfrac{3}{2}\sec x + C
        \end{flalign*}
        \item $\displaystyle\int\dfrac{\sec^2 x}{\tan x + 2}dx$
        \sol{}
        \begin{flalign*}
            I & = \int\dfrac{1}{\tan x + 2}d(\tan x) & \\
              & = \ln\vert\tan x + 2\vert + C
        \end{flalign*}
    \end{multicols}

    \begin{multicols}{2}
        \item $\displaystyle\int\cot2x\csc^3 2xdx$
        \sol{}
        \begin{flalign*}
            I & = \int\csc 2x \cot 2x \csc^2 2xdx     & \\
              & = \dfrac{1}{2}\int\csc^2 2xd(\csc 2x) & \\
              & = -\dfrac{1}{6}\csc^3 2x + C
        \end{flalign*}
        \vfill{}\null{}
        \item $\displaystyle\int\tan^3 x\sec^3 xdx$
        \sol{}
        \begin{flalign*}
            I & = \int\tan^2 x\sec^2 x\sec x\tan xdx                  & \\
              & = \int\left(\sec^2 x - 1\right)\sec^2 x\sec x\tan xdx & \\
              & = \int\left(\sec^4 x - \sec^2 x\right)d(\sec x)       & \\
              & = \dfrac{1}{5}\sec^5 x - \dfrac{1}{3}\sec^3 x + C
        \end{flalign*}
    \end{multicols}

    \item If the function $y = \ln x - \dfrac{3}{x}$, find $\dfrac{dy}{dx}$. Hence, find
          $\displaystyle\int\dfrac{3+x}{3x^2}dx$. \sol{}
          \begin{flalign*}
              y                       & = \ln x - \dfrac{3}{x}                   & \\
              \dfrac{dy}{dx}          & = \dfrac{1}{x} + \dfrac{3}{x^2}          & \\
                                      & = \dfrac{x+3}{x^2}                       & \\
                                      &                                            \\
              \int\dfrac{3+x}{3x^2}dx & = \dfrac{1}{3}\int\dfrac{x+3}{x^2}d(x^3) & \\
                                      & = \dfrac{1}{3}\ln x - \dfrac{1}{x} + C
          \end{flalign*}

    \item If the function $y = \dfrac{1}{\sqrt{4x^2 - 1}}$, find $\dfrac{dy}{dx}$. Hence,
          find $\displaystyle\int\dfrac{x}{\sqrt{{\left(4x^2-1\right)}^3}}$. \sol{}
          \begin{flalign*}
              y                                               & = \left[\left(4x^2 - 1\right)^{-\dfrac{1}{2}}\right]'                 & \\
                                                              & = -\dfrac{1}{2}\left(4x^2 - 1\right)^{-\dfrac{3}{2}}\cdot 8x          & \\
                                                              & = -\dfrac{4x}{\sqrt{{\left(4x^2-1\right)}^3}}                         & \\
                                                              &                                                                         \\
              \int\dfrac{x}{\sqrt{{\left(4x^2-1\right)}^3}}dx & = -\dfrac{1}{4}\int-\dfrac{4x}{\sqrt{{\left(4x^2-1\right)}^3}}d(4x^2) & \\
                                                              & = -\dfrac{1}{4\sqrt{4x^2-1}} + C
          \end{flalign*}

    \item Given the function $y = \dfrac{x^2 + 3}{1-x}$, and $\dfrac{dy}{dx} =
              \dfrac{1}{2}f(x)$, find $\displaystyle\int\left[3 - x^2 - f(x)\right]dx$.
          \sol{}
          \begin{flalign*}
              \dfrac{d}{dx}\left(\dfrac{x^2 + 3}{1-x}\right) & = \dfrac{1}{2}f(x)                                                                           & \\
              f(x)                                           & = 2\left[\dfrac{d}{dx}\left(\dfrac{x^2 + 3}{1-x}\right)\right]                               & \\
                                                             &                                                                                                \\
              \int\left[3 - x^2 - f(x)\right]dx              & = \int3dx - \int x^2dx - f(x)dx                                                              & \\
                                                             & = \int3dx - \int x^2dx - \int 2\left[\dfrac{d}{dx}\left(\dfrac{x^2 + 3}{1-x}\right)\right]dx & \\
                                                             & = 3x - \dfrac{x^3}{3} - \dfrac{2(x^2 + 3)}{1-x} + C
          \end{flalign*}

          \newpage

    \item Given the function $\dfrac{d}{dx}(x\ln x) = g(x)$, find
          $\displaystyle\int\left[g(x) - 2x\right]dx$. \sol{}
          \begin{flalign*}
              \int\left[g(x) - 2x\right]dx & = \int g(x)dx - \int 2xdx                  & \\
                                           & = \int \dfrac{d}{dx}(x\ln x)dx - \int 2xdx & \\
                                           & = x\ln x - x^2 + C
          \end{flalign*}

    \item The gradient of the tangent at any point on a curve is 3 times the
          $x$-coordinate of the point, and the curve passes through $(-2, 5)$. Find the
          equation of the curve. \sol{}
          \begin{flalign*}
              \dfrac{dy}{dx} & = 3x                  & \\
              dy             & = 3xdx                & \\
              y              & = \dfrac{3}{2}x^2 + C
          \end{flalign*}
          Given that the curve passes through $(-2, 5)$,
          \begin{flalign*}
              5 & = \dfrac{3(-2)^2}{2} + C & \\
              C & = 1
          \end{flalign*}
          Therefore, the equation of the curve is $y = \dfrac{3}{2}x^2 + 1$.

    \item The gradient of the tangent at any point on a curve is $\dfrac{dy}{dx} = 3x^2 -
              8x + 1$, and the curve intersect with $x$-axis at point $(2, 0)$, find the
          other point of intersection of the curve and the $x$-axis. \sol{}
          \begin{flalign*}
              \dfrac{dy}{dx} & = 3x^2 - 8x + 1          & \\
              y              & = \int (3x^2 - 8x + 1)dx & \\
                             & = x^3 - 4x^2 + x + C
          \end{flalign*}
          Given that the curve passes through $(2, 0)$,
          \begin{flalign*}
              0 & = 2^3 - 4(2)^2 + 2 + C & \\
              C & = -6
          \end{flalign*}
          Therefore, the equation of the curve is $y = x^3 - 4x^2 + x -6$.

          When the curve intersects with the $x$-axis, $y = 0$,
          \begin{flalign*}
              x^3 - 4x^2 + x + 6 & = 0 & \\
              (x-2)(x^2-2x-3)    & = 0 & \\
              (x-2)(x-3)(x+1)    & = 0
          \end{flalign*}
          Therefore, the curve also intersects with the $x$-axis at $(-1, 0)$ and $(3, 0)$.
\end{enumerate}
