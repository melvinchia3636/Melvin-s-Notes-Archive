\chapter{Definite Integrals}

\section{Concept of Definite Integrals and their Relationship with Indefinite Integrals}

\subsection*{Concept of Definite Integrals}

A lot of practical problems, for example finding area and volumes, can be
reduced to finding the limit of a certain type of sum. Let's take finding area
as an example to explain the method of solving this kind of problem, and hence
introduce the concept of definite integrals.

\begin{center}
    \includegraphics[scale=0.3]{assets/28-3.jpg}
\end{center}

Shown in the diagram above (shaded area) is the area bounded by the line $x =
    a$, $x = b$, $y = 0$, and the curve $y = f(x)$ where $f(x) \geq 0$. This kind
of graph is called the curved trapezoid.

\begin{center}
    \includegraphics[scale=0.3]{assets/28-1.jpg}
    \includegraphics[scale=0.3]{assets/28-2.jpg}
    \includegraphics[scale=0.3]{assets/28-4.jpg}
\end{center}

As shown in the diagram above, in order to find the area of this curved
trapezoid, we can split it into multiple small curved trapezoid, each of them
being substituted by their respective rectangular shape. As such, an
approximate value of the area of the curved trapezoid can be acquired by
summing up of the area of each rectangle. As the curved trapezoid is being
split into smaller and smaller pieces, the approximate value we get will get
closer and closer to its actual area.

With this concept in mind, we can split the interval $[a, b]$ into $n$ smaller
interval $[x_0, x_1]$, $[x_1, x_2]$, $\cdots$, $[x_{n-1}, x_n]$ where $x_0 =
    a$, $x_n = b$. From drawing lines that are perpendicular to the $x$-axis
through the points $x_1$, $x_2$, $\cdots$, $x_{n-1}$, we can split the curved
trapezoid into $n$ smaller curved trapezoid.

\begin{center}
    \includegraphics[scale=0.3]{assets/28-5.jpg}
\end{center}

Choose any point $\xi_i$ in the $i$-th interval $[x_{i-1}, x_i]$, then the area
$\Delta A$ of the $i$-th curved trapezoid can be approximated by the area of
the rectangle with width $\Delta x_i = x_i - x_{i-1}$ and height $f(\xi_i)$, as
shown in the diagram above, i.e.
\begin{cequation}
    \Delta A_i \approx f(\xi_i)\Delta x
\end{cequation}
And the approximated value of the area of the original curved trapezoid is the sum of the area of all the smaller rectangle, i.e.
\begin{cequation}
    A = \sum_{i=1}^n \Delta A_i \approx \sum_{i=1}^n f(\xi_i)\Delta x
\end{cequation}

As the number of smaller interval $n$ increases and the width of each interval
decreases, the approximated value of the area of the original curved trapezoid
gets closer and closer to its actual area. To find the value of $A$, we split
the interval $[a, b]$ into indefinitely many smaller interval such that $\Delta
    x \to 0$ (i.e. $n \to \infty$), hence the area of the original curved trapezoid
can be defined as the limit of the sum of the area of all the smaller
rectangle, i.e.
\begin{cequation}
    A = \lim_{n \to \infty} \sum_{i=1}^n f(\xi_i)\Delta x
\end{cequation}

This limit is called the definite integral of $f(x)$ from $a$ to $b$, and is
denoted by the symbol $\displaystyle\int_a^b f(x) dx$, i.e.
\begin{cequation}
    \int_a^b f(x) dx = \lim_{n \to \infty} \sum_{i=1}^n f(\xi_i)\Delta x
\end{cequation}
where $f(x)$ is called the integrand, $[a, b]$ is called the interval of integration, $a$ and $b$ are called the lower and upper limits of integration respectively.

If $f(x) \geq 0$ in the interval $[a, b]$, we know from the above discussion
that the value of the definite integral $\displaystyle\int_a^b f(x) dx$ is the
area of the curved trapezoid bounded by the curve $y = f(x)$, the $x$-axis, and
the lines $x = a$ and $x = b$, i.e. $A = \displaystyle\int_a^b f(x) dx$.

If $f(x) \leq 0$ in the interval $[a, b]$, as shown in the diagram above,
$f(\xi_i)\Delta x$ is the negative value of the area of the $i$-th smaller
rectangle. Hence, the definite integral $\displaystyle\int_a^b f(x) dx$ is
negative, and its absolute value is the area of the curved trapezoid bounded by
the curve $y = f(x)$, the $x$-axis, and the lines $x = a$ and $x = b$, i.e. $A
    = -\displaystyle\int_a^b f(x) dx$.

\newpage

\subsection*{The Relationship between Definite Integrals and Indefinite Integrals}

The definite integrals and the indefinite integrals has inseparable
relationship between them. Consider the case of finding the area of curved
trapezoid. Let $x_0 > a$ and $f(x) \geq 0$. The area of the curved trapezoid
bounded by the curve $y = f(x)$, the $x$-axis, and the lines $x = a$, $x = x_0$
and $y = 0$ is $A(x_0) = \displaystyle\int_a^{x_0} f(x) dx$.

When $x_0$ changes, the area $A(x_0)$ also changes. From the diagram above, we
know that
\begin{cequation}
    m\Delta x \leq A(x_0 + \Delta x) - A(x_0) \leq M\Delta x,
\end{cequation}
where $m$ and $M$ are the minimum and maximum values of $f(x)$ in the interval $[x_0, x_0 + \Delta x]$. Hence,
\begin{cequation}
    m \leq \dfrac{A(x_0 + \Delta x) - A(x_0)}{\Delta x} \leq M.
\end{cequation}
Apparently, as $\Delta x$ approaches 0, both $m$ and $M$ approach $f(x_0)$. Besides, from the definition of derivative, $\lim_{\Delta x \to 0} \dfrac{A(x_0 + \Delta x) - A(x_0)}{\Delta x} = A'(x_0)$. Hence, we get $A'(x_0) = f(x_0)$. This relational expression is true for any $x_0 > a$. In other words, $A'(x) = f(x)$, i.e. the derivative of the area function $A(x)$ is the integrand $f(x)$.

Let $\displaystyle\int f(x)d x = F(x) + C$, i.e. $F(x)$ is the primitive of
$f(x)$. Then, from $A'(x) = f(x) = F'(x)$, we get $A(x) = F(x) + C$, where $C$
is a constant. When $A(a) = 0$, $x = a$, we get $C = -F(a)$. Hence, $C =
    -F(a)$, i.e. for any $x_0 > a$, we get $A(x_0) = F(x_0) - F(a)$.

Let $x_0 = b$, we get $A(b) = \displaystyle\int_a^b f(x) dx = F(b) - F(a)$.
This relational expression is true for any continuous function $f(x)$. In order
to make the relational expression true for any $a$ and $b$, when $a > b$, the
following definition is made.
\begin{cequation}
    \int_a^b f(x) dx = -\int_b^a f(x) dx.
\end{cequation}

Above all are the relationship between definite integrals and indefinite
integrals, i.e. \vspace{-0.9em}
\begin{center}
    \framebox{

        \parbox[t][1.3cm]{9cm}{ \addvspace{0.2cm} \centering If $\displaystyle\int f(x) dx
                = F(x) + C$, then $\displaystyle\int_a^b f(x) dx = F(b) - F(a)$. }}
\end{center}
This relationship is called the fundamental theorem of calculus. Generally, we
express the expression as follows:
\begin{cequation}
    \int_a^b f(x) dx = \big[F(x)\big]_a^b = F(b) - F(a).
\end{cequation}
where $F(x)$ is any primitive of $f(x)$.

When finding the definite integral $\displaystyle\int_a^b f(x) dx$, we only
have to find any primitive $F(x)$ of $f(x)$. The other primitive of $f(x)$ can
be expressed as $F(x) + C$, where $C$ is a constant. Hence, the definite
integral $\displaystyle\int_a^b f(x) dx$ is independent of the choice of
primitive of $f(x)$. But,
\begin{flalign*}
     & \big[F(x) + C\big]_a^b    & \\
     & = [F(b) + C] - [F(a) + C] & \\
     & = F(b) - F(a)
\end{flalign*}
Hence, the value of the definite integral has nothing to do with the constant $C$. The value of the definite integral remains the same no matter which primitive of $f(x)$ is chosen.
\\\\
Note that $\big[F(x)\big]_a^b$ can also be written as $F(x)\big|_a^b$.

\newpage
\subsection{Practice 2}

\begin{enumerate}
    \begin{multicols}{2}
        \item $\displaystyle\int_2^8 x d x$
        \sol{}
        \begin{flalign*}
            I & = \left[\dfrac{1}{2}x^2\right]_2^8 & \\
              & = \dfrac{1}{2}(64 - 4)             & \\
              & = 30
        \end{flalign*}

        \item $\displaystyle\int_{-2}^4 x^3 d x$
        \sol{}
        \begin{flalign*}
            I & = \left[\dfrac{1}{4}x^4\right]_{-2}^4 & \\
              & = \dfrac{1}{4}(256 - 16)              & \\
              & = 60
        \end{flalign*}
    \end{multicols}
    \begin{multicols}{2}
        \item $\displaystyle\int_{-\pi}^\pi \cos x d x$
        \sol{}
        \begin{flalign*}
            I & = \bigg[\sin x\bigg]_{-\pi}^\pi & \\
              & = \sin\pi - \sin(-\pi)          & \\
              & = 0 - 0                         & \\
              & = 0
        \end{flalign*}

        \item $\displaystyle\int_0^{\frac{\pi}{4}} \sec ^2 x d x$
        \sol{}
        \begin{flalign*}
            I & = \bigg[\tan x\bigg]_0^{\frac{\pi}{4}} & \\
              & = \tan\dfrac{\pi}{4} - \tan 0          & \\
              & = 1 - 0                                & \\
              & = 1
        \end{flalign*}
    \end{multicols}
\end{enumerate}
\subsection{Exercise 28.1}

\begin{enumerate}
    \begin{multicols}{2}
        \item $\displaystyle\int_{-2}^4 x^2 d x$
        \sol{}
        \begin{flalign*}
            I & = \left[\dfrac{1}{3}x^3\right]_{-2}^4 & \\
              & = \dfrac{1}{3}(64 + 8)                & \\
              & = 24
        \end{flalign*}
        \item $\displaystyle\int_1^4 \dfrac{1}{x^2} d x$
        \sol{}
        \begin{flalign*}
            I & = \left[-\dfrac{1}{x}\right]_1^4 & \\
              & = -\dfrac{1}{4} + 1              & \\
              & = \dfrac{3}{4}
        \end{flalign*}
    \end{multicols}

    \begin{multicols}{2}
        \item $\displaystyle\int_4^9 \sqrt{x} d x$
        \sol{}
        \begin{flalign*}
            I & = \left[\dfrac{2}{3}x^{\frac{3}{2}}\right]_4^9 & \\
              & = \dfrac{2}{3}(27 - 8)                         & \\
              & = \dfrac{38}{3}
        \end{flalign*}

        \item $\displaystyle\int_1^{27} \dfrac{1}{\sqrt[3]{x^5}} d x$
        \sol{}
        \begin{flalign*}
            I & = \left[-\dfrac{3}{2}x^{-\frac{2}{3}}\right]_1^{27} & \\
              & = -\dfrac{3}{2}\left(\dfrac{1}{9} - 1\right)        & \\
              & = \dfrac{4}{3}
        \end{flalign*}
    \end{multicols}
    \newpage
    \begin{multicols}{2}
        \item $\displaystyle\int_0^{\frac{\pi}{3}} \cos x d x$
        \sol{}
        \begin{flalign*}
            I & = \bigg[\sin x\bigg]_0^{\frac{\pi}{3}} & \\
              & = \sin\dfrac{\pi}{3} - \sin 0          & \\
              & = \dfrac{\sqrt{3}}{2} - 0              & \\
              & = \dfrac{\sqrt{3}}{2}
        \end{flalign*}

        \item $\displaystyle\int_{-\frac{\pi}{4}}^{\frac{\pi}{2}} \sin x d x$
        \sol{}
        \begin{flalign*}
            I & = \bigg[-\cos x\bigg]_{-\frac{\pi}{4}}^{\frac{\pi}{2}} & \\
              & = -\cos\dfrac{\pi}{2} - (-\cos(-\frac{\pi}{4}))        & \\
              & = 0 + \dfrac{\sqrt{2}}{2}                              & \\
              & = \dfrac{\sqrt{2}}{2}
        \end{flalign*}
    \end{multicols}

    \begin{multicols}{2}
        \item $\displaystyle\int_0^2 e^x d x$
        \sol{}
        \begin{flalign*}
            I & = e^2 - e^0 & \\
              & = e^2 - 1
        \end{flalign*}
        \vfill{}\null{}
        \columnbreak{}
        \item $\displaystyle\int_{\frac{\pi}{4}}^{\frac{\pi}{2}} \operatorname{cosec}^2 x d x$
        \sol{}
        \begin{flalign*}
            I & = \bigg[-\cot x\bigg]_{\frac{\pi}{4}}^{\frac{\pi}{2}} & \\
              & = -\cot\frac{\pi}{2} + \cot\frac{\pi}{4}              & \\
              & = 0 + 1                                               & \\
              & = 1
        \end{flalign*}
    \end{multicols}

    \begin{multicols}{2}
        \item $\displaystyle\int_1^2 \dfrac{1}{x} d x$
        \sol{}
        \begin{flalign*}
            I & = = \bigg[\ln|x|\bigg]_1^2 & \\
              & = \ln 2 - \ln 1            & \\
              & = \ln 2
        \end{flalign*}

        \item $\displaystyle\int_0^2 \dfrac{1}{x+1} d x$
        \sol{}
        \begin{flalign*}
            I & = \bigg[\ln|x+1|\bigg]_0^2 & \\
              & = \ln 3 - \ln 1            & \\
              & = \ln 3                    & \\
        \end{flalign*}
    \end{multicols}
\end{enumerate}

\newpage


\section{Properties and Calculations of Definite Integrals}

\subsection*{Properties of Definite Integrals}

\noindent \hspace{1.2em}\textit{The definite integrals have the following basic properties:}

\begin{center}
    \framebox{

        \parbox[t][1.3cm]{10cm}{ \addvspace{0.2cm}\hspace{10pt}\textbf{Property 1}
            \hspace{10pt} $\displaystyle\int_a^b kf(x) dx = k\int_a^b f(x) dx$}}
\end{center}
\begin{center}
    \framebox{

    \parbox[t][1.3cm]{10cm}{ \addvspace{0.2cm}\hspace{10pt} \textbf{Property 2}
    \hspace{10pt} $\displaystyle\int_a^b [f(x) \pm g(x)] dx = \int_a^b f(x) dx \pm
        \int_a^b g(x) dx$}}
\end{center}
\begin{center}
    \framebox{

        \parbox[t][1.3cm]{10cm}{ \addvspace{0.25cm}\hspace{10pt}\textbf{Property 3}
            \hspace{10pt} $\displaystyle\int_c^a f(x) dx + \int_b^c f(x) dx = \int_b^a f(x)
                dx$}}
\end{center}

\subsection{Practice 3}

\begin{enumerate}
    \item Find the following definite integrals (Question 1 to 4):
          \begin{enumerate}
              \begin{multicols}{2}
                  \item $\displaystyle\int_1^4 \dfrac{2 x^2+3 x+2}{x} d x$
                  \sol{}
                  \begin{flalign*}
                      I & = \int_1^4 \left(2x + 3 + \dfrac{2}{x}\right) d x & \\
                        & = \bigg[x^2 + 3x + 2\ln|x|\bigg]_1^4              & \\
                        & = 16 + 12 + 2\ln 4 - 1 - 3 - 2\ln 1               & \\
                        & = 24 + 2\ln 4
                  \end{flalign*}

                  \item $\displaystyle\int_{-1}^1\left(3 e^{2 x}-5 x\right) d x$
                  \sol{}
                  \begin{flalign*}
                      I & = \int_{-1}^1\left(3 e^{2 x}-5 x\right) d x                          & \\
                        & = \bigg[\dfrac{3}{2}e^{2x} - \dfrac{5}{2}x^2\bigg]_{-1}^1            & \\
                        & = \dfrac{3}{2}e^2 - \dfrac{5}{2} - \dfrac{3}{2}e^{-2} + \dfrac{5}{2} & \\
                        & = \dfrac{3}{2}(e^2 - e^{-2})
                  \end{flalign*}
              \end{multicols}

              \begin{multicols}{2}
                  \item $\displaystyle\int_{-\frac{\pi}{3}}^{\frac{\pi}{3}}\left(2 \cos x-3 \sec ^2 x\right) d x$
                  \sol{}
                  \begin{flalign*}
                      I & = \bigg[2\sin x - 3\tan x\bigg]_{-\frac{\pi}{3}}^{\frac{\pi}{3}}                                                    & \\
                        & = 2\sin\dfrac{\pi}{3} - 3\tan\dfrac{\pi}{3} - 2\sin\left(-\dfrac{\pi}{3}\right) + 3\tan\left(-\dfrac{\pi}{3}\right) & \\
                        & = \sqrt{3} - 3\sqrt{3} + \sqrt{3} - 3\sqrt{3}                                                                       & \\
                        & = -4\sqrt{3}
                  \end{flalign*}

                  \item $\displaystyle\int_{-2}^4 f(x) d x, f(x)=\left\{\begin{array}{cc}x^2-2, & -2 \leq x<2 \\ 4-x, & 2 \leq x \leq 4\end{array}\right.$
                  \sol{}
                  \begin{flalign*}
                      I & = \int_{-2}^{2}(x^2 - 2)dx + \int_{2}^{4}(4 - x)dx                                     & \\
                        & = \bigg[\dfrac{1}{3}x^3 - 2x\bigg]_{-2}^{2} + \bigg[4x - \dfrac{1}{2}x^2\bigg]_{2}^{4} & \\
                        & = \dfrac{8}{3} - 4 + \dfrac{8}{3} - 4 + 16 - 8 - 8 + 2                                 & \\
                        & = -\dfrac{2}{3}
                  \end{flalign*}
              \end{multicols}
          \end{enumerate}
          \newpage

    \item Given that $\displaystyle\int_{-2}^5 f(x) d x=2, \displaystyle\int_{-2}^3 f(x)
              d x=-1, \displaystyle\int_3^4 g(x) d x=3 \text { and } \displaystyle\int_4^5
              g(x) d x=2$, find:
          \begin{enumerate}
              \item $\displaystyle\int_3^5 f(x) d x$;
                    \sol{}
                    \begin{flalign*}
                        \int_3^5 f(x) d x & = \int_{-2}^5 f(x) d x - \int_{-2}^3 f(x) d x & \\
                                          & = 2 - (-1)                                    & \\
                                          & = 3
                    \end{flalign*}

              \item $\displaystyle\int_3^5\left(\dfrac{1}{3} g(x)+\dfrac{1}{2} f(x)\right) d x$.
                    \sol{}
                    \begin{flalign*}
                        \int_3^5\left(\dfrac{1}{3} g(x)+\dfrac{1}{2} f(x)\right) d x & = \dfrac{1}{3}\int_3^5 g(x) d x + \dfrac{1}{2}\int_3^5 f(x) d x                                  & \\
                                                                                     & = \dfrac{1}{3}\left(\int_3^4 g(x) d x + \int_4^5 g(x) d x\right) + \dfrac{1}{2}\int_3^5 f(x) d x & \\
                                                                                     & = \dfrac{1}{3}(3 + 2) + \dfrac{1}{2}(3)                                                          & \\
                                                                                     & = \dfrac{19}{6}
                    \end{flalign*}
          \end{enumerate}

    \item Given that $f(x)=\sqrt{x^2+1}$, find $f^{\prime}(x)$. Hence, find
          $\displaystyle\int_0^1 \dfrac{x}{\sqrt{x^2+1}} d x$. \sol{}
          \begin{flalign*}
              f^{\prime}(x)                        & = \dfrac{1}{2}(x^2 + 1)^{-\frac{1}{2}}(2x) & \\
                                                   & = \dfrac{x}{\sqrt{x^2 + 1}}                & \\
              \\
              \int_0^1 \dfrac{x}{\sqrt{x^2+1}} d x & = \bigg[\sqrt{x^2 + 1}\bigg]_0^1           & \\
                                                   & = \sqrt{2} - 1
          \end{flalign*}
\end{enumerate}


\subsection*{Integration by Substitution}

In the last chapter, we have learned how to solve indefinite integrals using
the method of integration by substitution: $\displaystyle\int f(g(x))g'(x)dx =
    \int f(u)du$ where $u = g(x)$. From the basic theorem of calculus, the method
of integration by substitution of definite integrals can be derived:
\begin{center}
    \framebox{

        \parbox[t][1.3cm]{9cm}{ \addvspace{0.2cm} \centering $\displaystyle\int_a^b
                f(g(x))g'(x)dx = \int_{g(a)}^{g(b)} f(u)du$ }}
\end{center}
\vspace{0.9em}
The method of integration by substitution of definite integrals is similar to
that of indefinite integrals. The only difference is that the limits of
integration are changed accordingly.

\subsection{Practice 4}

\begin{enumerate}
    \begin{multicols}{2}
        \item $\displaystyle\int_0^3 4 e^{2 x} d x$
        \sol{}

        Let $u = 2x$, $du = 2dx$.

        When $x = 0$, $u = 0$.

        When $x = 3$, $u = 6$.
        \begin{flalign*}
            I & = \int_0^6 2 e^u d u    & \\
              & = \bigg[2 e^u\bigg]_0^6 & \\
              & = 2 e^6 - 2             & \\
              & = 2(e^6 - 1)
        \end{flalign*}

        \item $\displaystyle\int_1^3 \dfrac{x}{3 x^2+5} d x$
        \sol{}

        Let $u = 3x^2 + 5$, $du = 6xdx$.

        When $x = 1$, $u = 8$.

        When $x = 3$, $u = 32$.
        \begin{flalign*}
            I & = \dfrac{1}{6}\int_8^{32} \dfrac{1}{u} d u & \\
              & = \dfrac{1}{6} \bigg[\ln|u|\bigg]_8^{32}   & \\
              & = \dfrac{1}{6} \ln 32 - \dfrac{1}{6} \ln 8 & \\
              & = \dfrac{1}{6} \ln 4                       & \\
              & = \dfrac{1}{3} \ln 2
        \end{flalign*}
    \end{multicols}

    \begin{multicols}{2}
        \item $\displaystyle\int_0^4 \dfrac{x}{25-x^2} d x$
        \sol{}

        Let $u = 25 - x^2$, $du = -2xdx$.

        When $x = 0$, $u = 25$.

        When $x = 4$, $u = 9$.
        \begin{flalign*}
            I & = -\dfrac{1}{2}\int_{25}^9 \dfrac{1}{u} d u & \\
              & = -\dfrac{1}{2} \bigg[\ln|u|\bigg]_{25}^9   & \\
              & = -\dfrac{1}{2} \ln 9 + \dfrac{1}{2} \ln 25 & \\
              & = \dfrac{1}{2} \ln\dfrac{25}{9}             & \\
              & = \ln\dfrac{5}{3}
        \end{flalign*}

        \item $\displaystyle\int_0^\pi 2 \sin x \cos ^2 x d x$
        \sol{}

        Let $u = \cos x$, $du = -\sin xdx$.

        When $x = 0$, $u = 1$.

        When $x = \pi$, $u = -1$.
        \begin{flalign*}
            I & = 2\int_{-1}^1 u^2 d u                & \\
              & = 2\bigg[\dfrac{1}{3}u^3\bigg]_{-1}^1 & \\
              & = \dfrac{4}{3}
        \end{flalign*}
    \end{multicols}
\end{enumerate}

\newpage

\subsection{Exercise 28.2}

\begin{enumerate}
    \begin{multicols}{2}
        \item $\displaystyle\int_0^4\left(x^2-2 x\right) d x$
        \sol{}
        \begin{flalign*}
            I & = \bigg[\dfrac{1}{3}x^3 - x^2\bigg]_0^4 & \\
              & = \dfrac{64}{3} - 16                    & \\
              & = \dfrac{16}{3}
        \end{flalign*}
        \item $\displaystyle\int_1^4 \dfrac{2 x^2-3 \sqrt{x}+1}{x} d x$
        \sol{}
        \begin{flalign*}
            I & = \int_1^4 \left(2x - 3x^{-\frac{1}{2}} + x^{-1}\right) d x & \\
              & = \bigg[x^2 - 6\sqrt{x} + \ln|x|\bigg]_1^4                  & \\
              & = 16 - 12 + \ln 4 - 1 + 6 - \ln 1                           & \\
              & = 9 + \ln 4
        \end{flalign*}
    \end{multicols}

    \begin{multicols}{2}
        \item $\displaystyle\int_{-3}^3(x+3)^2 d x$
        \sol{}

        Let $u = x + 3$, $du = dx$.

        When $x = -3$, $u = 0$.

        When $x = 3$, $u = 6$.
        \begin{flalign*}
            I & = \int_0^6 u^2 d u                & \\
              & = \bigg[\dfrac{1}{3}u^3\bigg]_0^6 & \\
              & = 72
        \end{flalign*}

        \item $\displaystyle\int_{-1}^1(2+x)\left(2-x^2\right) d x$
        \sol{}
        \begin{flalign*}
            I & = \int_{-1}^1\left(4 - 2x^2 + 2x - x^3\right)dx                             & \\
              & = \bigg[4x - \dfrac{2}{3}x^3 + x^2 - \dfrac{1}{4}x^4\bigg]_{-1}^1           & \\
              & = 4 - \dfrac{2}{3} + 1 - \dfrac{1}{4} + 4 - \dfrac{2}{3} - 1 + \dfrac{1}{4} & \\
              & = \dfrac{20}{3}
        \end{flalign*}
    \end{multicols}

    \begin{multicols}{2}
        \item $\displaystyle\int_0^{\frac{\pi}{2}}(2 \sin 3 \theta-3 \cos 2 \theta) d \theta$
        \sol{}
        \begin{flalign*}
            I & = \bigg[-\dfrac{2}{3}\cos 3\theta - \dfrac{3}{2}\sin 2\theta\bigg]_0^{\frac{\pi}{2}} & \\
              & = \dfrac{2}{3} + \dfrac{3}{2} + \dfrac{2}{3}                                         & \\
              & = \dfrac{13}{6}
        \end{flalign*}
        \vfill{}\null{}
        \columnbreak{}

        \item $\displaystyle\int_0^{\frac{\pi}{3}} \tan \theta d \theta$
        \sol{}
        \begin{flalign*}
            I & = \int_0^{\frac{\pi}{3}} \dfrac{\sin \theta}{\cos \theta} d \theta &
        \end{flalign*}
        Let $u = \cos \theta$, $du = -\sin \theta d \theta$.

        When $\theta = 0$, $u = 1$.

        When $\theta = \dfrac{\pi}{3}$, $u = \dfrac{1}{2}$.
        \begin{flalign*}
            I & = -\int_1^{\frac{1}{2}} \dfrac{1}{u} d u & \\
              & = -\bigg[\ln|u|\bigg]_1^{\frac{1}{2}}    & \\
              & = -\ln\dfrac{1}{2} + \ln 1               & \\
              & = \ln 2
        \end{flalign*}
    \end{multicols}

    \newpage

    \begin{multicols}{2}
        \item $\displaystyle\int_0^1\left(e^x-1\right)^2 d x$
        \sol{}
        \begin{flalign*}
            I & = \int_0^1\left(e^{2x} - 2e^x + 1\right) d x    & \\
              & = \bigg[\dfrac{1}{2}e^{2x} - 2e^x + x\bigg]_0^1 & \\
              & = \dfrac{1}{2}e^2 - 2e + 1 - \dfrac{1}{2} + 2   & \\
              & = \dfrac{1}{2}e^2 - 2e + \dfrac{5}{2}           & \\
              & = \dfrac{e^2 - 4e + 5}{2}
        \end{flalign*}
        \vfill{}\null{}

        \item $\displaystyle\int_1^4 \dfrac{2}{4 x-1} d x$
        \sol{}

        Let $u = 4x - 1$, $du = 4dx$.

        When $x = 1$, $u = 3$.

        When $x = 4$, $u = 15$.
        \begin{flalign*}
            I & = \dfrac{1}{2}\int_3^{15} \dfrac{1}{u} d u & \\
              & = \dfrac{1}{2}\bigg[\ln|u|\bigg]_3^{15}    & \\
              & = \dfrac{1}{2}(\ln 15 - \ln 3)             & \\
              & = \dfrac{1}{2}\ln 5
        \end{flalign*}
    \end{multicols}

    \begin{multicols}{2}
        \item $\displaystyle\int_{-1}^3 \sqrt{2 x+3} d x$
        \sol{}

        Let $u = 2x + 3$, $du = 2dx$.

        When $x = -1$, $u = 1$.

        When $x = 3$, $u = 9$.
        \begin{flalign*}
            I & = \dfrac{1}{2}\int_1^9 \sqrt{u} d u           & \\
              & = \dfrac{1}{3}\bigg[u^{\frac{3}{2}}\bigg]_1^9 & \\
              & = \dfrac{1}{3}(27 - 1)                        & \\
              & = \dfrac{26}{3}
        \end{flalign*}

        \item $\displaystyle\int_1^2 \dfrac{1}{(2 x-1)^3} d x$
        \sol{}

        Let $u = 2x - 1$, $du = 2dx$.

        When $x = 1$, $u = 1$.

        When $x = 2$, $u = 3$.
        \begin{flalign*}
            I & = \dfrac{1}{2}\int_1^3 u^{-3} d u                & \\
              & = -\dfrac{1}{4}\bigg[u^{-2}\bigg]_1^3            & \\
              & = -\dfrac{1}{4}\left(\dfrac{1}{9} - 1\right)     & \\
              & = -\dfrac{1}{4} \cdot \left(-\dfrac{8}{9}\right) & \\
              & = \dfrac{2}{9}
        \end{flalign*}
    \end{multicols}

    \begin{multicols}{2}
        \item $\displaystyle\int_{-1}^1 x^2\left(x^3-1\right)^4 d x$
        \sol{}

        Let $u = x^3 - 1$, $du = 3x^2dx$.

        When $x = -1$, $u = -2$.

        When $x = 1$, $u = 0$.
        \begin{flalign*}
            I & = \dfrac{1}{3}\int_{-2}^0 u^4 d u     & \\
              & = \dfrac{1}{15}\bigg[u^5\bigg]_{-2}^0 & \\
              & = \dfrac{32}{15}
        \end{flalign*}
        \columnbreak{}

        \item $\displaystyle\int_1^6 x \sqrt{3 x-2} d x$
        \sol{}

        Let $u = 3x - 2$, $du = 3dx$, $x = \dfrac{u + 2}{3}$.

        When $x = 1$, $u = 1$.

        When $x = 6$, $u = 16$.
        \begin{flalign*}
            I & = \dfrac{1}{3}\int_1^{16} \dfrac{u + 2}{3}\sqrt{u} d u                                     & \\
              & = \dfrac{1}{9}\int_1^{16} \left(u^{\frac{3}{2}} + 2u^{\frac{1}{2}}\right) d u              & \\
              & = \dfrac{1}{9}\bigg[\dfrac{2}{5}u^{\frac{5}{2}} + \dfrac{4}{3}u^{\frac{3}{2}}\bigg]_1^{16} & \\
              & = \dfrac{1}{9}\left(\dfrac{2048}{5} + \dfrac{256}{3} - \dfrac{2}{5} - \dfrac{4}{3}\right)  & \\
              & = \dfrac{274}{5}
        \end{flalign*}
    \end{multicols}

    \newpage
    \item $\displaystyle\int_{-\frac{\pi}{4}}^{\frac{\pi}{4}} \sin ^2 \theta d \theta$
          \sol{}
          \begin{flalign*}
              I & = \dfrac{1}{2}\int_{-\frac{\pi}{4}}^{\frac{\pi}{4}} (1 - \cos 2\theta) d \theta              & \\
                & = \dfrac{1}{2}\bigg[\theta - \dfrac{1}{2}\sin 2\theta\bigg]_{-\frac{\pi}{4}}^{\frac{\pi}{4}} & \\
                & = \dfrac{1}{2}\left(\dfrac{\pi}{4} - \dfrac{1}{2} + \dfrac{\pi}{4} - \dfrac{1}{2}\right)     & \\
                & = \dfrac{\pi}{4} - \dfrac{1}{2}                                                              & \\
                & = \dfrac{\pi - 2}{4}
          \end{flalign*}

    \item $\displaystyle\int_3^5 \dfrac{1}{x^2-x-2} d x$
          \sol{}
          \begin{flalign*}
              I & = \int_3^5 \dfrac{1}{(x-2)(x+1)} d x &
          \end{flalign*}
          Let $\dfrac{1}{(x-2)(x+1)} = \dfrac{A}{x-2} + \dfrac{B}{x+1}$.
          \begin{flalign*}
              Ax + A + Bx - 2B    & = 1 \\
              (A + B)x + (A - 2B) & = 1
          \end{flalign*}
          \vspace{-2em}
          \begin{flalign*}
              \begin{cases}
                  A + B = 0 \\
                  A - 2B = 1
              \end{cases}
              \Rightarrow
              \begin{cases}
                  A = \dfrac{1}{3} \\
                  B = -\dfrac{1}{3}
              \end{cases}
          \end{flalign*}
          \vspace{-1em}
          \begin{flalign*}
              I & = \int_3^5 \left(\dfrac{1}{3(x-2)} - \dfrac{1}{3(x+1)}\right) d x       & \\
                & = \dfrac{1}{3}\int_3^5 \left(\dfrac{1}{x-2} - \dfrac{1}{x+1}\right) d x & \\
                & = \dfrac{1}{3}\bigg[\ln|x-2| - \ln|x+1|\bigg]_3^5                       & \\
                & = \dfrac{1}{3}\left(\ln 3 - \ln 6 - \ln 1 + \ln 4\right)                & \\
                & = \dfrac{1}{3}\ln 2
          \end{flalign*}

          \newpage

          \begin{multicols}{2}
              \item $\displaystyle\int_2^5 \dfrac{x}{x^3-x^2-x+1} d x$
              \sol{}
              \begin{flalign*}
                  I & = \int_2^5 \dfrac{x}{x^2(x-1) - (x-1)} d x & \\
                    & = \int_2^5 \dfrac{x}{(x^2 - 1)(x-1)} d x   & \\
                    & = \int_2^5 \dfrac{x}{(x + 1)(x-1)^2} d x
              \end{flalign*}
              Let $\dfrac{x}{(x + 1)(x-1)^2} = \dfrac{A}{x+1} + \dfrac{B}{x-1} + \dfrac{C}{(x-1)^2}$.
              \begin{flalign*}
                  Ax^2 - 2Ax + A + Bx^2 - B + Cx + C    & = x & \\
                  (A + B)x^2 + (-2A + C)x + (A - B + C) & = x
              \end{flalign*}
              \vspace{-2em}
              \begin{flalign*}
                  \begin{cases}
                      A + B = 0   \\
                      -2A + C = 1 \\
                      A - B + C = 0
                  \end{cases}
                  \Rightarrow
                  \begin{cases}
                      A = -\dfrac{1}{4} \\
                      B = \dfrac{1}{4}  \\
                      C = \dfrac{1}{2}
                  \end{cases}
              \end{flalign*}
              \vspace{-1em}
              \begin{flalign*}
                  I & = \int_2^5 \left(-\dfrac{1}{4(x+1)} + \dfrac{1}{4(x-1)} + \dfrac{1}{2(x-1)^2}\right) d x                       & \\
                    & = \left[-\dfrac{1}{4}\ln|x+1| + \dfrac{1}{4}\ln|x-1| - \dfrac{1}{2(x-1)}\right]_2^5                            & \\
                    & = -\dfrac{1}{4}\ln 6 + \dfrac{1}{4}\ln 4 - \dfrac{1}{8} + \dfrac{1}{4}\ln 3 - \dfrac{1}{4}\ln 1 + \dfrac{1}{2} & \\
                    & = -\dfrac{1}{4}\ln 6 + \dfrac{1}{4}\ln 4 - \dfrac{1}{8} + \dfrac{1}{4}\ln 3 + \dfrac{1}{2}                     & \\
                    & = \dfrac{1}{4}\ln 2 + \dfrac{3}{8}
              \end{flalign*}

              \item $\displaystyle\int_0^1 \dfrac{x^2}{x^2+2 x+1} d x$
              \sol{}
              \begin{flalign*}
                  I & = \int_0^1 \dfrac{x^2}{(x+1)^2} d x &
              \end{flalign*}
              Let $u = x + 1$, $du = dx$, $x = u - 1$.

              When $x = 0$, $u = 1$.

              When $x = 1$, $u = 2$.
              \begin{flalign*}
                  I & = \int_1^2 \dfrac{(u - 1)^2}{u^2} d u                         & \\
                    & = \int_1^2 \dfrac{u^2 - 2u + 1}{u^2} d u                      & \\
                    & = \int_1^2 \left(1 - \dfrac{2}{u} + \dfrac{1}{u^2}\right) d u & \\
                    & = \bigg[u - 2\ln|u| - \dfrac{1}{u}\bigg]_1^2                  & \\
                    & = 2 - 2\ln 2 - \dfrac{1}{2} - 1 + 2\ln 1 + 1                  & \\
                    & = \dfrac{3}{2} - 2\ln 2
              \end{flalign*}
          \end{multicols}

          \begin{multicols}{2}
              \item $\displaystyle\int_0^3 \dfrac{x}{\sqrt{25-x^2}} d x$
              \sol{}

              Let $u = 25 - x^2$, $du = -2xdx$.

              When $x = 0$, $u = 25$.

              When $x = 3$, $u = 16$.
              \begin{flalign*}
                  I & = \dfrac{1}{2}\int_{16}^{25} \dfrac{1}{\sqrt{u}} d u & \\
                    & = \bigg[\sqrt{u}\bigg]_{16}^{25}                     & \\
                    & = 5 - 4                                              & \\
                    & = 1
              \end{flalign*}

              \item $\displaystyle\int_0^{\frac{\pi}{6}} \sin ^2 \theta \cos \theta d \theta$
              \sol{}

              Let $u = \sin \theta$, $du = \cos \theta d \theta$.

              When $\theta = 0$, $u = 0$.

              When $\theta = \dfrac{\pi}{6}$, $u = \dfrac{1}{2}$.
              \begin{flalign*}
                  I & = \int_0^{\frac{1}{2}} u^2 d u                & \\
                    & = \bigg[\dfrac{1}{3}u^3\bigg]_0^{\frac{1}{2}} & \\
                    & = \dfrac{1}{24}
              \end{flalign*}
          \end{multicols}

          \newpage
    \item Given that $f(x)=\left\{\begin{array}{cc}2 x^2-1, & -2 \leq x \leq 2 \\ 3 x+1, & 2<x \leq 4\end{array}\right.$, find $\displaystyle\int_{-2}^4 f(x) d x$.
          \sol{}
          \begin{flalign*}
              I & = \int_{-2}^2 (2x^2 - 1) d x + \int_2^4 (3x + 1) d x                           & \\
                & = \bigg[\dfrac{2}{3}x^3 - x\bigg]_{-2}^2 + \bigg[\dfrac{3}{2}x^2 + x\bigg]_2^4 & \\
                & = \dfrac{16}{3} - 2 + \dfrac{16}{3} - 2 + 24 + 4 - 6 - 2                       & \\
                & = \dfrac{80}{3}
          \end{flalign*}

    \item Given that $\displaystyle\int_3^5 f(x) d x=6, \int_5^9 f(x) d x=18, \int_1^4
              g(x) d x=4$ and $\displaystyle\int_3^4 g(x) d x=-4$. Find:
          \begin{enumerate}
              \item $\displaystyle\int_1^3 g(x) d x$;
                    \sol{}
                    \begin{flalign*}
                        \int_1^3 g(x) d x & = \int_1^4 g(x) d x - \int_3^4 g(x) d x & \\
                                          & = 4 - (-4)                              & \\
                                          & = 8
                    \end{flalign*}

              \item $\displaystyle\int_1^3 f(3 x) d x$;
                    \sol{}

                    Let $u = 3x$, $du = 3dx$.

                    When $x = 1$, $u = 3$.

                    When $x = 3$, $u = 9$.
                    \begin{flalign*}
                        \int_1^3 f(3 x) d x & = \dfrac{1}{3}\int_3^9 f(u) d u                                 & \\
                                            & = \dfrac{1}{3}\int_3^5 f(u) d u + \dfrac{1}{3}\int_5^9 f(u) d u & \\
                                            & = \dfrac{1}{3}(6 + 18)                                          & \\
                                            & = 8
                    \end{flalign*}

              \item $\displaystyle\int_1^3[f(3 x)-3 g(x)] d x$.
                    \sol{}
                    \begin{flalign*}
                        \int_1^3[f(3 x)-3 g(x)] d x & = \int_1^3 f(3 x) d x - 3\int_1^3 g(x) d x & \\
                                                    & = 8 - 3 \cdot 8                            & \\
                                                    & = -16
                    \end{flalign*}
          \end{enumerate}
          \newpage

          \begin{multicols}{2}
              \item Given the function $y=x \sqrt{x+1}$, \\find $\dfrac{d y}{d x}$. Hence, find
              $\displaystyle\int_3^8 \dfrac{3 x+2}{\sqrt{x+1}} d x$. \sol{}
              \begin{flalign*}
                  \dfrac{d y}{d x}                       & = \sqrt{x + 1} + \dfrac{x}{2\sqrt{x + 1}}  & \\
                                                         & = \dfrac{2x + x + 2}{2\sqrt{x + 1}}        & \\
                                                         & = \dfrac{3x + 2}{2\sqrt{x + 1}}            & \\
                  \\
                  \int_3^8 \dfrac{3 x+2}{\sqrt{x+1}} d x & = 2\int_3^8 \dfrac{3 x+2}{2\sqrt{x+1}} d x & \\
                                                         & = 2\int_3^8 \dfrac{d y}{d x} d x           & \\
                                                         & = 2\left[x \sqrt{x + 1}\right]_3^8         & \\
                                                         & = 2\left(24 - 6\right)                     & \\
                                                         & = 36
              \end{flalign*}
              \vfill{}\null{}

              \item Given the function $y=\dfrac{x^2-1}{2 x+1}$, \\find $\dfrac{d y}{d x}$. Hence,
              find $\displaystyle\int_0^2 \dfrac{x^2+x+1}{4 x^2+4 x+1} d x$. \sol{}
              \begin{flalign*}
                  \dfrac{d y}{d x}                          & = \dfrac{(2x + 1)(2x) - (x^2 - 1)(2)}{(2x + 1)^2}             & \\
                                                            & = \dfrac{4x^2 + 2x - 2x^2 + 2}{(2x + 1)^2}                    & \\
                                                            & = \dfrac{2(x^2 + x + 1)}{(2x + 1)^2}                          & \\
                  \\
                  \int_0^2 \dfrac{x^2+x+1}{4 x^2+4 x+1} d x & = \dfrac{1}{2}\int_0^2 \dfrac{2(x^2 + x + 1)}{(2x + 1)^2} d x & \\
                                                            & = \dfrac{1}{2}\int_0^2 \dfrac{d y}{d x} d x                   & \\
                                                            & = \dfrac{1}{2}\left[\dfrac{x^2 - 1}{2x + 1}\right]_0^2        & \\
                                                            & = \dfrac{1}{2}\left(\dfrac{3}{5} - \dfrac{-1}{1}\right)       & \\
                                                            & = \dfrac{4}{5}
              \end{flalign*}
              \vfill\null{}
          \end{multicols}

    \item Given the function $y=x e^x-e^x$, find $\dfrac{d y}{d x}$. Hence, find
          $\displaystyle\int_1^4 2 x e^x d x$. \sol{}
          \begin{flalign*}
              \dfrac{d y}{d x}     & = x e^x + e^x - e^x                & \\
                                   & = x e^x                            & \\
              \\
              \int_1^4 2 x e^x d x & = 2\int_1^4 x e^x d x              & \\
                                   & = 2\int_1^4 \dfrac{d y}{d x} d x   & \\
                                   & = 2\left[x e^x - e^x\right]_1^4    & \\
                                   & = 2\left(4e^4 - e^4 - e + e\right) & \\
                                   & = 6e^4
          \end{flalign*}
\end{enumerate}

\newpage


\section{Area}

When we first introduced the concept of definite integrals, we have studied
that, if $f(x) \geq 0$ in the interval $a \leq x \leq b$, then the area of the
curved trapezoid bounded by the curve $y = f(x)$, the lines $x = a$ and $x =
    b$, the $x$-axis, and the $y$-axis is given by
\begin{cequation}
    A = \int_a^b f(x)dx
\end{cequation}

The applications of definite integrals are not limited to finding area of
curved trapezoid. In fact, it is widely used in different fields of
technologies. To solve real-life problems, the basic mindset is the steps
described in the definition: splitting, approximating, finding sum, and taking
limit. After understanding the concept, part of the steps can be skipped, and
the related definite integrals can be written out straight away.

Let's take finding the area of the curved trapezoid bounded by the line $x =
    a$, $x = b$, the $x$-axis, and the curve $y = f(x)$ as an example and do some
further elaboration. Since both sides of the curved trapezoid are the straight
lines $x = a$ and $x = b$, we can split the interval $a \leq x \leq b$ into
multiple smaller intervals, hence the original curved trapezoid is split into
multiple smaller curved trapezoid. As shown in the diagram below, take any one
of the smaller intervals, and express it as $[x, x + \Delta x]$, its
corresponding smaller curved trapezoid can be approximated by a rectangle with
width of $\Delta x$ and height of the function value $f(x)$ at the point $x$.
As such, the area of the smaller curved trapezoid is $\Delta A \approx
    f(x)\Delta x$.
\begin{center}
    \includegraphics[scale=0.3]{assets/28-8.png}
\end{center}

Summing up the approximated area of all the smaller curved trapezoid, then find
the limit of the sum when $\Delta x \to 0$, we can get the area of the original
curved trapezoid, i.e.
\begin{cequation}
    \sum\Delta A \approx \sum f(x)\Delta x\ \xrightarrow{\ \ \ \ \Delta x \to 0\ \ \ \ } \int_a^b f(x)dx
\end{cequation}

\begin{center}
    \includegraphics[scale=0.2]{assets/28-9.png}
\end{center}
The same can be applied if $f(x) \leq 0$ in the interval $a \leq x \leq b$, as
shown in the diagram above. The area of the curved trapezoid bounded by the
curve $y = f(x)$, the lines $x = a$ and $x = b$, the $x$-axis, and the $y$-axis
is given by
\begin{cequation}
    A = -\int_a^b f(x)dx
\end{cequation}

\begin{center}
    \includegraphics[scale=0.2]{assets/28-10.png}
\end{center}
If the target area $A$ is bounded by the lines $y = c$, $y = d$, the $y$-axis,
and the curve $x = f(y)$, where $f(y) \geq 0$ in the interval $c \leq y \leq
    d$, as shown in the diagram above, using the same concept, we can get the area
of the region
\begin{cequation}
    A = \int_c^d f(y)dy
\end{cequation}

\subsection{Practice 5}
Find the following indefinite integral:
\begin{enumerate}
    \begin{multicols}{2}
        \item $\displaystyle\int\sin2x\cos2x dx$
        \sol{}
        \begin{flalign*}
            I & = \dfrac{1}{2}\int\sin4x dx & \\
              & = -\dfrac{1}{8}\cos4x + C
        \end{flalign*}
        \vfill{}\null{}

        \item $\displaystyle\int\cos^2 2x dx$
        \sol{}
        \begin{flalign*}
            I & = \int\dfrac{1 + \cos 4x}{2} dx                    & \\
              & = \dfrac{1}{2}\int dx + \dfrac{1}{2}\int\cos 4x dx & \\
              & = \dfrac{1}{2}x + \dfrac{1}{8}\sin 4x + C
        \end{flalign*}
    \end{multicols}
    \begin{multicols}{2}
        \item $\displaystyle\int\sin^3 x dx$
        \sol{}
        \begin{flalign*}
            I & = \int\sin^2 x\sin x dx                 & \\
              & = \int(1 - \cos^2 x)\sin x dx           & \\
              & = \int\sin x dx - \int\cos^2 x\sin x dx
        \end{flalign*}
        Let $u = \cos x$, $du = -\sin xdx$.
        \begin{flalign*}
            I & = -\int \cos x dx + \int u^2 du    & \\
              & = -\sin x + \dfrac{1}{3}u^3 + C    & \\
              & = \dfrac{1}{3}\cos^3 x -\sin x + C
        \end{flalign*}

        \item $\displaystyle\int\cos^3 x dx$
        \sol{}
        \begin{flalign*}
            I & = \int\cos^2 x\cos x dx                 & \\
              & = \int(1 - \sin^2 x)\cos x dx           & \\
              & = \int\cos x dx - \int\sin^2 x\cos x dx
        \end{flalign*}
        Let $u = \sin x$, $du = \cos xdx$.
        \begin{flalign*}
            I & = \int \cos x dx - \int u^2 du      & \\
              & = \sin x - \dfrac{1}{3}u^3 + C      & \\
              & = \sin x - \dfrac{1}{3}\sin^3 x + C
        \end{flalign*}
    \end{multicols}
    \begin{multicols}{2}
        \item $\displaystyle\int\tan^4 x\sec^2 x dx$
        \sol{}

        Let $u = \tan x$, $du = \sec^2 xdx$.
        \begin{flalign*}
            I & = \int u^4 du              & \\
              & = \dfrac{u^5}{5} + C       & \\
              & = \dfrac{1}{5}\tan^5 x + C
        \end{flalign*}
        \vfill{}\null{}
        \columnbreak
        \item $\displaystyle\int\tan^4\dfrac{x}{2} dx$
        \sol{}
        \begin{flalign*}
            I & = \int\tan^2\dfrac{x}{2}\tan^2\dfrac{x}{2} dx                             & \\
              & = \int\left(\sec^2\dfrac{x}{2} - 1\right)\tan^2\dfrac{x}{2} dx            & \\
              & = \int\sec^2\dfrac{x}{2}\tan^2\dfrac{x}{2} dx - \int\tan^2\dfrac{x}{2} dx
        \end{flalign*}
        Let $u = \tan\dfrac{x}{2}$, $du = \dfrac{1}{2}\sec^2\dfrac{x}{2}dx$.
        \begin{flalign*}
            I & = 2\int u^2 du - \int \tan^2\dfrac{x}{2} dx                     & \\
              & = \dfrac{2u^3}{3} - \int \left(\sec^2\dfrac{x}{2} - 1\right) dx & \\
              & = \dfrac{2u^3}{3} - \int \sec^2\dfrac{x}{2} dx + \int dx        & \\
              & = \dfrac{2}{3}\tan^3\dfrac{x}{2} - 2\tan\dfrac{x}{2} + x + C
        \end{flalign*}
    \end{multicols}
\end{enumerate}

\newpage
\subsection{Practice 6}

\begin{enumerate}
    \begin{multicols}{2}
        \item Find the area of the region bounded by the curve $y = x^2 - 4x + 5$ and the
        line $y = x + 1$. \sol{}
        \begin{flalign*}
            x^2 - 5x + 4        & = 0 & \\
            (x - 4)(x - 1)      & = 0 & \\
            x = 4 \text{ or } x & = 1
        \end{flalign*}
        In the interval $1 \leq x \leq 4$, $x + 1 \geq x^2 - 4x + 5$
        \begin{flalign*}
            A & = \int_1^4 \left[(x + 1) - (x^2 - 4x + 5)\right] d x         & \\
              & = \int_1^4 (-x^2 + 5x - 4) d x                               & \\
              & = \bigg[-\dfrac{1}{3}x^3 + \dfrac{5}{2}x^2 - 4x\bigg]_1^4    & \\
              & = -\dfrac{64}{3} + 40 - 16 + \dfrac{1}{3} - \dfrac{5}{2} + 4 & \\
              & = \dfrac{9}{2}
        \end{flalign*}

        \item Find the area of the region bounded by the curve $x = 4y - y^2$ and the line $x
            - 2y + 3 = 0$. \sol{}
        \begin{flalign*}
            4y - y^2 - 2y + 3 & = 0 & \\
            -y^2 + 2y + 3     & = 0 & \\
            (y - 3)(y + 1)    & = 0
        \end{flalign*}
        In the interval $-1 \leq y \leq 3$, $4y - y^2 \geq 2y - 3$.
        \begin{flalign*}
            A & = \int_{-1}^3 \left[(4y - y^2) - (2y - 3)\right] d y & \\
              & = \int_{-1}^3 (-y^2 + 2y + 3) d y                    & \\
              & = \bigg[-\dfrac{1}{3}y^3 + y^2 + 3y\bigg]_{-1}^3     & \\
              & = -9 + 9 + 9 - \dfrac{1}{3} - 1 + 3                  & \\
              & = \dfrac{32}{3}
        \end{flalign*}
    \end{multicols}
\end{enumerate}


\subsection{Exercise 28.3}

\noindent \hspace{1.2em}\textit{Find the area of the region bounded by the following curves and lines:}
\begin{enumerate}
      \begin{multicols}{2}
            \item $y=3 x^2, x=2, x=5$, and $x$-axis
            \sol{}
            \begin{flalign*}
                  A & = \int_2^5 3x^2 d x    & \\
                    & = \left[x^3\right]_2^5 & \\
                    & = 125 - 8              & \\
                    & = 117
            \end{flalign*}

            \item $y=(x-1)^2, x=4, x$-axis, and $y$-axis
            \sol{}
            \begin{flalign*}
                  A & = \int_0^4 (x-1)^2 d x                 & \\
                    & = \left[\dfrac{1}{3}(x-1)^3\right]_0^4 & \\
                    & = \dfrac{1}{3}(3^3 + 1)                & \\
                    & = \dfrac{28}{3}
            \end{flalign*}
      \end{multicols}

      \vfill\null

      \begin{multicols}{2}
            \item $y=x^2+4 x-21$, and $x$-axis
            \begin{flalign*}
                  x^2 + 4x - 21        & = 0 & \\
                  (x + 7)(x - 3)       & = 0 & \\
                  x = -7 \text{ or } x & = 3
            \end{flalign*}
            In the interval $-7 \leq x \leq 3$, $x^2 + 4x - 21 \leq 0$.
            \begin{flalign*}
                  A & = \left|\int_{-7}^3 (x^2 + 4x - 21) d x\right|                       & \\
                    & = \left|\left[\dfrac{1}{3}x^3 + 2x^2 - 21x\right]_{-7}^3\right|      & \\
                    & = \left|9 + 18 - 63 - \left(-\dfrac{343}{3} + 98 + 147\right)\right| & \\
                    & = \dfrac{500}{3}
            \end{flalign*}

            \item $y=e^{2 x}, x=0, x=4$, and $x$-axis
            \sol{}
            \begin{flalign*}
                  A & = \int_0^4 e^{2x} d x                 & \\
                    & = \left[\dfrac{1}{2}e^{2x}\right]_0^4 & \\
                    & = \dfrac{1}{2}(e^8 - 1)
            \end{flalign*}
      \end{multicols}

      \vfill\null

      \begin{multicols}{2}
            \item $y=\sin \dfrac{x}{2}, 0 \leq x \leq 2 \pi$, and $x$-axis
            \sol{}
            \begin{flalign*}
                  A & = \int_0^{2 \pi} \sin \dfrac{x}{2} d x        & \\
                    & = \left[-2 \cos \dfrac{x}{2}\right]_0^{2 \pi} & \\
                    & = 4
            \end{flalign*}
            \vfill\null

            \columnbreak
            \item $y=\cos x, x=2 \pi, x$-axis, and $y$-axis
            \sol{}
            \begin{flalign*}
                  \cos x & = 0                                & \\
                  x      & = \dfrac{\pi}{2}, \dfrac{3 \pi}{2}
            \end{flalign*}
            In the interval $\dfrac{\pi}{2} \leq x \leq \dfrac{3 \pi}{2}$, $\cos x \leq 0$.
            \begin{flalign*}
                  A & = \int_{0}^{\frac{\pi}{2}} \cos x d x + \left|\int_{\frac{\pi}{2}}^{\frac{3 \pi}{2}} \cos x d x\right| + \int_{\frac{3 \pi}{2}}^{2 \pi} \cos x d x        & \\
                    & = \bigg[\sin x\bigg]_0^{\frac{\pi}{2}} + \left|\bigg[\sin x\bigg]_{\frac{\pi}{2}}^{\frac{3 \pi}{2}}\right| + \bigg[\sin x\bigg]_{\frac{3 \pi}{2}}^{2 \pi} & \\
                    & = 1 + 1 + 1 + 1 = 4
            \end{flalign*}
      \end{multicols}

      \newpage
      \begin{multicols}{2}
            \item $x=y^2, y=3$, and $y$-axis
            \sol{}
            \begin{flalign*}
                  A & = \int_0^3 y^2 d y                 & \\
                    & = \left[\dfrac{1}{3}y^3\right]_0^3 & \\
                    & = 9
            \end{flalign*}

            \vfill{}\null{}
            \columnbreak{}
            \item $x=9 y-y^3$, and $y$-axis
            \sol{}
            \begin{flalign*}
                  9y - y^3   & = 0                   & \\
                  y(9 - y^2) & = 0                   & \\
                  y = 0      & \text{ or } y = \pm 3
            \end{flalign*}
            In the interval $-3 \leq y \leq 0$, $9y - y^3 \leq 0$.
            \begin{flalign*}
                  A & = \left|\int_{-3}^0 (9y - y^3) d y\right| + \int_0^3 (9y - y^3) d y                                                       & \\
                    & = \left|\left[\dfrac{9}{2}y^2 - \dfrac{1}{4}y^4\right]_{-3}^0\right| + \left[\dfrac{9}{2}y^2 - \dfrac{1}{4}y^4\right]_0^3 & \\
                    & = \left|- \dfrac{81}{2} + \dfrac{81}{4}\right| + \dfrac{81}{2} - \dfrac{81}{4}                                            & \\
                    & = \dfrac{81}{2}
            \end{flalign*}
      \end{multicols}

      \begin{multicols}{2}
            \item $y=\dfrac{1}{x}, y=\dfrac{1}{2}, y=2$, and $y$-axis
            \sol{}
            \begin{flalign*}
                  A & = \int_{\frac{1}{2}}^2 \dfrac{1}{x} d x & \\
                    & = \bigg[\ln x\bigg]_{\frac{1}{2}}^2     & \\
                    & = \ln 2 - \ln \dfrac{1}{2}              & \\
                    & = 2 \ln 2
            \end{flalign*}

            \item $y^2=x, x=4$, and $x=16$
            \sol{}
            \begin{flalign*}
                  A & = 2 \int_4^{16} \sqrt{x} d x                        & \\
                    & = 2 \left[\dfrac{2}{3}x^{\frac{3}{2}}\right]_4^{16} & \\
                    & = 2 \left(\dfrac{128}{3} - \dfrac{16}{3}\right)     & \\
                    & = \dfrac{224}{3}
            \end{flalign*}
            \vfill{}\null{}
      \end{multicols}

      \begin{multicols}{2}
            \item Find the area of the region bounded by the curve \\$y=x^2-4$ and the line $y=3
            x$. \sol{}
                  \begin{flalign*}
                        x^2 - 4             & = 3x & \\
                        x^2 - 3x - 4        & = 0  & \\
                        (x - 4)(x + 1)      & = 0  & \\
                        x = 4 \text{ or } x & = -1
                  \end{flalign*}
                  In the interval $-1 \leq x \leq 4$, $x^2 - 4 \leq 3x$.
                  \begin{flalign*}
                        A & = \int_{-1}^4 (3x - x^2 + 4) d x                             & \\
                          & = \left[\dfrac{3}{2}x^2 - \dfrac{1}{3}x^3 + 4x\right]_{-1}^4 & \\
                          & = 24 - \dfrac{64}{3} + 16 - \dfrac{3}{2} - \dfrac{1}{3} + 4  & \\
                          & = \dfrac{125}{6}
                  \end{flalign*}
                  \item Find the area of the region bounded by the curve $y=x^2+2 x$ and the curve
            $y=12+4 x-x^2$.
                  \begin{flalign*}
                        x^2 + 2x            & = 12 + 4x - x^2 & \\
                        2x^2 - 2x - 12      & = 0             & \\
                        (x - 3)(2x + 4)     & = 0             & \\
                        x = 3 \text{ or } x & = -2
                  \end{flalign*}
                  In the interval $-2 \leq x \leq 3$, $x^2 + 2x \leq 12 + 4x - x^2$.
            \begin{flalign*}
                  A & = \int_{-2}^3 (12 + 4x - x^2 - x^2 - 2x) d x      & \\
                    & = \int_{-2}^3 (12 + 2x - 2x^2) d x                & \\
                    & = \left[12x + x^2 - \dfrac{2}{3}x^3\right]_{-2}^3 & \\
                    & = 36 + 9 - 18 + 24 - 4 - \dfrac{16}{3}            & \\
                    & = \dfrac{125}{3}
            \end{flalign*}
      \end{multicols}

      \begin{multicols}{2}
            \item Find the area of the region bounded by the curve $y=\sin x$ and $y=-2 \sin x$
            in the interval $0 \leq x \leq \pi$. \sol{}
            \begin{flalign*}
                  \sin x & = -2 \sin x & \\
                  \sin x & = 0         & \\
                  x      & = 0, \pi
            \end{flalign*}
            In the interval $0 \leq x \leq \pi$, $-2 \sin x \leq \sin x$.
            \begin{flalign*}
                  A & = \int_0^{\pi} (3 \sin x) d x   & \\
                    & = \bigg[-3 \cos x\bigg]_0^{\pi} & \\
                    & = 3 - (-3) = 6
            \end{flalign*}
            \vfill{}\null{}

            \item Find the area of the region bounded by the curve $y=e^x, y=e^{-2 x}$ and the
            lines $x=-2$ and $x=4$. \sol{}
            \begin{flalign*}
                  e^x & = e^{-2x} & \\
                  x   & = 0
            \end{flalign*}
            In the interval $-2 \leq x \leq 0$, $e^x \leq e^{-2x}$.

            In the interval $0 \leq x \leq 4$, $e^{-2x} \leq e^x$.
            \begin{flalign*}
                  A & = \int_{-2}^0 (e^{-2x} - e^x) d x + \int_0^4 (e^x - e^{-2x}) d x                               & \\
                    & = \left[-\dfrac{1}{2}e^{-2x} - e^x\right]_{-2}^0 + \left[e^x + \dfrac{1}{2}e^{-2x}\right]_0^4  & \\
                    & = -\dfrac{1}{2} - 1 + \dfrac{1}{2}e^{4} + e^{-2} + e^4 + \dfrac{1}{2}e^{-8} - 1 - \dfrac{1}{2} & \\
                    & = \dfrac{3}{2}e^4 + e^{-2} + \dfrac{1}{2}e^{-8} - 3
            \end{flalign*}
      \end{multicols}

      \vfill\null

      \item Find the area of the region bounded by the curve $y=x^3-10 x^2+28 x$ and the
            line $y=4 x$. \sol{}
            \begin{flalign*}
                  x^3 - 10x^2 + 28x   & = 4x                  & \\
                  x^3 - 10x^2 + 24x   & = 0                   & \\
                  x(x^2 - 10x + 24)   & = 0                   & \\
                  x = 0 \text{ or } x & = 4 \text{ or } x = 6
            \end{flalign*}
            In the interval $0 \leq x \leq 4$, $x^3 - 10x^2 + 28x \geq 4x$.

            In the interval $4 \leq x \leq 6$, $x^3 - 10x^2 + 28x \leq 4x$.
            \begin{flalign*}
                  A & = \int_0^4 (x^3 - 10x^2 + 28x - 4x) d x + \int_4^6 (4x - x^3 + 10x^2 - 28x) d x                                              & \\
                    & = \int_0^4 (x^3 - 10x^2 + 24x) d x + \int_4^6 (-x^3 + 10x^2 - 24x) d x                                                       & \\
                    & = \left[\dfrac{1}{4}x^4 - \dfrac{10}{3}x^3 + 12x^2\right]_0^4 + \left[-\dfrac{1}{4}x^4 + \dfrac{10}{3}x^3 - 12x^2\right]_4^6 & \\
                    & = 64 - \dfrac{640}{3} + 192 - 324 + 720 - 432 + 64 - \dfrac{640}{3} + 192                                                    & \\
                    & = \dfrac{148}{3}
            \end{flalign*}
            \vfill\null

            \newpage

      \item Find the area of the region bounded by the curve $x=8 y-y^2, x=16 y-y^2-48$,
            and $y$-axis. \sol{}
            \begin{flalign*}
                  8y - y^2 & = 16y - y^2 - 48 & \\
                  8y       & = 48             & \\
                  y        & = 6
            \end{flalign*}

      \item Find the area of the region bounded by the curve $x=2 y^2-8 y+10$ and
            $x=y^2-y$.
      \item Given that the curve $y=f(x)$ passes through point $(1,0)$, and the gradient of
            any point on the curve $(x, y)$ is $3 x^2-3$. Find the area of the region
            bounded by the curve, $x=2$ and $x$.
\end{enumerate}

\subsection{Practice 7}

\begin{enumerate}
    \item Find the volume of a cone with radius $r$ and height $h$ using definite
          integrals. \sol{}

          Let $y$ be the height of any cross section of the cone, and $x$ be the radius
          of the cross section.
          \begin{flalign*}
              \dfrac{y}{h} & = \dfrac{x}{r}   & \\
              y            & = \dfrac{h}{r} x
          \end{flalign*}
          \begin{flalign*}
              V & = \int_0^h \pi y^2 d x                                  & \\
                & = \int_0^h \pi \left(\dfrac{h}{r} x\right)^2 d x        & \\
                & = \pi \dfrac{h^2}{r^2} \int_0^h x^2 d x                 & \\
                & = \pi \dfrac{h^2}{r^2} \left[\dfrac{1}{3}x^3\right]_0^h & \\
                & = \pi \dfrac{h^2}{r^2} \cdot \dfrac{1}{3}h^3            & \\
                & = \dfrac{1}{3} \pi r^2 h
          \end{flalign*}

    \item Shown in the diagram below is the shaded region bounded by the ellipse
          $\dfrac{x^2}{a^2} + \dfrac{y^2}{b^2} = 1$, where $a > 0$ and $b > 0$. If the
          volume of the solid of revolution formed by rotating this region about the
          $x$-axis and the $y$-axis is $V_x$ and $V_y$ respectively,
          \begin{center}
              \includegraphics[width=0.3\textwidth]{assets/28-prac7-2.png}
          \end{center}
          \begin{enumerate}
              \item find $V_x$ and $V_y$. \sol{} \vspace{-0.8cm}
                    \begin{multicols}{2}
                        \begin{flalign*}
                            \dfrac{x^2}{a^2} + \dfrac{y^2}{b^2} & = 1                                    & \\
                            \dfrac{y^2}{b^2}                    & = 1 - \dfrac{x^2}{a^2}                 & \\
                            y^2                                 & = b^2\left(1 - \dfrac{x^2}{a^2}\right)
                        \end{flalign*}
                        \begin{flalign*}
                            V_x & = \int_{-a}^a \pi y^2 d x                                                  & \\
                                & = \pi b^2 \int_{-a}^a \left(1 - \dfrac{x^2}{a^2}\right) d x                & \\
                                & = \pi b^2 \left[x - \dfrac{x^3}{3a^2}\right]_{-a}^a                        & \\
                                & = \pi b^2 \left[a - \dfrac{a^3}{3a^2} - (-a) + \dfrac{(-a)^3}{3a^2}\right] & \\
                                & = \pi b^2 \left[a - \dfrac{a}{3} + a - \dfrac{a}{3}\right]                 & \\
                                & = \dfrac{4}{3} \pi a b^2
                        \end{flalign*}
                        \vfill\null\columnbreak
                        \begin{flalign*}
                            \dfrac{x^2}{a^2} + \dfrac{y^2}{b^2} & = 1                                    & \\
                            \dfrac{x^2}{a^2}                    & = 1 - \dfrac{y^2}{b^2}                 & \\
                            x^2                                 & = a^2\left(1 - \dfrac{y^2}{b^2}\right)
                        \end{flalign*}
                        \begin{flalign*}
                            V_y & = \int_{-b}^b \pi x^2 d y                                                  & \\
                                & = \pi a^2 \int_{-b}^b \left(1 - \dfrac{y^2}{b^2}\right) d y                & \\
                                & = \pi a^2 \left[y - \dfrac{y^3}{3b^2}\right]_{-b}^b                        & \\
                                & = \pi a^2 \left[b - \dfrac{b^3}{3b^2} - (-b) + \dfrac{(-b)^3}{3b^2}\right] & \\
                                & = \pi a^2 \left[b - \dfrac{b}{3} + b - \dfrac{b}{3}\right]                 & \\
                                & = \dfrac{4}{3} \pi a^2 b
                        \end{flalign*}
                    \end{multicols}
                    \vspace{-1cm}

              \item if $V_x = 3V_y$, find the value of $a:b$. \sol{}
                    \begin{flalign*}
                        \dfrac{4}{3} \pi a b^2 & = 3 \cdot \dfrac{4}{3} \pi a^2 b & \\
                        4\pi a b^2             & = 12 \pi a^2 b                   & \\
                        b                      & = 3a                             & \\
                        a:b                    & = 1:3
                    \end{flalign*}
          \end{enumerate}
\end{enumerate}

\subsection{Practice 8}

\begin{enumerate}
    \item Given that the line $x + y = a$ splits the circle $x^2 + y^2 = a^2$ into two
          parts, find the volume of the solid of revolution formed by rotating the
          smaller part of the circle about the $x$-axis.
    \item Given that a region is bounded by the curve $y^2 = 8 - x$ and $y^2 = x - 4$.
          Find the volume of the solid of revolution formed by rotating this region about
          the $x$-axis and the $y$-axis respectively.
\end{enumerate}

\subsection{Exercise 28.4}

\noindent \hspace{1.2em}\textit{Find the volume of the solid of revolution formed by rotating the regions bounded by the following curves and lines about the $x$-axis (Question 1 to 7):}

\begin{enumerate}
      \begin{multicols}{2}
            \item $y=\sqrt{x}, x=4, x=9$, and $x$-axis
            \sol{}
            \begin{flalign*}
                  V_x & = \int_{4}^{9} \pi \left( \sqrt{x} \right)^2 dx & \\
                      & = \pi \int_{4}^{9} x dx                         & \\
                      & = \pi \left[ \frac{x^2}{2} \right]_{4}^{9}      & \\
                      & = \pi \left( \frac{81}{2} - 8 \right)           & \\
                      & = \frac{65 \pi}{2}
            \end{flalign*}

            \item $y=3 x, x=4$, and $x$-axis
            \sol{}
            \begin{flalign*}
                  V_x & = \int_{0}^{4} \pi \left( 3 x \right)^2 dx & \\
                      & = \pi \int_{0}^{4} 9 x^2 dx                & \\
                      & = \pi \left[ 3 x^3 \right]_{0}^{4}         & \\
                      & = 192 \pi
            \end{flalign*}
      \end{multicols}

      \begin{multicols}{2}
            \item $y=x(x-2)$, and $x$-axis
            \sol{}
            \begin{flalign*}
                  V_x & = \pi \int_{0}^{2} x^2 (x-2)^2 dx                                 & \\
                      & = \pi \int_{0}^{2} x^2 (x^2 - 4x + 4) dx                          & \\
                      & = \pi \int_{0}^{2} x^4 - 4x^3 + 4x^2 dx                           & \\
                      & = \pi \left[ \frac{x^5}{5} - x^4 + \frac{4x^3}{3} \right]_{0}^{2} & \\
                      & = \pi\left[\dfrac{32}{5} - 16 + \dfrac{32}{3}\right]              & \\
                      & = \dfrac{16 \pi}{15}
            \end{flalign*}

            \item $x^2+y^2=4, x=0$, and $x=2$
            \sol{}
            \begin{flalign*}
                  V_x & = \pi \int_{0}^{2} \left( 4 - x^2 \right) dx    & \\
                      & = \pi \left[ 4x - \frac{x^3}{3} \right]_{0}^{2} & \\
                      & = \frac{16 \pi}{3}
            \end{flalign*}
      \end{multicols}

      \begin{multicols}{2}
            \item $y=\sin x, x=0, x=\pi$, and $x$-axis
            \sol{}
            \begin{flalign*}
                  V_x & = \pi \int_{0}^{\pi} \sin^2 x dx                               & \\
                      & = \pi \int_{0}^{\pi} \frac{1 - \cos 2x}{2} dx                  & \\
                      & = \frac{\pi}{2} \int_{0}^{\pi} (1 - \cos 2x) dx                & \\
                      & = \frac{\pi}{2} \left[ x - \frac{\sin 2x}{2} \right]_{0}^{\pi} & \\
                      & = \frac{\pi}{2} \left( \pi - 0 \right)                         & \\
                      & = \frac{\pi^2}{2}
            \end{flalign*}

            \item $y=e^x, x=-1, x=1$, and $x$-axis
            \sol{}
            \begin{flalign*}
                  V_x & = \pi \int_{-1}^{1} e^{2x} dx                         & \\
                      & = \pi \left[ \frac{e^{2x}}{2} \right]_{-1}^{1}        & \\
                      & = \pi \left( \frac{e^2}{2} - \frac{e^{-2}}{2} \right) & \\
                      & = \frac{\pi}{2} \left( e^2 - e^{-2} \right)
            \end{flalign*}
      \end{multicols}

      \newpage
      \item $y=x^3+x^2-2 x$, and $x$-axis
            \sol{}
            \begin{flalign*}
                  V_x & = \pi \int_{-2}^{1} \left( x^3 + x^2 - 2 x \right)^2 dx                                                                                                  & \\
                      & = \pi \int_{-2}^{1} (x^6 + x^4 + 4x^2 + 2x^5 - 4x^3 - 4x^4) dx                                                                                           & \\
                      & = \pi \int_{-2}^{1} (x^6 + 2x^5 - 3x^4 - 4x^3 + 4x^2) dx                                                                                                 & \\
                      & = \pi \left[ \frac{x^7}{7} + \frac{x^6}{3} - \dfrac{3}{5}x^5 - x^4 + \frac{4x^3}{3} \right]_{-2}^{1}                                                     & \\
                      & = \pi \left( \dfrac{1}{7} + \dfrac{1}{3} - \dfrac{3}{5} - 1 + \dfrac{4}{3} + \dfrac{128}{7} - \dfrac{64}{3} - \dfrac{96}{5} + 16 + \dfrac{32}{3} \right) & \\
                      & = \dfrac{162}{35} \pi
            \end{flalign*}
\end{enumerate}

\noindent \hspace{1.2em}\textit{Find the volume of the solid of revolution formed by rotating the regions bounded by the following curves and lines about the $y$-axis (Question 8 to 14):}

\begin{enumerate}[resume]
      \begin{multicols}{2}
            \item $y=x^3, y=8$, and $y$-axis
            \sol{}
            \begin{flalign*}
                  V_y & = \pi \int_{0}^{8} \left( \sqrt[3]{y} \right)^2 dy       & \\
                      & = \pi \int_{0}^{8} y^{\frac{2}{3}} dy                    & \\
                      & = \pi \left[ \frac{3}{5} y^{\frac{5}{3}} \right]_{0}^{8} & \\
                      & = \frac{3 \pi}{5} \left( 8^{\frac{5}{3}} - 0 \right)     & \\
                      & = \frac{96 \pi}{5}
            \end{flalign*}

            \item $x=\sqrt{y-1}, y=4$, and $y$-axis
            \sol{}
            \begin{flalign*}
                  V_y & = \pi \int_{1}^{4} \left( \sqrt{y-1} \right)^2 dy & \\
                      & = \pi \int_{1}^{4} (y-1) dy                       & \\
                      & = \pi \left[ \frac{y^2}{2} - y \right]_{1}^{4}    & \\
                      & = \pi \left( 8 - 4 - \frac{1}{2} + 1 \right)      & \\
                      & = \frac{9 \pi}{2}
            \end{flalign*}
      \end{multicols}

      \begin{multicols}{2}
            \item $y^2=x+3, y=2, x$-axis, and $y$-axis
            \sol{}
            \begin{flalign*}
                  V_y & = \pi \int_{0}^{2} \left( y^2 - 3 \right)^2 dy           & \\
                      & = \pi \int_{0}^{2} (y^4 - 6 y^2 + 9) dy                  & \\
                      & = \pi \left[ \frac{y^5}{5} - 2 y^3 + 9 y \right]_{0}^{2} & \\
                      & = \pi \left( \frac{32}{5} - 16 + 18 \right)              & \\
                      & = \frac{42 \pi}{5}
            \end{flalign*}

            \item $y^2=x+1$, and $y$-axis
            \sol{}
            \begin{flalign*}
                  V_y & = \pi \int_{-1}^{1} \left( y^2 - 1 \right)^2 dy                                    & \\
                      & = \pi \int_{-1}^{1} (y^4 - 2 y^2 + 1) dy                                           & \\
                      & = \pi \left[ \frac{y^5}{5} - \frac{2 y^3}{3} + y \right]_{-1}^{1}                  & \\
                      & = \pi \left( \frac{1}{5} - \frac{2}{3} + 1 + \frac{1}{5} - \frac{2}{3} + 1 \right) & \\
                      & = \frac{16 \pi}{15}
            \end{flalign*}
      \end{multicols}

      \newpage
      \begin{multicols}{2}
            \item $x^2-y^2=4, y=3$, and $x$-axis
            \sol{}
            \begin{flalign*}
                  V_x & = \pi \int_{0}^{3} (4+y^2) dy                   & \\
                      & = \pi \left[ 4y + \frac{y^3}{3} \right]_{0}^{3} & \\
                      & = \pi \left( 12 + 9 \right)                     & \\
                      & = 21 \pi
            \end{flalign*}

            \item $y=1-\sqrt{x}, x$-axis, and $y$-axis
            \sol{}
            \begin{flalign*}
                  V_x & = \pi \int_{0}^{1} \left( 1 - y \right)^4 dy                       & \\
                      & = \pi \int_{0}^{1} (1 - 4y + 6y^2 - 4y^3 + y^4) dy                 & \\
                      & = \pi \left[ y - 2y^2 + 2y^3 - y^4 + \frac{y^5}{5} \right]_{0}^{1} & \\
                      & = \pi \left( 1 - 2 + 2 - 1 + \frac{1}{5} \right)                   & \\
                      & = \frac{\pi}{5}
            \end{flalign*}
      \end{multicols}

      \item $y=\dfrac{1}{x}-1, y=1, x$-axis, and $y$-axis
            \sol{}
            \begin{flalign*}
                  V_x & = \pi \int_{0}^{1} \frac{1}{(y + 1)^2} dy
            \end{flalign*}
            Let $u = y + 1$, then $du = dy$. When $y = 0$, $u = 1$, when $y = 1$, $u = 2$.
            \begin{flalign*}
                  V_x & = \pi \int_{1}^{2} \frac{1}{u^2} du       & \\
                      & = \pi \left[ -\frac{1}{u} \right]_{1}^{2} & \\
                      & = \pi \left( -\frac{1}{2} + 1 \right)     & \\
                      & = \frac{\pi}{2}
            \end{flalign*}
      \item Given that a region is bounded by the curve $y=4-x^2$ and the $x$-axis. Find
            the volume of the solid of revolution formed by rotating this region about the
            $x$-axis and the $y$-axis respectively.
      \item Given that a region is bounded by the curve $y=5-\sqrt{x}, x$-axis, and
            $y$-axis. Find the volume of the solid of revolution formed by rotating this
            region about the $x$-axis and the $y$-axis respectively.
      \item Find the volume of the solid of revolution formed by rotating the region
            bounded by the curve $y^2=9 x$, the line $y=6$, and the $y$-axis about the the
            $y$-axis.
      \item Given that a region is bounded by the curve $y=x^2+1$, the line $x=-2, x=2$,
            and $x$-axis. Find the volume of the solid of revolution formed by rotating
            this region about the $x$-axis and the $y$-axis respectively.
      \item Find the volume of the solid of revolution formed by rotating the region
            bounded by the curve $y=x(6-x)$ and the line $y=3 x$ about the $x$-axis.
      \item Find the volume of the solid of revolution formed by rotating the region
            bounded by the curve $y=x^2$, the lines $x=1$ and $y=9$ about the $y$-axis.
      \item Given that a region is bounded by the curve $y^2=8 x$ and the line $y=2 x$.
            Find the volume of the solid of revolution formed by rotating this region about
            the $x$-axis and the $y$-axis respectively.
      \item Given that a region is bounded by the curve $y^2=8 x$ and $y=8 x^2$. Find the
            volume of the solid of revolution formed by rotating this region about the
            $x$-axis and the $y$-axis respectively.
      \item Given that a region is bounded by the curve $y^2=2 x$ and $y^2=12-4 x$. Find
            the volume of the solid of revolution formed by rotating this region about the
            $x$-axis and the $y$-axis respectively.
      \item Shown in the diagram below is the shaded region bounded by the ellipse
            $\dfrac{x^2}{a^2} + \dfrac{y^2}{b^2} = 1$ and the line $\dfrac{x}{a} +
                  \dfrac{y}{b} = 1$, where $a > 0$ and $b > 0$. If the volume of the solid of
            revolution formed by rotating this region about the $x$-axis and the $y$-axis
            is $V_x$ and $V_y$ respectively,
            \begin{enumerate}
                  \item find $V_x$ and $V_y$.
                  \item if $V_x = 2V_y$, find the value of $a:b$.
            \end{enumerate}

\end{enumerate}

\section{Revision Exercise 28}

\begin{enumerate}
      \begin{multicols}{2}
            \item $\displaystyle\int_0^a\left(2 x^2-3 x+2\right) d x$
            \sol{}
            \begin{flalign*}
                  I & =\left[\dfrac{2}{3} x^3-\dfrac{3}{2} x^2+2 x\right]_0^a & \\
                    & =\dfrac{2}{3} a^3-\dfrac{3}{2} a^2+2 a
            \end{flalign*}
            \vfill\null
            \item $\displaystyle\int_1^3\left(x^2+\dfrac{1}{x^3}\right) d x$
            \sol{}
            \begin{flalign*}
                  I & =\left[\dfrac{1}{3} x^3-\dfrac{1}{2 x^2}\right]_1^3 & \\
                    & = 9 - \dfrac{1}{18} - \dfrac{1}{3} + \dfrac{1}{2}   & \\
                    & = \dfrac{82}{9}
            \end{flalign*}
            \vfill\null
      \end{multicols}
      \vspace{-0.8cm}
      \vfill\null
      \begin{multicols}{2}
            \item $\displaystyle\int_{-\frac{\pi}{6}}^{\frac{\pi}{2}}(3 \sin \theta-2 \cos 2 \theta) d \theta$
            \sol{}
            \begin{flalign*}
                  I & =\bigg[-3 \cos \theta-\sin 2 \theta\bigg]_{-\frac{\pi}{6}}^{\frac{\pi}{2}}                                  & \\
                    & = -3 \cos \dfrac{\pi}{2} - \sin \pi + 3 \cos\left(-\dfrac{\pi}{6}\right) + \sin\left(-\dfrac{\pi}{3}\right) & \\
                    & = -3 \cdot 0 - 0 + \dfrac{3\sqrt{3}}{2} - \dfrac{\sqrt{3}}{2}                                               & \\
                    & = \sqrt{3}
            \end{flalign*}
            \vfill\null
            \item $\displaystyle\int_{-\frac{\pi}{4}}^{\frac{\pi}{4}}\left(3 \sec ^2 \theta+\tan ^2 \theta\right) d \theta$
            \sol{}
            \begin{flalign*}
                  I & =\int_{-\frac{\pi}{4}}^{\frac{\pi}{4}}\left(3 \sec ^2 \theta+\sec^2\theta-1\right) d \theta    & \\
                    & =\int_{-\frac{\pi}{4}}^{\frac{\pi}{4}}\left(4 \sec ^2 \theta-1\right) d \theta                 & \\
                    & =\bigg[4 \tan \theta-\theta\bigg]_{-\frac{\pi}{4}}^{\frac{\pi}{4}}                             & \\
                    & =4 \tan \dfrac{\pi}{4} - \dfrac{\pi}{4} - 4 \tan \left(-\dfrac{\pi}{4}\right) - \dfrac{\pi}{4} & \\
                    & =8-\dfrac{\pi}{2}
            \end{flalign*}
            \vfill\null
      \end{multicols}
      \vspace{-0.8cm}
      \vfill\null
      \begin{multicols}{2}
            \item $\displaystyle\int_0^{\ln 2} e^{3 x} d x$
            \sol{}
            \begin{flalign*}
                  I & =\left[\dfrac{1}{3} e^{3 x}\right]_0^{\ln 2} & \\
                    & =\dfrac{1}{3} e^{3\ln 2}-\dfrac{1}{3} e^0    & \\
                    & =\dfrac{8}{3} - \dfrac{1}{3}                 & \\
                    & =\dfrac{7}{3}
            \end{flalign*}
            \vfill\null
            \item $\displaystyle\int_1^3 \dfrac{2}{3 x-1} d x$
            \sol{}

            Let $u = 3x - 1$, then $du = 3dx$.

            When $x = 1$, $u = 2$.

            When $x = 3$, $u = 8$.
            \begin{flalign*}
                  I & =\dfrac{2}{3}\int_2^8 \dfrac{1}{u} d u & \\
                    & =\dfrac{2}{3}\bigg[\ln u\bigg]_2^8     & \\
                    & =\dfrac{2}{3}(\ln 8 - \ln 2)           & \\
                    & =\dfrac{4}{3}\ln 2
            \end{flalign*}
      \end{multicols}
      \vfill\null

      \newpage
      \begin{multicols}{2}
            \item $\displaystyle\int_1^{16} \dfrac{2 x+3}{\sqrt{x}} d x$
            \sol{}
            \begin{flalign*}
                  I & =\int_1^{16} 2 x^{\frac{1}{2}}+3 x^{-\frac{1}{2}} d x              & \\
                    & =\bigg[\dfrac{4}{3} x^{\frac{3}{2}}+6 x^{\frac{1}{2}}\bigg]_1^{16} & \\
                    & =\dfrac{256}{3} + 24 - \dfrac{4}{3} - 6                            & \\
                    & =102
            \end{flalign*}
            \vfill\null

            \item $\displaystyle\int_1^4 \dfrac{(\sqrt{x}-1)^2}{x} d x$
            \sol{}
            \begin{flalign*}
                  I & =\int_1^4 \dfrac{x-2 \sqrt{x}+1}{x} d x     & \\
                    & =\int_1^4 (1-2 x^{-\frac{1}{2}}+x^{-1}) d x & \\
                    & =\bigg[x-4 x^{\frac{1}{2}}+\ln x\bigg]_1^4  & \\
                    & =4 - 8 + 2\ln 2 - 1 + 4 - 0                 & \\
                    & =2\ln 2 - 1
            \end{flalign*}
      \end{multicols}
      \vfill\null

      \begin{multicols}{2}
            \item $\displaystyle\int_1^2\left(x+\dfrac{4}{x^2}\right)^2 d x$
            \sol{}
            \begin{flalign*}
                  I & =\int_1^2\left(x^2+\dfrac{8}{x}+\dfrac{16}{x^4}\right) d x                & \\
                    & =\bigg[\dfrac{1}{3} x^3+8 \ln x-\dfrac{16}{3 x^3}\bigg]_1^2               & \\
                    & =\dfrac{8}{3} + 8 \ln 2 - \dfrac{2}{3} - \dfrac{1}{3} - 0 + \dfrac{16}{3} & \\
                    & = 8 \ln 2 + 7
            \end{flalign*}
            \vfill\null
            \item $\displaystyle\int_0^1 \dfrac{x+1}{x^2+2 x+3} d x$
            \sol{}

            Let $u = x^2 + 2x + 3$, then $du = 2(x + 1)dx$.

            When $x = 0$, $u = 3$.

            When $x = 1$, $u = 6$.
            \begin{flalign*}
                  I & =\dfrac{1}{2}\int_3^6 \dfrac{1}{u} d u & \\
                    & =\dfrac{1}{2}\bigg[\ln u\bigg]_3^6     & \\
                    & =\dfrac{1}{2}(\ln 6 - \ln 3)           & \\
                    & =\dfrac{1}{2}\ln 2
            \end{flalign*}
      \end{multicols}
      \vfill\null

      \begin{multicols}{2}
            \item $\displaystyle\int_{-1}^2 \dfrac{5 x}{\left(1+x^2\right)^4} d x$
            \sol{}

            Let $u = 1 + x^2$, then $du = 2xdx$.

            When $x = -1$, $u = 2$.

            When $x = 2$, $u = 5$.
            \begin{flalign*}
                  I & =\dfrac{5}{2}\int_2^5 \dfrac{1}{u^4} d u                  & \\
                    & =\dfrac{5}{2}\bigg[-\dfrac{1}{3 u^3}\bigg]_2^5            & \\
                    & =\dfrac{5}{2}\left(-\dfrac{1}{375} + \dfrac{1}{24}\right) & \\
                    & =\dfrac{39}{400}
            \end{flalign*}
            \item $\displaystyle\int_0^2 \dfrac{x}{\sqrt{25-4 x^2}} d x$
            \sol{}

            Let $u = 25 - 4x^2$, then $du = -8xdx$.

            When $x = 0$, $u = 25$.

            When $x = 2$, $u = 9$.
            \begin{flalign*}
                  I & =-\dfrac{1}{8}\int_{25}^9 \dfrac{1}{\sqrt{u}} d u & \\
                    & =-\dfrac{1}{8}\bigg[2 \sqrt{u}\bigg]_{25}^9       & \\
                    & =-\dfrac{1}{8}(6 - 10)                            & \\
                    & =\dfrac{1}{2}
            \end{flalign*}
      \end{multicols}
      \vfill\null
      \newpage

      \begin{multicols}{2}
            \item $\displaystyle\int_2^4 \dfrac{3 x-2}{(2 x-3)^2} d x$
            \sol{}

            Let $\dfrac{3x - 2}{(2x - 3)^2} = \dfrac{A}{2x - 3} + \dfrac{B}{(2x - 3)^2}$.
            \begin{flalign*}
                  3x - 2 & = 2Ax - 3A + B   & \\
                         & = 2Ax + (B - 3A)
            \end{flalign*}
            Equating coefficients, $A = \dfrac{3}{2}$, $B = \dfrac{5}{2}$.
            \begin{flalign*}
                  I & =\dfrac{1}{2}\int_2^4 \left[\dfrac{3}{2 x-3}+\dfrac{5}{(2 x-3)^2}\right] d x
            \end{flalign*}
            Let $u = 2x - 3$, then $du = 2dx$.

            When $x = 2$, $u = 1$.

            When $x = 4$, $u = 5$.
            \begin{flalign*}
                  I & =\dfrac{1}{4}\int_1^5 \left[\dfrac{3}{u}+\dfrac{5}{u^2}\right] d u        & \\
                    & =\dfrac{1}{4}\bigg[3 \ln u-\dfrac{5}{u}\bigg]_1^5                         & \\
                    & =\dfrac{1}{4}\left(3 \ln 5 - \dfrac{5}{5} - 3 \ln 1 + \dfrac{5}{1}\right) & \\
                    & =\dfrac{3}{4}\ln 5 + 1
            \end{flalign*}

            \item $\displaystyle\int_2^4 \dfrac{2}{x^3-x} d x$
            \sol{}
            \begin{flalign*}
                  I & =\int_2^4 \dfrac{2}{x(x-1)(x+1)} d x
            \end{flalign*}
            Let $\dfrac{2}{x(x-1)(x+1)} = \dfrac{A}{x} + \dfrac{B}{x - 1} + \dfrac{C}{x + 1}$.
            \begin{flalign*}
                  2 & = A(x - 1)(x + 1) + Bx(x + 1) + Cx(x - 1) & \\
                    & = A(x^2 - 1) + B(x^2 + x) + C(x^2 - x)    & \\
                    & = (A + B + C)x^2 + (B - C)x + (-A)
            \end{flalign*}
            Equating coefficients, $A = -2$, $B = 1$, $C = 1$.
            \begin{flalign*}
                  I & =\int_2^4 \left[-\dfrac{2}{x}+\dfrac{1}{x-1}+\dfrac{1}{x+1}\right] d x & \\
                    & =\bigg[-2 \ln x+\ln |x-1|+\ln |x+1|\bigg]_2^4                          & \\
                    & = -2 \ln 4+\ln 3+\ln 5 + 2 \ln 2-\ln 1-\ln 3                           & \\
                    & = \ln \dfrac{5}{4}
            \end{flalign*}
      \end{multicols}
      \begin{multicols}{2}
            \item $\displaystyle\int_1^3 \dfrac{1}{x^3+2 x^2+x} d x$
            \sol{}
            \begin{flalign*}
                  I & =\int_1^3 \dfrac{1}{x(x+1)^2} d x
            \end{flalign*}
            Let $\dfrac{1}{x(x+1)^2} = \dfrac{A}{x} + \dfrac{B}{x + 1} + \dfrac{C}{(x + 1)^2}$.
            \begin{flalign*}
                  1 & = A(x + 1)^2 + Bx(x + 1) + Cx       & \\
                    & = A(x^2 + 2x + 1) + B(x^2 + x) + Cx & \\
                    & = (A + B)x^2 + (2A + B + C)x + A
            \end{flalign*}
            Equating coefficients, $A = 1$, $B = -1$, $C = -1$.
            \begin{flalign*}
                  I & =\int_1^3 \left[\dfrac{1}{x}-\dfrac{1}{x+1}-\dfrac{1}{(x+1)^2}\right] d x & \\
                    & =\bigg[\ln x-\ln |x+1|+\dfrac{1}{x+1}\bigg]_1^3                           & \\
                    & =\ln 3-\ln 4+\dfrac{1}{4}-\ln 1+\ln 2-\dfrac{1}{2}                        & \\
                    & =\ln \dfrac{3}{2}-\dfrac{1}{4}
            \end{flalign*}
            \columnbreak
            \item $\displaystyle\int_0^\pi(\sin \theta+\cos \theta)^2 d \theta$
            \sol{}
            \begin{flalign*}
                  I & =\int_0^\pi (\sin ^2 \theta+2 \sin \theta \cos \theta+\cos ^2) \theta d \theta & \\
                    & =\int_0^\pi (1 + \sin 2\theta) d \theta                                        & \\
                    & =\bigg[\theta - \dfrac{1}{2}\cos 2\theta\bigg]_0^\pi                           & \\
                    & =\pi - \dfrac{1}{2}\cos 2\pi - 0 + \dfrac{1}{2}\cos 0                          & \\
                    & =\pi
            \end{flalign*}
            \vfill\null
      \end{multicols}

      \begin{multicols}{2}
            \item $\displaystyle\int_0^{\frac{\pi}{3}} \sec ^2 \theta \tan \theta d \theta$
            \sol{}

            Let $u = \sec \theta$, then $du = \sec \theta \tan \theta d\theta$.

            When $\theta = 0$, $u = 1$.

            When $\theta = \dfrac{\pi}{3}$, $u = 2$.
            \begin{flalign*}
                  I & =\int_1^2 u d u                    & \\
                    & =\left[\dfrac{1}{2} u^2\right]_1^2 & \\
                    & =2 - \dfrac{1}{2}                  & \\
                    & =\dfrac{3}{2}
            \end{flalign*}

            \item $\displaystyle\int_{-\frac{\pi}{2}}^{\frac{\pi}{2}} \sin ^2 \theta \cos \theta d \theta$
            \sol{}

            Let $u = \sin \theta$, then $du = \cos \theta d\theta$.

            When $\theta = -\dfrac{\pi}{2}$, $u = -1$.

            When $\theta = \dfrac{\pi}{2}$, $u = 1$.
            \begin{flalign*}
                  I & =\int_{-1}^1 u^2 d u                       & \\
                    & =\left[\dfrac{1}{3} u^3\right]_{-1}^1      & \\
                    & =\dfrac{1}{3} - \left(-\dfrac{1}{3}\right) & \\
                    & =\dfrac{2}{3}
            \end{flalign*}
      \end{multicols}

      \begin{multicols}{2}
            \item $\displaystyle\int_0^1 \dfrac{e^x}{e^x+1} d x$
            \sol{}

            Let $u = e^x + 1$, then $du = e^x dx$.

            When $x = 0$, $u = 2$.

            When $x = 1$, $u = e + 1$.
            \begin{flalign*}
                  I & =\int_2^{e+1} \dfrac{1}{u} d u & \\
                    & =\bigg[\ln u\bigg]_2^{e+1}     & \\
                    & =\ln (e+1) - \ln 2             & \\
                    & =\ln \dfrac{e+1}{2}
            \end{flalign*}
            \vfill\null

            \item $\displaystyle\int_{\frac{\pi}{6}}^{\frac{\pi}{3}} \dfrac{\sec ^2 \theta}{\tan \theta} d \theta$
            \sol{}

            Let $u = \tan\theta$, then $du = \sec^2\theta d\theta$.

            When $\theta = \dfrac{\pi}{6}$, $u = \dfrac{\sqrt{3}}{3}$.

            When $\theta = \dfrac{\pi}{3}$, $u = \sqrt{3}$.
            \begin{flalign*}
                  I & =\int_{\frac{\sqrt{3}}{3}}^{\sqrt{3}} \dfrac{1}{u} d u & \\
                    & =\bigg[\ln u\bigg]_{\frac{\sqrt{3}}{3}}^{\sqrt{3}}     & \\
                    & =\ln \sqrt{3} - \ln \dfrac{\sqrt{3}}{3}                & \\
                    & =\ln 3
            \end{flalign*}
      \end{multicols}

      \item Given that $\displaystyle\int_0^4 f(x) d x=2, \displaystyle\int_0^3 g(x) d
                  x=4$, and $\displaystyle\int_3^8 g(x) d x=12$. Find the value of
            $\displaystyle\int_0^8\left[f\left(\dfrac{x}{2}\right)-2 g(x)\right] d x$.
            \sol{} \vspace{-1cm}
            \begin{multicols}{2}
                  \begin{flalign*}
                        I & = \int_0^8 f\left(\dfrac{x}{2}\right) d x - \int_0^8 2 g(x) d x &
                  \end{flalign*}
                  Let $u = \dfrac{x}{2}$, then $du = \dfrac{1}{2}dx$.

                  When $x = 0$, $u = 0$.

                  When $x = 8$, $u = 4$.
                  \begin{flalign*}
                        \int_0^8 f\left(\dfrac{x}{2}\right) d x & = 2\int_0^4 f(u) d u & \\
                                                                & = 2 \cdot 2          & \\
                                                                & = 4
                  \end{flalign*}
                  \columnbreak

                  \begin{flalign*}
                        \int_0^8 2 g(x) d x & = 2\left[\int_0^3 g(x) d x + \int_3^8 g(x) d x\right] & \\
                                            & = 2(4 + 12)                                           & \\
                                            & = 32
                  \end{flalign*}
                  \begin{flalign*}
                        I & = 4 - 32 & \\
                          & = -28
                  \end{flalign*}
                  \vfill\null
            \end{multicols}

      \item Given the function $y=(x+3) \sqrt{2 x-3}$, find $\dfrac{d y}{d x}$. Hence, find
            $\displaystyle\int_2^6 \dfrac{x}{\sqrt{2 x-3}} d x$. \sol{}
            \begin{flalign*}
                  \dfrac{d y}{d x} & = \dfrac{1}{\sqrt{2x - 3}}(x + 3) + \sqrt{2x - 3} & \\
                                   & = \dfrac{3x}{\sqrt{2x - 3}}                       &
            \end{flalign*}
            \begin{flalign*}
                  I & = \dfrac{1}{3}\int_2^6 \dfrac{3x}{\sqrt{2x - 3}} d x & \\
                    & = \dfrac{1}{3}\bigg[(x + 3)\sqrt{2x - 3}\bigg]_2^6   & \\
                    & = \dfrac{1}{3}(27 - 5)                               & \\
                    & = \dfrac{22}{3}
            \end{flalign*}

      \item Given the function $y=x \ln x$, find $\dfrac{d y}{d x}$. Hence, find the
            following definite integrals: \sol{}
            \begin{flalign*}
                  \dfrac{d y}{d x} & = \ln x + 1 &
            \end{flalign*}
            \begin{multicols}{2}
                  \begin{enumerate}
                        \item $\displaystyle\int_1^4 \ln x d x$
                              \sol{}
                              \begin{flalign*}
                                    I & = \int_1^4 \left[(\ln x + 1) - 1\right] d x   & \\
                                      & = \int_1^4 (\ln x + 1) d x - \int_1^4 d x     & \\
                                      & = \bigg[x \ln x\bigg]_1^4 - \bigg[x\bigg]_1^4 & \\
                                      & = 4\ln 4 - 1\ln 1 - 4 + 1                     & \\
                                      & = 8\ln 2 - 3
                              \end{flalign*}
                              \vfill\null
                        \item $\displaystyle\int_1^4 \ln (2 x) d x$
                              \sol{}

                              Let $u = 2x$, then $du = 2dx$.

                              When $x = 1$, $u = 2$.

                              When $x = 4$, $u = 8$.
                              \begin{flalign*}
                                    I & = \dfrac{1}{2}\int_2^8 \ln u d u                                       & \\
                                      & = \dfrac{1}{2}\left[\int_2^8(\ln u + 1) - \int_2^8 d u\right]          & \\
                                      & = \dfrac{1}{2}\left(\bigg[u \ln u\bigg]_2^8 - \bigg[u\bigg]_2^8\right) & \\
                                      & = \dfrac{1}{2}\left(8\ln 8 - 2\ln 2 - 8 + 2\right)                     & \\
                                      & = 11\ln 2 - 3
                              \end{flalign*}
                  \end{enumerate}
            \end{multicols}

      \item Find the area of the region bounded by the curve $y=\dfrac{1}{x+1}$, the lines
            $x=1, x=7$, and the $x$-axis. \sol{}
            \begin{flalign*}
                  A & = \int_1^7 \dfrac{1}{x + 1} d x & \\
                    & = \bigg[\ln (x + 1)\bigg]_1^7   & \\
                    & = \ln 8 - \ln 2 = 2\ln 2
            \end{flalign*}

            \begin{multicols}{2}
                  \item Find the area of the region bounded by the curve \\ $y=\dfrac{3}{x}$ and the
                  line $y=4-x$. \sol{}
                  \begin{flalign*}
                        4 - x          & = \dfrac{3}{x}    & \\
                        4x - x^2       & = 3               & \\
                        x^2 - 4x + 3   & = 0               & \\
                        (x - 1)(x - 3) & = 0               & \\
                        x = 1          & \text{ or } x = 3
                  \end{flalign*}
                  \begin{flalign*}
                        A & = \int_1^3 \left(4 - x - \dfrac{3}{x}\right) d x         & \\
                          & = \bigg[4x - \dfrac{1}{2}x^2 - 3\ln x\bigg]_1^3          & \\
                          & = 12 - \dfrac{9}{2} - 3\ln 3 - 4 + \dfrac{1}{2} + 3\ln 1 & \\
                          & = 4 - 3\ln 3
                  \end{flalign*}
                  \vfill\null

                  \item Find the area of the region bounded by the curve $x=y^2-5 y$ and the line
                  $x+7y=24$. \sol{}
                  \begin{flalign*}
                        y^2 - 5y       & = 24 - 7y         & \\
                        y^2 + 2y - 24  & = 0               & \\
                        (y + 6)(y - 4) & = 0               & \\
                        y = -6         & \text{ or } y = 4
                  \end{flalign*}
                  \begin{flalign*}
                        A & = \int_{-6}^4 \left(24 - 7y - y^2 + 5y\right) d y & \\
                          & = \int_{-6}^4 (24 - 2y - y^2) d y                 & \\
                          & = \bigg[24y - y^2 - \dfrac{1}{3}y^3\bigg]_{-6}^4  & \\
                          & = 96 - 16 - \dfrac{64}{3} + 144 + 36 - 72         & \\
                          & = \dfrac{500}{3}
                  \end{flalign*}
                  \vfill\null
            \end{multicols}
            \vspace{-0.8cm}
            \vfill\null

      \item Find the area of the region bounded by the curves $y=x^2$ and $y^3=x$. \sol{}
            \begin{flalign*}
                  x^2        & = \sqrt[3]{x}     & \\
                  x^6        & = x               & \\
                  x^6 - x    & = 0               & \\
                  x(x^5 - 1) & = 0               & \\
                  x = 0      & \text{ or } x = 1
            \end{flalign*}
            \begin{flalign*}
                  A & = \int_0^1 \left(\sqrt[3]{x} - x^2\right) d x                   & \\
                    & = \bigg[\dfrac{3}{4}x^{\frac{4}{3}} - \dfrac{1}{3}x^3\bigg]_0^1 & \\
                    & = \dfrac{3}{4} - \dfrac{1}{3}                                   & \\
                    & = \dfrac{5}{12}
            \end{flalign*}
            \vfill\null

            \newpage
      \item Shown in the diagram below is the shaded region bounded by the curves $y=\ln x,
                  y=\ln (2 x-1)$, and the line $y=3$. Find the area of this region.
            \begin{center}
                  \includegraphics[scale=0.15]{assets/28-rev-28.png}
            \end{center}
            \sol{}
            \begin{flalign*}
                  y   & = \ln(2x - 1)        & \\
                  e^y & = 2x - 1             & \\
                  x   & = \dfrac{e^y + 1}{2} &
            \end{flalign*}
            \begin{flalign*}
                  y & = \ln x & \\
                  x & = e^y
            \end{flalign*}
            \begin{flalign*}
                  A & = \int_0^3 \left(e^y - \dfrac{e^y + 1}{2}\right) d y & \\
                    & = \int_0^3 \left(\dfrac{e^y - e^y - 1}{2}\right) d y & \\
                    & = \dfrac{1}{2}\int_0^3 (e^y - 1) d y                 & \\
                    & = \dfrac{1}{2}\bigg[e^y - y\bigg]_0^3                & \\
                    & = \dfrac{1}{2}(e^3 - 3 - 1)                          & \\
                    & = \dfrac{1}{2}(e^3 - 4)                              & \\
                    & = \dfrac{1}{2}e^3 - 2
            \end{flalign*}

            \newpage
      \item Find the area of the region bounded by the curves $x=y^3-y$ and $x=y-y^2$.
            \sol{}
            \begin{flalign*}
                  y^3 - y           & = y - y^2                & \\
                  y^3 + y^2 - 2y    & = 0                      & \\
                  y(y^2 + y - 2)    & = 0                      & \\
                  y(y + 2)(y - 1)   & = 0                      & \\
                  y = 0 \text{ or } & y = -2 \text{ or } y = 1
            \end{flalign*}
            \begin{flalign*}
                  A & = \int_{-2}^0 \left(y^3 - y - y + y^2\right) d y + \int_0^1 \left(y - y^2 - y^3 + y\right) d y                          & \\
                    & = \int_{-2}^0 (y^3 + y^2 - 2y) d y + \int_0^1 (-y^3 - y^2 + 2y) d y                                                     & \\
                    & = \bigg[\dfrac{1}{4}y^4 + \dfrac{1}{3}y^3 - y^2\bigg]_{-2}^0 + \bigg[-\dfrac{1}{4}y^4 - \dfrac{1}{3}y^3 + y^2\bigg]_0^1 & \\
                    & = -4 + \dfrac{8}{3} + 4 - \dfrac{1}{4} - \dfrac{1}{3} + 1                                                               & \\
                    & = \dfrac{37}{12}
            \end{flalign*}

      \item Shown in the diagram below is the shaded region bounded by the curves $y=\sin
                  x$ 及 $y=\sin 2 x$ in the interval $0 \leq x \leq \pi$. Find the area of this
            region.
            \begin{center}
                  \includegraphics[scale=0.2]{assets/28-rev-30.png}
            \end{center}
            \sol{}
            \begin{flalign*}
                  \sin x                 & = 2 \sin x \cos x                      & \\
                  2\sin x\cos x - \sin x & = 0                                    & \\
                  \sin x(2\cos x - 1)    & = 0                                    & \\
                  \sin x = 0 \text{ or } & \cos x = \dfrac{1}{2}                  & \\
                  x = 0 \text{ or }      & x = \dfrac{\pi}{3} \text{ or } x = \pi
            \end{flalign*}
            \begin{flalign*}
                  A & = \int_0^{\frac{\pi}{3}} (\sin 2x - \sin x) d x + \int_{\frac{\pi}{3}}^{\pi} (\sin x - \sin 2x) d x                             & \\
                    & = \bigg[-\dfrac{1}{2}\cos 2x + \cos x\bigg]_0^{\frac{\pi}{3}} + \bigg[-\cos x + \dfrac{1}{2}\cos 2x\bigg]_{\frac{\pi}{3}}^{\pi} & \\
                    & = \dfrac{1}{4} + \dfrac{1}{2} + \dfrac{1}{2} - 1 + 1 + \dfrac{1}{2} + \dfrac{1}{2} + \dfrac{1}{4} = \dfrac{5}{2}
            \end{flalign*}

            \begin{multicols}{2}
                  \item Find the volume of the solid of revolution formed by rotating the region
                  bounded by the curve $y=\dfrac{1}{x+2}$, the line $x=2$, and two axes about the
                  $x$-axis. \sol{}
                  \begin{flalign*}
                        V & = \int_0^2 \pi\left(\dfrac{1}{x + 2}\right)^2 d x & \\
                          & = \pi\int_0^2 \dfrac{1}{(x + 2)^2} d x
                  \end{flalign*}
                  Let $u = x + 2$, then $du = dx$.

                  When $x = 0$, $u = 2$.

                  When $x = 2$, $u = 4$.
                  \begin{flalign*}
                        V & = \pi\int_2^4 \dfrac{1}{u^2} d u               & \\
                          & = \pi\bigg[-\dfrac{1}{u}\bigg]_2^4             & \\
                          & = \pi\left(-\dfrac{1}{4} + \dfrac{1}{2}\right) & \\
                          & = \dfrac{\pi}{4}
                  \end{flalign*}
                  \vfill\null
                  \item Find the volume of the solid of revolution formed by rotating the region
                  bounded by the curve $y=e^x-3$ and the two axes about the $x$-axis. \sol{}
                  \begin{flalign*}
                        e^x - 3 & = 0     & \\
                        e^x     & = 3     & \\
                        x       & = \ln 3
                  \end{flalign*}
                  \begin{flalign*}
                        V & = \int_0^{\ln 3} \pi(e^x - 3)^2 d x                                                   & \\
                          & = \pi\int_0^{\ln 3} (e^{2x} - 6e^x + 9) d x                                           & \\
                          & = \pi\int_0^{\ln 3} e^{2x} d x - \pi\int_0^{\ln 3} 6e^x d x + \pi\int_0^{\ln 3} 9 d x & \\
                  \end{flalign*}
                  Let $u = 2x$, then $du = 2dx$.

                  When $x = 0$, $u = 0$.

                  When $x = \ln 3$, $u = 2\ln 3$.
                  \begin{flalign*}
                        V & = \dfrac{\pi}{2}\int_0^{2\ln 3} e^u d u - 6\pi\int_0^{\ln 3} e^x d x + 9\pi\int_0^{\ln 3} d x            & \\
                          & = \dfrac{\pi}{2}\bigg[e^u\bigg]_0^{2\ln 3} - 6\pi\bigg[e^x\bigg]_0^{\ln 3} + 9\pi\bigg[x\bigg]_0^{\ln 3} & \\
                          & = \dfrac{\pi}{2}(9 - 1) - 6\pi(3 - 1) + 9\pi\ln 3                                                        & \\
                          & = 4\pi - 12\pi + 9\pi\ln 3                                                                               & \\
                          & = \pi(9\ln 3 - 8)
                  \end{flalign*}
                  \vfill\null
            \end{multicols}
            \vfill\null

      \item Find the volume of the solid of revolution formed by rotating the region
            bounded by the curve $x=y^2-3 y$ and the $y$-axis about the $y$-axis. \sol{}
            \begin{flalign*}
                  V & = \int_{0}^{3} \pi(y^2 - 3y)^2 d y                                & \\
                    & = \pi\int_{0}^{3} (y^4 - 6y^3 + 9y^2) d y                         & \\
                    & = \pi\bigg[\dfrac{1}{5}y^5 - \dfrac{3}{2}y^4 + 3y^3\bigg]_{0}^{3} & \\
                    & = \pi\left(\dfrac{243}{5} - \dfrac{243}{2} + 81\right)            & \\
                    & = \dfrac{81\pi}{10}
            \end{flalign*}
            \vfill\null

            \newpage
            \begin{multicols}{2}
                  \item Find the volume of the solid of revolution formed by rotating the region
                  bounded by the curve $y=x^2$ and the line $y=x+2$ about the $x$-axis. \sol{}
                  \begin{flalign*}
                        x^2               & = x + 2 & \\
                        x^2 - x - 2       & = 0     & \\
                        (x - 2)(x + 1)    & = 0     & \\
                        x = 2 \text{ or } & x = -1
                  \end{flalign*}
                  \vspace{-0.5cm}
                  \begin{flalign*}
                        V & = \int_{-1}^2 \pi\left[(x + 2)^2 - x^4\right] d x                                             & \\
                          & = \pi\int_{-1}^2 (x^2 + 4x + 4 - x^4) d x                                                     & \\
                          & = \pi\bigg[-\dfrac{1}{5}x^5 + \dfrac{1}{3}x^3 + 2x^2 + 4x\bigg]_{-1}^2                        & \\
                          & = \pi\left(-\dfrac{32}{5} + \dfrac{8}{3} + 8 + 8 - \dfrac{1}{5} + \dfrac{1}{3} - 2 + 4\right) & \\
                          & = \dfrac{72\pi}{15}
                  \end{flalign*}
                  \vfill\null

                  \item Find the volume of the solid of revolution formed by rotating the region
                  bounded by the curve $y^2=x+9$ and the line $y=x+3$ about the $y$-axis. \sol{}
                  \begin{flalign*}
                        y^2 - 9           & = y - 3 & \\
                        y^2 - y - 6       & = 0     & \\
                        (y - 3)(y + 2)    & = 0     & \\
                        y = 3 \text{ or } & y = -2
                  \end{flalign*}
                  \vspace{-0.5cm}
                  \begin{flalign*}
                        V & = \int_{-2}^3 \pi\left[(y^2 - 9)^2 - (y - 3)^2\right] d y                                     & \\
                          & = \pi\int_{-2}^3 (y^4 - 18y^2 + 81 - y^2 + 6y - 9) d y                                        & \\
                          & = \pi\int_{-2}^3 (y^4 - 19y^2 + 6y + 72) d y                                                  & \\
                          & = \pi\bigg[\dfrac{1}{5}y^5 - \dfrac{19}{3}y^3 + 3y^2 + 72y\bigg]_{-2}^3                       & \\
                          & = \pi\left(\dfrac{243}{5} - 171 + 27 + 216 + \dfrac{32}{5} - \dfrac{152}{3} - 12 + 144\right) & \\
                          & = \dfrac{625\pi}{3}
                  \end{flalign*}
                  \vfill\null
            \end{multicols}
            \vspace{-0.8cm}

      \item Given that a region is bounded by the curve $y^2=8 x$ and $y=x^2$. Find the
            volume of the solid of revolution formed by rotating this region about the
            $x$-axis and the $y$-axis respectively. \sol{}
            \begin{flalign*}
                  x^4               & = 8x  & \\
                  x(x^3 - 8)        & = 0   & \\
                  x = 0 \text{ or } & x = 2
            \end{flalign*}
            \vspace{-0.8cm}
            \begin{flalign*}
                  V_x & = \pi\int_0^2 (8x - x^4) d x                & \\
                      & = \pi\bigg[4x^2 - \dfrac{1}{5}x^5\bigg]_0^2 & \\
                      & = \pi\left(16 - \dfrac{32}{5}\right)        & \\
                      & = \dfrac{48\pi}{5}
            \end{flalign*}
            When $x = 2$, $y = 4$.
            \begin{flalign*}
                  V_y & = \pi\int_0^4 \left(y - \dfrac{y^4}{64}\right) d y       & \\
                      & = \pi\bigg[\dfrac{1}{2}y^2 - \dfrac{1}{320}y^5\bigg]_0^4 & \\
                      & = \pi\left(8 - \dfrac{1024}{320}\right)                  & \\
                      & = \dfrac{24\pi}{5}
            \end{flalign*}

      \item Shown in the diagram below is the shaded region bounded by the curve $y =
                  2\cos\pi x$, the line $y = 3x$, and the $y$-axis.
            \begin{center}
                  \includegraphics[scale=0.15]{assets/28-rev-37.png}
            \end{center}
            \begin{enumerate}
                  \item Prove that the $x$-coordinate of point $A$ is $\dfrac{1}{3}$. \prooff{}

                        When $x = \dfrac{1}{3}$, $y = 2\cos\left(\dfrac{\pi}{3}\right) = 1$.

                        When $x = \dfrac{1}{3}$, $y = 3\left(\dfrac{1}{3}\right) = 1$.

                        Since point $A$ is the point of intersection of the two curves, and both
                        functions give the same $y$-value when $x = \dfrac{1}{3}$, the $x$-coordinate
                        of point $A$ is $\dfrac{1}{3}$. \qed
                  \item Find the volume of the solid of revolution formed by rotating this region about
                        the $x$-axis. \sol{}
                        \begin{flalign*}
                              V & = \pi\int_0^{\frac{1}{3}} (4\cos^2\pi x - 9x^2) d x                                 & \\
                                & = \pi\int_0^{\frac{1}{3}} (2 - 9x^2) d x + 2\pi\int_0^{\frac{1}{3}} \cos 2\pi x d x
                        \end{flalign*}
                        Let $u = 2\pi x$, then $du = 2\pi dx$.

                        When $x = 0$, $u = 0$, when $x = \dfrac{1}{3}$, $u = \dfrac{2\pi}{3}$.
                        \begin{flalign*}
                              V & = \pi\int_0^{\frac{1}{3}} (2 - 9x^2) d x + \int_0^{\frac{2\pi}{3}} \cos u d u      & \\
                                & = \pi\bigg[2x - 3x^3\bigg]_0^{\frac{1}{3}} + \bigg[\sin u\bigg]_0^{\frac{2\pi}{3}} & \\
                                & = \pi\left(\dfrac{2}{3} - \dfrac{1}{9}\right) + \dfrac{\sqrt{3}}{2}                & \\
                                & = \dfrac{5\pi}{9} + \dfrac{\sqrt{3}}{2}
                        \end{flalign*}
            \end{enumerate}
      \item Given that a region is bounded by the curve $xy = 12$, the line $x = 4$, and $y
                  = 6$. Find the volume of the solid of revolution formed by rotating this region
            about the $x$-axis and the $y$-axis respectively. \sol{} \vspace{-0.8cm}
            \begin{multicols}{2}
                  \begin{flalign*}
                        V_x & = \pi\int_2^4 \left(36 - \dfrac{144}{x^2}\right) d x & \\
                            & = \pi\bigg[36x + \dfrac{144}{x}\bigg]_2^4            & \\
                            & = \pi\left(144 + 36 - 72 - 72\right)                 & \\
                            & = 36\pi
                  \end{flalign*}

                  \begin{flalign*}
                        V_y & = \pi\int_3^6 \left(16 - \dfrac{144}{y^2}\right) d y & \\
                            & = \pi\bigg[16y + \dfrac{144}{y}\bigg]_3^6            & \\
                            & = \pi\left(96 + 24 - 48 - 48\right)                  & \\
                            & = 24\pi
                  \end{flalign*}
            \end{multicols}
\end{enumerate}