\chapter{Projections}

Given two vectors $\vec{v}$ and $\vec{u}$. Construct a vector $\vec{w_1}$ from the terminal
point of $\vec{u}$ perpendicular to $\vec{v}$. The vector that starts from the
initial point of $\vec{u}$ and ends at the intersection of the line and
$\vec{v}$ is called the \textbf{projection of $\vec{u}$ onto $\vec{v}$}, which
is also known as the \textbf{vector component of $\vec{u}$ along $\vec{v}$}.

\begin{center}
    \begin{tikzpicture}[scale=1.8]
        %draw vector u and v
        \draw[->] (0, 0) -- (2, 2);
        \draw[->] (0, 0) -- (3, 0);
        %draw vector w_1, w_2
        \draw[->, thick] (0, 0) -- (2, 0);
        \draw[->, thick] (0, 0) -- (0, 2);
        %draw dashed line
        \draw[-, dashed] (2, 2) -- (2, 0);
        %label vectors
        \node[above left] at (1, 1) {$\vec{u}$};
        \node[below] at (1, 0) {$\vec{w_1}$};
        \node[below] at (2.5, 0) {$\vec{v}$};
        \node[left] at (0, 1) {$\vec{w_2}$};
    \end{tikzpicture}
\end{center}

\noindent From the diagram, it is not hard to see that $\vec{u} = \vec{w_1} + \vec{w_2}$

\noindent Hence, the vector component of $\vec{u}$ orthogonal to $\vec{v}$ is given by $\vec{w_2} = \vec{u} - \vec{w_1}$

\noindent Let $\vec{w_1} = t\vec{v}$, for some scalar $t$.

\noindent Then $\vec{w_2} = \vec{u} - t\vec{v}$ is orthogonal to $\vec{v}$, which implies that $\vec{w_2} \cdot \vec{v} = 0$.
\begin{align*}
    \vec{w_2} \cdot \vec{v}                        & = (\vec{u} - t\vec{v}) \cdot \vec{v} = 0              \\
    \vec{u} \cdot \vec{v} - t\vec{v} \cdot \vec{v} & = 0                                                   \\
    t                                              & = \frac{\vec{u} \cdot \vec{v}}{\vec{v} \cdot \vec{v}}
\end{align*}
Therefore, The projection of $\vec{u}$ onto $\vec{v}$ is given by \[proj_{\vec{v}}\vec{u} = \vec{w_1} = t\vec{v} = \left(\dfrac{\vec{u} \cdot \vec{v}}{\vec{v} \cdot \vec{v}}\right)\vec{v} = \left(\frac{\vec{u} \cdot \vec{v}}{\norm{\vec{v}}^2}\right)\vec{v} = \left(\frac{\vec{u} \cdot \vec{v}}{\norm{\vec{v}}}\right)\dfrac{\vec{v}}{\norm{\vec{v}}}\] where $\dfrac{\vec{u} \cdot \vec{v}}{\norm{\vec{v}}}$ is the \textbf{scalar projection} of $\vec{u}$ onto $\vec{v}$, denoted by $comp_{\vec{v}}\vec{u}$.

\newpage
\noindent\textbf{Example 1. } Find the projection of $\vec{u} = \langle 6, 7 \rangle$ onto $\vec{v} = \langle 1, 4 \rangle$. Hence, find the vector component of $\vec{u}$ orthogonal to $\vec{v}$.
\begin{align*}
    proj_{\vec{v}}\vec{u} & = \left(\frac{\vec{u} \cdot \vec{v}}{\norm{\vec{v}}^2}\right)\vec{v} \\
                          & = \left(\frac{6(1) + 7(4)}{1^2 + 4^2}\right)\langle 1, 4 \rangle     \\
                          & = \left(\frac{34}{17}\right)\langle 1, 4 \rangle                     \\
                          & = \langle 2, 8 \rangle                                               \\
    \\
    \vec{w_2}             & = \vec{u} - \vec{w_1}                                                \\
                          & = \langle 6, 7 \rangle - \langle 2, 8 \rangle                        \\
                          & = \langle 4, -1 \rangle
\end{align*}
\noindent\textbf{Example 2. } Find the projection of $\vec{u} = 2i + 3j$ onto $\vec{v} = 5i + j$. Hence, find the vector component of $\vec{u}$ orthogonal to $\vec{v}$.
\begin{align*}
    proj_{\vec{v}}\vec{u} & = \left(\frac{\vec{u} \cdot \vec{v}}{\norm{\vec{v}}^2}\right)\vec{v}             \\
                          & = \left(\frac{2(5) + 3(1)}{5^2 + 1^2}\right)(5i + j)                             \\
                          & = \left(\frac{13}{26}\right)(5i + j)                                             \\
                          & = \left(\frac{5}{2}\right)i + \left(\frac{1}{2}\right)j                          \\
    \\
    \vec{w_2}             & = \vec{u} - \vec{w_1}                                                            \\
                          & = (2i + 3j) - \left(\left(\frac{5}{2}\right)i + \left(\frac{1}{2}\right)j\right) \\
                          & = \left(-\frac{1}{2}\right)i + \left(\frac{5}{2}\right)j
\end{align*}

\newpage
\noindent\textbf{Example 3. } Find the scalar projection of the force $\vec{F} = 4i - 2j + 3k$ in the direction of the vector $v = i - j + 2k$.

\sol{}
\begin{align*}
    comp_{\vec{v}}\vec{F} & = \frac{\vec{F} \cdot \vec{v}}{\norm{\vec{v}}}             \\
                          & = \frac{4(1) + (-2)(-1) + 3(2)}{\sqrt{1^2 + (-1)^2 + 2^2}} \\
                          & = \frac{4 + 2 + 6}{\sqrt{6}}                               \\
                          & = \frac{12}{\sqrt{6}}                                      \\
                          & = 2\sqrt{6}
\end{align*}

\noindent\textbf{Notes:} Selected exercises are mixed in the exercises of the previous chapters.