\chapter{Cylindrical Coordinates}

Cylindrical coordinates are an extension of polar coordinates into three
dimensions. A point $P(x, y, z)$ in space is represented by the ordered triple
$(r, \theta, z)$, called the \textbf{cylindrical coordinates} of $P$. Here,
$(r, \theta)$ are the polar representation of the projection of $P$ in the
$xy$-plane, and $z$ is the direct distance from the $(r, \theta)$ to $P$.
\begin{center}
    \tdplotsetmaincoords{60}{110}
    \begin{tikzpicture}[tdplot_main_coords,scale=0.8]
        %draw x, y and z axis
        \draw[->] (0, 0, 0) -- (5, 0, 0);
        \draw[->] (0, 0, 0) -- (0, 5, 0);
        \draw[->] (0, 0, 0) -- (0, 0, 5);
        %draw point P
        \draw[fill] (4, 4, 4) circle [radius=0.05];
        %label point P
        \node[left] at (4, 4, 4) {$P$};
        \node[above=0.5em, right] at (4, 4, 4) {$(x,y,z)$};
        \node[below right] at (4, 4, 4) {$(r,\theta,z)$};
        %draw line connecting P to xy-plane
        \draw[-, dashed] (4, 4, 4) -- (4, 4, 0);
        %draw line connecting P to x-axis
        \draw[-, dashed] (4, 4, 0) -- (4, 0, 0);
        %draw line connecting P to y-axis
        \draw[-, dashed] (4, 4, 0) -- (0, 4, 0);
        %draw line connecting P to origin
        \draw[-, dashed] (4, 4, 0) -- (0, 0, 0) node [above,midway] {$r$};
        %draw arc
        \tdplotdefinepoints(0, 0, 0)(4, 0, 0)(4, 4, 0)
        \tdplotdrawpolytopearc[->,thick]{1}{below}{$\theta$}
        %label x, y and z axis
        \node[below left] at (5, 0, 0) {$x$};
        \node[right] at (0, 5, 0) {$y$};
        \node[above] at (0, 0, 5) {$z$};
        %curly braces for x and y
        \draw [decorate,decoration={brace,mirror,amplitude=5pt},yshift=-5pt]
        (4, 0, 0) -- (4, 4, 0) node [black,midway,below=0.5em] {\footnotesize $y$};
        \draw [decorate,decoration={brace,mirror,amplitude=5pt},xshift=-5pt]
        (0, 0, 0) -- (4, 0, 0) node [black,midway,left=0.6em,above=0.3em] {\footnotesize $x$};
    \end{tikzpicture}
\end{center}

To convert from Cartesian coordinates to cylindrical coordinates, we use the
following equations. \[r^2 = x^2 + y^2 \qquad \tan\theta = \frac{y}{x} \qquad z = z\]

To convert from cylindrical coordinates to Cartesian coordinates, we use the
following equations. \[x = r\cos\theta \qquad y = r\sin\theta \qquad z = z\]

\noindent\textbf{Example 1. } Convert $\left(4, \dfrac{5\pi}{6}, 3\right)$ from cylindrical coordinates to Cartesian coordinates.
\begin{align*}
    x & = r\cos\theta                      \\
      & = 4\cos\left(\frac{5\pi}{6}\right) \\
      & = -2\sqrt{3}                       \\
    y & = r\sin\theta                      \\
      & = 4\sin\left(\frac{5\pi}{6}\right) \\
      & = 2                                \\
    z & = z                                \\
      & = 3
\end{align*}
Hence, $\left(-2\sqrt{3}, 2, 3\right)$ is the Cartesian coordinates of $\left(4, \dfrac{5\pi}{6}, 3\right)$.
~\\\\
\noindent\textbf{Example 2. } Convert $\left(1, \sqrt{3}, 2\right)$ from Cartesian coordinates to cylindrical coordinates.
\begin{align*}
    r      & = \sqrt{x^2 + y^2}                         \\
           & = \sqrt{1^2 + \sqrt{3}^2}                  \\
           & = 2                                        \\
    \theta & = \tan^{-1}\left(\frac{y}{x}\right)        \\
           & = \tan^{-1}\left(\frac{\sqrt{3}}{1}\right) \\
           & = \frac{\pi}{3}                            \\
    z      & = z                                        \\
           & = 2
\end{align*}
Hence, $\left(2, \dfrac{\pi}{3}, 2\right)$ is the cylindrical coordinates of $\left(1, \sqrt{3}, 2\right)$.
