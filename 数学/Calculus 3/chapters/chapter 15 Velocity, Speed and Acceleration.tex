\chapter{Velocity, Speed and Acceleration}

If $x(t)$, $y(t)$ and $z(t)$ are differentiable functions, then the velocity of
a vector-valued function $\vec{r}(t) = x(t)\hat{\imath} + y(t)\hat{\jmath} +
    z(t)\hat{k}$ is \[\vec{v}(t) = (\vec{r})'(t) = x'(t)\hat{\imath} + y'(t)\hat{\jmath} + z'(t)\hat{k}\]

The speed of a vector-valued function $\vec{r}(t) = x(t)\hat{\imath} +
    y(t)\hat{\jmath} + z(t)\hat{k}$ is \[|\vec{v}(t)| = \sqrt{x'(t)^2 + y'(t)^2 + z'(t)^2}\]

The acceleration of a vector-valued function $\vec{r}(t) = x(t)\hat{\imath} +
    y(t)\hat{\jmath} + z(t)\hat{k}$ is \[\vec{a}(t) = (\vec{v})'(t) = x''(t)\hat{\imath} + y''(t)\hat{\jmath} + z''(t)\hat{k}\]

The path of a projectile launched from an initial height $h$ with initial speed
$v_0$ at an angle of elevation $\theta$ is given by \[\vec{r}(t) = v_0\cos\theta t\hat{\imath} + \left[h + (v_0\sin\theta) t - \frac{1}{2}gt^2\right]\hat{\jmath}\]

Since this topic is relatively straightforward also, there will be no examples.
:)

\newpage

\section*{Selected Exercises}
\textit{Source: Larson Calculus 11th Ed. Exercise 12.3}

\subsection*{Projectile Motion}
In Exercises 27-32, use the model for projectile motion, assuming there is no
air resistance and $g=9.8$ meters per second per second.
\begin{enumerate}
    \setcounter{enumi}{26}
    \item  A baseball is hit from a height of 1 meter above the ground with an initial
          speed of 40 feet per second and at an angle of $22^{\circ}$ above the
          horizontal. Find the maximum height reached by the baseball. Determine whether
          it will clear a 3-meters-high fence located 105 meters from home plate.

          \sol{}  Given that $h = 1$, $v_0 = 40$, $\theta = 22^{\circ}$ and $g = 9.8$,
          \begin{align*}
              \vec{r}(t) & = v_0\cos\theta t\hat{\imath} + \left[h + (v_0\sin\theta) t - \frac{1}{2}gt^2\right]\hat{\jmath}               \\
                         & = 40\cos{22^{\circ}} t\hat{\imath} + \left[1 + (40\sin{22^{\circ}}) t - \frac{1}{2}(9.8)t^2\right]\hat{\jmath}
          \end{align*}
          The velocity vector is
          \begin{align*}
              \vec{v}(t) & = (\vec{r})'(t) = 40\cos{22^{\circ}}\hat{\imath} + (40\sin{22^{\circ}} - 9.8t)\hat{\jmath}
          \end{align*}
          The maximum height is reached when the vertical component of the velocity is $0$.
          \begin{align*}
              40\sin{22^{\circ}} - 9.8t & = 0 \implies t = \frac{40\sin{22^{\circ}}}{9.8} \approx 1.53\ \text{seconds}
          \end{align*}
          The maximum height is
          \begin{align*}
              y & = 1 + (40\sin{22^{\circ}})\left(\frac{40\sin{22^{\circ}}}{9.8}\right) - \frac{1}{2}(9.8)\left(\frac{40\sin{22^{\circ}}}{9.8}\right)^2 \\
                & \approx 12.46\ \text{meters}
          \end{align*}
          The ball is 105 meters from home plate when $x(t) = 105$.
          \begin{align*}
              40\cos{22^{\circ}} t & = 105 \implies t = \frac{105}{40\cos{22^{\circ}}}  \approx 2.83\ \text{seconds}
          \end{align*}
          At this time, the height of the ball is
          \begin{align*}
              y & = 1 + (40\sin{22^{\circ}})\left(\frac{105}{40\cos{22^{\circ}}}\right) - \frac{1}{2}(9.8)\left(\frac{105}{40\cos{22^{\circ}}}\right)^2 \\
                & \approx 4.15\ \text{meters}
          \end{align*}
          Hence, the ball will clear the fence. \hfill$\blacksquare$

    \item Determine the maximum height and range of a projectile fired at a height of 2
          meters above the ground with an initial speed of 300 meters per second and at
          angle of $45^{\circ}$ above the horizontal.

          \sol{}  Given that $h = 2$, $v_0 = 300$, $\theta = 45^{\circ}$ and $g = 9.8$,
          \begin{align*}
              \vec{r}(t) & = v_0\cos\theta t\hat{\imath} + \left[h + (v_0\sin\theta) t - \frac{1}{2}gt^2\right]\hat{\jmath}                 \\
                         & = 300\cos{45^{\circ}} t\hat{\imath} + \left[2 + (300\sin{45^{\circ}}) t - \frac{1}{2}(9.8)t^2\right]\hat{\jmath} \\
                         & = 150\sqrt{2} t\hat{\imath} + \left[2 + 150\sqrt{2} t - 4.9t^2\right]\hat{\jmath}
          \end{align*}
          The velocity vector is
          \begin{align*}
              \vec{v}(t) & = (\vec{r})'(t)                                              \\
                         & = 150\sqrt{2}\hat{\imath} + (150\sqrt{2} - 9.8t)\hat{\jmath}
          \end{align*}
          The maximum height is reached when the vertical component of the velocity is $0$.
          \begin{align*}
              150\sqrt{2} - 9.8t & = 0                           \\
              t                  & = \frac{150\sqrt{2}}{9.8}     \\
                                 & \approx 21.65\ \text{seconds}
          \end{align*}
          The maximum height is
          \begin{align*}
              y & = 2 + (150\sqrt{2})\left(\frac{150\sqrt{2}}{9.8}\right) - 4.9\left(\frac{150\sqrt{2}}{9.8}\right)^2 \\
                & \approx 2,297.92\ \text{meters}
          \end{align*}
          The projectile hits the ground when $y(t) = 0$.
          \begin{align*}
              2 + (150\sqrt{2}) t - 4.9t^2 & = 0                                                           \\
              t                            & = \frac{-150\sqrt{2} - \sqrt{150^2(2) - 4(-4.9)(2)}}{2(-4.9)} \\
                                           & \approx 43.302\ \text{seconds}
          \end{align*}
          Hence, the range is
          \begin{align*}
              x & = 150\sqrt{2} \left(\frac{-150\sqrt{2} - \sqrt{150^2(2) - 4(-4.9)(2)}}{2(-4.9)}\right) \approx 9185.67\ \text{meters}
          \end{align*} \hfill$\blacksquare$

          \newpage
    \item A baseball, hit 1 meter above the ground, leaves the bat at an angle of
          $45^{\circ}$ and is caught by an outfielder 1 feet above the ground and 100
          feet from home plate. What is the initial speed of the ball, and how high does
          it rise?

          \sol{}  Given that $h = 1$, $\theta = 45^{\circ}$ and $g = 9.8$,
          \begin{align*}
              \vec{r}(t) & = v_0\cos\theta t\hat{\imath} + \left[h + (v_0\sin\theta) t - \frac{1}{2}gt^2\right]\hat{\jmath}                 \\
                         & = v_0\cos{45^{\circ}} t\hat{\imath} + \left[1 + (v_0\sin{45^{\circ}}) t - \frac{1}{2}(9.8)t^2\right]\hat{\jmath} \\
                         & = \frac{v_0}{\sqrt{2}} t\hat{\imath} + \left[1 + \frac{v_0}{\sqrt{2}} t - 4.9t^2\right]\hat{\jmath}
          \end{align*}
          When the ball is caught, $x(t) = 100$ and $y(t) = 1$.
          \begin{align*}
              \frac{v_0}{\sqrt{2}} t & = 100 \implies t = \frac{100\sqrt{2}}{v_0}\ \cdots\ (1)
          \end{align*}
          \vspace{-2.4em}
          \begin{align*}
              1 + \frac{v_0}{\sqrt{2}} t - 4.9t^2 & = 1                                     \\
              4.9t^2 - \frac{v_0}{\sqrt{2}} t     & = 0                                     \\
              t(4.9t - \frac{v_0}{\sqrt{2}})      & = 0                                     \\
              t                                   & = \frac{v_0}{4.9\sqrt{2}} \ \cdots\ (2)
          \end{align*}
          Equating $(1)$ and $(2)$,
          \begin{align*}
              \frac{100\sqrt{2}}{v_0} & = \frac{v_0}{4.9\sqrt{2}}                                  \\
              v_0^2                   & = 980                                                      \\
              v_0                     & = 14\sqrt{5}\ \text{meters per second}\ \text{($v_0 > 0$)}
          \end{align*}
          Hence,
          \begin{align*}
              \vec{r}(t) & = \frac{14\sqrt{5}}{\sqrt{2}} t\hat{\imath} + \left[1 + \frac{14\sqrt{5}}{\sqrt{2}} t - 4.9t^2\right]\hat{\jmath} \\
                         & = 7\sqrt{10} t\hat{\imath} + \left[1 + 7\sqrt{10} t - 4.9t^2\right]\hat{\jmath}
          \end{align*}
          The maximum height is reached when the vertical component of the velocity is $0$.
          \begin{align*}
              7\sqrt{10} - 9.8t & = 0                          \\
              t                 & = \frac{7\sqrt{10}}{9.8}     \\
                                & \approx 2.26\ \text{seconds}
          \end{align*}
          The maximum height is
          \begin{align*}
              y & = 1 + (7\sqrt{10})\left(\frac{7\sqrt{10}}{9.8}\right) - 4.9\left(\frac{7\sqrt{10}}{9.8}\right)^2 \\
                & = 26\ \text{meters}
          \end{align*} \hfill$\blacksquare$

    \item A baseball player at second base throws a ball 28 meters to the player at first
          base. The ball is released at a point 1.5 meters above the ground with an
          initial speed of 80 kilometres per hour and at an angle of $15^{\circ}$ above
          the horizontal. At what height does the player at first base catch the ball?

          \sol{}  Given that $h = 1.5$, $v_0 = 80$km/h$=80\times\dfrac{1000}{3600}=22\dfrac{2}{9}$m/s, $\theta = 15^{\circ}$ and $g = 9.8$,
          \begin{align*}
              \vec{r}(t) & = v_0\cos\theta t\hat{\imath} + \left[h + (v_0\sin\theta) t - \frac{1}{2}gt^2\right]\hat{\jmath}                                         \\
                         & = 22\dfrac{2}{9}\cos{15^{\circ}} t\hat{\imath} + \left[1.5 + (22\dfrac{2}{9}\sin{15^{\circ}}) t - \frac{1}{2}(9.8)t^2\right]\hat{\jmath} \\
                         & = 22\dfrac{2}{9}\cos{15^{\circ}} t\hat{\imath} + \left[1.5 + 22\dfrac{2}{9}\sin{15^{\circ}} t - 4.9t^2\right]\hat{\jmath}
          \end{align*}
          When the ball is caught, $x(t) = 28$
          \begin{align*}
              22\dfrac{2}{9}\cos{15^{\circ}} t & = 28 \implies t = \frac{28}{22\dfrac{2}{9}\cos{15^{\circ}}} \approx 1.305\ \text{seconds}
          \end{align*}
          The height of the ball is
          \begin{align*}
              y & = 1.5 + 22\dfrac{2}{9}\sin{15^{\circ}} \left(\frac{28}{22\dfrac{2}{9}\cos{15^{\circ}}}\right) - 4.9\left(\frac{28}{22\dfrac{2}{9}\cos{15^{\circ}}}\right)^2 \\
                & \approx 0.66\ \text{meters}
          \end{align*} \hfill$\blacksquare$

          \newpage
    \item Eliminate the parameter $t$ from the position vector for the motion of a
          projectile to show that the rectangular equation is $$ y=-\frac{g \sec ^2
                  \theta}{2 v_0^2} x^2+(\tan \theta) x+h $$

          \sol{}
          \begin{align*}
              \vec{r}(t) & = v_0\cos\theta t\hat{\imath} + \left[h + (v_0\sin\theta) t - \frac{1}{2}gt^2\right]\hat{\jmath}
          \end{align*}
          Let $x = v_0\cos\theta t$ and $y = h + (v_0\sin\theta) t - \dfrac{1}{2}gt^2$.
          \begin{align*}
              x & = v_0\cos\theta t                                                                                               \\
              t & = \frac{x}{v_0\cos\theta}                                                                                       \\
              y & = h + (v_0\sin\theta) t - \frac{1}{2}gt^2                                                                       \\
                & = h + (v_0\sin\theta) \left(\frac{x}{v_0\cos\theta}\right) - \frac{1}{2}g\left(\frac{x}{v_0\cos\theta}\right)^2 \\
                & = h + \frac{v_0\sin\theta}{v_0\cos\theta} x - \frac{1}{2}\frac{g}{v_0^2\cos^2\theta} x^2                        \\
                & = h + (\tan\theta) x - \frac{1}{2}\frac{g}{v_0^2\cos^2\theta} x^2                                               \\
                & = -\frac{g \sec ^2 \theta}{2 v_0^2} x^2+(\tan \theta) x+h
          \end{align*} \hfill$\blacksquare$

    \item The path of a ball is given by the rectangular equation $$ y=x-0.0245 x^2 \text
              {. } $$ Use the result of Exercise 31 to find the position vector. Then find
          the speed and direction of the ball at the point at which it has travelled 15
          meters horizontally.

          \sol{}  Comparing the equation with the equation in Exercise 31,
          \begin{align*}
              \tan \theta                       & = 1                                                \\
              \theta                            & = 45^{\circ}                                       \\
              -\frac{g \sec ^2 \theta}{2 v_0^2} & = -0.0245                                          \\
              \frac{9.8}{v_0^2}                 & = 0.0245                                           \\
              v_0^2                             & = 400                                              \\
              v_0                               & = 20\ \text{meters per second}\ \text{($v_0 > 0$)}
          \end{align*}
          Hence, the position vector is
          \begin{align*}
              \vec{r}(t) & = v_0\cos\theta t\hat{\imath} + \left[h + (v_0\sin\theta) t - \frac{1}{2}gt^2\right]\hat{\jmath}               \\
                         & = 20\cos{45^{\circ}} t\hat{\imath} + \left[0 + (20\sin{45^{\circ}}) t - \frac{1}{2}(9.8)t^2\right]\hat{\jmath} \\
                         & = 10\sqrt{2} t\hat{\imath} + \left[10\sqrt{2} t - 4.9t^2\right]\hat{\jmath}
          \end{align*}
          The time taken to travel 15 meters horizontally is
          \begin{align*}
              10\sqrt{2} t & = 15 \implies t = \frac{3\sqrt{2}}{2} \approx 2.12\ \text{seconds}
          \end{align*}
          The speed vector is
          \begin{align*}
              \vec{v}(t) & = (\vec{r})'(t) = 10\sqrt{2}\hat{\imath} + (10\sqrt{2} - 9.8t)\hat{\jmath}
          \end{align*}
          The direction of the ball is
          \begin{align*}
              \vec{v}\left(\frac{3\sqrt{2}}{2}\right) & = 10\sqrt{2}\hat{\imath} + (10\sqrt{2} - 9.8\left(\frac{3\sqrt{2}}{2}\right))\hat{\jmath} \\
                                                      & = 10\sqrt{2}\hat{\imath} - 14.7\sqrt{2}\hat{\jmath}
          \end{align*}
          Hence, the speed of the ball is
          \begin{align*}
              \lVert\vec{v}\left(\frac{3\sqrt{2}}{2}\right)\rVert & = \sqrt{(10\sqrt{2})^2 + (-14.7\sqrt{2})^2} \\
                                                                  & = 25.14\ \text{meters per second}
          \end{align*} \hfill$\blacksquare$
\end{enumerate}

\newpage