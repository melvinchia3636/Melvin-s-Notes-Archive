\subsection{Practice 2}

\noindent \hspace{1.2em}\textit{
    Determine which intervals the following functions is an increasing function or a decreasing function.
}
\begin{enumerate}
    \begin{multicols}{2}
        \item $f(x)=x^2+2 x-3$
        \sol{}
        \begin{flalign*}
            f'(x)  & = 2x+2 & \\
            2x + 2 & = 0    & \\
            x      & = -1   &
        \end{flalign*}
        In the interval $(-\infty,-1)$, $f'(x)<0$, so $f(x)$ is a decreasing function in the interval $(-\infty,-1]$.

        In the interval $(-1,\infty)$, $f'(x)>0$, so $f(x)$ is an increasing function
        in the interval $[-1,\infty)$. \vfill\null

                        \item $f(x)=x^3-x^2-x+1$
                        \sol{}
                        \begin{flalign*}
                            f'(x)       & = 3x^2-2x-1                  & \\
                            3x^2-2x-1   & = 0                          & \\
                            (3x+1)(x-1) & = 0                          & \\
                            x = 1       & \text{ or } x = -\frac{1}{3} &
                        \end{flalign*}
                        In the interval $\left(-\infty,-\dfrac{1}{3}\right)$, $f'(x)>0$, so $f(x)$ is an increasing function in the interval $\bigg(-\infty,-\dfrac{1}{3}\bigg]$.

        In the interval $\left(-\dfrac{1}{3},1\right)$, $f'(x)<0$, so $f(x)$ is a
        decreasing function in the interval $\bigg[-\dfrac{1}{3},1\bigg]$.

        In the interval $\left(1,\infty\right)$, $f'(x)>0$, so $f(x)$ is an increasing
        function in the interval $\bigg[1,\infty\bigg)$.
    \end{multicols}

\end{enumerate}