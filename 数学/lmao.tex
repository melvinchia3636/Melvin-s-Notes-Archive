\documentclass[9pt]{article}
\usepackage{graphicx} % Required for inserting images
\usepackage{xeCJK}
\usepackage{amssymb}
\usepackage[fleqn]{amsmath}
\usepackage{multicol}
\usepackage{titlesec}
\usepackage[T1]{fontenc}
\usepackage{enumitem}
\usepackage{tikz}
\usepackage{setspace}
\usepackage[a4paper, total={7.5in, 10in}]{geometry}
\usepackage{etoolbox}

\setlength{\columnseprule}{1pt}
\setlength{\columnsep}{24pt}

\title{第十二章 数列与级数}
\author{何胜威}
\date{14/2/2023}

\begin{document}

\doublespacing

\newcommand{\sol}{\noindent\textbf{解:}}
\newcommand{\example}{\noindent\textbf{例子:}}

\maketitle

\def\thesection{12.\arabic{section}}

\begin{multicols}{2}
    \begin{small}

        \section{数列}

        \subsection{有限数列与无限数列}

        \begin{center}
            \begin{tabular}{@{}p{13.3em}p{0.1em}p{15.2em}@{}}
                \textbf{数列(sequence)}              & \textbf{:} & 按照某种法则排列的数。 \\
                \multicolumn{3}{@{}l@{}}{\example{}2, 4, 6, 8, 10}                         \\
                \textbf{有限数列(finite sequence)}   & \textbf{:} & 指拥有有限项数的数列。 \\
                \multicolumn{3}{@{}l@{}}{\example{}1, 2, 3, 4, 5}                          \\
                \textbf{无限数列(infinite sequence)} & \textbf{:} & 指拥有无限项数的数列。 \\
                \multicolumn{3}{@{}l@{}}{\example{}1, 2, 3, $\ldots$, $n$, $\ldots$}
            \end{tabular}
        \end{center}

        \subsection{项}

        \begin{center}
            \begin{tabular}{@{}lp{0.1em}p{15.2em}@{}}
                \textbf{项数(term)}       & \textbf{:} & 数列中项的总数,通常以$n$表示($n \in \mathbb{Z}^+$)。 \\
                \textbf{首项(first term)} & \textbf{:} & 数列中的第一项。                                      \\
                \textbf{末项(last term)}  & \textbf{:} & 数列中的最后一项,在无限数列中没末项。                \\
            \end{tabular}
        \end{center}

        \noindent\example{}1, 2, 3, 4, 5

        \sol{} 此数列的首项为1,末项为5,项数为5。

        \subsection{通项}

        \begin{center}
            \begin{tabular}{@{}p{12.8em}p{0.1em}p{12em}@{}}
                \textbf{通项(general term)}        & \textbf{:} & 数列各项$a_n$与项数$n$之间的关系。               \\
                \textbf{通项公式(general formula)} & \textbf{:} & 以一个公式表示各项数列$a_n$与项数$n$之间的关系。 \\

            \end{tabular}
        \end{center}
        \hfill \break
        \noindent\example{}写出从1到8各整数的10倍加1的数列,并求此数列的首项,末项及其通项公式。

        \sol{}

        \noindent\textbf{所求数列有8项:}11, 21, 31, 41, 51, 61, 71, 81

        \noindent$\therefore$ 此数列的首项为11, 末项为81, 通项公式为$a_n = 10n+1$。

        \subsection*{练习12.1.1-12.1.3}

        \begin{enumerate}
            \item 写出从1到10各整数的倒数的数列,并求此数列的首项,末项及其通项公式。

                  \sol{}

                  \textbf{所求数列有10项:}$\frac{1}{1}$, $\frac{1}{2}$, $\frac{1}{3}$, $\frac{1}{4}$, $\frac{1}{5}$, $\frac{1}{6}$, $\frac{1}{7}$, $\frac{1}{8}$, $\frac{1}{9}$, $\frac{1}{10}$

                  $\therefore$ 此数列的首项为1,末项为$\frac{1}{10}$,通项公式为$a_n$ = $\frac{1}n$

            \item 根据下列各数列的通项公式,写出其首5项:

                  \begin{enumerate}
                      \item $a_n = 2n+1$

                            \sol{}

                            数列的首5项是:
                            \begin{flalign*}
                                a_{1} & = 2(1)+1 = 3  \\
                                a_{2} & = 2(2)+1 = 5  \\
                                a_{3} & = 2(3)+1 = 7  \\
                                a_{4} & = 2(4)+1 = 9  \\
                                a_{5} & = 2(5)+1 = 11
                            \end{flalign*}
                            $\therefore$ 此数列的首5项为3, 5, 7, 9, 11。

                      \item $a_n = n(n+1)$

                            \sol{}

                            数列的首5项是:
                            \begin{flalign*}
                                a_{1} & = 1(1+1) = 2  \\
                                a_{2} & = 2(2+1) = 6  \\
                                a_{3} & = 3(3+1) = 12 \\
                                a_{4} & = 4(4+1) = 20 \\
                                a_{5} & = 5(5+1) = 30
                            \end{flalign*}
                            $\therefore$ 此数列的首5项为2, 6, 12, 20, 30。

                  \end{enumerate}

            \item 写出下列各数列的一个通项公式:

                  \begin{enumerate}
                      \item 3, 7, 11, 15, $\ldots$

                            \sol{}

                            将各项的数拆解分析后可得:
                            \begin{flalign*}
                                a_{1} & = 3               \\
                                a_{2} & = 7 = 3+4(2-1)    \\
                                a_{3} & = 11 = 3+4(3-1)   \\
                                a_{4} & = 15 = 3+4(4-1)   \\
                                      & \ \ \vdots        \\
                                a_n   & = 3+4(n-1) = 4n-1
                            \end{flalign*}
                            \noindent$\therefore$ 此数列的通项公式为4n-1。

                      \item 2, 5, 10, 17, $\ldots$

                            \sol{}

                            将各项的数拆解分析后可得:
                            \begin{flalign*}
                                a_{1} & = 2 = 1^{2}+1  \\
                                a_{2} & = 5 = 2^{2}+1  \\
                                a_{3} & = 10 = 3^{2}+1 \\
                                a_{4} & = 17 = 4^{2}+1 \\
                                      & \ \ \vdots     \\
                                a_n   & = n^{2}+1
                            \end{flalign*}
                            \noindent$\therefore$ 此数列的通项公式为$n^{2}$+1。

                  \end{enumerate}

            \item 一个数列的通项公式是$a_n$ = $\frac{2^n}{n+1}$,写出它的首5项。

                  \sol{}

                  数列的首5项为:
                  \begin{flalign*}
                      a_{1} & = \frac{2^{1}}{1+1} = 1            \\
                      a_{2} & = \frac{2^{2}}{2+1} = \frac{4}{3}  \\
                      a_{3} & = \frac{2^{3}}{3+1} = 2            \\
                      a_{4} & = \frac{2^{4}}{4+1} = \frac{16}{5} \\
                      a_{5} & = \frac{2^{5}}{5+1} = \frac{16}{3}
                  \end{flalign*}
                  \noindent$\therefore$ 此数列的首5项为1, $\frac{4}{3}$, 2, $\frac{16}{5}$, $\frac{16}{3}$。

            \item 写出数列1, 8, 27, 64, $\ldots$的一个通项公式。

                  \noindent\textbf{解:}

                  \textbf{将各项的数拆解分析后可得:}
                  \begin{flalign*}
                      a_{1} & = 1 = 1^{3}  \\
                      a_{2} & = 8 = 2^{3}  \\
                      a_{3} & = 27 = 3^{3} \\
                      a_{4} & = 64 = 4^{3} \\
                            & \ \ \vdots   \\
                      a_n   & = n^{3}
                  \end{flalign*}
                  \noindent$\therefore$ 此数列的通项公式为$a_n$ = $n^{3}$

        \end{enumerate}

        \subsection{有限级数与无限级数}

        \begin{center}
            \begin{tabular}{@{}p{11.7em}p{0.1em}p{14em}@{}}
                \textbf{级数(series)}              & \textbf{:} & 将数列中的项依次用加号连接的函数。       \\
                \multicolumn{3}{@{}l@{}}{\example{}1 + 2 + 4 + 6 + 7 + 8}                                  \\
                \textbf{有限级数(finite series)}   & \textbf{:} & 指数列中由有限的项组成,再用加号连接。   \\
                \multicolumn{3}{@{}l@{}}{\example{}1 + 2 + 3 + 4 + 5}                                      \\
                \textbf{无限级数(infinite series)} & \textbf{:} & 指数列中由无限多的项组成,再用加号连接。 \\
                \multicolumn{3}{@{}l@{}}{\example{}1 + 2 + 3 + 4 + 5 + \ldots}                             \\
                \textbf{求和符号(summation)}       & \textbf{:} & 求和一般使用$\sum$符号表示,读作sigma。
            \end{tabular}
        \end{center}
        \hfill\break
        \example{} 求 $\sum\limits_{n = 5}^{10}\frac{n^{2}}{2}$ 的首项、末项和项数。

        \sol{}
        \begin{flalign*}
            a_{5} = \frac{5^{2}}{2}   & = \frac{25}{2}       \\
            a_{10} = \frac{10^{2}}{2} & = \frac{100}{2} = 50 \\
            n                         & = 6
        \end{flalign*}
        \noindent$\therefore$$\sum\limits_{n = 5}^{10}$$\frac{n^{2}}{2}$的首项为$\frac{25}{2}$,末项为50,项数为6。

        \example{} 以$\sum$符号表示下列各级数:

        \begin{enumerate}[label = (\alph*)]

            \item 2 + 2 + 2 + 2 + 2 + 2

                  \sol{}
                  \begin{flalign*}
                       & a_{1} = a_{2} = \ldots = a_{6} = 2
                  \end{flalign*}
                  $\therefore$ 2 + 2 + 2 + 2 + 2 + 2 = $\sum\limits_{n = 1}^{6}2$。

            \item $3 + 3^{2} + 3^{3} + \ldots + 3^{20}$

                  \sol{}
                  \begin{flalign*}
                       & a_{1} = 3^{1}, a_{2} = 3^{2}, a_{3} = 3^{3}, \ldots, a_{20} = 3^{20}
                  \end{flalign*}

                  $\therefore$ 3 + $3^{2}$ + $3^{3}$ + $\ldots$ + $3^{20}$ = $\sum\limits_{n = 1}^{20}3^n$。

            \item $1 - \frac{1}{2}$ + $\frac{1}{4}$ $- \frac{1}{8}$ + $\frac{1}{16}$

                  \sol{}
                  \begin{flalign*}
                       & a_{1} = \frac{1}{(-2)^{1-1}}, a_{2} = \frac{1}{(-2)^{2-1}}, a_{3} = \frac{1}{(-2)^{3-1}}, \\
                       & a_{4} = \frac{1}{(-2)^{4-1}}, a_{5} = \frac{1}{(-2)^{5-1}}
                  \end{flalign*}

                  $\therefore$ $1 - \frac{1}{2}$ + $\frac{1}{4}$ $- \frac{1}{8}$ + $\frac{1}{16}$ = $\sum\limits_{n = 1}^{5}$$\frac{1}{(-2)^n}$。

            \item 2 $\times$ 4 + 4 $\times$ 7 + 6 $\times$ 10 + 8 $\times$ 13 + 10 $\times$ 16

                  \sol{}
                  \begin{flalign*}
                       & a_{1} = [2(1)][3(1)], a_{2} = [2(2)][3(2)], a_{3} = [2(3)][3(3)], \\
                       & a_{4} = [2(4)][3(4)],  a_{5} = [2(5)][3(5)]
                  \end{flalign*}

                  $\therefore$ $2 \times 4 + 4 \times 7 + 6 \times 10 + 8 \times 13 + 10 \times 16 = \sum\limits_{n = 1}^{5}2n(3n+1)$。
        \end{enumerate}

        \section{等差数列与等差级数}

        \subsection{等差数列}

        若数列中每一项(首项除外)减去前一项所得的差均相等,则该数列为\textbf{等差数列(arithmetic sequence)}。
        \example{} 1, 3, 5, 7, 9。

        \subsection{公差}

        \noindent\textbf{公差(common difference)}为等差数列中相等的差,通常以${d}$表示。

        \textbf{以${d}$表示等差数列的公差}
        $$d = a_n - a_{n-1}$$

        \noindent\textbf{注:}公差可为负数。

        \noindent\example{}求等差数列1, 3, 5, 7, 9的公差。

        \sol{}${d}$ = 3 - 1 = 2

        $\therefore$ 此等差数列的公差为2。

        \subsection{等差通项公式}

        \noindent\textbf{等差数列通项公式推导}
        \begin{flalign*}
            \because\  & d = a_n - a_{n-1}         \\
                       & a_n = a_{n-1} + d         \\
                       & a_{2} = a_{1} + d         \\
                       & a_{3} = a_{1} + 2d \ldots
        \end{flalign*}
        \noindent 依照此规律可得,
        $$a_n = a + (n-1)d$$

        \noindent\textbf{等差数列项数推导}
        \begin{flalign*}
            \because\  & a_n = a + (n-1)d      \\
                       & a_n = a + dn - d      \\
                       & dn  = a_n - a + d     \\
                       & n = \frac{a_n-a+d}{d}
        \end{flalign*}
        \noindent 由此可得,
        $$n = \frac{a_n-a}{d} + 1$$

        \noindent\textbf{注:}此公式为额外推导而得,为方便计算求等差数列项数题型,解题时必须先行使用等差数列通项公式进行推导。

        \subsection*{练习12.2.1-12.2.3}

        \begin{enumerate}

            \item 等差数列9,6,3,\ldots 的第几项是 -21?

            \item 求从1到100所有5的倍数之和。

            \item 若等差数列13, 21, 29, \ldots 的首$n$项之和是910,求$n$的值。

            \item 若一等差数列的首$n$项之和$S_n = n^2 + 3n$,求
                  \begin{enumerate}
                      \item 首项;
                      \item 公差;
                      \item 第10项;
                      \item 从第5项到第10项之和。
                  \end{enumerate}

            \item -56, -50, -44, \ldots 是一个等差数列。问从第1项加到第几项,其和才开始是正值?

        \end{enumerate}

        \section{等比数列与等比级数}

        \subsection{等比数列}

        若数列中每一项(首项除外)除以前一项所得的商均相等,则此数列为一\textbf{等比数列(arithmetic mean):}。
        \example{}1, 2, 4, 8, 16。

        \subsection{公比}

        \textbf{公比(common ratio):}等差数列中相等的差商,通常以${r}$表示。

        \noindent \textbf{以${r}$表示等比数列的公比}
        $$r = \frac{a_n}{a_{n-1}}$$

        \noindent\textbf{注:}公比可为负数。\\

        \noindent\example{}求等比数列1, 2, 4, 8, 16的公比。

        \sol{}

        \noindent r = $\frac{2}{1}$ = 2

        \noindent $\therefore$ 此等差数列的公比为2。

        \subsection{等比通项公式}

        \noindent\textbf{等比数列通项公式推导}
        \begin{flalign*}
            \because  r & = \frac{a_{2}}{a_{1}}    \\
            a_n         & = a_{n-1} \times r^{n-1} \\
                        & = a_{1} \times r^{2-1}   \\
            a_{3}       & = a_{1} \times r^{3-1}   \\
                        & \ \ \vdots
        \end{flalign*}
        \noindent 依照此规律可得,
        $$a_n = ar^{n-1}$$

        \noindent\textbf{等比数列项数推导}
        \begin{flalign*}
            \because a_n       & = ar^{n-1}    \\
            \Rightarrow  n - 1 & = log_{ar}a_n
        \end{flalign*}
        \noindent 由此可得,
        $$n = log_{ar}a_n + 1$$

        \noindent\textbf{注:}此公式为额外推导而得,为方便计算求等比数列项数题型,解题时必须先行使用等比数列通项公式进行推导。

        \subsection*{练习12.3.1-12.3.3}

        \begin{enumerate}
            \item 等比数列2, 6, 18, \ldots 的第几项是486?

            \item 已知一等比数列的第3项是972,第8项是128,求此数列的首项,公比及第5项。

            \item 求3与3888之间的三个数,使这五个数形成一个等比数列。
        \end{enumerate}

        \subsection{等比中项}

        \textbf{等比中项(geometric mean)}为两数之间的另一个数,且满足等比数列的定义并形成等比数列,通常以${G}$表示,但不建议使用简写。

        \noindent\textbf{等比中项公式推导}

        \noindent 设一数列为${x}$, ${y}$,并求出等比中项。 \\
        \begin{flalign*}
            \because r  & = \frac{a_n}{a_{n-1}} \\
            \frac{G}{x} & = \frac{y}{G}         \\
            G^{2}       & = xy
        \end{flalign*}

        \noindent 由此可得,
        $$ G = \pm\sqrt{xy} $$

        \subsection*{练习12.3.4}

        \begin{enumerate}
            \item 求3与27的等比中项。

            \item 若三个数16,${x}$,9成一等比数列,求${x}$的值。

            \item 若7,${x}$,$\frac{36}{7}$,${y}$,$\frac{1296}{363}$成一等比数列,求${x}$及${y}$的值。

            \item 若一等比数列的首三项是${x+12}$,${x+4}$,${x-2}$成一等比数列,求
                  \begin{enumerate}
                      \item 公比;

                      \item ${x}$的值;

                      \item 第6项。
                  \end{enumerate}
        \end{enumerate}

        \subsection{等比求和公式}

        将等比数列中的每项相加所得的式子称为\textbf{等比级数(geometric progression / geometric series)}
        \hfill\break

        \noindent\textbf{等比求和公式推导}

        \noindent 设$S_n$表示某等比数列之和可得,
        \begin{flalign*}
            S_n                                        & = a_{1} + a_{2} + \ldots + a_{n-1} + a_n              \\
            S_n                                        & = a + ar + \ldots + ar^{n-1} + ar^n \ldots      & (1) \\
            \text{$(1)\times r$ 得 }\             rS_n & = ar + ar^{2} + \ldots + ar^{n-1} + ar^n \ldots & (2) \\
            \text{$(1)-(2)$得 }\ S_n - rS_n            & = a - ar^n                                            \\
            S_n(1 - r)                                 & = a(1 - r^n)
        \end{flalign*}
        \noindent 由此可得,
        $$S_n = \frac{a(1 - r^n)}{1 - r} \qquad (r \neq 1)$$
        \noindent 当r = 1时,$S_n = a + ar + \ldots + ar^{n-1} + ar^n$  \\
        $S_n = a + a + \ldots + a + a$ (共有n个a)\\
        \noindent 由此可得,
        $$S_n = na \qquad (r = 1)$$

        \subsection*{练习12.3.5}

        \begin{enumerate}
            \item 求等比级数4 + 8 + 16 + $\ldots$的首6项之和。

            \item 若等比数列108,72,48,$\ldots$的首$n$项之和是281$\frac{1}{3}$,求$n$的值。

            \item 若一等比数列的首项是7,公比是3,级数之和是847,求
                  \begin{enumerate}
                      \item 项数;

                      \item 末项。
                  \end{enumerate}
        \end{enumerate}

        \subsection{无穷等比级数的和}

        \textbf{无穷等比级数(infinite geometric series)}指等比数列中由无限多的项组成,再用加号连接。
        \example{}2, 4, 8, 16, $\ldots$

        \noindent\textbf{无穷等比级数之和公式推导}

        \noindent 此公式推导将会展示三种推导方法。

        \noindent\textbf{方法1:}

        \noindent 以概念理解:

        \noindent 已知,
        $$S_n = \frac{a(1 - r^n)}{1 - r} \qquad (r \neq 1)$$
        \noindent 由于$r \neq 1$且n趋近于无穷,因此$r^n$最终的值将会等于0。

        \noindent 由此可得,
        $$S_{\infty} = \frac{a}{1-r} \qquad (\vert {r}\vert  < 1)$$

        \noindent\textbf{方法2:}

        \noindent 以图示理解:

        $\therefore 1 + \frac{1}{2} + \frac{1}{4} + \ldots = 2$

        \noindent\textbf{方法3:}

        \noindent 正规公式推导:

        \noindent 已知,
        $$S_n = \frac{a(1 - r^n)}{1 - r}, (r \neq 1)$$

        \noindent 当n趋近于无穷(正无穷)时,
        \begin{flalign*}
            \lim_{n\to+\infty} S_n & = \lim_{n\to+\infty} \frac{a(1 - r^n)}{1 - r} \qquad (\vert {r}\vert < 1) \\
            S_{\infty}             & = \frac{a(1 - 0)}{1 - r}                                                  \\
            S_{\infty}             & = \frac{a}{1 - r} (1 - 0)                                                 \\
        \end{flalign*}
        \noindent 由此可得,
        $$S_{\infty} = \frac{a}{1-r},(\vert {r}\vert  < 1)$$

        \subsection*{练习12.3.6}

        \begin{enumerate}
            \item 求无穷等比级数36 + 12 + 4 + $\ldots$之和。

            \item 使用无穷等比级数的求和公式将下列循环小数化为分数:
                  \begin{enumerate}
                      \item 0.\.{2}\.{3}

                      \item  0.2\.{1}\.{3}
                  \end{enumerate}

        \end{enumerate}

        \section{简易特殊级数之和}
    \end{small}
\end{multicols}
\end{document}