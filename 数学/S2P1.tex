\documentclass[a4paper]{report}
\usepackage[fleqn]{amsmath}
\usepackage{amssymb}
\usepackage[fleqn]{cases}


\title{Senior 2 Math Part I}
\author{Melvin Chia}


\begin{document}
	\maketitle


	\chapter{Sequence and Series}


	\section{Sequence and Series}


	\subsection{Practice 1}


	\begin{enumerate}
		\item Find the first 5 terms of the sequence $a_{n} = \frac{2^{n}}{n+1}$.

			\textbf{Ans.} $a_{1} = \frac{2}{2}= 1, a_{2} = \frac{4}{3}, a_{3} = \frac{8}{4}
			, a_{4} = \frac{16}{5}, a_{5} = \frac{32}{6}$

		\item Write the general term of the sequence 1, 8, 27, 64, \ldots

			\textbf{Ans.} $a_{n} = n^{3}$
	\end{enumerate}

	\subsection{Practice 2}


	\begin{enumerate}
		\item Express the series $\sum_{n=1}^{10}{n^2+1}$ in the form of numbers.

			\begin{flalign*}
				\textbf{Ans.} & \sum_{n=1}^{10}{n^2+1}                                       &  \\
				              & = (1^{2}+1) + (2^{2}+1) + (3^{2}+1) + (4^{2}+1) + (5^{2}+1)  &  \\
				              & + (6^{2}+1) + (7^{2}+1) + (8^{2}+1) + (9^{2}+1) + (10^{2}+1) &  \\
				              & = 2 + 5 + 10 + 17 + 26 + 37 + 50 + 65 + 82 + 101
			\end{flalign*}

		\item Write the first term, last term and the number of terms of the series
			$\sum_{n=1}^{10}{(3^n-2^n)}$.

			\begin{flalign*}
				\textbf{Ans.} & First\ term = (3^{1}-2^{1}) = 1      &  \\
				              & Last\ term = (3^{10}-2^{10}) = 59049 &  \\
				              & Number\ of\ terms = 10
			\end{flalign*}

		\item Express the series $2\times5 + 3\times7 + 4\times9 + \ldots + 15\times31$
			in the form of $\sum$.

			\begin{flalign*}
				\noindent \textbf{Ans.}                                       \\
				a_{1}                                                        & = 2\times5 = 10        \\
				a_{2}                                                        & = 3\times7 = 21        \\
				a_{3}                                                        & = 4\times9 = 36        \\
				a_{4}                                                        & = 5\times11 = 55       \\
				                                                             & \vdots                 \\
				a_{15}                                                       & = 15\times31 = 465     \\
				\therefore 2&\times5 + 3\times7 + 4\times9 + \ldots + 15\times31 = \sum_{n=1}^{15}a_{n}
			\end{flalign*}
	\end{enumerate}

	\subsection{Exercise 12.1}


	\begin{enumerate}

	\item Find the general term of the following sequences.

	\begin{enumerate}
		\item 5, 8, 11, 14, \ldots

			\textbf{Ans.} $a_{n} = 3n+2$

		\item 2, 4, 8, 16, \ldots

			\textbf{Ans.} $a_{n} = 2^{n}$

		\item $\frac{2}{1}, \frac{3}{2}, \frac{4}{3}, \frac{5}{4}, \ldots$

			\textbf{Ans.} $a_{n} = \frac{n+1}{n}$

		\item $\frac{2}{5}, \frac{4}{7}, \frac{6}{9}, \frac{8}{11}, \ldots$

			\textbf{Ans.} $a_{n} = \frac{2n}{2n+1}$
	\end{enumerate}

	\item Find the first 5 terms of the following sequences.

	\begin{enumerate}
		\item $a_{n} = 2n+3$

			\textbf{Ans.}
			$a_{1} = 2\times1+3 = 5, a_{2} = 2\times2+3 = 7, a_{3} = 2\times3+3 = 9, a_{4}
			= 2\times4+3 = 11, a_{5} = 2\times5+3 = 13$

		\item $a_{n} = n(n-2)$

			\textbf{Ans.}
			$a_{1} = 1\times(-1) = -1, a_{2} = 2\times0 = 0, a_{3} = 3\times1 = 3, a_{4}
			= 4\times2 = 8, a_{5} = 5\times3 = 15$

		\item $a_{n} = \frac{n}{2n+1}$

			\textbf{Ans.}
			$a_{1} = \frac{1}{2\times1+1}= \frac{1}{3}, a_{2} = \frac{2}{2\times2+1}= \frac{2}{5}
			, a_{3} = \frac{3}{2\times3+1}= \frac{3}{7}, a_{4} = \frac{4}{2\times4+1}=
			\frac{4}{9}, a_{5} = \frac{5}{2\times5+1}= \frac{5}{11}$

		\item $a_{n} = (-3)^{n}$

			\textbf{Ans.}
			$a_{1} = (-3)^{1} = -3, a_{2} = (-3)^{2} = 9, a_{3} = (-3)^{3} = -27, a_{4}
			= (-3)^{4} = 81, a_{5} = (-3)^{5} = -243$
	\end{enumerate}

	\item Express the following series in the form of numbers.

	\begin{enumerate}
		\item $\sum_{n=1}^{5}{n(n+3)}$

			\begin{flalign*}
				\textbf{Ans.} & \sum_{n=1}^{5}{n(n+3)}                                           &  \\
				              & = (1\times4) + (2\times5) + (3\times6) + (4\times7) + (5\times8) &  \\
				              & = 4 + 10 + 18 + 28 + 40                                          &  \\
			\end{flalign*}

		\item $\sum_{n=2}^{6}{\frac{1}{3^{n}}}$

			\begin{flalign*}
				\textbf{Ans.} & \sum_{n=2}^{6}{\frac{1}{3^{n}}}                                                       &  \\
				              & = \frac{1}{3^{2}}+ \frac{1}{3^{3}}+ \frac{1}{3^{4}}+ \frac{1}{3^{5}}+ \frac{1}{3^{6}} &  \\
				              & = \frac{1}{9}+ \frac{1}{27}+ \frac{1}{81}+ \frac{1}{243}+ \frac{1}{729}
			\end{flalign*}

		\item $\sum_{n=1}^{6}{\frac{1}{n(2n+1)}}$

			\begin{flalign*}
				\textbf{Ans.} & \sum_{n=1}^{6}{\frac{1}{n(2n+1)}}                                                  &  \\
				              & = \frac{1}{1(2\times1+1)}+ \frac{1}{2(2\times2+1)}+ \frac{1}{3(2\times3+1)}        &  \\
				              & + \frac{1}{4(2\times4+1)}+ \frac{1}{5(2\times5+1)}+ \frac{1}{6(2\times6+1)}        &  \\
				              & = \frac{1}{3}+ \frac{1}{10}+ \frac{1}{21}+ \frac{1}{36}+ \frac{1}{55}+ \frac{1}{78}
			\end{flalign*}

		\item $\sum_{n=2}^{5}{\frac{1}{n^{2}+2}}$

			\begin{flalign*}
				\textbf{Ans.} & \sum_{n=2}^{5}{\frac{1}{n^{2}+2}}                              &  \\
				              & = \frac{1}{4+2}+ \frac{1}{9+2}+ \frac{1}{16+2}+ \frac{1}{25+2} &  \\
				              & = \frac{1}{6}+ \frac{1}{11}+ \frac{1}{18}+ \frac{1}{27}
			\end{flalign*}

			\end {enumerate}

		\item Find the first term, last term and the number of terms of the
			following series.

			\begin{enumerate}
				\item $\sum_{n=3}^{10}{2^2}$

					\textbf{Ans.} $a_{3} = 2^{2} = 4, a_{10}= 2^{2} = 4, n = 10-3+1 = 8$

				\item $\sum_{n=1}^{8}{\frac{n+2}{n}}$

					\textbf{Ans.}
					$a_{1} = \frac{1+2}{1}= \frac{3}{1}= 3, a_{8}= \frac{8+2}{8}= \frac{10}{8}
					= \frac{5}{4}, n = 8-1+1 = 8$

				\item $\sum_{n=1}^{10}{3n^2-n}$

					\textbf{Ans.}
					$a_{1} = 3\times1^{2}-1 = 2, a_{10}= 3\times10^{2}-10 = 290, n = 10-1+1
					= 10$

				\item $\sum_{n=9}^{14}{n^2(n-7)}$

					\textbf{Ans.}
					$a_{9} = 9^{2}(9-7) = 9^{2}\times2 = 162, a_{14}= 14^{2}(14-7) = 14^{2}
					\times7 = 2744, n = 14-9+1 = 6$
			\end{enumerate}

		\item Express the following series in the form of $\sum$.

			\begin{enumerate}
				\item $1+\frac{1}{2}+\frac{1}{3}+\ldots+\frac{1}{30}$

					\begin{flalign*}
						\textbf{Ans.}a_{1} & = 1                                                                     \\
						a_{2}              & = \frac{1}{2}                                                           \\
						a_{3}              & = \frac{1}{3}                                                           \\
						\vdots              \\
						a_{30}             & = \frac{1}{30}                                                          \\
						\therefore 1       & +\frac{1}{2}+\frac{1}{3}+\ldots+\frac{1}{30}= \sum_{n=1}^{30}{\frac{1}{n}}
					\end{flalign*}

				\item $1^{3} + 2^{3} + 3^{3} + \ldots + 50^{3}$

					\begin{flalign*}
						\textbf{Ans.}a_{1} & = 1^{3}                                               \\
						a_{2}              & = 2^{3}                                               \\
						a_{3}              & = 3^{3}                                               \\
						\vdots              \\
						a_{50}             & = 50^{3}                                              \\
						\therefore 1^{3}   & + 2^{3} + 3^{3} + \ldots + 50^{3} = \sum_{n=1}^{50}{n^3}
					\end{flalign*}

				\item $1 - \frac{1}{2}+ \frac{1}{4}- \frac{1}{8}+ \frac{1}{16}$

					\begin{flalign*}
						\textbf{Ans.}a_{1} & = (-\frac{1}{2})^{1-1}                                                                      \\
						a_{2}              & = (-\frac{1}{2})^{2-1}                                                                      \\
						a_{3}              & = (-\frac{1}{2})^{3-1}                                                                      \\
						a_{4}              & = (-\frac{1}{2})^{4-1}                                                                      \\
						a_{5}              & = (-\frac{1}{2})^{5-1}                                                                      \\
						\therefore 1       & - \frac{1}{2}+ \frac{1}{4}- \frac{1}{8}+ \frac{1}{16}= \sum_{n=1}^{5}{(-\frac{1}{2})^{n-1}}
					\end{flalign*}

				\item $2\times4 + 4\times7 + 6\times10 + 8\times13 + 10\times16$

					\begin{flalign*}
						\textbf{Ans.}a_{1} & = 2\times1\times(3\times1+1)                                                       \\
						a_{2}              & = 2\times2\times(3\times2+1)                                                       \\
						a_{3}              & = 2\times3\times(3\times3+1)                                                       \\
						a_{4}              & = 2\times4\times(3\times4+1)                                                       \\
						a_{5}              & = 2\times5\times(3\times5+1)                                                       \\
						\therefore 2       & \times4 + 4\times7 + 6\times10 + 8\times13 + 10\times16 = \sum_{n=1}^{5}{2n(3n+1)}
					\end{flalign*}
			\end{enumerate}
	\end{enumerate}

	\section{Arithmetic Progression}


	General term of an Arithmetic Progression (AP) is given by

	\[
		a_{n} = a_{1} + (n-1)d
	\]

	where $a_{1}$ is the first term, $d$ is the common difference and $n$ is the number
	of terms.

	\subsection{Practice 3}


	\begin{enumerate}
		\item Find the number of terms of the AP $-4 - 2\frac{3}{4}- 1\frac{1}{2}- \frac{1}{4}
			+ \ldots + 16$.

			\begin{flalign*}
				\noindent \textbf{Ans.} \\
				a_{1}                  & = -4                    \\
				a_{n}                  & = 16                    \\
				d                      & = -2\frac{3}{4}- (-4)   \\
				                       & = -2\frac{3}{4}+ 4      \\
				                       & = \frac{5}{4}           \\
				16                     & = -4 + (n-1)\frac{5}{4} \\
				20                     & = \frac{5}{4}(n-1)      \\
				80                     & = 5(n-1)                \\
				n-1                    & = 16                    \\
				n                      & = 17
			\end{flalign*}

		\item Given that $a_{2} = 4$ and $a_{6} = -8$, find the 10th term of the AP.

			\begin{flalign*}
				\noindent \textbf{Ans.} \\
				a_{2}                  & = 4  \\
				a + (2-1)d             & = 4  \\
				a_{6}                  & = -8 \\
				a + (6-1)d             & = -8 \\
			\end{flalign*}
			\begin{numcases}
				{} a + d &= 4\\ a + 5d &= -8
			\end{numcases}
			\begin{flalign*}
				(2) - (1): 4d     & =-12             \\
				d                 & = -3             \\
				a + (-3)          & = 4              \\
				a                 & = 7              \\
				\therefore a_{10} & = 7 + (10-1)(-3) \\
				                  & = 7 - 27         \\
				                  & = -20
			\end{flalign*}

		\item How many multiples of 7 are there between 50 and 500?

			\begin{flalign*}
				\noindent \textbf{Ans.} \\
				a_{1}                  & = 56  \\
				a_{n}                  & = 497 \\
				d                      & = 7   \\
				497 = 56 + (n-1)7       \\
				441 = 7(n-1)            \\
				n-1                    & = 63  \\
				n                      & = 64
			\end{flalign*}

		\item Find 5 numbers between 30 and 54 such that these numbers form an AP.

			\begin{flalign*}
				\noindent \textbf{Ans.} \\
				a_{1}                  & = 30                                                 \\
				a_{7}                  & = 54                                                 \\
				54                     & = 30 + (7-1)d                                        \\
				24                     & = 6d                                                 \\
				d                      & = 4                                                  \\
				\\
				\therefore\            & These\ 5\ numbers\ are\ 34,\ 38,\ 42,\ 46,\ and\ 50.
			\end{flalign*}
	\end{enumerate}

	\subsection*{Arithmetic mean}


	If A is in between x and y, and x, A, y are in AP, then

	\[
		A = \frac{x+y}{2}
	\]

	\subsection{Practice 4}


	\begin{enumerate}
		\item If 9, x, 17 are in AP, find x.

			\begin{flalign*}
				\noindent \textbf{Ans.} \\
				x                      & = \frac{9+17}{2} \\
				                       & = \frac{26}{2}   \\
				                       & = 13
			\end{flalign*}

		\item Find the arithmetic mean of 26 and -11.

			\begin{flalign*}
				\noindent \textbf{Ans.} \\
				A                      & = \frac{26-11}{2} \\
				                       & = \frac{15}{2}    \\
			\end{flalign*}

		\item Find x and y when 3, x, 12, y, 21 are in AP.

			\begin{flalign*}
				\noindent \textbf{Ans.} \\
				x                      & = \frac{3+12}{2}  \\
				                       & = \frac{15}{2}    \\
				y                      & = \frac{12+21}{2} \\
				                       & = \frac{33}{2}    \\
			\end{flalign*}
	\end{enumerate}

	\subsection*{Summation of Arithmetic Progression}


	The summation formula for AP is given by

	\[
		S_{n} = \frac{n}{2}(2a + (n-1)d)
	\]

	or

	\[
		S_{n} = \frac{n}{2}(a_{1} + a_{n})
	\]

	\subsection{Practice 5}


	\begin{enumerate}
		\item Find the sum of the first 16 terms of the AP $22 + 18 + 14 + 10 + \ldots$

			\begin{flalign*}
				\noindent \textbf{Ans.} \\
				a_{1}                  & = 22                                   \\
				n                      & = 16                                   \\
				d                      & = -4                                   \\
				S_{n}                  & = \frac{16}{2}(2\times22 + (-4)(16-1)) \\
				                       & = \frac{16}{2}(44 + (-4)(15))          \\
				                       & = \frac{16}{2}(44 - 60)                \\
				                       & = \frac{16}{2}(-16)                    \\
				                       & = -128
			\end{flalign*}

		\item If the sum of AP $23 + 19 + 15 + \ldots$ is 72, find the number of terms.

			\begin{flalign*}
				\noindent \textbf{Ans.} \\
				a_{1}                  & = 23                                 \\
				S_{n}                  & = 72                                 \\
				d                      & = -4                                 \\
				72                     & = \frac{n}{2}(2\times23 + (-4)(n-1)) \\
				72                     & = \frac{n}{2}(46 + (-4)(n-1))        \\
				144                    & = n(46 + (-4)(n-1))                  \\
				144                    & = n(46 - 4n + 4)                     \\
				144                    & = n(50 - 4n)                         \\
				144                    & = 50n - 4n^{2}                       \\
				72                     & = 25n - 2n^{2}                       \\
				2n^{2} - 25n + 72      & = 0                                  \\
				(n - 8)(2n - 9)        & = 0                                  \\
				n                      & = 8                                  \\
			\end{flalign*}

		\item Given that $S_{n} = 2n + 3n^{2}$, find the first term and the common
			difference of the AP.

			\begin{flalign*}
				\noindent \textbf{Ans.} \\
				S_{n}                  & = 2n + 3n^{2}              \\
				2n + 3n^{2}            & = \frac{n}{2}(2a + (n-1)d) \\
				4n + 6n^{2}            & = n(2a + (n-1)d)           \\
				4n + 6n^{2}            & = 2na + (n-1)nd            \\
				4n + 6n^{2}            & = 2na + n^{2}d - nd        \\
				4n + 6n^{2}            & = (2a-d)n + dn^{2}         \\
				\\
				Compar                 & ing\ both\ sides,          \\
				2a-d                   & = 4                        \\
				a                      & = 6                        \\
				d                      & = 2                        \\
			\end{flalign*}
	\end{enumerate}
\end{document}