\documentclass{report}
\usepackage{setspace}
\usepackage[a4paper, total={7in, 9in}]{geometry}
\usepackage[fleqn]{amsmath}
\usepackage{empheq}
\usepackage{amssymb}
\usepackage{gensymb}
\usepackage[fleqn]{cases}
\usepackage{multicol}
\usepackage{color}
\usepackage{stix}
\usepackage{chngcntr}

\counterwithout{equation}{chapter}
\setlength{\columnseprule}{1pt}
\setlength{\columnsep}{24pt}
\setcounter{chapter}{11}
\hfuzz=100pt

\begin{document}
\newcommand{\sol}[1]{

  \noindent \textbf{sol{}.}
}
\begin{titlepage}
  \raggedleft{}
  \rule{1pt}{\textheight}
  \hspace{0.02\textwidth}
  \parbox[b]{0.75\textwidth}{

  {\Huge\bfseries Solution Book of \\[0.5\baselineskip] Mathematic}\\[2\baselineskip]
  {\large\textit{Ssnior 2 Part I}}\\[4\baselineskip]
  {\Large\textsc{MELVIN CHIA}}

  \vspace{0.5\textheight}

  {\noindent Written on 9 October 2022}\\[\baselineskip]
  }

\end{titlepage}

\doublespacing{}
\tableofcontents
\singlespacing{}
\newpage

\begin{multicols}{2}

  \chapter{Sequence and Series}

  \section{Sequence and Series}

  \subsection{Practice 1}

  \begin{enumerate}
    \item Find the first 5 terms of the sequence $a_{n} = \frac{2^{n}}{n+1}$.

          \textbf{sol{}.} $a_{1} = \frac{2}{2}= 1, a_{2} = \frac{4}{3}, a_{3} = \frac{8}{4}
            , a_{4} = \frac{16}{5}, a_{5} = \frac{32}{6}$

    \item Write the general term of the sequence $1, 8, 27, 64, \cdots$

          \textbf{sol{}.} $a_{n} = n^{3}$
  \end{enumerate}

  \subsection{Practice 2}

  \begin{enumerate}
    \item Express the series $\sum_{n=1}^{10}{n^2+1}$ in the form of numbers.

          \begin{flalign*}
            \textbf{sol{}.} & \sum_{n=1}^{10}{n^2+1}                          & \\
                            & = (1^{2}+1) + (2^{2}+1) + (3^{2}+1) + (4^{2}+1)   \\ & + (5^{2}+1) + (6^{2}+1) + (7^{2}+1) \\ & + (8^{2}+1) + (9^{2}+1) + (10^{2}+1) &  \\
                            & = 2 + 5 + 10 + 17 + 26 + 37 + 50 + 65             \\ & + 82 + 101
          \end{flalign*}

    \item Write the first term, last term and the number of terms of the series
          $\sum_{n=1}^{10}{(3^n-2^n)}$.

          \begin{flalign*}
            \textbf{sol{}.} & First\ term = (3^{1}-2^{1}) = 1      & \\
                            & Last\ term = (3^{10}-2^{10}) = 59049 & \\
                            & Number\ of\ terms = 10
          \end{flalign*}

    \item Express the series $2\times5 + 3\times7 + 4\times9 + \cdots + 15\times31$ in
          the form of $\sum$.

          \begin{flalign*}
            \noindent \textbf{sol{}.}                                          \\
            a_{1}        & = 2\times5 = 10                                     \\
            a_{2}        & = 3\times7 = 21                                     \\
            a_{3}        & = 4\times9 = 36                                     \\
            a_{4}        & = 5\times11 = 55                                    \\
                         & \vdots                                              \\
            a_{15}       & = 15\times31 = 465                                  \\
            \therefore 2 & \times5 + 3\times7 + 4\times9 + \cdots + 15\times31 \\ & = \sum_{n=1}^{15}a_{n}
          \end{flalign*}
  \end{enumerate}

  \subsection{Exercise 12.1}
  \begin{enumerate}

    \item Find the general term of the following sequences.

          \begin{enumerate}
            \item 5, 8, 11, 14, \ldots

                  \textbf{sol{}.} $a_{n} = 3n+2$

            \item 2, 4, 8, 16, \ldots

                  \textbf{sol{}.} $a_{n} = 2^{n}$

            \item $\frac{2}{1}, \frac{3}{2}, \frac{4}{3}, \frac{5}{4}, \cdots$

                  \textbf{sol{}.} $a_{n} = \frac{n+1}{n}$

            \item $\frac{2}{5}, \frac{4}{7}, \frac{6}{9}, \frac{8}{11}, \cdots$

                  \textbf{sol{}.} $a_{n} = \frac{2n}{2n+1}$
          \end{enumerate}

    \item Find the first 5 terms of the following sequences.

          \begin{enumerate}
            \item $a_{n} = 2n+3$

                  \textbf{sol{}.}
                  $a_{1} = 2\times1+3 = 5, a_{2} = 2\times2+3 = 7, a_{3} = 2\times3+3 = 9, a_{4}
                    = 2\times4+3 = 11, a_{5} = 2\times5+3 = 13$

            \item $a_{n} = n(n-2)$

                  \textbf{sol{}.}
                  $a_{1} = 1\times(-1) = -1, a_{2} = 2\times0 = 0, a_{3} = 3\times1 = 3, a_{4}
                    = 4\times2 = 8, a_{5} = 5\times3 = 15$

            \item $a_{n} = \frac{n}{2n+1}$

                  \textbf{sol{}.}
                  $a_{1} = \frac{1}{2\times1+1}= \frac{1}{3}, a_{2} = \frac{2}{2\times2+1}= \frac{2}{5}
                    , a_{3} = \frac{3}{2\times3+1}= \frac{3}{7}, a_{4} = \frac{4}{2\times4+1}=
                    \frac{4}{9}, a_{5} = \frac{5}{2\times5+1}= \frac{5}{11}$

            \item $a_{n} = {{(-3)}}^{n}$

                  \textbf{sol{}.}
                  $a_{1} = {(-3)}^{1} = -3, a_{2} = {(-3)}^{2} = 9, a_{3} = {(-3)}^{3} = -27, a_{4}
                      = {(-3)}^{4} = 81, a_{5} = {(-3)}^{5} = -243$
          \end{enumerate}

    \item Express the following series in the form of numbers.

          \begin{enumerate}
            \item $\sum_{n=1}^{5}{n(n+3)}$

                  \begin{flalign*}
                    \textbf{sol{}.} & \sum_{n=1}^{5}{n(n+3)}                              & \\
                                    & = (1\times4) + (2\times5) + (3\times6) + (4\times7)   \\ & + (5\times8) &  \\
                                    & = 4 + 10 + 18 + 28 + 40                             & \\
                  \end{flalign*}

            \item $\sum_{n=2}^{6}{\frac{1}{3^{n}}}$

                  \begin{flalign*}
                    \textbf{sol{}.} & \sum_{n=2}^{6}{\frac{1}{3^{n}}}                                                       & \\
                                    & = \frac{1}{3^{2}}+ \frac{1}{3^{3}}+ \frac{1}{3^{4}}+ \frac{1}{3^{5}}+ \frac{1}{3^{6}} & \\
                                    & = \frac{1}{9}+ \frac{1}{27}+ \frac{1}{81}+ \frac{1}{243}+ \frac{1}{729}
                  \end{flalign*}

            \item $\sum_{n=1}^{6}{\frac{1}{n(2n+1)}}$

                  \begin{flalign*}
                    \textbf{sol{}.} & \sum_{n=1}^{6}{\frac{1}{n(2n+1)}}                                                   & \\
                                    & = \frac{1}{1(2\times1+1)}+ \frac{1}{2(2\times2+1)}                                    \\                                                                                                                                                                                                                                                                                                                                                                                       & + \frac{1}{3(2\times3+1)} +
                    \frac{1}{4(2\times4+1)}                                                                                 \\ &+ \frac{1}{5(2\times5+1)}+ \frac{1}{6(2\times6+1)}        &  \\
                                    & = \frac{1}{3}+ \frac{1}{10}+ \frac{1}{21}+ \frac{1}{36}+ \frac{1}{55}+ \frac{1}{78}
                  \end{flalign*}

            \item $\sum_{n=2}^{5}{\frac{1}{n^{2}+2}}$

                  \begin{flalign*}
                    \textbf{sol{}.} & \sum_{n=2}^{5}{\frac{1}{n^{2}+2}}                              & \\
                                    & = \frac{1}{4+2}+ \frac{1}{9+2}+ \frac{1}{16+2}+ \frac{1}{25+2} & \\
                                    & = \frac{1}{6}+ \frac{1}{11}+ \frac{1}{18}+ \frac{1}{27}
                  \end{flalign*}

          \end{enumerate}

    \item Find the first term, last term and the number of terms of the following series.

          \begin{enumerate}
            \item $\sum_{n=3}^{10}{2^2}$

                  \textbf{sol{}.} $a_{3} = 2^{2} = 4, a_{10}= 2^{2} = 4, n = 10-3+1 = 8$

            \item $\sum_{n=1}^{8}{\frac{n+2}{n}}$

                  \textbf{sol{}.}
                  $a_{1} = \frac{1+2}{1}= \frac{3}{1}= 3, a_{8}= \frac{8+2}{8}= \frac{10}{8}
                    = \frac{5}{4}, n = 8-1+1 = 8$

            \item $\sum_{n=1}^{10}{3n^2-n}$

                  \textbf{sol{}.}
                  $a_{1} = 3\times1^{2}-1 = 2, a_{10}= 3\times10^{2}-10 = 290, n = 10-1+1
                    = 10$

            \item $\sum_{n=9}^{14}{n^2(n-7)}$

                  \textbf{sol{}.}
                  $a_{9} = 9^{2}(9-7) = 9^{2}\times2 = 162, a_{14}= 14^{2}(14-7) = 14^{2}
                    \times7 = 2744, n = 14-9+1 = 6$
          \end{enumerate}

    \item Express the following series in the form of $\sum$.

          \begin{enumerate}
            \item $1+\frac{1}{2}+\frac{1}{3}+\cdots+\frac{1}{30}$
                  \sol{}
                  \begin{flalign*}
                    a_{1}        & = 1                                                                        \\
                    a_{2}        & = \frac{1}{2}                                                              \\
                    a_{3}        & = \frac{1}{3}                                                              \\
                    \vdots                                                                                    \\
                    a_{30}       & = \frac{1}{30}                                                             \\
                    \therefore 1 & +\frac{1}{2}+\frac{1}{3}+\cdots+\frac{1}{30}= \sum_{n=1}^{30}{\frac{1}{n}}
                  \end{flalign*}

            \item $1^{3} + 2^{3} + 3^{3} + \cdots + 50^{3}$
                  \sol{}
                  \begin{flalign*}
                    a_{1}            & = 1^{3}                                                  \\
                    a_{2}            & = 2^{3}                                                  \\
                    a_{3}            & = 3^{3}                                                  \\
                    \vdots                                                                      \\
                    a_{50}           & = 50^{3}                                                 \\
                    \therefore 1^{3} & + 2^{3} + 3^{3} + \cdots + 50^{3} = \sum_{n=1}^{50}{n^3}
                  \end{flalign*}

            \item $1  - \frac{1}{2}+ \frac{1}{4}- \frac{1}{8}+ \frac{1}{16}$
                  \sol{}
                  \begin{flalign*}
                    a_{1}        & = {(-\frac{1}{2})}^{1-1}                              \\
                    a_{2}        & = {(-\frac{1}{2})}^{2-1}                              \\
                    a_{3}        & = {(-\frac{1}{2})}^{3-1}                              \\
                    a_{4}        & = {(-\frac{1}{2})}^{4-1}                              \\
                    a_{5}        & = {(-\frac{1}{2})}^{5-1}                              \\
                    \therefore 1 & - \frac{1}{2}+ \frac{1}{4}- \frac{1}{8}+ \frac{1}{16} \\ & = \sum_{n=1}^{5}{{(-\frac{1}{2})}^{n-1}}
                  \end{flalign*}

            \item $2\times4 + 4\times7 + 6\times10 + 8\times13 + 10\times16$
                  \sol{}
                  \begin{flalign*}
                    a_{1}        & = 2\times1\times(3\times1+1)               \\
                    a_{2}        & = 2\times2\times(3\times2+1)               \\
                    a_{3}        & = 2\times3\times(3\times3+1)               \\
                    a_{4}        & = 2\times4\times(3\times4+1)               \\
                    a_{5}        & = 2\times5\times(3\times5+1)               \\
                    \therefore 2 & \times4 + 4\times7 + 6\times10 + 8\times13 \\ & + 10\times16 = \sum_{n=1}^{5}{2n(3n+1)}
                  \end{flalign*}
          \end{enumerate}
  \end{enumerate}

  \section{Arithmetic Progression}

  General term of an Arithmetic Progression (AP) is given by

  \[
    a_{n} = a_{1} + (n-1)d
  \]

  where $a_{1}$ is the first term, $d$ is the common difference and $n$ is the
  number of terms.

  \subsection{Practice 3}

  \begin{enumerate}
    \item Find the number of terms of the AP $-4 - 2\frac{3}{4}- 1\frac{1}{2}-
            \frac{1}{4} + \cdots + 16$.

          \begin{flalign*}
            a_{1} & = -4                    \\
            a_{n} & = 16                    \\
            d     & = -2\frac{3}{4}- (-4)   \\
                  & = -2\frac{3}{4}+ 4      \\
                  & = \frac{5}{4}           \\
            16    & = -4 + (n-1)\frac{5}{4} \\
            20    & = \frac{5}{4}(n-1)      \\
            80    & = 5(n-1)                \\
            n-1   & = 16                    \\
            n     & = 17
          \end{flalign*}

    \item Given that $a_{2} = 4$ and $a_{6} = -8$, find the 10th term of the AP. \sol{}
          \begin{flalign*}
            a_{2}      & = 4  \\
            a + (2-1)d & = 4  \\
            a_{6}      & = -8 \\
            a + (6-1)d & = -8 \\
          \end{flalign*}
          \begin{numcases}
            {} a + d &= 4\\ a + 5d &= -8
          \end{numcases}
          \begin{flalign*}
            (2)  - (1): 4d    & =-12               \\
            d                 & = -3               \\
            a + {(-3)}        & = 4                \\
            a                 & = 7                \\
            \therefore a_{10} & = 7 + (10-1){(-3)} \\
                              & = 7  - 27          \\
                              & = -20
          \end{flalign*}

    \item How many multiples of 7 are there between 50 and 500? \sol{}
          \begin{flalign*}
            a_{1} & = 56      \\
            a_{n} & = 497     \\
            d     & = 7       \\
            497 = 56 + (n-1)7 \\
            441 = 7(n-1)      \\
            n-1   & = 63      \\
            n     & = 64
          \end{flalign*}

    \item Find 5 numbers between 30 and 54 such that these numbers form an AP. \sol{}
          \begin{flalign*}
            a_{1}        & = 30                                       \\
            a_{7}        & = 54                                       \\
            54           & = 30 + (7-1)d                              \\
            24           & = 6d                                       \\
            d            & = 4                                        \\
            \\
            \therefore\  & These\ 5\ numbers\ are\ 34,\ 38,\ 42,\ 46, \\
                         & and\ 50.
          \end{flalign*}
  \end{enumerate}

  \subsection*{Arithmetic mean}

  If A is in between x and y, and x, A, y are in AP, then

  \[
    A = \frac{x+y}{2}
  \]

  \subsection{Practice 4}

  \begin{enumerate}
    \item If 9, x, 17 are in AP, find x. \sol{}
          \begin{flalign*}
            x & = \frac{9+17}{2} \\
              & = \frac{26}{2}   \\
              & = 13
          \end{flalign*}

    \item Find the arithmetic mean of 26 and -11. \sol{}
          \begin{flalign*}
            A & = \frac{26-11}{2} \\
              & = \frac{15}{2}    \\
          \end{flalign*}

    \item Find x and y when 3, x, 12, y, 21 are in AP. \sol{}
          \begin{flalign*}
            x & = \frac{3+12}{2}  \\
              & = \frac{15}{2}    \\
            y & = \frac{12+21}{2} \\
              & = \frac{33}{2}    \\
          \end{flalign*}
  \end{enumerate}

  \subsection*{Summation of Arithmetic Progression}

  The summation formula for AP is given by

  \[
    S_{n} = \frac{n}{2}(2a + (n-1)d)
  \]

  or

  \[
    S_{n} = \frac{n}{2}(a_{1} + a_{n})
  \]

  \subsection{Practice 5}

  \begin{enumerate}
    \item Find the sum of the first 16 terms of the AP $22 + 18 + 14 + 10 + \cdots$
          \sol{}
          \begin{flalign*}
            a_{1} & = 22                                   \\
            n     & = 16                                   \\
            d     & = -4                                   \\
            S_{n} & = \frac{16}{2}(2\times22 + (-4)(16-1)) \\
                  & = \frac{16}{2}(44 + (-4)(15))          \\
                  & = \frac{16}{2}(44  - 60)               \\
                  & = \frac{16}{2}(-16)                    \\
                  & = -128
          \end{flalign*}

    \item If the sum of AP $23 + 19 + 15 + \cdots$ is 72, find the number of terms.
          \sol{}
          \begin{flalign*}
            a_{1}              & = 23                                 \\
            S_{n}              & = 72                                 \\
            d                  & = -4                                 \\
            72                 & = \frac{n}{2}(2\times23 + (-4)(n-1)) \\
            72                 & = \frac{n}{2}(46 + (-4)(n-1))        \\
            144                & = n(46 + (-4)(n-1))                  \\
            144                & = n(46  - 4n + 4)                    \\
            144                & = n(50  - 4n)                        \\
            144                & = 50n  - 4n^{2}                      \\
            72                 & = 25n  - 2n^{2}                      \\
            2n^{2}  - 25n + 72 & = 0                                  \\
            (n  - 8)(2n  - 9)  & = 0                                  \\
            n                  & = 8                                  \\
          \end{flalign*}

    \item Given that $S_{n} = 2n + 3n^{2}$, find the first term and the common difference
          of the AP. \sol{}
          \begin{flalign*}
            S_{n}       & = 2n + 3n^{2}              \\
            2n + 3n^{2} & = \frac{n}{2}(2a + (n-1)d) \\
            4n + 6n^{2} & = n(2a + (n-1)d)           \\
            4n + 6n^{2} & = 2na + (n-1)nd            \\
            4n + 6n^{2} & = 2na + n^{2}d  - nd       \\
            4n + 6n^{2} & = (2a-d)n + dn^{2}         \\
            \\
            Compar      & ing\ both\ sides,          \\
            2a-d        & = 4                        \\
            a           & = 6                        \\
            d           & = 2                        \\
          \end{flalign*}
  \end{enumerate}

  \subsection{Exercise 12.2}

  \begin{enumerate}
    \item Find the 10th terms of the AP $5, 13, 21, \cdots$ \sol{}
          \begin{flalign*}
            a_{1}  & = 5                 \\
            n      & = 10                \\
            d      & = 8                 \\
            a_{10} & = 5 + (10-1)\times8 \\
                   & = 5 + 72            \\
                   & = 77
          \end{flalign*}

    \item Find the 8th term of the AP $5, 4\frac{1}{4}, 3\frac{1}{2}, 2\frac{3}{4},
            \cdots$ \sol{}
          \begin{flalign*}
            a_{1} & = 5                           \\
            n     & = 8                           \\
            d     & = -\frac{3}{4}                \\
            a_{8} & = 5 + (8-1)\times-\frac{3}{4} \\
                  & = 5  - \frac{3}{4}\times7     \\
                  & = 5  - \frac{21}{4}           \\
                  & = -\frac{1}{4}
          \end{flalign*}

    \item Find the number of terms of the following AP.

          \begin{enumerate}

            \item $4, 9, \ldots, 64$
                  \sol{}
                  \begin{flalign*}
                    a_{1} & = 4                \\
                    a_{n} & = 64               \\
                    d     & = 5                \\
                    64    & = 4 + (n-1)\times5 \\
                    60    & = 5(n-1)           \\
                    12    & = n  - 1           \\
                    n     & = 13
                  \end{flalign*}

            \item $4\frac{1}{3}, 3\frac{2}{3}, 3, \ldots, -10\frac{1}{3}$
                  \sol{}
                  \begin{flalign*}
                    a_{1}          & = 4\frac{1}{3}                           \\
                    a_{n}          & = -10\frac{1}{3}                         \\
                    d              & = -\frac{2}{3}                           \\
                    -10\frac{1}{3} & = 4\frac{1}{3} + (n-1)\times-\frac{2}{3} \\
                    -\frac{31}{3}  & = \frac{13}{3}  - \frac{1}{3}(n-1)       \\
                    -31            & = 13  - 2n + 2                           \\
                    -46            & = 2n                                     \\
                    n              & = 23
                  \end{flalign*}

          \end{enumerate}

    \item The 6th term of an AP is 43, and its 10th term is 75. Find the first term and
          common difference of this AP. \sol{}
          \begin{flalign*}
            a_{6}      & = 43              \\
            a_{10}     & = 75              \\
            43         & = a + (6-1)d      \\
            75         & = a + (10-1)d     \\
            32         & = 4d              \\
            d          & = 8               \\
            43         & = a + 5\times8    \\
            43         & = a + 40          \\
            3          & = a               \\
            a          & = 3               \\
            \therefore & \ a_1 = 3,\ d = 8
          \end{flalign*}

    \item The 7th term of an AP is -10, and the 12th term -25, find the 15th term of this
          AP. \sol{}
          \begin{flalign*}
            a_{7}  & = -10                \\
            a_{12} & = -25                \\
            -10    & = a + (7-1)d         \\
            -25    & = a + (12-1)d        \\
            -15    & = 5d                 \\
            d      & = -3                 \\
            -10    & = a + 6\times-3      \\
            -10    & = a  - 18            \\
            a      & = 8                  \\
            a_{15} & = 8 + (15-1)\times-3 \\
                   & = 8  - 42            \\
                   & = -34
          \end{flalign*}

    \item How many multiples of 7 are there between 100 and 200? \sol{}
          \begin{flalign*}
            a     & = 105                \\
            d     & = 7                  \\
            a_{n} & = 196                \\
            196   & = 105 + (n-1)\times7 \\
            91    & = 7(n-1)             \\
            13    & = n  - 1             \\
            n     & = 14
          \end{flalign*}

    \item Find the arithmetic mean o fthe following number pairs.

          \begin{enumerate}

            \item $(8, 20)$
                  \sol{}
                  \begin{flalign*}
                     & \frac{8 + 20}{2} = 14
                  \end{flalign*}

            \item $(-9, 17)$
                  \sol{}
                  \begin{flalign*}
                     & \frac{-9 + 17}{2} = 4
                  \end{flalign*}

          \end{enumerate}

    \item Find 5 numbers between 22 and 58 such that these 7 numbers are in AP. \sol{}
          \begin{flalign*}
            a_{1}      & = 22                                         \\
            a_{7}      & = 58                                         \\
            58         & = 22 + (7-1)d                                \\
            36         & = 6d                                         \\
            d          & = 6                                          \\
            \therefore & \ These\ 5\ numbers\ are\ 22, 28, 34, 40, 46
          \end{flalign*}

    \item Find the sum of first 20 terms of AP $12+15+18+\cdots$ \sol{}
          \begin{flalign*}
            a_{1}  & = 12                                      \\
            n      & = 20                                      \\
            d      & = 3                                       \\
            S_{20} & = \frac{20}{2}(2\times12 + (20-1)\times3) \\
                   & = 10(24 + 57)                             \\
                   & = 10(81)                                  \\
                   & = 810
          \end{flalign*}

    \item Find the sum of first 12 terms of the AP $18 + 10 + 2 - 6 - \dots$ \sol{}
          \begin{flalign*}
            a_{1}  & = 18                                       \\
            n      & = 12                                       \\
            d      & = -8                                       \\
            S_{12} & = \frac{12}{2}(2\times18 + (12-1)\times-8) \\
                   & = 6(36  - 88)                              \\
                   & = 6(-52)                                   \\
                   & = -312
          \end{flalign*}

    \item Find the sum of first 14 terms of the AP $\frac{1}{6} + \frac{4}{3} +
            \frac{5}{2} + \cdots$ \sol{}
          \begin{flalign*}
            a_{1}  & = \frac{1}{6}                                                \\
            n      & = 14                                                         \\
            d      & = \frac{7}{6}                                                \\
            S_{14} & = \frac{14}{2}(2\times\frac{1}{6} + (14-1)\times\frac{7}{6}) \\
                   & = 7(\frac{1}{3} + \frac{91}{6})                              \\
                   & = 7\times\frac{93}{6}                                        \\
                   & = 7\times\frac{31}{2}                                        \\
                   & = \frac{217}{2}
          \end{flalign*}

    \item Find the sum of all the multiples of 13 in between 200 and 800. \sol{}
          \begin{flalign*}
            a_{1}  & = 208                                       \\
            a_{n}  & = 793                                       \\
            d      & = 13                                        \\
            793    & = 208 + (n-1)\times13                       \\
            585    & = 13(n-1)                                   \\
            45     & = n  - 1                                    \\
            n      & = 46                                        \\
            \\
            S_{46} & = \frac{46}{2}(2\times208 + (46-1)\times13) \\
                   & = 23(416 + 585)                             \\
                   & = 23(1001)                                  \\
                   & = 23023
          \end{flalign*}

    \item If the sum of first n terms of the AP $-3, -7, -11, \cdots$ is -903, find the
          value of n. \sol{}
          \begin{flalign*}
            a_1               & = -3                                   \\
            d                 & = -4                                   \\
            -903              & = \frac{n}{2}(2\times{(-3)}  - 4(n-1)) \\
            -1806             & = -2n -4n^2                            \\
            4n^2 + 2n  - 1806 & = 0                                    \\
            2n^2 + n  - 903   & = 0                                    \\
            (n-21)(2n+43)     & = 0                                    \\
            n                 & = 21, -43(invalid)                     \\
            \therefore\ n     & =21
          \end{flalign*}

    \item Given that the first 3 terms of an AP are $x,\ 3x-4,\ 2x+7$, find:

          \begin{enumerate}

            \item The value of x \sol{}
                  \begin{flalign*}
                    3x-4 & = \frac{x + 2x + 7}{2} \\
                    6x-8 & = 3x + 7               \\
                    3x   & = 15                   \\
                    x    & = 5
                  \end{flalign*}

            \item The common difference \sol{}
                  \begin{flalign*}
                    a_1 & = x = 5                     \\
                    a_2 & = 3x-4 = 3\times5  - 4 = 11 \\
                    d   & = 11  - 5                   \\
                        & = 6
                  \end{flalign*}

            \item The sum of first 10 terms. \sol{}
                  \begin{flalign*}
                    a_1    & = x = 5                                  \\
                    n      & = 10                                     \\
                    d      & = 6                                      \\
                    S_{10} & = \frac{10}{2}(2\times5 + (10-1)\times6) \\
                           & = 5(10 + 54)                             \\
                           & = 5(64)                                  \\
                           & = 320
                  \end{flalign*}

          \end{enumerate}

    \item Let the sum of the first n terms of an AP to be $S_n = \frac{n(n+1)}{4}$, find:

          \begin{enumerate}

            \item The first term \sol{}
                  \begin{flalign*}
                    \frac{n(n+1)}{4} & = \frac{n}{2}(2a + (n-1)d) \\
                    n(n+1)           & = 2n(2a+dn-d)              \\
                    n^2 + n          & = 4na + 2dn^2  - 2nd       \\
                    n^2 + n          & = 2dn^2 + (4a  - 2d)n      \\
                    \\
                    Comparing        & \ both\ sides,             \\
                    2d               & = 1                        \\
                    d                & = \frac{1}{2}              \\
                    4a  - 2d         & = 1                        \\
                    4a  - 1          & = 1                        \\
                    4a               & = 2                        \\
                    a                & = \frac{1}{2}              \\
                  \end{flalign*}

            \item The common difference \sol{}
                  \begin{flalign*}
                    d & = \frac{1}{2}
                  \end{flalign*}gg

            \item The 6th terms \sol{}
                  \begin{flalign*}
                    a_1 & = \frac{1}{2}                          \\
                    n   & = 6                                    \\
                    d   & = \frac{1}{2}                          \\
                    a_6 & = \frac{1}{2} + (6-1)\times\frac{1}{2} \\
                        & = \frac{1}{2} + \frac{5}{2}            \\
                        & = 3
                  \end{flalign*}

            \item The sum from 6th term to 10th term \sol{}
                  \begin{flalign*}
                    a             & = \frac{1}{2}                                                \\
                    d             & = \frac{1}{2}                                                \\
                    \\
                    S_{10}        & = \frac{10}{2}(2\times\frac{1}{2} + (10-1)\times\frac{1}{2}) \\
                                  & = \frac{10}{2}(1 + \frac{9}{2})                              \\
                                  & = 5\times\frac{11}{2}                                        \\
                                  & = \frac{55}{2}                                               \\
                    \\
                    S_5           & = \frac{5}{2}(2\times\frac{1}{2} + (5-1)\times\frac{1}{2})   \\
                                  & = \frac{5}{2}(1 + 2)                                         \\
                                  & = \frac{15}{2}                                               \\
                    \\
                    S_{10}  - S_6 & = \frac{55}{2}  - \frac{15}{2}                               \\
                                  & = \frac{40}{2}                                               \\
                                  & = 20
                  \end{flalign*}

          \end{enumerate}

    \item Given three numbers in an AP, the sum of these three numbers is 30, and the sum
          of square of these numbers is 318, find these three numbers. \sol{}
          \begin{flalign*}
            a_1 + a_2 + a_3                     & = 30                  \\
            a_1^2 + a_2^2 + a_3^2               & = 318                 \\
            a_2  - a_1                          & = a_3  - a_2          \\
            a_1  - 2a_2 + a_3                   & = 0                   \\
            3a_2                                & = 30                  \\
            a_2                                 & = 10                  \\
            a_1  - 20 + a_3                     & = 0                   \\
            a_1 + a_3                           & = 20                  \\
            a_3                                 & = 20  - a_1           \\
            a_1^2 + 100 + {(20  - a_1)}^2       & = 318                 \\
            a_1^2 + 100 + 400 + a_1^2  - 40a_1  & = 318                 \\
            2a_1^2  - 40a_1 + 182               & = 0                   \\
            a_1^2  - 20a_1 + 91                 & = 0                   \\
            (a_1-7)(a_1-13)                     & = 0                   \\
            a_1 = 7 or a_1                      & = 13                  \\
            \\
            \therefore\ These\ three\ numbers\  & are\ 7,\ 10,\ and\ 13
          \end{flalign*}

    \item Find the sum of all the numbers between 100 and 200 that are both the multiples
          of 2 and 3. \sol{}
          \begin{flalign*}
            a_1     & = 102                                      \\
            d       & = 6                                        \\
            a_n     & = 198                                      \\
            198     & = 102 + (n-1)\times6                       \\
            96      & = 6(n-1)                                   \\
            6n  - 6 & = 96                                       \\
            6n = 102                                             \\
            n       & = 17                                       \\
            \\
            S_{17}  & = \frac{17}{2}(2\times102 + (17-1)\times6) \\
                    & = \frac{17}{2}(204 + 96)                   \\
                    & = \frac{17}{2}(300)                        \\
                    & = 150\times17                              \\
                    & = 2550
          \end{flalign*}

    \item Given an AP $-100-96-92-\cdots$:

          \begin{enumerate}

            \item Find the term where the number become positive. \sol{}
                  \begin{flalign*}
                    a_1                       & = -100 \\
                    d                         & = 4    \\
                    a_n = -100 + (n-1)\times4 & > 0    \\
                    -100 + 4n  - 4            & > 0    \\
                    4n                        & > 104  \\
                    n                         & > 26   \\
                    \\
                    \therefore\ n = 27        &
                  \end{flalign*}

            \item Find the term where the sum of this AP becomes positive. \sol{}
                  \begin{align*}
                    S_n = \frac{n}{2}(2(-100) + (n-1)\times(4)) & > 0  \\
                    \frac{n}{2}(-200 + 4n  - 4)                 & > 0  \\
                    \frac{n}{2}(-204 + 4n)                      & > 0  \\
                    n(2n  - 102)                                & > 0  \\
                    n(n  - 51)                                  & > 0  \\
                    n                                           & > 51 \\
                    \\
                    \therefore\ n = 52                          &
                  \end{align*}

          \end{enumerate}

    \item Find the first negative term of the AP $20, 19\frac{1}{5}, 18\frac{2}{5},
            \cdots$ \sol{}
          \begin{flalign*}
            a_1                                  & = 20           \\
            d                                    & = -\frac{4}{5} \\
            a_n = 20 + (n-1)\times(-\frac{4}{5}) & < 0            \\
            100  - 4n + 4                        & < 0            \\
            4n                                   & > 104          \\
            n                                    & > 26           \\
            \\
            \therefore\ n = 27                   &
          \end{flalign*}

    \item Given an AP $10+9\frac{1}{5}+8\frac{2}{5}+\cdots$, what is the first negative
          term? When will the sum of the terms become negative, and what's the value of
          it? \sol{}
          \begin{flalign*}
            a_n = 10 + (n-1)\times(-\frac{4}{5})                     & < 0             \\
            10  - \frac{4}{5}(n  - 1)                                & < 0             \\
            50  - 4n + 4                                             & < 0             \\
            -4n                                                      & < -54           \\
            n                                                        & > 13\frac{1}{2} \\
            \\
            \therefore\ n                                            & = 14            \\
            \\
            S_n = \frac{n}{2}(2\times10 + (n-1)\times(-\frac{4}{5})) & < 0             \\
            \frac{n}{2}(20-\frac{4}{5}(n-1))                         & < 0             \\
            20n-\frac{4}{5}(n^2  - n)                                & < 0             \\
            100n  - 4n^2 + 4n                                        & < 0             \\
            25n  - n^2 + n                                           & < 0             \\
            26n  - n^2                                               & < 0             \\
            n(n  - 26)                                               & > 0             \\
            n                                                        & > 26            \\
            \\
            \therefore\ n                                            & = 27            \\
          \end{flalign*}
          \begin{flalign*}
            S_{27} & = \frac{27}{2}(2\times10 + (27-1)\times(-\frac{4}{5})) \\
                   & = \frac{27}{2}(20  - \frac{4}{5}(27-1))                \\
                   & = \frac{27}{2}(20  - \frac{4}{5}(26))                  \\
                   & = \frac{27}{2}\times(-\frac{4}{5})                     \\
                   & = -\frac{54}{5}
          \end{flalign*}
          \begin{flalign*}
             & \therefore\ The \ first \ negative \ term \ is\ the \ 14th \\
             & \ \ \ \ \ term                                             \\
             & \therefore\ The\ first\ term\ where\ the\ sum\ of\ the     \\
             & \ \ \ \ \ terms\ becomes\ negative\ is\ the\ 27th          \\
             & \ \ \ \ \ term                                             \\
             & \therefore\ The\ value\ of\ the\ sum\ of\ the\ terms       \\
             & \ \ \ \ \ when\ it\ becomes\ negative\ is\ -\frac{54}{5}
          \end{flalign*}

    \item Given a polygon which all their internal angles are in AP. The common
          difference of this AP is 6\degree, the largest angle is 135\degree. How many
          sides does this polygon have? \sol{}
          \begin{flalign*}
            a_1                                       & = 135            \\
            d                                         & = -6             \\
            \frac{n}{2}(2\times135 + (n-1)\times(-6)) & = 180(n-2)       \\
            n(270-6(n-1))                             & = 360(n-2)       \\
            n(276-6n)                                 & = 360n  - 720    \\
            276n  - 6n^2                              & = 360n  - 720    \\
            46n  - n^2                                & = 60n  - 120     \\
            n^2 + 14n  - 120                          & = 0              \\
            (n+20)(n-6)                               & = 0              \\
            n                                         & = -20\ (invalid) \\
            n                                         & = 6              \\
            \therefore\ The\ number\ of\              & sides\ is\ 6
          \end{flalign*}

    \item Given an AP which its 5th term is 3 and the sum of its first 10 terms is
          $26\frac{1}{4}$. Whcih term in this AP is 0? \sol{}
          \begin{flalign*}
            a_5 = a + (5-1)d                    & = 3                      \\
            a + 4d                              & = 3                      \\
            S_{10} = \frac{10}{2}(2a + (10-1)d) & = 26\frac{1}{4}          \\
            5(2a + 9d)                          & = 26\frac{1}{4}          \\
            20(2a+9d)                           & = 105                    \\
            4(2a+9d)                            & = 21                     \\
            8a + 36d                            & = 21                     \\
            8a + 32d                            & = 24                     \\
            4d                                  & = -3                     \\
            d                                   & = -\frac{3}{4}           \\
            a                                   & = 3 + \frac{3}{4}\times4 \\
                                                & = 6
            \\
            a_n = 6 + (n-1)\times(-\frac{3}{4}) & = 0                      \\
            6  - \frac{3}{4}(n-1)               & = 0                      \\
            24  - 3n + 3                        & = 0                      \\
            3n                                  & = 27                     \\
            n                                   & = 9
          \end{flalign*}

    \item Given that the sum of the first 6 terms of an AP is 96, and the sum of the
          first 20 terms is 3 times the sum of the first 10 terms of this AP. Find the
          first term and the 10th term of it. \sol{}
          \begin{flalign*}
            S_6 = \frac{6}{2}(2a + (6-1)d) & = 96                     \\
            3(2a + 5d)                     & = 96                     \\
            2a + 5d                        & = 32                     \\
            S_{20}                         & = 3S_{10}                \\
            \frac{20}{2}(2a + (20-1)d)     & = 3\times\frac{10}{2}(2a \\
                                           & + (10-1)d)               \\
            10(2a + 19d)                   & = 15(2a + 9d)            \\
            2(2a + 19d)                    & = 3(2a + 9d)             \\
            4a + 38d                       & = 6a + 27d               \\
            2a  - 11d                      & = 0                      \\
            16d                            & = 32                     \\
            d                              & = 2                      \\
            a                              & = \frac{11\times2}{2}    \\
                                           & = 11                     \\
            a_{10}                         & = 11 + (10-1)\times2     \\
                                           & = 29
          \end{flalign*}

    \item Given that $5^2\times5^4\times5^6\times\cdots\times5^{2n} = {(0.04)}^{-28}$,
          find the value of n. \sol{}
          \begin{flalign*}
            {(0.04)}^{-28}                             & = \frac{1}{25}^{-28} \\
                                                       & = {(5^(-2))}^{-28}   \\
                                                       & = 5^{56}             \\
            \because n^a \times n^b                    & = n^{a+b}            \\
            2+4+6+\cdots+2n                            & = 56                 \\
            S_n = \frac{n}{2}(2\times2 + (n-1)\times2) & = 56                 \\
            n(4 + 2(n-1))                              & = 112                \\
            n(2 + 2n)                                  & = 112                \\
            2n^2 + 2n                                  & = 112                \\
            n^2 + n  - 56                              & = 0                  \\
            (n+8)(n-7)                                 & = 0                  \\
            n                                          & = -8\ (invalid)      \\
            n                                          & = 7                  \\
          \end{flalign*}

    \item Given that the 9th term of an AP is double the 5th term of it. Find the ratio
          of the sum of first 9 terms and the sum of first 5 terms of the AP. \sol{}
          \begin{flalign*}
            a_9                  & = 2a_5                                           \\
            a + (9-1)d           & = 2(a + (5-1)d)                                  \\
            a + 8d               & = 2a + 8d                                        \\
            a                    & = 0                                              \\
            S_9 : S_5            & = \frac{9}{2}(2a + a_9) : \frac{5}{2}(2a + a_5)  \\
                                 & = \frac{9}{2}(2a + 2a_5) : \frac{5}{2}(2a + a_5) \\
                                 & = 9(a + a_5) : \frac{5}{2}(2a + a_5)             \\
            \frac{S_9}{S_5}      & = \frac{9(a + a_5)}{\frac{5}{2}(2a + a_5)}       \\
                                 & = \frac{18(a + a_5)}{5(2a + a_5)}                \\
                                 & = \frac{18\times a_5}{5\times a_5}               \\
                                 & = \frac{18}{5}
            \\
            \therefore S_9 : S_5 & = 18 : 5
          \end{flalign*}
  \end{enumerate}

  \section{Geometric Progression}

  The general formula of a geometric progression (GP) is given by
  \[
    a_n = a_1\times r^{n-1}
  \]

  where $a_1$ is the first term, $r$ is the common ratio, and $n$ is the number
  of terms.

  \subsection{Practice 6}

  \begin{enumerate}

    \item Find the 6th term of the GP $12, -18, 27, \cdots$ \sol{}
          \begin{flalign*}
            a_1 & = 12                             \\
            r   & = \frac{-18}{12}                 \\
                & = -\frac{3}{2}                   \\
            a_6 & = 12 \times (-\frac{3}{2})^{6-1} \\
                & = 12 \times (-\frac{3}{2})^5     \\
                & = 12 \times (-\frac{243}{32})    \\
                & = -\frac{729}{8}
          \end{flalign*}

    \item Find the number of terms of GP $\frac{1}{64} - \frac{1}{32} + \frac{1}{16} -
            \frac{1}{8} + \cdots - 512$ \sol{}
          \begin{flalign*}
            a_1       & = \frac{1}{64}                       \\
            r         & = \frac{-\frac{1}{32}}{\frac{1}{64}} \\
                      & = -2                                 \\
            -512      & = \frac{1}{64}(-2)^{n-1}             \\
            (-2)^9    & = \frac{1}{2^6}(-2)^{n-1}            \\
            (-2)^{15} & = (-2)^{n-1}                         \\
            n-1       & = 15                                 \\
            n         & = 16
          \end{flalign*}

    \item The 5th term of a GP is 3, and its 9th term is $\frac{1}{27}$, find the first
          term and the common ratio of this GP. \sol{}
          \begin{flalign*}
            a_5 & = ar^4 = 3                        \\
            a_9 & = ar^8 = \frac{1}{27}             \\
            r^4 & = \frac{1}{27} \times \frac{1}{3} \\
                & = \frac{1}{81}                    \\
            r   & = \frac{1}{3}                     \\
            a_1 & = 3 \times 81                     \\
                & = 243
          \end{flalign*}

    \item Find 5 numbers between $\frac{1}{2}$ and $\\frac{1}{128}$ such that these 7
          numbers are in GP. \sol{}
          \begin{flalign*}
            a_1                     & = \frac{1}{2}                                                                    \\
            n                       & = 7                                                                              \\
            \frac{1}{128}           & = \frac{1}{2}r^{7-1}                                                             \\
            r^6                     & = \frac{1}{64}                                                                   \\
            r                       & = \frac{1}{2}                                                                    \\
            \\
            \therefore\ These \ 5\  & numbers\ are\ \frac{1}{4}, \frac{1}{8}, \frac{1}{16}, \frac{1}{32}, \frac{1}{64}
          \end{flalign*}

  \end{enumerate}

  \subsubsection* {Geometric Mean}

  The geometric mean G of two numbers $x$ and $y$ is given by
  \begin{flalign*}
    \frac{G}{x} & = \frac{G}{y}     \\
    G^2         & = xy              \\
    G           & = \mp\sqrt[2]{xy}
  \end{flalign*}

  \subsection{Practice 7}

  Find the geometric mean of $\frac{27}{8}$ and $\frac{2}{3}$. \sol{x}
  \begin{flalign*}
    G & = \pm\sqrt[2]{\frac{27}{8}\times\frac{2}{3}} \\
      & = \pm\sqrt[2]{\frac{9}{4}}                   \\
      & = \pm\frac{3}{2}
  \end{flalign*}

  \subsection*{Summation of Geometric Progression}

  The sum of $n$ terms of a GP is given by
  \[
    S_n = \frac{a_1(1-r^n)}{1-r}\ (r \neq 1)
  \]

  \subsection{Practice 8}

  \begin{enumerate}

    \item Find the sum of the first 8 terms of GP $3+6+12+\cdots$ \sol{}
          \begin{flalign*}
            a_1 & = 3                    \\
            r   & = \frac{6}{3}          \\
                & = 2                    \\
            n   & = 8                    \\
            S_n & = \frac{3(1-2^8)}{1-2} \\
                & = \frac{3(1-256)}{1-2} \\
                & = 3\times255           \\
                & = 765
          \end{flalign*}

    \item Find the sum of the GP $1+\sqrt{3}+3+\cdots+81$ \sol{}
          \begin{flalign*}
            a_1          & = 1                                     \\
            r            & = \sqrt{3}                              \\
            81           & = 1\times(\sqrt{3})^{n-1}               \\
            3^4          & = (\sqrt{3})^{n-1}                      \\
            (\sqrt{3})^8 & = (\sqrt{3})^{n-1}                      \\
            n-1          & = 8                                     \\
            n            & = 9                                     \\
            S_n          & = \frac{1(1-(\sqrt{3})^9)}{1-\sqrt{3}}  \\
                         & = \frac{1-81\sqrt{3}}{1-\sqrt{3}}       \\
                         & = \frac{(1-81\sqrt{3})(1+\sqrt{3})}{-2} \\
                         & = \frac{1-81\sqrt{3}+\sqrt{3}-243}{-2}  \\
                         & = \frac{-242-80\sqrt(3)}{-2}            \\
                         & = 121 + 40\sqrt{3}
          \end{flalign*}

    \item Given that the sum of the first n terms of GP $4\frac{4}{5}, 1\frac{3}{5},
            \frac{8}{15}, \cdots$ is $7\frac{145}{729}$, find n. \sol{}
          \begin{flalign*}
            a_1                                                     & = \frac{24}{5}                                              \\
            r                                                       & = \frac{8}{5}\times\frac{5}{24}                             \\
                                                                    & = \frac{1}{3}                                               \\
            S_n                                                     & = \frac{24}{5}\times\frac{1-(\frac{1}{3})^n}{1-\frac{1}{3}} \\
            \frac{5248}{729}                                        & = \frac{24}{5}\times\frac{1-(\frac{1}{3})^n}{\frac{2}{3}}   \\
            \frac{5248}{729} \times \frac{5}{24} \times \frac{2}{3} & = 1-(\frac{1}{3})^n                                         \\
            \frac{6560}{6561}                                       & = 1-(\frac{1}{3})^n                                         \\
            -\frac{1}{6561}                                         & = -(\frac{1}{3})^n                                          \\
            (\frac{1}{3})^8                                         & = (\frac{1}{3})^n                                           \\
            n                                                       & = 8
          \end{flalign*}

  \end{enumerate}

  \subsection* {Summation of Infinite Geometric Progression}

  The sum of infinite GP is given by
  \[
    S_\infty = \frac{a_1}{1-r}\ (-1 < r < 1)
  \]

  \subsection{Practice 9}

  \begin{enumerate}

    \item Find the sum of the following infinite GP.

          \begin{enumerate}

            \item $16+8+4+\cdots$
                  \sol{}
                  \begin{flalign*}
                    a_1      & = 16                       \\
                    r        & = \frac{8}{16}             \\
                             & = \frac{1}{2}              \\
                    S_\infty & = \frac{16}{1-\frac{1}{2}} \\
                             & = \frac{16}{\frac{1}{2}}   \\
                             & = 32
                  \end{flalign*}

            \item $18-12+8+\cdots$
                  \sol{}
                  \begin{flalign*}
                    a_1      & = 18                       \\
                    r        & = \frac{8}{-12}            \\
                             & = -\frac{2}{3}             \\
                    S_\infty & = \frac{18}{1+\frac{2}{3}} \\
                             & = \frac{18}{\frac{5}{3}}   \\
                             & = \frac{54}{5}             \\
                  \end{flalign*}

            \item $1+\frac{3}{4}+\frac{9}{16}+\cdots$
                  \sol{}
                  \begin{flalign*}
                    a_1      & = 1                              \\
                    r        & = \frac{9}{16}\times\frac{16}{9} \\
                             & = \frac{3}{4}                    \\
                    S_\infty & = \frac{1}{1-\frac{3}{4}}        \\
                             & = \frac{1}{\frac{1}{4}}          \\
                             & = 4
                  \end{flalign*}

            \item $\sqrt{2}+1+\frac{1}{\sqrt{2}}+\cdots$
                  \sol{}
                  \begin{flalign*}
                    a_1      & = \sqrt{2}                                   \\
                    r        & = \frac{1}{\sqrt{2}}                         \\
                    S_\infty & = \frac{\sqrt{2}}{1-\frac{1}{\sqrt{2}}}      \\
                             & = \frac{\sqrt{2}}{\frac{\sqrt2-1}{\sqrt{2}}} \\
                             & = \frac{2}{\sqrt2-1}                         \\
                             & = 2(\sqrt2 + 1)
                  \end{flalign*}

          \end{enumerate}

    \item Convert the following recurring decimals to fraction using the summation of
          inifinite geometric series.

          \begin{enumerate}

            \item $0.\overline{3}$
                  \sol{}
                  \begin{flalign*}
                    a_1            & = 0.3               \\
                    r              & = 0.1               \\
                    S_\infty       & = \frac{0.3}{1-0.1} \\
                                   & = \frac{0.3}{0.9}   \\
                                   & = \frac{1}{3}       \\
                    \\
                    \therefore
                    0.\overline{3} & = \frac{1}{3}
                  \end{flalign*}

            \item $0.5\overline{3}$
                  \sol{}
                  \begin{flalign*}
                    a_1                      & = 0.03                        \\
                    r                        & = 0.01                        \\
                    S_\infty                 & = \frac{0.03}{1-0.01}         \\
                                             & = \frac{0.03}{0.99}           \\
                                             & = \frac{3}{99}                \\
                    \\
                    \therefore 0.5\overline3 & = \frac{5}{10} + \frac{3}{99} \\
                                             & = \frac{53}{99}               \\
                  \end{flalign*}

          \end{enumerate}

  \end{enumerate}

  \subsection{Exercise 12.3}

  \begin{enumerate}

    \item Find the 10th term of the GP $2, 4, 8, \cdots$ \sol{}
          \begin{flalign*}
            a_1    & = 2               \\
            r      & = \frac{4}{2}     \\
                   & = 2               \\
            a_{10} & = 2\times2^{10-1} \\
                   & = 2\times512      \\
                   & = 1024
          \end{flalign*}

    \item Find the 8th term of the GP $243, -162, 108, \cdots$ \sol{}
          \begin{flalign*}
            a_1   & = 243                             \\
            r     & = \frac{-162}{243}                \\
                  & = -\frac{2}{3}                    \\
            a_{8} & = 243\times{(-\frac{2}{3})}^{8-1} \\
                  & = 243\times(-\frac{128}{2187})    \\
                  & = -\frac{128}{9}
          \end{flalign*}

    \item Find the number of terms of the following GP.

          \begin{enumerate}

            \item $8, 4, 2, 1, \ldots, \frac{1}{64}$
                  \sol{}
                  \begin{flalign*}
                    a_1           & = 8                          \\
                    r             & = \frac{4}{8}                \\
                                  & = \frac{1}{2}                \\
                    \frac{1}{64}  & = 8\times(\frac{1}{2})^{n-1} \\
                    \frac{1}{512} & = (\frac{1}{2})^{n-1}        \\
                    \frac{1}{2^9} & = (\frac{1}{2})^{n-1}        \\
                    n-1           & = 9                          \\
                    n             & = 10
                  \end{flalign*}

            \item $6, -18, 54, \ldots, -13122$
                  \sol{}
                  \begin{flalign*}
                    a_1      & = 6                   \\
                    r        & = \frac{-18}{6}       \\
                             & = -3                  \\
                    -13122   & = 6\times{(-3)}^{n-1} \\
                    -2187    & = {(-3)}^{n-1}        \\
                    {(-3)}^7 & = {(-3)}^{n-1}        \\
                    n-1      & = 7                   \\
                    n        & = 8
                  \end{flalign*}

            \item $54, 36, 24, \dots, 3\frac{13}{81}$
                  \sol{}
                  \begin{flalign*}
                    a_1                                & = 54                          \\
                    r                                  & = \frac{36}{54}               \\
                                                       & = \frac{2}{3}                 \\
                    \frac{256}{81}                     & = 54\times(\frac{2}{3})^{n-1} \\
                    \frac{256}{81} \times \frac{1}{54} & = (\frac{2}{3})^{n-1}         \\
                    \frac{128}{2187}                   & = (\frac{2}{3})^{n-1}         \\
                    (\frac{2}{3})^7                    & = (\frac{2}{3})^{n-1}         \\
                    n-1                                & = 7                           \\
                    n                                  & = 8
                  \end{flalign*}

          \end{enumerate}

    \item Given that the 2nd term of a GP is 12, and its 4th term is 108, find the first
          term and the common ratio of it. \sol{}
          \begin{flalign*}
            a_2            & = ar = 12                                 \\
            a_4            & = ar^3 = 109                              \\
            r^2            & = 9                                       \\
            r              & = \pm3                                    \\
            a_1            & = \pm4                                    \\
            \therefore a_1 & = 4, r = 3 \text{\ or\ } a_1 = -4, r = -3
          \end{flalign*}

    \item Given that the 3rd term of an GP is $1\frac{1}{3}$, and its 8th term is
          $-10\frac{1}{8}$. Find the 5th term of this AP. \sol{}
          \begin{flalign*}
            a_3 & = ar^2 = \frac{4}{3}                   \\
            a_8 & = ar^7 = -\frac{81}{8}                 \\
            r^5 & = -\frac{81}{8}\times\frac{3}{4}       \\
                & = -\frac{243}{32}                      \\
                & = {(-\frac{3}{2})}^5                   \\
            r   & = -\frac{3}{2}                         \\
            a   & = \frac{4}{3}\times\frac{4}{9}         \\
                & = \frac{16}{27}                        \\
            a_5 & = \frac{16}{27}\times{(\frac{3}{2})}^4 \\
                & = \frac{16}{27}\times\frac{81}{16}     \\
                & = 3
          \end{flalign*}

    \item Find the geometric mean of 2 and 18. \sol{}
          \begin{flalign*}
            G & = \pm\sqrt[2]{2\times18} \\
              & = \pm\sqrt[2]{36}        \\
              & = \pm6
          \end{flalign*}

    \item Given that x+12, x+4 and x-2 are in GP, find the value of x and the common
          ratio of this GP. \sol{}
          \begin{flalign*}
            x+4           & = \pm\sqrt{(x+12)(x-2)} \\
            x^2 + 8x + 16 & = x^2 + 10x  - 24       \\
            2x            & = 40                    \\
            x             & = 20                    \\
            a_1           & = 20+12 = 32            \\
            a_2           & = 20+4 = 24             \\
            r             & = \frac{24}{32}         \\
                          & = \frac{3}{4}           \\
          \end{flalign*}

    \item Find 3 numbers between 14 and 224 such that these 5 numbersare in GP. \sol{}
          \begin{flalign*}
            a_1                    & = 14                        \\
            a_5                    & = 224                       \\
            244                    & = 14\times r^4              \\
            16                     & = r^4                       \\
            (\pm2)^4               & = r^4                       \\
            r                      & = \pm2                      \\
            \\
            \therefore\ These\ 3\  & numbers\ are\ 28,\ 56,\ 112 \\
            or\ -28,\              & 56,\ -112
          \end{flalign*}

    \item Calculate the sum of the first 6 terms of the GP $2+6+18+\cdots$ \sol{}
          \begin{flalign*}
            a_1 & = 2                    \\
            r   & = \frac{6}{2}          \\
                & = 3                    \\
            S_6 & = \frac{2(1-3^6)}{1-3} \\
                & = \frac{2(1-729)}{-2}  \\
                & = 728
          \end{flalign*}

    \item Calculate the sum of the first 8 terms of the GP $32-16+8-\cdots$ \sol{}
          \begin{flalign*}
            a_1 & = 32                                          \\
            r   & = \frac{-16}{32}                              \\
                & = -\frac{1}{2}                                \\
            S_8 & = \frac{32(1-(\frac{1}{2})^8)}{1+\frac{1}{2}} \\
                & = \frac{32(1-\frac{1}{256})}{\frac{3}{2}}     \\
                & = 32\times\frac{255}{256}\times\frac{2}{3}    \\
                & = \frac{85}{4}
          \end{flalign*}

    \item Find the sum of the GP $14-28+56-\cdots+3584$ \sol{}
          \begin{flalign*}
            a_1    & = 14                          \\
            r      & = \frac{-28}{14} = -2         \\
            3584   & = 14\times(-2)^{n-1}          \\
            256    & = (-2)^{n-1}                  \\
            (-2)^8 & = (-2)^{n-1}                  \\
            n-1    & = 8                           \\
            n      & = 9                           \\
            S_9    & = \frac{14(1-(-2)^9)}{1-(-2)} \\
                   & = \frac{14(1+512)}{3}         \\
                   & = \frac{14\times513}{3}       \\
                   & = 2394
          \end{flalign*}

    \item If the first term of a GP is 7, its common ratio is 3, and the sum of its terms
          is 847, find the number of terms and the last term of this GP. \sol{}
          \begin{flalign*}
            a_1      & = 7                          \\
            r        & = 3                          \\
            S_n      & = \frac{7(1-3^n)}{1-3} = 847 \\
            7(1-3^n) & = -1694                      \\
            1-3^n    & = -242                       \\
            3^n      & = 243                        \\
            3^n      & = 3^5                        \\
            n        & = 5                          \\
            a_5      & = 7\times3^4 = 567
          \end{flalign*}

    \item Find the sum of the following infinite GP.

          \begin{enumerate}

            \item $24+18+13\frac{1}{2}+\cdots$
                  \sol{}
                  \begin{flalign*}
                    a_1      & = 24                          \\
                    r        & = \frac{18}{24} = \frac{3}{4} \\
                    S_\infty & = \frac{24}{1-\frac{3}{4}}    \\
                             & = \frac{24}{\frac{1}{4}}      \\
                             & = 96
                  \end{flalign*}

            \item $27-9+3-1+\cdots$
                  \sol{}
                  \begin{flalign*}
                    a_1      & = 27                           \\
                    r        & = \frac{-9}{27} = -\frac{1}{3} \\
                    S_\infty & = \frac{27}{1+\frac{1}{3}}     \\
                             & = \frac{27}{\frac{4}{3}}       \\
                             & = \frac{81}{4}
                  \end{flalign*}

            \item $2-\frac{1}{2}+\frac{1}{8}-\frac{1}{32}+\cdots$
                  \sol{}
                  \begin{flalign*}
                    a_1      & = 2                                     \\
                    r        & = \frac{-\frac{1}{2}}{2} = -\frac{1}{4} \\
                    S_\infty & = \frac{2}{1+\frac{1}{4}}               \\
                             & = \frac{2}{\frac{5}{4}}                 \\
                             & = \frac{8}{5}
                  \end{flalign*}

          \end{enumerate}

    \item Given an infinite GP which has a sum of 24 and first term of 30, find the
          common difference. \sol{}
          \begin{flalign*}
            a_1      & = 30             \\
            S_\infty & = 24             \\
            24       & = \frac{30}{1-r} \\
            24(1-r)  & = 30             \\
            24-24r   & = 30             \\
            -24r     & = 6              \\
            r        & = -\frac{1}{4}
          \end{flalign*}

    \item Convert the following recurrring decimals into fractions.

          \begin{enumerate}

            \item $0.\overline{45}$
                  \sol{}
                  \begin{flalign*}
                    a_1          & = 0.45                         \\
                    r            & = 0.01                         \\
                    S_\infty     & = \frac{0.45}{1-0.01}          \\
                                 & = \frac{0.45}{0.99}            \\
                                 & = \frac{45}{99}                \\
                                 & = \frac{5}{11}                 \\
                    \\
                    \therefore\  & 0.\overline{45} = \frac{5}{11}
                  \end{flalign*}

            \item $0.\overline{037}$
                  \sol{}
                  \begin{flalign*}
                    a_1          & = 0.037                         \\
                    r            & = 0.001                         \\
                    S_\infty     & = \frac{0.037}{1-0.001}         \\
                                 & = \frac{0.037}{0.999}           \\
                                 & = \frac{37}{999}                \\
                                 & = \frac{1}{27}                  \\
                    \\
                    \therefore\  & 0.\overline{037} = \frac{1}{27}
                  \end{flalign*}

            \item $0.2\overline{18}$
                  \sol{}
                  \begin{flalign*}
                    a_1                          & = 0.018                    \\
                    r                            & = 0.01                     \\
                    S_\infty                     & = \frac{0.018}{1-0.01}     \\
                                                 & = \frac{0.018}{0.99}       \\
                                                 & = \frac{18}{990}           \\
                                                 & = \frac{1}{55}             \\
                    \\
                    \therefore\ 0.2\overline{18} & = \frac{1}{5}+\frac{1}{55} \\
                                                 & = \frac{12}{55}
                  \end{flalign*}

            \item $1.\overline{3}$
                  \sol{}
                  \begin{flalign*}
                    a_1                        & = 0.3               \\
                    r                          & = 0.1               \\
                    S_\infty                   & = \frac{0.3}{1-0.1} \\
                                               & = \frac{0.3}{0.9}   \\
                                               & = \frac{1}{3}       \\
                    \\
                    \therefore\ 1.\overline{3} & = 1+\frac{1}{3}     \\
                                               & = \frac{4}{3}
                  \end{flalign*}

          \end{enumerate}

    \item Three integers are in GP, summing up to 42 while accumulating up to 512, find
          these three integers. \sol{}
          \begin{flalign*}
            a_1 + a_2 + a_3            & = 42                      \\
            a_1a_2a_3                  & = 512                     \\
            a_2                        & = \pm\sqrt{a_1a_3}        \\
            a_1a_3                     & = a_2^2                   \\
            a_2^3                      & = 512                     \\
            a_2                        & = \sqrt[3]{512}           \\
                                       & = 8                       \\
            a_1a_3                     & = 64                      \\
            a_3                        & = \frac{64}{a_1}          \\
            a_1 + 8 + \frac{64}{a_1}   & = 42                      \\
            a_1 + \frac{64}{a_1}       & = 34                      \\
            a_1^2 + 64                 & = 34a_1                   \\
            a_1^2  - 34a_1 + 64        & = 0                       \\
            (a_1-32)(a_1-2)            & = 0                       \\
            a_1                        & = 32\ or\ a_1 = 2         \\
            \\
            \therefore\ These\ three\  & integers\ are\ 2,\ 8,\ 32
          \end{flalign*}

    \item The sum of first 6 term of a GP is 9 times the sum of first 3 terms. Find the
          common ratio. \sol{}
          \begin{flalign*}
            S_6                  & = 9S_3                        \\
            \frac{a(1-r^6)}{1-r} & = 9\times\frac{a(1-r^3)}{1-r} \\
            a(1-r^6)             & = 9a(1-r^3)                   \\
            1-r^6                & = 9(1-r^3)                    \\
                                 & = 9-9r^3                      \\
            r^6  - 9r^3 + 8      & = 0                           \\
            (r^3  - 8)(r^3  - 1) & = 0                           \\
            r^3                  & = 8\ or\ r^3 = 1              \\
            r                    & = 1\ (invalid)                \\
            r                    & = 2
          \end{flalign*}

    \item Given a GP, its first term is 16, last term is $\frac{1}{2}$ and its sum is
          $31\frac{1}{2}$, find its common ratio and number of terms. \sol{}
          \begin{flalign*}
            a_1                 & = 16                     \\
            \frac{1}{2}         & = 16r^{n-1}              \\
            \frac{1}{32}        & = r^{n-1}                \\
                                & = r^n \times \frac{1}{r} \\
            r^n                 & = \frac{r}{32}           \\
            \frac{63}{2}        & = \frac{16(1-r^n)}{1-r}  \\
            63(1-r)             & = 32(1-r^n)              \\
            63-63r              & = 32-32r^n               \\
            -31                 & = 32r^n  - 63r           \\
            -31                 & = r  - 63r               \\
            -31                 & = -62r                   \\
            r                   & = \frac{1}{2}            \\
            (\frac{1}{2})^{n-1} & = \frac{1}{32}           \\
                                & = (\frac{1}{2})^5        \\
            n-1                 & = 5                      \\
            n                   & = 6                      \\
          \end{flalign*}

    \item Given a GP, its 3rd term is 6 less than its 2nd term, ant its 2nd term is 9
          less than its 1st term. Find the 4th term and the sum of the first 4 terms.
          \sol{}
          \begin{flalign*}
            Let\ x   & =a_2                                           \\
            a_3      & = x-6                                          \\
            a_1      & = x+9                                          \\
            x        & = \pm\sqrt{(x-6)(x+9)}                         \\
            x^2      & = x^2 + 3x  - 54                               \\
            3x  - 54 & = 0                                            \\
            x        & = 18                                           \\
            a_2      & = 18                                           \\
            a_1      & = 27                                           \\
            r        & = \frac{12}{18}                                \\
                     & = \frac{2}{3}                                  \\
            a_4      & = 27\times(\frac{2}{3})^3                      \\
                     & = 8                                            \\
            S_4      & = \frac{27(1-(\frac{16}{3})^4)}{1-\frac{2}{3}} \\
                     & = \frac{27(1-\frac{8}{81})}{\frac{1}{3}}       \\
                     & = 81\times\frac{65}{81}                        \\
                     & = 65
          \end{flalign*}

    \item GIven an infinite GP, its common ratio is positive and the sum of it is 9. The
          sum of the first two terms is 5, find the 4th term. \sol{}
          \begin{flalign*}
            S_\infty                   & = \frac{a}{1-r} = 9        \\
            a                          & = 9(1-r)                   \\
                                       & = 9-9r                     \\
            S_2                        & = \frac{a(1-r^2)}{1-r} = 5 \\
            a  - ar^2                  & = 5  - 5r                  \\
            9-9r  - (9-9r)r^2          & = 5  - 5r                  \\
            9-9r  - 9r^2 + 9r^3        & = 5  - 5r                  \\
            4  - 4r  - 9r^2 + 9r^3     & = 0                        \\
            4(1-r)-9r^2(1-r)           & = 0                        \\
            (4-9r^2)(1-r)              & = 0                        \\
            (9r^2  - 4)(r-1)           & = 0                        \\
            (3r^2 + 2)(3r^2  - 2)(r-1) & = 0                        \\
            r                          & = 1\ (invalid)             \\
            r                          & = -\frac{2}{3}(invalid)    \\
            r                          & = \frac{2}{3}              \\
            a                          & = 9(1-\frac{2}{3})         \\
                                       & = 3                        \\
            a_4                        & = 3(\frac{2}{3})^3         \\
                                       & = 3\times\frac{8}{27}      \\
                                       & = \frac{8}{9}
          \end{flalign*}

    \item If $x + 1, x - 2, \frac{1}{2}x$ are the first three terms of an infinite GP,
          find:

          \begin{enumerate}

            \item The value of x \sol{}
                  \begin{flalign*}
                    x  - 2         & = \pm\sqrt{(x+1)(\frac{1}{2}x)} \\
                    x^2  - 4x + 4  & = \frac{1}{2}x(x+1)             \\
                    2x^2  - 8x + 8 & = x^2 + x                       \\
                    x^2  - 9x + 8  & = 0                             \\
                    (x-8)(x-1)     & = 0                             \\
                    x              & = 8\ or\ x = 1                  \\
                  \end{flalign*}

            \item The common ratio \sol{}
                  \begin{flalign*}
                    When\ x & = 8,              \\
                    r       & = \frac{8-2}{8+1} \\
                            & = \frac{6}{9}     \\
                            & = \frac{2}{3}     \\
                    \\
                    When\ x & = 1,              \\
                    r       & = \frac{1-2}{1+1} \\
                            & = -\frac{1}{2}    \\
                  \end{flalign*}

            \item The sum of the GP \sol{}
                  \begin{flalign*}
                    When\ x  & = 8,                      \\
                    S_\infty & = \frac{a}{1-r}           \\
                             & = \frac{9}{1-\frac{2}{3}} \\
                             & = 9\times3                \\
                             & = 27                      \\
                    \\
                    When\ x  & = 1,                      \\
                    S_\infty & = \frac{a}{1-r}           \\
                             & = \frac{2}{1+\frac{1}{2}} \\
                             & = 2\times\frac{2}{3}      \\
                             & = \frac{4}{3}
                  \end{flalign*}

          \end{enumerate}

  \end{enumerate}

  \section{Simple Summation of Special Series}

  \noindent Sum formula of natural number:
  \[ \sum_{i=1}^n k = \frac{n(n+1)}{2} \]

  \noindent Sum formula of square of natural number:
  \[ \sum_{i=1}^n k^2 = \frac{n(n+1)(2n+1)}{6} \]

  \noindent Sum formula of cube of natural number:
  \[ \sum_{i=1}^n k^3 = \left[\frac{n(n+1)}{2}\right]^2 \]

  \subsection{Practice 10}

  \begin{enumerate}

    \item Find the sum of the following series.

          \begin{enumerate}

            \item $\sum_{k=1}^8 3k$
                  \sol{}
                  \begin{flalign*}
                    \sum_{k=1}^8 3k & = 3\sum_{k=1}^8 k           \\
                                    & = 3\times\frac{8(8+1)}{2}   \\
                                    & = 3\times\frac{8\times9}{2} \\
                                    & = 3\times\frac{72}{2}       \\
                                    & = 3\times36                 \\
                                    & = 108
                  \end{flalign*}

            \item $\sum_{k=1}^{12} k^2$
                  \sol{}
                  \begin{flalign*}
                    \sum_{k=1}^{12} k^2 & = \frac{12(12+1)(2\times12+1)}{6} \\
                                        & = \frac{12\times13\times25}{6}    \\
                                        & = 650
                  \end{flalign*}

            \item $\sum_{k=3}^{10} (2k-3)$
                  \sol{}
                  \begin{flalign*}
                     & \sum_{k=3}^{10} (2k-3)                                          \\
                     & = 2\sum_{k=3}^{10} k  - \sum_{k=3}^{10} 3                       \\
                     & = 2\left[\sum_{k=1}^{10} k  - \sum_{k=1}^{2} k\right]  - (30-6) \\
                     & = 2\left[\frac{10(10+1)}{2}  - \frac{2(2+1)}{2}\right]  - 8     \\
                     & = 2(55  - 3)  - 24                                              \\
                     & = 2\times52  - 24                                               \\
                     & = 104  - 24                                                     \\
                     & = 80
                  \end{flalign*}

            \item $\sum_{k=7}^{13} 3k^2$
                  \sol{}
                  \begin{flalign*}
                     & \sum_{k=7}^{13} 3k^2                                                             \\
                     & = 3\left[\sum_{k=1}^{13} k^2  - \sum_{k=1}^6 k^2\right]                          \\
                     & = 3\times\left[\frac{13(13+1)(2\times13+1)}{6}\right.                            \\
                     & \ \ \ \  - \left.\frac{6(6+1)(2\times6+1)}{6}\right]                             \\
                     & = 3\times\left[\frac{13\times14\times27}{6}  - \frac{6\times7\times13}{6}\right] \\
                     & = 3\times\left[\frac{4914}{6}  - \frac{546}{6}\right]                            \\
                     & = 3\times\frac{4368}{6}                                                          \\
                     & = 3\times728                                                                     \\
                     & = 2184
                  \end{flalign*}

          \end{enumerate}

    \item Given that the nth term of a series is n (n+3), find the sum of the first 20
          terms of the series. \sol{}
          \begin{flalign*}
             & \sum_{k=1}^{20} k(k+3)                                        \\
             & = \sum_{k=1}^{20} k^2 + 3k                                    \\
             & = \sum_{k=1}^{20} k^2 + 3\sum_{k=1}^{20} k                    \\
             & = \frac{20(20+1)(2\times20+1)}{6} + 3\times\frac{20(20+1)}{2} \\
             & = \frac{20\times21\times41}{6} + 3\times\frac{20\times21}{2}  \\
             & = 2870 + 630                                                  \\
             & = 3500
          \end{flalign*}

    \item Find the sum of series $1\times3 + 2\times4 + 3\times5 + \cdots + n(n+2)$.
          \sol{}
          \begin{flalign*}
             & \sum_{k=1}^n k(k+2)                                \\
             & = \sum_{k=1}^n k^2 + 2k                            \\
             & = \sum_{k=1}^n k^2 + 2\sum_{k=1}^n k               \\
             & = \frac{n(n+1)(2n+1)}{6} + 2\times\frac{n(n+1)}{2} \\
             & = \frac{n(n+1)(2n+1)}{6} + n(n+1)                  \\
             & = \frac{n(n+1)(2n+1)+6n(n+1)}{6}                   \\
             & = \frac{n(n+1)(2n+7)}{6}
          \end{flalign*}

  \end{enumerate}

  \subsection{Exercise 12.4}

  \begin{enumerate}

    \item Find the sum of the following series.

          \begin{enumerate}

            \item $\sum{k=1}^8 5k^2$
                  \sol{}
                  \begin{flalign*}
                    \sum_{k=1}^8 5k^2 & = 5\sum_{k=1}^8 k^2                   \\
                                      & = 5\times\frac{8(8+1)(2\times8+1)}{6} \\
                                      & = 5\times\frac{8\times9\times17}{6}   \\
                                      & = 5\times\frac{1368}{6}               \\
                                      & = 5\times204                          \\
                                      & = 1020
                  \end{flalign*}

            \item $\sum_{k=1}^{9} k^3$
                  \sol{}
                  \begin{flalign*}
                    \sum_{k=1}^{9} k^3 & = \left[\frac{9(9+1)}{2}\right]^2 \\
                                       & = 45^2                            \\
                                       & = 2025
                  \end{flalign*}

            \item $\sum_{n=1}^{10} (3n-5)$
                  \sol{}
                  \begin{flalign*}
                    \sum_{n=1}^{10} (3n-5) & = 3\sum_{n=1}^{10} n  - 5\sum_{n=1}^{10} 1 \\
                                           & = 3\times\frac{10(10+1)}{2}  - 5\times10   \\
                                           & = 3\times\frac{10\times11}{2}  - 5\times10 \\
                                           & = 3\times55  - 50                          \\
                                           & = 3\times5  - 50                           \\
                                           & = 165  - 50                                \\
                                           & = 115
                  \end{flalign*}

            \item $\sum_{k=3}^{6} 2k^3$
                  \sol{}
                  \begin{flalign*}
                    \sum_{k=3}^{6} 2k^3 & = 2\sum_{k=3}^{6} k^3                                   \\
                                        & = 2\left(\sum_{k=1}^6 k^3  - \sum_{k=1}^2 k^3\right)    \\
                                        & = 2\left\{\left[\frac{6(6+1)}{2}\right]^2\right.        \\
                                        & \ \ \ \ \left.- \left[\frac{2(2+1)}{2}\right]^2\right\} \\
                                        & = 2(21^2  - 3^2)                                        \\
                                        & = 2(441  - 9)                                           \\
                                        & = 2\times432                                            \\
                                        & = 864
                  \end{flalign*}

            \item $\sum_{k=6}^{10} (2k^2 + 3)$
                  \sol{}
                  \begin{flalign*}
                     & \sum_{k=6}^{10}(2k^2 + 3)                                                        \\
                     & = 2\sum_{k=6}^{10} k^2 + 3\sum_{k=6}^{10} 1                                      \\
                     & = 2\left(\sum_{k=1}^{10} k^2  - \sum_{k=1}^5 k^2\right)                          \\
                     & \ \ \ \ + 3\times(10-5)                                                          \\
                     & = 2\times\left[\frac{10\times11\times21}{6}  - \frac{5\times6\times11}{6}\right] \\
                     & \ \ \ \ + 3\times5                                                               \\
                     & = 2\times\left[\frac{2310}{6}  - \frac{330}{6}\right] + 3\times5                 \\
                     & = 2\times\frac{1980}{6} + 3\times5                                               \\
                     & = 2\times330 + 3\times5                                                          \\
                     & = 660 + 15                                                                       \\
                     & = 675
                  \end{flalign*}

            \item $\sum_{n=11}^{15} (n^2 + 2n)$
                  \sol{}
                  \begin{flalign*}
                     & \sum_{n=11}^{15}(n^2 + 2n)                                                  \\
                     & = \sum_{n=11}^{15} n^2 + 2\sum_{n=11}^{15} n                                \\
                     & = \left[\sum_{n=1}^{15} n^2  - \sum_{n=1}^{10} n^2\right]                   \\
                     & \ \ \ \ + 2\left[\sum_{n=1}^{15} n  - \sum_{n=1}^{10} n\right]              \\
                     & = \left[\frac{15\times16\times31}{6}  - \frac{10\times11\times21}{6}\right] \\
                     & \ \ \ \ + 2\left[\frac{15\times16}{2}  - \frac{10\times11}{2}\right]        \\
                     & = 985
                  \end{flalign*}

            \item $\sum_{n=2}^{6} n(n^2  - n + 1)$
                  \sol{}
                  \begin{flalign*}
                     & \sum_{n=2}^{6} n(n^2  - n + 1)                                                                           \\
                     & = \sum_{n=2}^{6} n^3  - \sum_{n=2}^{6} n^2 + \sum_{n=2}^{6} n                                            \\
                     & = \left[\sum_{n=1}^6 n^3  - \sum_{n=1}^1 n^3\right]  - \left[\sum_{n=1}^6 n^2  - \sum_{n=1}^1 n^2\right] \\
                     & \ \ \ \ + \left[\sum_{n=1}^6 n  - \sum_{n=1}^1 n\right]                                                  \\
                     & = \left[\left(\frac{6\times7}{2}\right)^2  - \left(\frac{1\times2}{2}\right)^2\right]                    \\
                     & \ \ \ \  - \left(\frac{6\times7\times13}{6}\  - \frac{1\times2\times3}{6}\right)                         \\
                     & \ \ \ \ + \left(\frac{6\times7}{2}\  - \frac{1\times2}{2}\right)                                         \\
                     & = 21^2-1^2  - (7\times13-1) + (3\times7-1)                                                               \\
                     & = 440  - 90 + 20                                                                                         \\
                     & = 370
                  \end{flalign*}

          \end{enumerate}

    \item Fiven that the nth term of a series is $3n^2 + n$, find the sum of the first 10
          terms of the series. \sol{}
          \begin{flalign*}
            \sum_{n=1}^{10} 3n^2 + n & = 3\sum_{n=1}^{10} n^2 + \sum_{n=1}^{10} n                                       \\
                                     & = 3\left(\frac{10\times11\times21}{6}\right) + \left(\frac{10\times11}{2}\right) \\
                                     & = 3\times\frac{2310}{6} + \frac{110}{2}                                          \\
                                     & = 3\times385 + 55                                                                \\
                                     & = 1210
          \end{flalign*}

    \item Find the sum of first nth term of series $1\times3+2\times7+3\times11+\cdots$
          \sol{}
          \begin{flalign*}
             & \sum_{n=1}^{n} n\times(4n-1)                                            \\
             & = 4\sum_{n=1}^{n} n^2  - \sum_{n=1}^{n} n                               \\
             & = 4\left(\frac{n(n+1)(2n+1)}{6}\right)  - \left(\frac{n(n+1)}{2}\right) \\
             & = \frac{4n(n+1)(2n+1)-3n(n+1)}{6}                                       \\
             & = \frac{n(n+1)(8n+1)}{6}
          \end{flalign*}

    \item Find the sum fo the series $1^2 + 3^2 + 5^2 + \cdots + 15^2$ \sol{}
          \begin{flalign*}
            \sum_{n=1}^{8} (2n-1)^2 & = \sum_{n=1}^{8} (4n^2-4n+1)                                                       \\
                                    & = 4\sum_{n=1}^{8} n^2  - 4\sum_{n=1}^{8} n + \sum_{n=1}^{8} 1                      \\
                                    & = 4\left(\frac{8\times9\times17}{6}\right)  - 4\left(\frac{8\times9}{2}\right) + 8 \\
                                    & = 4\times204  - 4\times36 + 8                                                      \\
                                    & = 816  - 144 + 8                                                                   \\
                                    & = 680
          \end{flalign*}

  \end{enumerate}

  \section{Revision Exerise 12}

  \begin{enumerate}

    \item Express the following series in form of $\sum$.

          \begin{enumerate}

            \item $\frac{1}{2}+\frac{3}{4}+\frac{5}{6}+\cdots+\frac{49}{50}$
                  \sol{}
                  \begin{flalign*}
                    a_1          & = \frac{2\times1-1}{2\times1}                                                              \\
                    a_2          & = \frac{2\times2-1}{2\times2}                                                              \\
                    a_3          & = \frac{2\times3-1}{2\times3}                                                              \\
                                 & \vdots                                                                                     \\
                    a_{25}       & = \frac{2\times25-1}{2\times25}                                                            \\
                    \therefore\  & \frac{1}{2}+\frac{3}{4}+\frac{5}{6}+\cdots+\frac{49}{50} = \sum_{n=1}^{25} \frac{2n-1}{2n}
                  \end{flalign*}

            \item $6-7+8-9+\cdots$
                  \sol{}
                  \begin{flalign*}
                    a_1          & = (-1)^6\times6                               \\
                    a_2          & = (-1)^7\times7                               \\
                    a_3          & = (-1)^8\times8                               \\
                                 & \vdots                                        \\
                    a_n          & = (-1)^n n\
                    \therefore\  & 6-7+8-9+\cdots = \sum_{n=1}^{\infty} (-1)^n n
                  \end{flalign*}

            \item $2\times5+3\times7+4\times9+\cdots+15\times31$
                  \sol{}
                  \begin{flalign*}
                    a_1          & = (1+1)(2\times1+3)                          \\
                    a_2          & = (2+1)(2\times2+3)                          \\
                    a_3          & = (3+1)(2\times3+3)                          \\
                                 & \vdots                                       \\
                    a_{14}       & = (14+1)(2\times14+3)                        \\
                    \therefore\  & 2\times5+3\times7+4\times9+\cdots+15\times31 \\
                                 & = \sum_{n=1}^{14} (n+1)(2n+3)
                  \end{flalign*}

          \end{enumerate}

    \item Given a general formula $a_n = \frac{3^n}{2n-3}$, state the first 5 terms of
          the sequence. \sol{}
          \begin{flalign*}
            a_1 & = \frac{3^1}{2\times1-3} = -3            \\
            a_2 & = \frac{3^2}{2\times2-3} = 9             \\
            a_3 & = \frac{3^3}{2\times3-3} = 9             \\
            a_4 & = \frac{3^4}{2\times4-3} = \frac{81}{5}  \\
            a_5 & = \frac{3^5}{2\times5-3} = \frac{243}{7}
          \end{flalign*}

    \item Express the series $\sum_{k=1}^{10} (2k^2-3)$ \sol{}
          \begin{flalign*}
             & \ \ \ \ \sum_{k=1}^{10} (2k^2-3)                                    \\
             & = (2\times1^2  - 3) + (2\times2^2  - 3) + (2\times3^2  - 3)         \\
             & \ \ \ \ + (2\times4^2  - 3) + (2\times5^2  - 3) + (2\times6^2  - 3) \\
             & \ \ \ \ + (2\times7^2  - 3) + (2\times8^2  - 3) + (2\times9^2  - 3) \\
             & \ \ \ \ + (2\times10^2  - 3)                                        \\
             & = -1 + 5 + 15 + 29 + 47 + 69 + 95 + 125                             \\
             & \ \ \ \ + 159 + 197                                                 \\
          \end{flalign*}

    \item State the first term, last term and the number of terms of theh series
          $\sum_{k=3}^7 (3^k-2^k-k)$ \sol{}
          \begin{flalign*}
            a_3 & = 3^3-2^3-3 = 27-8-3 = 16       \\
            a_7 & = 3^7-2^7-7 = 2187-128-7 = 2052 \\
            n   & = 5
          \end{flalign*}

    \item Find the number of terms of the AP
          $-4-2\frac{3}{4}-1{1}{2}-\frac{1}{4}+\cdots+16$ \sol{}
          \begin{flalign*}
            a       & = -4                    \\
            d       & = \frac{5}{4}           \\
            16      & = -4 + (n-1)\frac{5}{4} \\
            20      & = \frac{5}{4}(n-1)      \\
            5n  - 5 & = 80                    \\
            5n      & = 85                    \\
            n       & = 17
          \end{flalign*}

    \item If x+1, 2x+1, x-3 are the first 3 terms of AP, find:

          \begin{enumerate}

            \item The value of x \sol{}
                  \begin{flalign*}
                    2x+1 & = \frac{x+1+x-3}{2} \\
                    4x+2 & = 2x-2              \\
                    2x   & = -4                \\
                    x    & = -2
                  \end{flalign*}

            \item Sum from the 10th term to the 20th term \sol{}
                  \begin{flalign*}
                    a_1 & = -1                                 \\
                    a_2 & = -3                                 \\
                    r   & = -2                                 \\
                    S   & = S_{20}-S_9                         \\
                        & = \frac{20}{2}(-2+(20-1)(-2))        \\
                        & \ \ \ \  - \frac{9}{2}(-2+(9-1)(-2)) \\
                        & = 10\times(-40)  - 9\times(-9)       \\
                        & = -400 + 81                          \\
                        & = -319
                  \end{flalign*}

          \end{enumerate}

    \item Find 4 numbers between 28 and -12 such that these 6 numbers form an AP. \sol{}
          \begin{flalign*}
            a_1                 & = 28                           \\
            a_n                 & = -12                          \\
            n                   & = 6                            \\
            -12                 & = 28 + 5d                      \\
            5d                  & = 40                           \\
            d                   & = 8                            \\
            \\
            \therefore\ These\  & 4\ numbers\ are\ -4, 4, 12, 20
          \end{flalign*}

    \item Find the sum of the following AP.

          \begin{enumerate}

            \item $7 + 11 + 15 + \cdots$ up to the 10th term
                  \sol{}
                  \begin{flalign*}
                    a_1    & = 7                              \\
                    d      & = 4                              \\
                    n      & = 10                             \\
                    S_{10} & = \frac{10}{2}(2\times7+(10-1)4) \\
                           & = 5(14+36)                       \\
                           & = 250
                  \end{flalign*}

            \item $20 + 18\frac{1}{2} + 17 + \cdots$ up to the 16tm term
                  \sol{}
                  \begin{flalign*}
                    a_1    & = 20                                           \\
                    d      & = -\frac{3}{2}                                 \\
                    n      & = 16                                           \\
                    S_{16} & = \frac{16}{2}(2\times20+(16-1)(-\frac{3}{2})) \\
                           & = 8(40  - \frac{45}{2})                        \\
                           & = 8\times\frac{35}{2}                          \\
                           & = 140
                  \end{flalign*}

            \item $2\sqrt2+3\sqrt2+4\sqrt2+\cdots+13\sqrt2$
                  \sol{}
                  \begin{flalign*}
                    a_1    & = 2\sqrt2                                   \\
                    d      & = \sqrt2                                    \\
                    n      & = 12                                        \\
                    S_{12} & = \frac{12}{2}(2\times2\sqrt2+(12-1)\sqrt2) \\
                           & = 6(4\sqrt2+11\sqrt2)                       \\
                           & = 6\times15\sqrt2                           \\
                           & = 90\sqrt2
                  \end{flalign*}

          \end{enumerate}

    \item Given an AP which the sum of the first n terms $S_n = n(1+2n)$, find:

          \begin{enumerate}

            \item First term \sol{}
                  \begin{flalign*}
                    \frac{n}{2}(2a+(n-1)d) & = n(1+2n)    \\
                    n(2a+(n-1d))           & = 2n(1+2n)   \\
                    2an + dn^2  - dn       & = 2n  - 4n^2 \\
                    (2a  - d)n + dn^2      & = 2n  - 4n^2 \\
                    \\
                    Comparing\ bot         & h\ sides,    \\
                    a                      & = 3          \\
                    d                      & = 4
                  \end{flalign*}

            \item Common Difference \sol{}
                  \begin{flalign*}
                     & According\ to\ the\ sol{}.\ of\ (a), \\
                     & d = 4
                  \end{flalign*}

            \item Sum of the first 20 terms. \sol{}
                  \begin{flalign*}
                           & According\ to\ the\ sol{}.\ of\ (a), \\
                           & a = 3                                \\
                           & d = 4                                \\
                           & n = 20                               \\
                    S_{20} & = \frac{20}{2}(2\times3+(20-1)4)     \\
                           & = 10(6+76)                           \\
                           & = 10\times82                         \\
                           & = 820
                  \end{flalign*}

          \end{enumerate}

    \item Given an AP $33+27+21+\cdots$

          \begin{enumerate}

            \item If the first sum of the first n terms is 105, find the value of n. \sol{}
                  \begin{flalign*}
                    a_1                         & = 33              \\
                    d                           & = -6              \\
                    105 = \frac{n}{2}(2\times33 & +(n-1)\times(-6)) \\
                    210                         & = n(66-(n-1)6)    \\
                    35                          & = 11n  - n^2 + n  \\
                    n^2  - 12n + 35             & = 0               \\
                    (n-7)(n-5)                  & = 0               \\
                    n                           & = 7\ or\ n = 5
                  \end{flalign*}

            \item If the sum of the first n terms is negative value, find the minimum value of n.
                  \sol{}
                  \begin{flalign*}
                    a_1                                    & = 33          \\
                    d                                      & = -6          \\
                    \frac{n}{2}(2\times33+(n-1)\times(-6)) & < 0           \\
                    n(66- 6n + 6)                          & < 0           \\
                    12n-n^2                                & < 0           \\
                    n(12-n)                                & < 0           \\
                    n                                      & > 12          \\
                    \\
                    \therefore\ The\ minimum\ value\       & of\ n\ is\ 13
                  \end{flalign*}

          \end{enumerate}

    \item Find the sum of the numbers between 150 and 300 that are multiple of both 5 and
          3. \sol{}
          \begin{flalign*}
            a_1 & = 165                                   \\
            a_n & = 285                                   \\
            d   & = 15                                    \\
            285 & = 165 + (n-1)\times15                   \\
            8   & = n-1                                   \\
            n   & = 9                                     \\
            \\
            S_9 & = \frac{9}{2}(2\times165+(9-1)\times15) \\
                & = \frac{9}{2}\times450                  \\
                & = 2025
          \end{flalign*}

    \item Find the sum of all the numbers between 100 and 200 that can be divided by 2 or
          3. \sol{}
          \begin{flalign*}
            a_1            & = 102                                    \\
            a_n            & = 198                                    \\
            \\
            \text{When\ }d & = 2,                                     \\
            198            & = 102 + (n-1)\times2                     \\
            48             & = n-1                                    \\
            n              & = 49                                     \\
            S_{49}         & = \frac{49}{2}(2\times102+(49-1)\times2) \\
                           & = \frac{49}{2}\times(204+96)             \\
                           & = 7350                                   \\
            \\
            \text{When\ }d & = 3,                                     \\
            198            & = 102 + (n-1)\times3                     \\
            32             & = n-1                                    \\
            n              & = 33                                     \\
            S_{33}         & = \frac{33}{2}(2\times102+(33-1)\times3) \\
                           & = \frac{33}{2}\times(204+96)             \\
                           & = 4950                                   \\
            \\
            \text{When\ }d & = 6,                                     \\
            198            & = 102 + (n-1)\times6                     \\
            16             & = n-1                                    \\
            n              & = 17                                     \\
            S_{17}         & = \frac{17}{2}(2\times102+(17-1)\times6) \\
                           & = \frac{17}{2}\times(204+96)             \\
                           & = 2550                                   \\
            \\
            \therefore\ S  & = 7350 + 4950  - 2550                    \\
                           & = 9750
          \end{flalign*}

    \item Find the sum of the numbers between 50 and 100 that cannot be divided by 5.
          \sol{}
          \begin{flalign*}
            \text{When\ }d & = 1,                                    \\
            a_1            & = 51                                    \\
            a_n            & = 99                                    \\
            99             & = 51 + (n-1)\times1                     \\
            48             & = n-1                                   \\
            n              & = 49                                    \\
            S_{49}         & = \frac{49}{2}(2\times51+(49-1)\times1) \\
                           & = \frac{49}{2}\times(102+48)            \\
                           & = 3675                                  \\
            \\
            \text{When\ }d & = 5,                                    \\
            a_1            & = 55                                    \\
            a_n            & = 95                                    \\
            95             & = 55 + (n-1)\times5                     \\
            8              & = n-1                                   \\
            n              & = 9                                     \\
            S_{9}          & = \frac{9}{2}(2\times55+(9-1)\times5)   \\
                           & = \frac{9}{2}\times(110+40)             \\
                           & = 675                                   \\
            \\
            \therefore\ S  & = 3675  - 675                           \\
                           & = 3000
          \end{flalign*}

    \item Which term is the first negative term of the AP
          $20+16\frac{1}{4}+12\frac{1}{2}+\cdots$? \sol{}
          \begin{flalign*}
            a_1                                 & = 20            \\
            d                                   & = -\frac{15}{4} \\
            a_n = 20  - (n-1)\times\frac{15}{4} & < 0             \\
            80-15(n-1)                          & < 0             \\
            16  - 3n + 3                        & < 0             \\
            3n > 19                                               \\
            n > 6\frac{1}{3}                                      \\
            \\
            \therefore\ The\ first\ negativ     & e\ term\ is\ 7
          \end{flalign*}

    \item Three numbers are in AP, thier sum is 15 while the sum of the square of these
          numbers is 83. Find this three numbers. \sol{}
          \begin{flalign*}
            a_1 + a_2 + a_3              & = 15                  \\
            a_1^2 + a_2^2 + a_3^2        & = 83                  \\
            a_2  - a_1                   & = a_3  - a_2          \\
            a_1 + a_3                    & = 2a_2                \\
            3a_2                         & = 15                  \\
            a_2                          & = 5                   \\
            a_3                          & = 10  - a_1           \\
            a_1^2 + a_3^2                & = 83  - 25            \\
                                         & = 58                  \\
            a_1^2 + (10-a_1)^2           & = 58                  \\
            a_1^2 + 100  - 20a_1 + a_1^2 & = 58                  \\
            2a_1^2  - 20a_1 + 100        & = 58                  \\
            2a_1^2  - 20a_1 + 42         & = 0                   \\
            a_1^2  - 10a_1 + 21          & = 0                   \\
            (a_1  - 7)(a_1  - 3)         & = 0                   \\
            a_1                          & = 7\ or\ a_1 = 3      \\
            \\
            \therefore\ The\ three\      & numbers\ are\ 7, 5, 3
          \end{flalign*}

    \item Find the sum of the series $18^2-17^2+16^2-15^2+14^2-13^2+\cdots+2^2-1^2$
          \sol{}
          \begin{flalign*}
             & \ \ \ \ 18^2  - 17^2 + 16^2  - 15^2 + \cdots + 2^2  - 1^2                     \\
             & = (18^2  - 17^2) + (16^2  - 15^2) + \cdots + (2^2  - 1^2)                     \\
             & = ({(2\times9)}^2  - {(2\times9-1)}^2) + ({(2\times8)}^2  - {(2\times8-1)}^2) \\
             & \ \ \ \ + \cdots + ((2\times1)^2  - (2\times1-1)^2)                           \\
             & = \sum_{n=1}^9 \left[{(2n)}^2  - {(2n-1)}^2\right]                            \\
             & = \sum_{n=1}^9 (4n  - 1)                                                      \\
             & = 4\sum_{n=1}^9 n  - \sum_{n=1}^9 1                                           \\
             & = 4\times\frac{9\times10}{2}  - 9                                             \\
             & = 180  - 9                                                                    \\
             & = 171
          \end{flalign*}

    \item State the general formula of the series $20, -10, 5, -2\frac{1}{2}, \cdots$
          \sol{}
          \begin{flalign*}
            a_1 & = 20                       \\
            r   & = -\frac{1}{2}             \\
            a_n & = 20{(-\frac{1}{2})}^{n-1} \\
          \end{flalign*}

    \item Given three integers x-3, x+1, 4x-2 that are in GP. If the sum of this GP is S,
          common ratio is r, find the value of S+r. \sol{}
          \begin{flalign*}
            x+1              & = \pm\sqrt{(x-3)(4x-2)}              \\
            x^2 + 2x + 1     & = 4x^2  - 14x + 6                    \\
            3x^2  - 16^x + 5 & = 0                                  \\
            (3x-1)(x-5)      & = 0                                  \\
            x                & = 5\ or\ x = \frac{1}{3}             \\
            \\
            a_1              & = x-3 = 5-3 = 2                      \\
            a_2              & = x+1 = 5+1 = 6                      \\
            a_3              & = 4x-2 = 4(5)-2 = 18                 \\
            \\
            S                & = a_1 + a_2 + a_3                    \\
                             & = 2 + 6 + 18                         \\
                             & = 26                                 \\
            \\
            r                & = \frac{a_3}{a_2} = \frac{18}{6} = 3 \\
            \\
            \therefore\ S+r  & = 26 + 3                             \\
                             & = 29
          \end{flalign*}

    \item Find the geometric mean of $\frac{1}{3}$ and $\frac{1}{5}$ \sol{}
          \begin{flalign*}
            G & = \pm\sqrt{\frac{1}{3}\times\frac{1}{5}} \\
              & = \pm\sqrt{\frac{1}{15}}                 \\
              & = \pm\frac{1}{\sqrt{15}}                 \\
              & = \pm\frac{\sqrt{15}}{15}
          \end{flalign*}

    \item Find 5 numbers between $-\frac{1}{4}$ and $-\frac{1}{256}$ such that these 7
          numbers form a GP. \sol{}
          \begin{flalign*}
            a_1                           & = -\frac{1}{4}                                                             \\
            n                             & = 7                                                                        \\
            -\frac{1}{256}                & = -\frac{1}{4}r^6                                                          \\
            \frac{1}{64}                  & = r^6                                                                      \\
            \left(\pm\frac{1}{2}\right)^6 & = r^6                                                                      \\
            r                             & = \pm\frac{1}{2}                                                           \\
            \\
                                          & \text{When\ } r = \frac{1}{2},                                             \\
                                          & \text{These 5 numbers are}                                                 \\
                                          & \frac{1}{8},\ \frac{1}{16},\ \frac{1}{32},\ \frac{1}{64},\ \frac{1}{128}   \\
            \\
                                          & \text{When\ } r = -\frac{1}{2},                                            \\
                                          & \text{These 5 numbers are}                                                 \\
                                          & \frac{1}{8},\ -\frac{1}{16},\ \frac{1}{32},\ -\frac{1}{64},\ \frac{1}{128}
          \end{flalign*}

    \item Find the sum of the series $\sum_{n=5}^{15} n^2(3n+1)$ \sol{}
          \begin{flalign*}
            \sum_{n=5}^{15} n^2(3n+1) & = \sum_{n=5}^{15} n^3 + \sum_{n=5}^{15} 3n^2                                             \\
                                      & = 3\sum_{n=5}^{15} n^3 + \sum_{n=5}^{15} n^2                                             \\
                                      & = 3\left[\sum_{n=1}^{15} n^3  - \sum_{n=1}^{4} n^3\right]                                \\
                                      & \ \ \ \ + \left[\sum_{n=1}^{15} n^2  - \sum_{n=1}^{4} n^2\right]                         \\
                                      & = 3\left[\left(\frac{15\times16}{2}\right)^2  - \left(\frac{4\times5}{2}\right)^2\right] \\
                                      & \ \ \ \ + \left[\frac{15\times16\times31}{6}  - \frac{4\times5\times9}{6}\right]         \\
                                      & = 3\left[(15\times8)^2  - (2\times5)^2\right]                                            \\
                                      & \ \ \ \ + 1240  - 30                                                                     \\
                                      & = 3(14400  - 100) + 1210                                                                 \\
                                      & = 42900 + 1210                                                                           \\
                                      & = 44110
          \end{flalign*}

    \item Find the sum of the series $5^2 + 7^2 + 9^2 + \cdots + 25^2$ \sol{}
          \begin{flalign*}
             & \ \ \ \ \sum_{n=1}^{11} (2n+3)^2                                                        \\
             & = \sum_{n=1}^{11} 4n^2 + 12n + 9                                                        \\
             & = 4\sum_{n=1}^{11} n^2 + 12\sum_{n=1}^{11} n + 11                                       \\
             & = 4\left[\frac{11\times12\times23}{6}\right] + 12\left[\frac{11\times12}{2}\right] + 99 \\
             & = 2024 + 792 + 99                                                                       \\
             & = 2915
          \end{flalign*}

    \item Find the sum of the series
          $2\times3+3\times12+4\times27+\cdots+(n+1)\times3n^2$ \sol{}
          \begin{flalign*}
             & \ \ \ \ \sum_{n=1}^{n} (n+1)3n^2                                         \\
             & = \sum_{n=1}^{n} 3n^3 + \sum_{n=1}^{n} 3n^2                              \\
             & = 3\left[\sum_{n=1}^n n^3 + \sum_{n=1}^n n^2\right]                      \\
             & = 3\left[\left(\frac{n(n+1)}{2}\right)^2 + \frac{n(n+1)(2n+1)}{6}\right] \\
             & = 3\left[\frac{n^2(n+1)^2}{4} + \frac{n(n+1)(2n+1)}{6}\right]            \\
             & = 3\left[\frac{3n^2(n+1)^2 + 2n(n+1)(2n+1)}{12}\right]                   \\
             & = \frac{n(n+1)\left[3n^2+3n+4n+2\right]}{4}                              \\
             & = \frac{n(n+1)\left[3n^2+7n+2\right]}{4}                                 \\
             & = \frac{n(n+1)(n+2)(3n+1)}{4}                                            \\
          \end{flalign*}

  \end{enumerate}

  \chapter{System of Equations}

  \section{System of Equations with Two Variables}

  \subsection{Practice 1}

  Solve the following system of equations.

  \begin{enumerate}
    \item \[
            \begin{cases}
              2x  - 3y & = 11 \\
              xy       & = -5
            \end{cases}
          \]
          \sol{}
          \setcounter{equation}{0}
          \begin{empheq}[left=\empheqlbrace]{align}
            2x  -3y  & = 11 \\
            xy & = -5
          \end{empheq}
          \begin{flalign*}
            (2)                                   & \Rightarrow y = -\frac{5}{x}                  &  & (3) \\
            \text{Sub (3) into (1)}               & \Rightarrow 2x-\frac{15}{x}            = 11            \\
                                                  & 2x^2  - 15                              = 11x          \\
                                                  & 2x^2  - 11x  - 15                        = 0           \\
                                                  & (2x  - 5)(x-3)                          = 0            \\
                                                  & x = 3\ or\ x = \frac{5}{2}                             \\
            \text{Sub x = 3 into (2)}             & \Rightarrow y = -\frac{5}{3}                           \\
            \text{Sub x = $\frac{5}{2}$ into (2)} & \Rightarrow y = -\frac{5}{\frac{5}{2}}                 \\
                                                  & \Rightarrow y = -\frac{5}{5}                           \\
                                                  & \Rightarrow y = -1                                     \\
            \\
            \therefore \left\{\begin{array}{l}
                                x = 3 \\
                                y = -\frac{5}{3}
                              \end{array}\right.    & or\ \left\{\begin{array}{l}
                                                                   x = \frac{5}{2} \\
                                                                   y = -1
                                                                 \end{array}\right.
          \end{flalign*}

    \item \[
            \begin{cases}
              3x + y     & = 5 \\
              x^2  - 2xy & = 8
            \end{cases}
          \]
          \sol{}
          \setcounter{equation}{0}
          \begin{empheq}[left=\empheqlbrace]{align}
            3x + y  & = 5 \\
            x^2  -2xy & = 8
          \end{empheq}
          \begin{flalign*}{3}
            (1)                                    & \Rightarrow y = 5  - 3x                             &  & (3) \\
            \text{Sub (3) into (2)}                & \Rightarrow x^2  - 2x(5-3x)                     = 8          \\
                                                   & x^2  - 10x + 6x^2                               = 8          \\
                                                   & 7x^2  - 10x + 8                                 = 0          \\
                                                   & (7x + 4)(x  - 2)                                = 0          \\
                                                   & x = -\frac{4}{7} \ or\ x = 2                                 \\
            \text{Sub $x = -\frac{4}{7}$ into (1)} & \Rightarrow y = 5  - 3\left(-\frac{4}{7}\right)              \\
                                                   & \Rightarrow y = \frac{47}{7}                                 \\
            \text{Sub $x = 2$ into (1)}            & \Rightarrow y = -1                                           \\
            \\
            \therefore \left\{\begin{array}{l}
                                x = -\frac{4}{7} \\
                                y = \frac{47}{7}
                              \end{array}\right.     & or\ \left\{\begin{array}{l}
                                                                    x = 2 \\
                                                                    y = -1
                                                                  \end{array}\right.
          \end{flalign*}
  \end{enumerate}

  \subsection{Exercise 13.1}

  Solve the following system of equations.

  \begin{enumerate}
    \item \[
            \begin{cases}
              x  - y & = 1 \\
              xy     & = 6
            \end{cases}
          \]
          \sol{}
          \setcounter{equation}{0}
          \begin{empheq}[left=\empheqlbrace]{align}
            x  -y  & = 1 \\
            xy & = 6
          \end{empheq}
          \begin{flalign*}
            (1)                                & \Rightarrow y = x  - 1                    &  & (3) \\
            \text{Sub (3) into (2)}            & \Rightarrow x(x-1)                    = 6          \\
                                               & x^2-x                               = 6            \\
                                               & x^2-x-6                             = 0            \\
                                               & (x+2)(x-3)                          = 0            \\
                                               & x = -2 \ or\ x = 3                                 \\
            \text{Sub $x = -2$ into (1)}       & \Rightarrow y = -2  - 1                            \\
                                               & \Rightarrow y = -3                                 \\
            \text{Sub $x = 3$ into (1)}        & \Rightarrow y = 3  - 1                             \\
                                               & \Rightarrow y = 2                                  \\
            \\
            \therefore \left\{\begin{array}{l}
                                x = -2 \\
                                y = -3
                              \end{array}\right. & or\ \left\{\begin{array}{l}
                                                                x = 3 \\
                                                                y = 2
                                                              \end{array}\right.
          \end{flalign*}

    \item \[
            \begin{cases}
              3x-y=4 \\
              xy=4
            \end{cases}
          \]
          \sol{}
          \setcounter{equation}{0}
          \begin{empheq}[left=\empheqlbrace]{align}
            3x-y  & = 4 \\
            xy & = 4
          \end{empheq}
          \begin{flalign*}
            (1)                                    & \Rightarrow y = 3x  - 4                         &  & (3) \\
            \text{Sub (3) into (2)}                & \Rightarrow x(3x-4)                    = 4               \\
                                                   & 3x^2-4x                               = 4                \\
                                                   & 3x^2-4x-4                             = 0                \\
                                                   & (3x+2)(x-2)                          = 0                 \\
                                                   & x = -\frac{2}{3} \ or\ x = 2                             \\
            \text{Sub $x = -\frac{2}{3}$ into (1)} & \Rightarrow y = 3\left(-\frac{2}{3}\right)  - 4          \\
                                                   & \Rightarrow y = -6                                       \\
            \text{Sub $x = 2$ into (1)}            & \Rightarrow y = 3(2)  - 4                                \\
                                                   & \Rightarrow y = 2                                        \\
            \\
            \therefore \left\{\begin{array}{l}
                                x = -\frac{2}{3} \\
                                y = -6
                              \end{array}\right.     & or\ \left\{\begin{array}{l}
                                                                    x = 2 \\
                                                                    y = 2
                                                                  \end{array}\right.
          \end{flalign*}

    \item \[
            \begin{cases}
              3x+4y=-39 \\
              xy=30
            \end{cases}
          \]
          \sol{}
          \setcounter{equation}{0}
          \begin{empheq}[left=\empheqlbrace]{align}
            3x+4y  & = -39 \\
            xy & = 30
          \end{empheq}
          \begin{flalign*}
            (2)                                & \Rightarrow y = \frac{30}{x}                   &  & (3) \\
            \text{Sub (3) into (1)}            & \Rightarrow 3x+4\frac{30}{x}             = -39          \\
                                               & 3x^2+120                             = -39x             \\
                                               & 3x^2+39x+120                          = 0               \\
                                               & x^2 + 13x + 40                        = 0               \\
                                               & (x+5)(x+8)                            = 0               \\
                                               & x = -5 \ or\ x = -8                                     \\
            \text{Sub $x = -5$ into (1)}       & \Rightarrow y = \frac{30}{-5}  - 39                     \\
                                               & \Rightarrow y = -6                                      \\
            \text{Sub $x = -8$ into (1)}       & \Rightarrow y = \frac{30}{-8}  - 39                     \\
                                               & \Rightarrow y = -\frac{15}{4}                           \\
            \\
            \therefore \left\{\begin{array}{l}
                                x = -5 \\
                                y = -6
                              \end{array}\right. & or\ \left\{\begin{array}{l}
                                                                x = -8 \\
                                                                y = -\frac{15}{4}
                                                              \end{array}\right.
          \end{flalign*}

    \item \[
            \begin{cases}
              y = 2x+3 \\
              y = x^2-2x+1
            \end{cases}
          \]
          \sol{}
          \setcounter{equation}{0}
          \begin{empheq}[left=\empheqlbrace]{align}
            y  & = 2x+3 \\
            y & = x^2
          \end{empheq}
          \begin{flalign*}
            (1) = (2)                          & \Rightarrow 2x+3 = x^2          \\
                                               & \Rightarrow x^2  - 2x  - 3 = 0  \\
                                               & \Rightarrow (x+1)(x-3) = 0      \\
                                               & \Rightarrow x = -1 \ or\ x = 3  \\
            \text{Sub $x = -1$ into (1)}       & \Rightarrow y = 2(-1)+3         \\
                                               & \Rightarrow y = 1               \\
            \text{Sub $x = 3$ into (1)}        & \Rightarrow y = 2(3)+3          \\
                                               & \Rightarrow y = 9               \\
            \\
            \therefore \left\{\begin{array}{l}
                                x = -1 \\
                                y = 1
                              \end{array}\right. & or\ \left\{\begin{array}{l}
                                                                x = 3 \\
                                                                y = 9
                                                              \end{array}\right.
          \end{flalign*}

    \item \[
            \begin{cases}
              x-y = 1 \\
              x^2 + y^2 = 25
            \end{cases}
          \]
          \sol{}
          \setcounter{equation}{0}
          \begin{empheq}[left=\empheqlbrace]{align}
            x-y  & = 1 \\
            x^2 + y^2 & = 25
          \end{empheq}
          \begin{flalign*}
            (1)                                & \Rightarrow x = y+1                 &  & (3) \\
            \text{Sub (3) into (2)}            & \Rightarrow {(y+1)}^2 + y^2 = 25             \\
                                               & \Rightarrow y^2 + 2y + 1 + y^2 = 25          \\
                                               & \Rightarrow 2y^2 + 2y = 24                   \\
                                               & \Rightarrow y^2 + y = 12                     \\
                                               & \Rightarrow y^2 + y  - 12 = 0                \\
                                               & \Rightarrow (y+4)(y-3) = 0                   \\
                                               & \Rightarrow y = -4 \ or\ y = 3               \\
            \text{Sub $y = -4$ into (1)}       & \Rightarrow x = -4+1                         \\
                                               & \Rightarrow x = -3                           \\
            \text{Sub $y = 3$ into (1)}        & \Rightarrow x = 3+1                          \\
                                               & \Rightarrow x = 4                            \\
            \\
            \therefore \left\{\begin{array}{l}
                                x = -3 \\
                                y = -4
                              \end{array}\right. & or\ \left\{\begin{array}{l}
                                                                x = 4 \\
                                                                y = 3
                                                              \end{array}\right.
          \end{flalign*}

    \item \[
            \begin{cases}
              5x-y = 3 \\
              y^2  - 6x^2 = 25
            \end{cases}
          \]
          \sol{}
          \setcounter{equation}{0}
          \begin{empheq}[left=\empheqlbrace]{align}
            5x-y  & = 3 \\
            y^2  -6x^2 & = 25
          \end{empheq}
          \begin{flalign*}
            (1)                                     & \Rightarrow y = 5x-3                      & (3) \\
            \text{Sub (3) into (2)}                 & \Rightarrow {(5x-3)}^2  - 6x^2 = 25             \\
                                                    & \Rightarrow 25x^2  - 30x + 9                    \\
                                                    & \ \ \ \ \   - 6x^2 = 25                         \\
                                                    & \Rightarrow 19x^2  - 30x + 16 = 0               \\
                                                    & \Rightarrow (19x+8)(x-2) = 0                    \\
                                                    & \Rightarrow x = -\frac{8}{19} \ or\ x = 2       \\
            \text{Sub $x = -\frac{8}{19}$ into (1)} & \Rightarrow y = 5(-\frac{8}{19})-3              \\
                                                    & \Rightarrow y = -\frac{97}{19}                  \\
            \text{Sub $x = 2$ into (1)}             & \Rightarrow y = 7                               \\
            \\
            \therefore \left\{\begin{array}{l}
                                x = -\frac{8}{19} \\
                                y = -\frac{97}{19}
                              \end{array}\right.      & or\ \left\{\begin{array}{l}
                                                                     x = 2 \\
                                                                     y = 7
                                                                   \end{array}\right.
          \end{flalign*}

    \item \[
            \begin{cases}
              x+y=3 \\
              (x+2)(y+3) = 12
            \end{cases}
          \]
          \sol{}
          \setcounter{equation}{0}
          \begin{empheq}[left=\empheqlbrace]{align}
            x+y  & = 3 \\
            (x+2) (y+3) & = 12
          \end{empheq}
          \begin{flalign*}
            (1)                                & \Rightarrow x = 3-y                 &  & (3) \\
            \text{Sub (3) into (2)}            & \Rightarrow (3-y+2)(y+3) = 12                \\
                                               & \Rightarrow (5-y)(y+3) = 12                  \\
                                               & \Rightarrow 5y + 15  - y^2 -3y = 12          \\
                                               & \Rightarrow 2y  - y^2 = -3                   \\
                                               & \Rightarrow y^2  - 2y -3 = 0                 \\
                                               & \Rightarrow (y+1)(y-3) = 0                   \\
                                               & \Rightarrow y = -1 \ or\ y = 3               \\
            \text{Sub $y = -1$ into (1)}       & \Rightarrow x = 4                            \\
            \text{Sub $y = 3$ into (1)}        & \Rightarrow x = 0                            \\
            \\
            \therefore \left\{\begin{array}{l}
                                x = 4 \\
                                y = -1
                              \end{array}\right. & or\ \left\{\begin{array}{l}
                                                                x = 0 \\
                                                                y = 3
                                                              \end{array}\right.
          \end{flalign*}

    \item \[
            \begin{cases}
              5x-6y=-1 \\
              25x^2+36y^2=61
            \end{cases}
          \]
          \sol{}
          \setcounter{equation}{0}
          \begin{empheq}[left=\empheqlbrace]{align}
            5x-6y  & = -1 \\
            25x^2+36y^2 & = 61
          \end{empheq}
          \begin{flalign*}
            (1)                                    & \Rightarrow y = \frac{5x+1}{6}                        & (3) \\
            \text{Sub (3) into (2)}                & \Rightarrow 25x^2 + 36{\left(\frac{5x+1}{6}\right)}^2       \\
                                                   & \ \ \ \ \  = 61                                             \\
                                                   & \Rightarrow 25x^2 + 36{\left(\frac{5x+1}{6}\right)}^2       \\
                                                   & \ \ \ \ \ + 36 = 61                                         \\
                                                   & \Rightarrow 25x^2 + 25x^2+10x                               \\
                                                   & \ \ \ \ \ +1 = 61                                           \\
                                                   & \Rightarrow 50x^2 + 10x = 60                                \\
                                                   & \Rightarrow 5x^2 + x  - 6 = 0                               \\
                                                   & \Rightarrow (5x+6)(x-1) = 0                                 \\
                                                   & \Rightarrow x = -\frac{6}{5} \ or\ x = 1                    \\
            \text{Sub $x = -\frac{6}{5}$ into (1)} & \Rightarrow y = \frac{5(-\frac{6}{5})+1}{6}                 \\
                                                   & \Rightarrow y = -\frac{5}{6}                                \\
            \text{Sub $x = 1$ into (1)}            & \Rightarrow y = \frac{5(1)+1}{6}                            \\
                                                   & \Rightarrow y = \frac{6}{6}                                 \\
                                                   & \Rightarrow y = 1                                           \\
            \\
            \therefore \left\{\begin{array}{l}
                                x = -\frac{6}{5} \\
                                y = -\frac{5}{6}
                              \end{array}\right.     & or\ \left\{\begin{array}{l}
                                                                    x = 1 \\
                                                                    y = 1
                                                                  \end{array}\right.
          \end{flalign*}

    \item \[
            \begin{cases}
              x+4y = 5 \\
              2x^2+21xy+27y^2 = 0
            \end{cases}
          \]
          \sol{}
          \setcounter{equation}{0}
          \begin{empheq}[left=\empheqlbrace]{align}
            x+4y  & = 5 \\
            2x^2+21xy + 27y^2 &= 0
          \end{empheq}
          \begin{flalign*}
            (1)                                & \Rightarrow x = 5-4y                                    & (3) \\
            \text{Sub (3) into (2)}            & \Rightarrow 2\left(5-4y\right)^2 + 21\left(5-4y\right)y       \\
                                               & \ \ \ \ \ + 27y^2 = 0                                         \\
                                               & \Rightarrow 2\left(25-40y+16y^2\right)                        \\
                                               & \ \ \ \ \ + 105y-84y^2 + 27y^2 = 0                            \\
                                               & \Rightarrow 50  - 80y + 32y^2 + 105y                          \\
                                               & \ \ \ \ \   - 57y^2 = 0                                       \\
                                               & \Rightarrow 25y^2  - 25y  - 50 = 0                            \\
                                               & \Rightarrow y^2  - y  - 2                                     \\
                                               & \Rightarrow (y+1)(y-2) = 0                                    \\
                                               & \Rightarrow y = -1 \ or\ y = 2                                \\
            \text{Sub $y = -1$ into (1)}       & \Rightarrow x = 5-4(-1) = 9                                   \\
            \text{Sub $y = 2$ into (1)}        & \Rightarrow x = 5-4(2) = -3                                   \\
            \\
            \therefore \left\{\begin{array}{l}
                                x = 9 \\
                                y = -1
                              \end{array}\right. & or\ \left\{\begin{array}{l}
                                                                x = -3 \\
                                                                y = 2
                                                              \end{array}\right.
          \end{flalign*}

    \item \[
            \begin{cases}
              \frac{x}{3}  - \frac{y}{10} = \frac{5}{6} \\
              x(y-2) = 2y+3
            \end{cases}
          \]
          \sol{}
          \setcounter{equation}{0}
          \begin{empheq}[left=\empheqlbrace]{align}
            \frac{x}{3}-\frac{y}{10} & = \frac{5}{6} \\
            x (y-2) & = 2y+3
          \end{empheq}
          \begin{flalign*}
            (1)                                    & \Rightarrow 10x  - 3y = 25                             & (3) \\
            (2)                                    & \Rightarrow x = \frac{2y+3}{y-2}                       & (4) \\
            \text{Sub (4) int {3}}                 & \Rightarrow 10\left(\frac{2y+3}{y-2}\right)  - 3y = 25       \\
                                                   & \Rightarrow 10(2y+3)  - 3y(y-2)                              \\
                                                   & \ \ \ \ \ = 25(y-2)                                          \\
                                                   & \Rightarrow 20y + 30  - 3y^2 + 6y                            \\
                                                   & \ \ \ \ \ = 25y  - 50                                        \\
                                                   & \Rightarrow 3y^2  - y  - 80 = 0                              \\
                                                   & \Rightarrow (y+5)(3y-16) = 0                                 \\
                                                   & \Rightarrow y = -5 \ or\ y = \frac{16}{3}                    \\
            \text{Sub $y = -5$ into (1)}           & \Rightarrow 10x  - 3(-5) = 25                                \\
                                                   & \Rightarrow 10x + 15 = 25                                    \\
                                                   & \Rightarrow 10x = 10                                         \\
                                                   & \Rightarrow x = 1                                            \\
            \text{Sub $y = \frac{16}{3}$ into (1)} & \Rightarrow 10x  - 3\left(\frac{16}{3}\right) = 25           \\
                                                   & \Rightarrow 10x = 41                                         \\
                                                   & \Rightarrow x = \frac{41}{10}                                \\
            \\
            \therefore \left\{\begin{array}{l}
                                x = 1 \\
                                y = -5
                              \end{array}\right.     & or\ \left\{\begin{array}{l}
                                                                    x = \frac{41}{10} \\
                                                                    y = \frac{16}{3}
                                                                  \end{array}\right.
          \end{flalign*}
  \end{enumerate}

  \section{System of Equations with Three Variables}

  \subsection{Practice 2}

  Solve the system of equation \[
    \begin{cases}
      x+2y-z = -5 \\
      2x-y+z = 6  \\
      x-y-3z = -3
    \end{cases}
  \] \sol{}
  \setcounter{equation}{0}
  \begin{empheq}[left=\empheqlbrace]{align}
    x+2y-z & = -5 \\
    2x-y+z & = 6 \\
    x-y-3z& = -3
  \end{empheq}
  \begin{flalign*}
    (1) \times 3                             & \Rightarrow 3x + 6y  - 3z = -15 & (4)  \\
    (2) \times 3                             & \Rightarrow 6x  - 3y + 3z = 18  & (5)  \\
    (3) + (5)                                & \Rightarrow 7x  - 4y = 15       & (6)  \\
    (4) + (5)                                & \Rightarrow 9x + 3y = 3         & (7)  \\
    (6) \times 3                             & \Rightarrow 21x  - 12y = 45     & (8)  \\
    (7) \times 4                             & \Rightarrow 36x + 12y = 12      & (9)  \\
    (8) + (9)                                & \Rightarrow 57x = 57            & (10) \\
                                             & \Rightarrow x = 1                      \\
    \text{Sub $x = 1$ into (7)}              & \Rightarrow -4y = 8                    \\
                                             & \Rightarrow y = -2                     \\
    \text{Sub $y = -2$ and $x = 1$ into (1)} & \Rightarrow -z = -2                    \\
                                             & \Rightarrow z = 2
    \\
    \therefore\ x = 1, y = -2, z = 2
  \end{flalign*}

  \subsection{Exercise 13.2}

  Solve the following system of equations.

  \begin{enumerate}

    \item \[
            \begin{cases}
              x + y  - z = 1   \\
              2x  - 3y + z = 0 \\
              2x + y + 2z = 5
            \end{cases}
          \]
          \sol{}
          \setcounter{equation}{0}
          \begin{empheq}[left=\empheqlbrace]{align}
            x + y  -z & = 1 \\
            2x  -3y + z & = 0 \\
            2x + y + 2z & = 5
          \end{empheq}
          \begin{flalign*}
            (1) \times 2                            & \Rightarrow 2x + 2y  - 2z = 2 & (4) \\
            (4)  - (3)                              & \Rightarrow y  - 4z = -3      & (5) \\
            (3)  - (2)                              & \Rightarrow 4y + z = 5        & (6) \\
            (5) \times 4                            & \Rightarrow 4y  - 16z = -12   & (7) \\
            (6)  - (7)                              & \Rightarrow 17z = 17                \\
                                                    & \Rightarrow z = 1                   \\
            \text{Sub $z = 1$ into (5)}             & \Rightarrow y = 1                   \\
            \text{Sub $y = 1$ and $z = 1$ into (1)} & \Rightarrow z = 1                   \\
            \\
            \therefore\ x = 1, y = 1, z = 1
          \end{flalign*}

    \item \[
            \begin{cases}
              x  - 2y = 5      \\
              2x + y  - 3z = 8 \\
              x + 4y  - z = 0
            \end{cases}
          \]
          \sol{}
          \setcounter{equation}{0}
          \begin{empheq}[left=\empheqlbrace]{align}
            x -2y & = 5 \\
            2x + y -3z & = 8 \\
            x + 4y -z & = 0
          \end{empheq}
          \begin{flalign*}
            (3) \times 3                 & \Rightarrow 3x + 12y  - 3z = 0 & (4) \\
            (4)  - (2)                   & \Rightarrow x + 11y = -8       & (5) \\
            (5)  - (1)                   & \Rightarrow 13y = -13                \\
                                         & \Rightarrow y = -1                   \\
            \text{Sub $y = -1$ into (1)} & \Rightarrow x + 2 = 5                \\
                                         & \Rightarrow x = 3                    \\
            \text{Sub $x = 3$}           &                                      \\
            \text{and $y = -1$ into (2)} & \Rightarrow -3z = 3                  \\
                                         & \Rightarrow z = -1
            \\
            \therefore\ x = 3, y = -1, z = -1
          \end{flalign*}

    \item \[
            \begin{cases}
              x + y = z  - 5 \\
              y + z = x  - 3 \\
              z + x = y + 1
            \end{cases}
          \]
          \sol{}
          \setcounter{equation}{0}
          \begin{empheq}[left=\empheqlbrace]{align}
            x + y & = z- 5 \\
            y + z & = x- 3 \\
            z + x & = y + 1
          \end{empheq}
          \begin{flalign*}
            (1)                          & \Rightarrow x + y  - z = -5 & (4) \\
            (2)                          & \Rightarrow -x + y + z = -3 & (5) \\
            (3)                          & \Rightarrow x  - y + z = 1  & (6) \\
            (4) + (5)                    & \Rightarrow 2y = -8               \\
                                         & \Rightarrow y = -4                \\
            (5) + (6)                    & \Rightarrow 2z = -2               \\
                                         & \Rightarrow z = -1                \\
            \text{Sub $y = -4$}          &                                   \\
            \text{and $z = -1$ into (2)} & \Rightarrow x  - 3 = -5           \\
                                         & \Rightarrow x = -2                \\
            \\
            \therefore\ x = -2, y = -4, z = -1
          \end{flalign*}

    \item \[
            \begin{cases}
              x + 4y + 2z = 4  \\
              2x  - 2y + z = 4 \\
              x  - 2y + 3z = 3 \\
            \end{cases}
          \]
          \sol{}
          \setcounter{equation}{0}
          \begin{empheq}[left=\empheqlbrace]{align}
            x + 4y + 2z & = 4 \\
            2x  -2y + z &= 4 \\
            x  -2y + 3z & = 3
          \end{empheq}
          \begin{flalign*}
            (1) \times 2                                                & \Rightarrow 2x + 8y + 4z = 8  & (4) \\
            (3) \times 2                                                & \Rightarrow 2x  - 4y + 6z = 6 & (5) \\
            (4)  - (2)                                                  & \Rightarrow 10y + 3z = 4      & (6) \\
            (5)  - (4)                                                  & \Rightarrow -12y + 2z = -2    & (7) \\
            (6) \times 2                                                & \Rightarrow 20y + 6z = 8      & (8) \\
            (7) \times 3                                                & \Rightarrow -36y + 6z = -6    & (9) \\
            (8)  - (9)                                                  & \Rightarrow 56y = 14                \\
                                                                        & \Rightarrow y = \frac{1}{4}         \\
            \text{Sub $y = \frac{1}{4}$ into (6)}                       & \Rightarrow 6z = 3                  \\
                                                                        & \Rightarrow z = \frac{1}{2}         \\
            \text{Sub $y = \frac{1}{4}$ and $z = \frac{1}{2}$ into (1)} & \Rightarrow x + 1 + 1 = 4           \\
                                                                        & \Rightarrow x = 2                   \\
            \\
            \therefore\ x = 2, y = \frac{1}{4}, z = \frac{1}{2}
          \end{flalign*}

    \item \[
            \begin{cases}
              x - y - z = 0 \\
              3x + 2y = 13  \\
              y - 3z = -1
            \end{cases}
          \]
          \sol{}
          \setcounter{equation}{0}
          \begin{empheq}[left=\empheqlbrace]{align}
            x -y -z & = 0 \\
            3x + 2y & = 13 \\
            y -3z & = -1
          \end{empheq}
          \begin{flalign*}
            (3)                         & \Rightarrow y = 3z - 1           & (4) \\
            \text{Sub (4) into (1)}     & \Rightarrow x - (3z - 1) - z = 0       \\
                                        & \Rightarrow x - 4z = -1          & (5) \\
            \text{Sub (4) into (2)}     & \Rightarrow 3x + 2(3z - 1) = 13        \\
                                        & \Rightarrow 3x + 6z = 15         & (6) \\
            (5) \times 3                & \Rightarrow 3x - 12z = -3        & (7) \\
            (6) - (7)                   & \Rightarrow 18z = 18                   \\
                                        & \Rightarrow z = 1                      \\
            \text{Sub $z = 1$ into (4)} & \Rightarrow y = 2                      \\
            \text{Sub $z = 1$ into (5)} & \Rightarrow x - 4 = -1                 \\
                                        & \Rightarrow x = 3                      \\
            \\
            \therefore\ x = 3, y = 2, z = 1
          \end{flalign*}

    \item \[
            \begin{cases}
              2x + 2y - z = -1 \\
              x + 3y + z = -8  \\
              3x - 2y + 3z = 9
            \end{cases}
          \]
          \sol{}
          \setcounter{equation}{0}
          \begin{empheq}[left=\empheqlbrace]{align}
            2x + 2y -z & = -1 \\
            x + 3y + z & = -8 \\
            3x -2y + 3z & = 9
          \end{empheq}
          \begin{flalign*}
            (1) \times 3                 & \Rightarrow 6x + 6y - 3z = -3  & (4) \\
            (2) \times 3                 & \Rightarrow 3x + 9y + 3z = -24 & (5) \\
            (3) + (4)                    & \Rightarrow 9x + 4y = 6        & (6) \\
            (4) + (5)                    & \Rightarrow 9x + 15y = -27     & (7) \\
            (7) - (6)                    & \Rightarrow 11y = -33                \\
                                         & \Rightarrow y = -3                   \\
            \text{Sub $y = -3$ into (6)} & \Rightarrow 9x = 18                  \\
                                         & \Rightarrow x = 2                    \\
            \text{Sub $x = 2$}           &                                      \\
            \text{and $y = -3$ into (2)} & \Rightarrow -7 + z = -8              \\
                                         & \Rightarrow z = -1                   \\
            \\
            \therefore\ x = 2, y = -3, z = -1
          \end{flalign*}

    \item \[
            \begin{cases}
              \frac{3}{x} + \frac{1}{y} + \frac{4}{z} = 0 \\
              \frac{1}{x} + \frac{4}{y} - \frac{2}{z} = 4 \\
              \frac{2}{x} - \frac{3}{y} - \frac{1}{z} = -11
            \end{cases}
          \]
          \sol{}
          \setcounter{equation}{0}
          \begin{empheq}[left=\empheqlbrace]{align}
            \frac{3}{x} + \frac{1}{y} + \frac{4}{z} & = 0 \\
            \frac{1}{x} +\frac{4}{y} -\frac{2}{z} & = 4 \\
            \frac{2}{x} -\frac{3}{y} -\frac{1}{z} & = -11
          \end{empheq}
          \begin{flalign*}
                                         & \text{Let $u = \frac{1}{x}$, $v = \frac{1}{y}$, $w = \frac{1}{z}$}        \\
            (1)                          & \Rightarrow 3u + v + 4w = 0                                        & (4)  \\
            (2)                          & \Rightarrow u + 4v - 2w = 4                                        & (5)  \\
            (3)                          & \Rightarrow 2u - 3v - w = -11                                      & (6)  \\
            (5) \times 2                 & \Rightarrow 2u + 8v - 4w = 8                                       & (7)  \\
            (6) \times 4                 & \Rightarrow 8u - 12v - 4w = -44                                    & (8)  \\
            (4) + (7)                    & \Rightarrow 5u + 9v = 8                                            & (9)  \\
            (4) + (8)                    & \Rightarrow 11u - 11v = -44                                               \\
                                         & \Rightarrow u - v = -4                                             & (10) \\
            (10) \times 5                & \Rightarrow 5u - 5v = -20                                          & (11) \\
            (9) - (11)                   & \Rightarrow 14v = 28                                               & (12) \\
                                         & \Rightarrow v = 2                                                         \\
            \text{Sub $v = 2$ into (10)} & \Rightarrow u = -2                                                        \\
            \text{Sub $u = -2$}          &                                                                           \\
            \text{and $v = 2$ into (4)}  & \Rightarrow -4 + 4w = 0                                                   \\
                                         & \Rightarrow w = 1                                                         \\
            \\
            \therefore\ u = -2, v = 2, w = 1                                                                         \\
            \therefore\ x = -\frac{1}{2}, y = \frac{1}{2}, z = 1
          \end{flalign*}
  \end{enumerate}

\end{multicols}
\end{document}