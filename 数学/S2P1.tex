\documentclass{report}
\usepackage[a4paper, total={7in, 9in}]{geometry}
\usepackage[fleqn]{amsmath}
\usepackage{amssymb}
\usepackage{gensymb}
\usepackage[fleqn]{cases}
\usepackage{multicol}
\usepackage{color}
\setlength{\columnseprule}{1pt}
\setlength{\columnsep}{24pt}

\title{Senior 2 Math Part I}
\author{Melvin Chia}


\begin{document}
	\maketitle

  \begin{multicols}{2}

	\chapter{Sequence and Series}


	\section{Sequence and Series}


	\subsection{Practice 1}

	\begin{enumerate}
		\item Find the first 5 terms of the sequence $a_{n} = \frac{2^{n}}{n+1}$.

			\textbf{Sol.} $a_{1} = \frac{2}{2}= 1, a_{2} = \frac{4}{3}, a_{3} = \frac{8}{4}
			, a_{4} = \frac{16}{5}, a_{5} = \frac{32}{6}$

		\item Write the general term of the sequence 1, 8, 27, 64, \ldots

			\textbf{Sol.} $a_{n} = n^{3}$
	\end{enumerate}

	\subsection{Practice 2}


	\begin{enumerate}
		\item Express the series $\sum_{n=1}^{10}{n^2+1}$ in the form of numbers.

			\begin{flalign*}
				\textbf{Sol.} & \sum_{n=1}^{10}{n^2+1}                                       &  \\
                      & = (1^{2}+1) + (2^{2}+1) + (3^{2}+1) + (4^{2}+1) \\ & + (5^{2}+1) + (6^{2}+1) + (7^{2}+1) \\ & + (8^{2}+1) + (9^{2}+1) + (10^{2}+1) &  \\
                      & = 2 + 5 + 10 + 17 + 26 + 37 + 50 + 65 \\ & + 82 + 101
			\end{flalign*}

		\item Write the first term, last term and the number of terms of the series
			$\sum_{n=1}^{10}{(3^n-2^n)}$.

			\begin{flalign*}
				\textbf{Sol.} & First\ term = (3^{1}-2^{1}) = 1      &  \\
				              & Last\ term = (3^{10}-2^{10}) = 59049 &  \\
				              & Number\ of\ terms = 10
			\end{flalign*}

		\item Express the series $2\times5 + 3\times7 + 4\times9 + \ldots + 15\times31$
			in the form of $\sum$.

			\begin{flalign*}
				\noindent \textbf{Sol.}                                       \\
				a_{1}                                                        & = 2\times5 = 10        \\
				a_{2}                                                        & = 3\times7 = 21        \\
				a_{3}                                                        & = 4\times9 = 36        \\
				a_{4}                                                        & = 5\times11 = 55       \\
				                                                             & \vdots                 \\
				a_{15}                                                       & = 15\times31 = 465     \\
        \therefore 2&\times5 + 3\times7 + 4\times9 + \ldots + 15\times31 \\ & = \sum_{n=1}^{15}a_{n}
			\end{flalign*}
	\end{enumerate}

	\subsection{Exercise 12.1}


	\begin{enumerate}

	\item Find the general term of the following sequences.

	\begin{enumerate}
		\item 5, 8, 11, 14, \ldots

			\textbf{Sol.} $a_{n} = 3n+2$

		\item 2, 4, 8, 16, \ldots

			\textbf{Sol.} $a_{n} = 2^{n}$

		\item $\frac{2}{1}, \frac{3}{2}, \frac{4}{3}, \frac{5}{4}, \ldots$

			\textbf{Sol.} $a_{n} = \frac{n+1}{n}$

		\item $\frac{2}{5}, \frac{4}{7}, \frac{6}{9}, \frac{8}{11}, \ldots$

			\textbf{Sol.} $a_{n} = \frac{2n}{2n+1}$
	\end{enumerate}

	\item Find the first 5 terms of the following sequences.

	\begin{enumerate}
		\item $a_{n} = 2n+3$

			\textbf{Sol.}
			$a_{1} = 2\times1+3 = 5, a_{2} = 2\times2+3 = 7, a_{3} = 2\times3+3 = 9, a_{4}
			= 2\times4+3 = 11, a_{5} = 2\times5+3 = 13$

		\item $a_{n} = n(n-2)$

			\textbf{Sol.}
			$a_{1} = 1\times(-1) = -1, a_{2} = 2\times0 = 0, a_{3} = 3\times1 = 3, a_{4}
			= 4\times2 = 8, a_{5} = 5\times3 = 15$

		\item $a_{n} = \frac{n}{2n+1}$

			\textbf{Sol.}
			$a_{1} = \frac{1}{2\times1+1}= \frac{1}{3}, a_{2} = \frac{2}{2\times2+1}= \frac{2}{5}
			, a_{3} = \frac{3}{2\times3+1}= \frac{3}{7}, a_{4} = \frac{4}{2\times4+1}=
			\frac{4}{9}, a_{5} = \frac{5}{2\times5+1}= \frac{5}{11}$

		\item $a_{n} = (-3)^{n}$

			\textbf{Sol.}
			$a_{1} = (-3)^{1} = -3, a_{2} = (-3)^{2} = 9, a_{3} = (-3)^{3} = -27, a_{4}
			= (-3)^{4} = 81, a_{5} = (-3)^{5} = -243$
	\end{enumerate}

	\item Express the following series in the form of numbers.

	\begin{enumerate}
		\item $\sum_{n=1}^{5}{n(n+3)}$

			\begin{flalign*}
				\textbf{Sol.} & \sum_{n=1}^{5}{n(n+3)}                                           &  \\
                      & = (1\times4) + (2\times5) + (3\times6) + (4\times7) \\ & + (5\times8) &  \\
				              & = 4 + 10 + 18 + 28 + 40                                          &  \\
			\end{flalign*}

		\item $\sum_{n=2}^{6}{\frac{1}{3^{n}}}$

			\begin{flalign*}
				\textbf{Sol.} & \sum_{n=2}^{6}{\frac{1}{3^{n}}}                                                       &  \\
				              & = \frac{1}{3^{2}}+ \frac{1}{3^{3}}+ \frac{1}{3^{4}}+ \frac{1}{3^{5}}+ \frac{1}{3^{6}} &  \\
				              & = \frac{1}{9}+ \frac{1}{27}+ \frac{1}{81}+ \frac{1}{243}+ \frac{1}{729}
			\end{flalign*}

		\item $\sum_{n=1}^{6}{\frac{1}{n(2n+1)}}$

			\begin{flalign*}
				\textbf{Sol.} & \sum_{n=1}^{6}{\frac{1}{n(2n+1)}}                                                  &  \\
                      & = \frac{1}{1(2\times1+1)}+ \frac{1}{2(2\times2+1)} \\ & + \frac{1}{3(2\times3+1)} +   
                      \frac{1}{4(2\times4+1)} \\ &+ \frac{1}{5(2\times5+1)}+ \frac{1}{6(2\times6+1)}        &  \\
				              & = \frac{1}{3}+ \frac{1}{10}+ \frac{1}{21}+ \frac{1}{36}+ \frac{1}{55}+ \frac{1}{78}
			\end{flalign*}

		\item $\sum_{n=2}^{5}{\frac{1}{n^{2}+2}}$

			\begin{flalign*}
				\textbf{Sol.} & \sum_{n=2}^{5}{\frac{1}{n^{2}+2}}                              &  \\
				              & = \frac{1}{4+2}+ \frac{1}{9+2}+ \frac{1}{16+2}+ \frac{1}{25+2} &  \\
				              & = \frac{1}{6}+ \frac{1}{11}+ \frac{1}{18}+ \frac{1}{27}
			\end{flalign*}

			\end {enumerate}

		\item Find the first term, last term and the number of terms of the
			following series.

			\begin{enumerate}
				\item $\sum_{n=3}^{10}{2^2}$

					\textbf{Sol.} $a_{3} = 2^{2} = 4, a_{10}= 2^{2} = 4, n = 10-3+1 = 8$

				\item $\sum_{n=1}^{8}{\frac{n+2}{n}}$

					\textbf{Sol.}
					$a_{1} = \frac{1+2}{1}= \frac{3}{1}= 3, a_{8}= \frac{8+2}{8}= \frac{10}{8}
					= \frac{5}{4}, n = 8-1+1 = 8$

				\item $\sum_{n=1}^{10}{3n^2-n}$

					\textbf{Sol.}
					$a_{1} = 3\times1^{2}-1 = 2, a_{10}= 3\times10^{2}-10 = 290, n = 10-1+1
					= 10$

				\item $\sum_{n=9}^{14}{n^2(n-7)}$

					\textbf{Sol.}
					$a_{9} = 9^{2}(9-7) = 9^{2}\times2 = 162, a_{14}= 14^{2}(14-7) = 14^{2}
					\times7 = 2744, n = 14-9+1 = 6$
			\end{enumerate}

		\item Express the following series in the form of $\sum$.

			\begin{enumerate}
				\item $1+\frac{1}{2}+\frac{1}{3}+\ldots+\frac{1}{30}$
				\\~\\\noindent \textbf{Sol.}
					\begin{flalign*}
						a_{1} & = 1                                                                     \\
						a_{2}              & = \frac{1}{2}                                                           \\
						a_{3}              & = \frac{1}{3}                                                           \\
						\vdots              \\
						a_{30}             & = \frac{1}{30}                                                          \\
						\therefore 1       & +\frac{1}{2}+\frac{1}{3}+\ldots+\frac{1}{30}= \sum_{n=1}^{30}{\frac{1}{n}}
					\end{flalign*}

				\item $1^{3} + 2^{3} + 3^{3} + \ldots + 50^{3}$
				\\~\\\noindent \textbf{Sol.}
					\begin{flalign*}
						a_{1} & = 1^{3}                                               \\
						a_{2}              & = 2^{3}                                               \\
						a_{3}              & = 3^{3}                                               \\
						\vdots              \\
						a_{50}             & = 50^{3}                                              \\
						\therefore 1^{3}   & + 2^{3} + 3^{3} + \ldots + 50^{3} = \sum_{n=1}^{50}{n^3}
					\end{flalign*}

				\item $1 - \frac{1}{2}+ \frac{1}{4}- \frac{1}{8}+ \frac{1}{16}$
				\\~\\\noindent \textbf{Sol.}
					\begin{flalign*}
						a_{1} & = (-\frac{1}{2})^{1-1}                                                                      \\
						a_{2}              & = (-\frac{1}{2})^{2-1}                                                                      \\
						a_{3}              & = (-\frac{1}{2})^{3-1}                                                                      \\
						a_{4}              & = (-\frac{1}{2})^{4-1}                                                                      \\
						a_{5}              & = (-\frac{1}{2})^{5-1}                                                                      \\
            \therefore 1       & - \frac{1}{2}+ \frac{1}{4}- \frac{1}{8}+ \frac{1}{16}\\ & = \sum_{n=1}^{5}{(-\frac{1}{2})^{n-1}}
					\end{flalign*}

				\item $2\times4 + 4\times7 + 6\times10 + 8\times13 + 10\times16$
				\\~\\\noindent \textbf{Sol.}
					\begin{flalign*}
						a_{1} & = 2\times1\times(3\times1+1)                                                       \\
						a_{2}              & = 2\times2\times(3\times2+1)                                                       \\
						a_{3}              & = 2\times3\times(3\times3+1)                                                       \\
						a_{4}              & = 2\times4\times(3\times4+1)                                                       \\
						a_{5}              & = 2\times5\times(3\times5+1)                                                       \\
            \therefore 2       & \times4 + 4\times7 + 6\times10 + 8\times13 \\ & + 10\times16 = \sum_{n=1}^{5}{2n(3n+1)}
					\end{flalign*}
			\end{enumerate}
	\end{enumerate}

	\section{Arithmetic Progression}


	General term of an Arithmetic Progression (AP) is given by

	\[
		a_{n} = a_{1} + (n-1)d
	\]

	where $a_{1}$ is the first term, $d$ is the common difference and $n$ is the number
	of terms.

	\subsection{Practice 3}


	\begin{enumerate}
		\item Find the number of terms of the AP $-4 - 2\frac{3}{4}- 1\frac{1}{2}- \frac{1}{4}
			+ \ldots + 16$.

			\begin{flalign*}
				a_{1}                  & = -4                    \\
				a_{n}                  & = 16                    \\
				d                      & = -2\frac{3}{4}- (-4)   \\
				                       & = -2\frac{3}{4}+ 4      \\
				                       & = \frac{5}{4}           \\
				16                     & = -4 + (n-1)\frac{5}{4} \\
				20                     & = \frac{5}{4}(n-1)      \\
				80                     & = 5(n-1)                \\
				n-1                    & = 16                    \\
				n                      & = 17
			\end{flalign*}

		\item Given that $a_{2} = 4$ and $a_{6} = -8$, find the 10th term of the AP.
		\\~\\\noindent \textbf{Sol.}
			\begin{flalign*}
				a_{2}                  & = 4  \\
				a + (2-1)d             & = 4  \\
				a_{6}                  & = -8 \\
				a + (6-1)d             & = -8 \\
			\end{flalign*}
			\begin{numcases}
				{} a + d &= 4\\ a + 5d &= -8
			\end{numcases}
			\begin{flalign*}
				(2) - (1): 4d     & =-12             \\
				d                 & = -3             \\
				a + (-3)          & = 4              \\
				a                 & = 7              \\
				\therefore a_{10} & = 7 + (10-1)(-3) \\
				                  & = 7 - 27         \\
				                  & = -20
			\end{flalign*}

		\item How many multiples of 7 are there between 50 and 500?
		\\~\\\noindent \textbf{Sol.}
			\begin{flalign*}
				a_{1}                  & = 56  \\
				a_{n}                  & = 497 \\
				d                      & = 7   \\
				497 = 56 + (n-1)7       \\
				441 = 7(n-1)            \\
				n-1                    & = 63  \\
				n                      & = 64
			\end{flalign*}

		\item Find 5 numbers between 30 and 54 such that these numbers form an AP.
		\\~\\\noindent \textbf{Sol.}
			\begin{flalign*}
				a_{1}                  & = 30                                                 \\
				a_{7}                  & = 54                                                 \\
				54                     & = 30 + (7-1)d                                        \\
				24                     & = 6d                                                 \\
				d                      & = 4                                                  \\
				\\
				\therefore\            & These\ 5\ numbers\ are\ 34,\ 38,\ 42,\ 46,\\
				&and\ 50.
			\end{flalign*}
	\end{enumerate}

	\subsection*{Arithmetic mean}


	If A is in between x and y, and x, A, y are in AP, then

	\[
		A = \frac{x+y}{2}
	\]

	\subsection{Practice 4}


	\begin{enumerate}
		\item If 9, x, 17 are in AP, find x.
		\\~\\\noindent \textbf{Sol.}
			\begin{flalign*}
				x                      & = \frac{9+17}{2} \\
				                       & = \frac{26}{2}   \\
				                       & = 13
			\end{flalign*}

		\item Find the arithmetic mean of 26 and -11.
		\\~\\\noindent \textbf{Sol.}
			\begin{flalign*}
				A                      & = \frac{26-11}{2} \\
				                       & = \frac{15}{2}    \\
			\end{flalign*}

		\item Find x and y when 3, x, 12, y, 21 are in AP.
		\\~\\\noindent \textbf{Sol.}
			\begin{flalign*}
				x                      & = \frac{3+12}{2}  \\
				                       & = \frac{15}{2}    \\
				y                      & = \frac{12+21}{2} \\
				                       & = \frac{33}{2}    \\
			\end{flalign*}
	\end{enumerate}

	\subsection*{Summation of Arithmetic Progression}


	The summation formula for AP is given by

	\[
		S_{n} = \frac{n}{2}(2a + (n-1)d)
	\]

	or

	\[
		S_{n} = \frac{n}{2}(a_{1} + a_{n})
	\]

	\subsection{Practice 5}


	\begin{enumerate}
		\item Find the sum of the first 16 terms of the AP $22 + 18 + 14 + 10 + \ldots$
		\\~\\\noindent \textbf{Sol.}
			\begin{flalign*}
				a_{1}                  & = 22                                   \\
				n                      & = 16                                   \\
				d                      & = -4                                   \\
				S_{n}                  & = \frac{16}{2}(2\times22 + (-4)(16-1)) \\
				                       & = \frac{16}{2}(44 + (-4)(15))          \\
				                       & = \frac{16}{2}(44 - 60)                \\
				                       & = \frac{16}{2}(-16)                    \\
				                       & = -128
			\end{flalign*}

		\item If the sum of AP $23 + 19 + 15 + \ldots$ is 72, find the number of terms.
		\\~\\\noindent \textbf{Sol.}
			\begin{flalign*}
				a_{1}                  & = 23                                 \\
				S_{n}                  & = 72                                 \\
				d                      & = -4                                 \\
				72                     & = \frac{n}{2}(2\times23 + (-4)(n-1)) \\
				72                     & = \frac{n}{2}(46 + (-4)(n-1))        \\
				144                    & = n(46 + (-4)(n-1))                  \\
				144                    & = n(46 - 4n + 4)                     \\
				144                    & = n(50 - 4n)                         \\
				144                    & = 50n - 4n^{2}                       \\
				72                     & = 25n - 2n^{2}                       \\
				2n^{2} - 25n + 72      & = 0                                  \\
				(n - 8)(2n - 9)        & = 0                                  \\
				n                      & = 8                                  \\
			\end{flalign*}

		\item Given that $S_{n} = 2n + 3n^{2}$, find the first term and the common
			difference of the AP.
			\\~\\\noindent \textbf{Sol.}
			\begin{flalign*}
				S_{n}                  & = 2n + 3n^{2}              \\
				2n + 3n^{2}            & = \frac{n}{2}(2a + (n-1)d) \\
				4n + 6n^{2}            & = n(2a + (n-1)d)           \\
				4n + 6n^{2}            & = 2na + (n-1)nd            \\
				4n + 6n^{2}            & = 2na + n^{2}d - nd        \\
				4n + 6n^{2}            & = (2a-d)n + dn^{2}         \\
				\\
				Compar                 & ing\ both\ sides,          \\
				2a-d                   & = 4                        \\
				a                      & = 6                        \\
				d                      & = 2                        \\
			\end{flalign*}
	\end{enumerate}

  \subsection{Exercise 12.2}
  
  \begin{enumerate}
    \item Find the 10th terms of the AP $5, 13, 21, \ldots$
	\\~\\\noindent \textbf{Sol.}
      \begin{flalign*}
        a_{1}& = 5\\
        n& = 10\\
        d& = 8\\
        a_{10}& = 5 + (10-1)\times8\\
        & = 5 + 72\\
        & = 77
      \end{flalign*}

    \item Find the 8th term of the AP $5, 4\frac{1}{4}, 3\frac{1}{2}, 2\frac{3}{4}, \ldots$
	\\~\\\noindent \textbf{Sol.}
      \begin{flalign*}
        a_{1}& = 5\\
        n& = 8\\
        d& = -\frac{3}{4}\\
        a_{8}& = 5 + (8-1)\times-\frac{3}{4}\\
        & = 5 - \frac{3}{4}\times7\\
        & = 5 - \frac{21}{4}\\
        & = -\frac{1}{4}
      \end{flalign*}

    \item Find the number of terms of the following AP.

      \begin{enumerate}

        \item $4, 9, \ldots, 64$
		\\~\\\noindent \textbf{Sol.} 
            \begin{flalign*}
              a_{1}& = 4\\
              a_{n}& = 64\\
              d& = 5\\
              64& = 4 + (n-1)\times5\\
              60& = 5(n-1)\\
              12& = n - 1\\
              n& = 13
            \end{flalign*}

          \item $4\frac{1}{3}, 3\frac{2}{3}, 3, \ldots, -10\frac{1}{3}$
		  \\~\\\noindent \textbf{Sol.}
            \begin{flalign*}
              a_{1}& = 4\frac{1}{3}\\
              a_{n}& = -10\frac{1}{3}\\
              d& = -\frac{2}{3}\\
              -10\frac{1}{3}& = 4\frac{1}{3} + (n-1)\times-\frac{2}{3}\\
              -\frac{31}{3}& = \frac{13}{3} - \frac{1}{3}(n-1)\\
              -31 &= 13 - 2n + 2\\
              -46 &= 2n\\
              n& = 23
            \end{flalign*}

          \end{enumerate}
      
        \item The 6th term of an AP is 43, and its 10th term is 75. Find the first term and common difference of this AP.
		\\~\\\noindent \textbf{Sol.}
          \begin{flalign*}
            a_{6}& = 43\\
            a_{10}& = 75\\
            43& = a + (6-1)d\\
            75& = a + (10-1)d\\
            32& = 4d\\
            d& = 8\\
            43& = a + 5\times8\\
            43& = a + 40\\
            3& = a\\
            a& = 3\\
            \therefore&\ a_1 = 3,\ d = 8
          \end{flalign*}

        \item The 7th term of an AP is -10, and the 12th term -25, find the 15th term of this AP.
		\\~\\\noindent \textbf{Sol.}
          \begin{flalign*}
            a_{7}& = -10\\
            a_{12}& = -25\\
            -10& = a + (7-1)d\\
            -25& = a + (12-1)d\\
            -15& = 5d\\
            d& = -3\\
            -10& = a + 6\times-3\\
            -10& = a - 18\\
            a &= 8\\
            a_{15}& = 8 + (15-1)\times-3\\
            & = 8 - 42\\
            & = -34
          \end{flalign*}

        \item How many multiples of 7 are there between 100 and 200?
		\\~\\\noindent \textbf{Sol.}
          \begin{flalign*}
            a &= 105\\
            d &= 7\\
            a_{n}& = 196\\
            196& = 105 + (n-1)\times7\\
            91& = 7(n-1)\\
            13& = n - 1\\
            n& = 14
          \end{flalign*}

        \item Find the arithmetic mean o fthe following number pairs.

          \begin{enumerate}

            \item $(8, 20)$
			\\~\\\noindent \textbf{Sol.}
              \begin{flalign*}
                &\frac{8 + 20}{2} = 14
              \end{flalign*}

            \item $(-9, 17)$
			\\~\\\noindent \textbf{Sol.}
              \begin{flalign*}
                &\frac{-9 + 17}{2} = 4
              \end{flalign*}

          \end{enumerate}

        \item Find 5 numbers between 22 and 58 such that these 7 numbers are in AP.
		\\~\\\noindent \textbf{Sol.}
          \begin{flalign*}
            a_{1}& = 22\\
            a_{7}& = 58\\
            58 &= 22 + (7-1)d\\
            36 &= 6d\\
            d& = 6\\
            \therefore&\ These\ 5\ numbers\ are\ 22, 28, 34, 40, 46
          \end{flalign*}

        \item Find the sum of first 20 terms of AP $12+15+18+\ldots$
		\\~\\\noindent \textbf{Sol.}
          \begin{flalign*}
            a_{1}& = 12\\
            n& = 20\\
            d& = 3\\
            S_{20} &= \frac{20}{2}(2\times12 + (20-1)\times3)\\
            & = 10(24 + 57)\\
            & = 10(81)\\
            & = 810
          \end{flalign*}
 
        \item Find the sum of first 12 terms of the AP $18 + 10 + 2 - 6 - \dots$
		\\~\\\noindent \textbf{Sol.}
          \begin{flalign*}
            a_{1}& = 18\\
            n& = 12\\
            d& = -8\\
            S_{12} &= \frac{12}{2}(2\times18 + (12-1)\times-8)\\
            & = 6(36 - 88)\\
            & = 6(-52)\\
            & = -312
          \end{flalign*}

        \item Find the sum of first 14 terms of the AP $\frac{1}{6} + \frac{4}{3} + \frac{5}{2} + \ldots$
		\\~\\\noindent \textbf{Sol.}
          \begin{flalign*}
            a_{1}& = \frac{1}{6}\\
            n& = 14\\
            d& = \frac{7}{6}\\
            S_{14} &= \frac{14}{2}(2\times\frac{1}{6} + (14-1)\times\frac{7}{6})\\
            & = 7(\frac{1}{3} + \frac{91}{6})\\
            & = 7\times\frac{93}{6}\\
            & = 7\times\frac{31}{2}\\
            & = \frac{217}{2}
          \end{flalign*}

        \item Find the sum of all the multiples of 13 in between 200 and 800.
		\\~\\\noindent \textbf{Sol.}
          \begin{flalign*}
            a_{1}& = 208\\
            a_{n}& = 793\\
            d& = 13\\
            793 &= 208 + (n-1)\times13\\
            585 &= 13(n-1)\\
            45 &= n - 1\\
            n& = 46\\
            \\
            S_{46} &= \frac{46}{2}(2\times208 + (46-1)\times13)\\
            & = 23(416 + 585)\\
            & = 23(1001)\\
            & = 23023
          \end{flalign*}

        \item If the sum of first n terms of the AP $-3, -7, -11, \ldots$ is -903, find the value of n.
		\\~\\\noindent \textbf{Sol.}
          \begin{flalign*}
            a_1 &= -3\\
            d &= -4\\
            -903 &= \frac{n}{2}(2\times(-3) - 4(n-1))\\
            -1806 &= -2n -4n^2\\
            4n^2 + 2n - 1806 &= 0\\
            2n^2 + n - 903 &= 0\\
            (n-21)(2n+43) &= 0\\
            n& = 21, -43(invalid)\\
            \therefore\ n&=21
          \end{flalign*}

        \item Given that the first 3 terms of an AP are $x,\ 3x-4,\ 2x+7$, find:

          \begin{enumerate}

            \item The value of x
			\\~\\\noindent \textbf{Sol.}
              \begin{flalign*}
                3x-4 &= \frac{x + 2x + 7}{2}\\
                6x-8 &= 3x + 7\\
                3x &= 15\\
                x& = 5
              \end{flalign*}

            \item The common difference
			\\~\\\noindent \textbf{Sol.}
              \begin{flalign*}
                a_1 &= x = 5\\
                a_2 &= 3x-4 = 3\times5 - 4 = 11\\
                d &= 11 - 5\\
                & = 6
              \end{flalign*}

            \item The sum of first 10 terms.
			\\~\\\noindent \textbf{Sol.}
              \begin{flalign*}
                a_1 &= x = 5\\
                n &= 10\\
                d &= 6\\
                S_{10} &= \frac{10}{2}(2\times5 + (10-1)\times6)\\
                & = 5(10 + 54)\\
                & = 5(64)\\
                & = 320
              \end{flalign*}

          \end{enumerate}

        \item Let the sum of the first n terms of an AP to be $S_n = \frac{n(n+1)}{4}$, find:

          \begin{enumerate}
            
            \item The first term
			\\~\\\noindent \textbf{Sol.}
              \begin{flalign*}
                \frac{n(n+1)}{4} &= \frac{n}{2}(2a + (n-1)d)\\
                n(n+1) &= 2n(2a+dn-d)\\
                n^2 + n &= 4na + 2dn^2 - 2nd\\
                n^2 + n &= 2dn^2 + (4a - 2d)n\\
                \\
                Comparing&\ both\ sides,\\
                2d &= 1\\
                d &= \frac{1}{2}\\
                4a - 2d &= 1\\
                4a - 1 &= 1\\
                4a &= 2\\
                a& = \frac{1}{2}\\
              \end{flalign*}

            \item The common difference
			\\~\\\noindent \textbf{Sol.}
              \begin{flalign*}
                d &= \frac{1}{2}
              \end{flalign*}gg

            \item The 6th terms
			\\~\\\noindent \textbf{Sol.} 
              \begin{flalign*}
                a_1 &= \frac{1}{2}\\
                n &= 6\\
                d &= \frac{1}{2}\\
                a_6 &= \frac{1}{2} + (6-1)\times\frac{1}{2}\\
                &= \frac{1}{2} + \frac{5}{2}\\
                &= 3
              \end{flalign*}

            \item The sum from 6th term to 10th term
			\\~\\\noindent \textbf{Sol.}
              \begin{flalign*}
                a &= \frac{1}{2}\\
                d &= \frac{1}{2}\\
                \\
                S_{10} &= \frac{10}{2}(2\times\frac{1}{2} + (10-1)\times\frac{1}{2})\\
                & = \frac{10}{2}(1 + \frac{9}{2})\\
                & = 5\times\frac{11}{2}\\
                & = \frac{55}{2}\\
                \\
                S_5 &= \frac{5}{2}(2\times\frac{1}{2} + (5-1)\times\frac{1}{2})\\
                &= \frac{5}{2}(1 + 2)\\
                &= \frac{15}{2}\\
                \\
                S_{10} - S_6 &= \frac{55}{2} - \frac{15}{2}\\
                & = \frac{40}{2}\\
                & = 20
              \end{flalign*}

          \end{enumerate}

        \item Given three numbers in an AP, the sum of these three numbers is 30, and the sum of square of these numbers is 318, find these three numbers.
		\\~\\\noindent \textbf{Sol.}
          \begin{flalign*}
            a_1 + a_2 + a_3 &= 30\\
            a_1^2 + a_2^2 + a_3^2 &= 318\\
            a_2 - a_1 &= a_3 - a_2\\
            a_1 - 2a_2 + a_3 &= 0\\
            3a_2 &= 30\\
            a_2 &= 10\\
            a_1 - 20 + a_3 &= 0\\
            a_1 + a_3 &= 20\\
            a_3 &= 20 - a_1\\
            a_1^2 + 100 + (20 - a_1)^2 &= 318\\
            a_1^2 + 100 + 400 + a_1^2 - 40a_1 &= 318\\
            2a_1^2 - 40a_1 + 182 &= 0\\
            a_1^2 - 20a_1 + 91 &= 0\\
            (a_1-7)(a_1-13) &= 0\\
            a_1 = 7 or a_1 &= 13\\
            \\
            \therefore\ These\ three\ numbers\ &are\ 7,\ 10,\ and\ 13
          \end{flalign*}

        \item Find the sum of all the numbers between 100 and 200 that are both the multiples of 2 and 3.
		\\~\\\noindent \textbf{Sol.}
          \begin{flalign*}
            a_1 &= 102\\
            d &= 6\\
            a_n &= 198\\
            198 &= 102 + (n-1)\times6\\
            96 &= 6(n-1)\\
            6n - 6 &= 96\\
            6n = 102\\
            n &= 17\\
            \\
            S_{17} &= \frac{17}{2}(2\times102 + (17-1)\times6)\\
            & = \frac{17}{2}(204 + 96)\\
            & = \frac{17}{2}(300)\\
            & = 150\times17\\
            & = 2550
          \end{flalign*}

        \item Given an AP $-100-96-92-\ldots$:

          \begin{enumerate}

            \item Find the term where the number become positive.
			\\~\\\noindent \textbf{Sol.}
              \begin{flalign*}
                a_1 &= -100\\
                d &= 4\\
                a_n = -100 + (n-1)\times4 &> 0\\
                -100 + 4n - 4 &> 0\\
                4n &> 104\\
                n &> 26\\
                \\
                \therefore\ n = 27&
              \end{flalign*}

            \item Find the term where the sum of this AP becomes positive.
			  \\~\\\noindent \textbf{Sol.}
              \begin{align*}
                S_n = \frac{n}{2}(2(-100) + (n-1)\times(4)) &> 0\\
                \frac{n}{2}(-200 + 4n - 4) &> 0\\
                \frac{n}{2}(-204 + 4n) &> 0\\
                n(2n - 102) &> 0\\
                n(n - 51) &> 0\\
                n &> 51\\
                \\
                \therefore\ n = 52&
              \end{align*}

          \end{enumerate}

        \item Find the first negative term of the AP $20, 19\frac{1}{5}, 18\frac{2}{5}, \ldots$
          \\~\\\noindent \textbf{Sol.}
          \begin{flalign*}
            a_1 &= 20\\
            d &= -\frac{4}{5}\\
            a_n = 20 + (n-1)\times(-\frac{4}{5}) &< 0\\
            100 - 4n + 4 &< 0\\
            4n &> 104\\
            n &> 26\\
            \\
            \therefore\ n = 27&
          \end{flalign*}

        \item Given an AP $10+9\frac{1}{5}+8\frac{2}{5}+\ldots$, what is the first negative term? When will the sum of the terms become negative, and what's the value of it?
          \\~\\\noindent \textbf{Sol.}
          \begin{flalign*}
            a_1 &= 10\\
            d &= -\frac{4}{5}\\
            a_n = 10 + (n-1)\times(-\frac{4}{5}) &< 0\\
            10 - \frac{4}{5}(n - 1) &< 0\\
            50 - 4n + 4 &< 0\\
            -4n &< -54\\
            n &> 13\frac{1}{2}\\
            \\
            \therefore\ n &= 14\\
            \\
            S_n = \frac{n}{2}(2\times10 + (n-1)\times(-\frac{4}{5})) &< 0\\
            \frac{n}{2}(20-\frac{4}{5}(n-1)) &< 0\\
            20n-\frac{4}{5}(n^2 - n) &< 0\\
            100n - 4n^2 + 4n &< 0\\
            25n - n^2 + n &< 0\\
            26n - n^2 &< 0\\
            n(n - 26) &> 0\\
            n &> 26\\
            \\
            \therefore\ n &= 27\\
          \end{flalign*}
          \begin{flalign*}
            S_{27} &= \frac{27}{2}(2\times10 + (27-1)\times(-\frac{4}{5}))\\
            & = \frac{27}{2}(20 - \frac{4}{5}(27-1))\\
            & = \frac{27}{2}(20 - \frac{4}{5}(26))\\
            & = \frac{27}{2}\times(-\frac{4}{5})\\
            & = -\frac{54}{5}
          \end{flalign*}
          \begin{flalign*}
            &\therefore\ The \ first \ negative \ term \ is\ the \ 14th\\
            &\ \ \ \ \ term\\
            &\therefore\ The\ first\ term\ where\ the\ sum\ of\ the\\
            &\ \ \ \ \ terms\ becomes\ negative\ is\ the\ 27th\\
            &\ \ \ \ \ term\\
            &\therefore\ The\ value\ of\ the\ sum\ of\ the\ terms\\
            &\ \ \ \ \ when\ it\ becomes\ negative\ is\ -\frac{54}{5}
          \end{flalign*}

        \item Given a polygon which all their internal angles are in AP. The common difference of this AP is 6\degree, the largest angle is 135\degree. How many sides does this polygon have?
          \\~\\\noindent \textbf{Sol.}
          \begin{flalign*}
            a_1 &= 135\\
            d &= -6\\
              \frac{n}{2}(2\times135 + (n-1)\times(-6)) &= 180(n-2)\\
              n(270-6(n-1)) &= 360(n-2)\\
              n(276-6n) &= 360n - 720\\
              276n - 6n^2 &= 360n - 720\\
              46n - n^2 &= 60n - 120\\
              n^2 + 14n - 120 &= 0\\
              (n+20)(n-6) &= 0\\
              n &= -20\ (invalid)\\
              n &= 6\\
              \therefore\ The\ number\ of\ &sides\ is\ 6
          \end{flalign*}

  \end{enumerate}

\end{multicols}
\end{document}
