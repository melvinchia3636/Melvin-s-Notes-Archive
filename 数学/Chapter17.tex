% chktex-file 2% chktex-file 29
% chktex-file 13
\documentclass{report}
\usepackage{setspace}
\usepackage[a4paper, total={7in, 10in}]{geometry}
\usepackage[fleqn]{amsmath}
\usepackage{empheq}
\usepackage{amssymb}
\usepackage{amsthm}
\usepackage{gensymb}
\usepackage[fleqn]{cases}
\usepackage{multicol}
\usepackage{color}
\usepackage{stix}
\usepackage{chngcntr}
\usepackage{tikz}
\usepackage{enumitem}
\usepackage{pgfplots}
\usepackage{etoolbox}
\usepackage{tikz-3dplot}
\usepackage{tkz-euclide}
\usepackage{graphicx}
\usepackage{enumitem}

\def\nswe#1#2#3{#1\,$#2^\circ\,#3'$}
\graphicspath{ {./assets/} }
\usetikzlibrary{calc,matrix,arrows}
\usetikzlibrary{decorations.pathmorphing,patterns, calligraphy, perspective,backgrounds}

\tikzset{
    right angle quadrant/.code={
            \pgfmathsetmacro\quadranta{{1,1,-1,-1}[#1-1]}     % Arrays for selecting quadrant
            \pgfmathsetmacro\quadrantb{{1,-1,-1,1}[#1-1]}},
    right angle quadrant=1, % Make sure it is set, even if not called explicitly
    right angle length/.code={\def\rightanglelength{#1}},   % Length of symbol
    right angle length=2ex, % Make sure it is set...
    right angle symbol/.style n args={3}{
            insert path={
                    let \p0 = ($(#1)!(#3)!(#2)$) in     % Intersection
                    let \p1 = ($(\p0)!\quadranta*\rightanglelength!(#3)$), % Point on base line
                    \p2 = ($(\p0)!\quadrantb*\rightanglelength!(#2)$) in % Point on perpendicular line
                    let \p3 = ($(\p1)+(\p2)-(\p0)$) in  % Corner point of symbol
                    (\p1) -- (\p3) -- (\p2)
                }
        }
}

\counterwithout{equation}{chapter}
\setlength{\columnseprule}{1pt}
\setlength{\columnsep}{24pt}
\setcounter{chapter}{16}
\hfuzz=100pt

\newcommand{\pgfplotsdrawaxis}{\pgfplots@draw@axis}
\makeatother
\pgfplotsset{only axis on top/.style={axis on top=false, after end axis/.code={
                    \pgfplotsset{axis line style=opaque, ticklabel style=opaque, tick style={thick,opaque},
                        grid=none}\pgfplotsdrawaxis}}}

\newtheorem{theorem}{Theorem}

\begin{document}\makeatletter
\newcommand{\newparallel}{\mathrel{\mathpalette\new@parallel\relax}}
\newcommand{\new@parallel}[2]{%
    \begingroup
    \sbox\z@{$#1T$}% get the height of an uppercase letter
    \resizebox{!}{\ht\z@}{\raisebox{\depth}{$\m@th#1/\mkern-5mu/$}}%
    \endgroup
}
\makeatother

\newcommand{\planelineinter}[5]% a, b, c, p as {a_x,a_y,a_z}, coordinate name
{   \foreach \a [count=\k] in {#1}
        { \ifthenelse{\k=1}{\xdef\tempxa{\a}}
            \ifthenelse{\k=2}{\xdef\tempya{\a}}
            \ifthenelse{\k=3}{\xdef\tempza{\a}}
        }
    \foreach \b [count=\k] in {#2}
        { \ifthenelse{\k=1}{\xdef\tempxb{\b}}
            \ifthenelse{\k=2}{\xdef\tempyb{\b}}
            \ifthenelse{\k=3}{\xdef\tempzb{\b}}
        }
    \foreach \c [count=\k] in {#3}
        { \ifthenelse{\k=1}{\xdef\tempxc{\c}}
            \ifthenelse{\k=2}{\xdef\tempyc{\c}}
            \ifthenelse{\k=3}{\xdef\tempzc{\c}}
        }
    \foreach \p [count=\k] in {#4}
        { \ifthenelse{\k=1}{\xdef\tempxp{\p}}
            \ifthenelse{\k=2}{\xdef\tempyp{\p}}
            \ifthenelse{\k=3}{\xdef\tempzp{\p}}
        }
    \pgfmathsetmacro{\abx}{\tempxb-\tempxa}
    \pgfmathsetmacro{\aby}{\tempyb-\tempya}
    \pgfmathsetmacro{\abz}{\tempzb-\tempza}
    \pgfmathsetmacro{\acx}{\tempxc-\tempxa}
    \pgfmathsetmacro{\acy}{\tempyc-\tempya}
    \pgfmathsetmacro{\acz}{\tempzc-\tempza}
    \pgfmathsetmacro{\nx}{\aby*\acz-\abz*\acy}
    \pgfmathsetmacro{\ny}{\abz*\acx-\abx*\acz}
    \pgfmathsetmacro{\nz}{\abx*\acy-\aby*\acx}
    \pgfmathsetmacro{\d}{(\nx+\ny+\nz)/(\nx*\tempxp+\ny*\tempyp+\nz*\tempzp)}
    \path (0,0,0) -- (#4) coordinate[pos=\d] (#5);
}

% golden ratio and inverse golden ratio
\pgfmathsetmacro{\gr}{(1+sqrt(5))/2}
\pgfmathsetmacro{\igr}{2/(1+sqrt(5))}

%choose axis angles
\newcommand{\xangle}{0}
\newcommand{\yangle}{90}
\newcommand{\zangle}{225}

%choose axis lengths
\newcommand{\xlength}{1}
\newcommand{\ylength}{1}
\newcommand{\zlength}{0.5}

\pgfmathsetmacro{\xx}{\xlength*cos(\xangle)}
\pgfmathsetmacro{\xy}{\xlength*sin(\xangle)}
\pgfmathsetmacro{\yx}{\ylength*cos(\yangle)}
\pgfmathsetmacro{\yy}{\ylength*sin(\yangle)}
\pgfmathsetmacro{\zx}{\zlength*cos(\zangle)}
\pgfmathsetmacro{\zy}{\zlength*sin(\zangle)}

\newcommand{\sol}[1]{

    \noindent \textbf{Sol.}
}
\newcommand{\prooff}[1]{

    \noindent \textbf{Proof.}
}
\newcommand\m[1]{\begin{pmatrix}#1\end{pmatrix}}
\newcommand\vm[1]{\begin{vmatrix}#1\end{vmatrix}}
\newenvironment{amatrix}[1]{%
    \left(\begin{array}{@{}*{#1}{c}|c@{}}
        }{%
    \end{array}\right)
}
\newenvironment{cequation}{
    \makeatletter
    \setbool{@fleqn}{false}
    \makeatother
    \begin{equation*}
        }{\end{equation*}}

\begin{titlepage}
    \raggedleft{}
    \rule{1pt}{\textheight}
    \hspace{0.02\textwidth}
    \parbox[b]{0.75\textwidth}{

    {\Huge\bfseries Solution Book of \\[0.5\baselineskip] Mathematic}\\[2\baselineskip]
    {\large\textit{Ssnior 2 Part I}}\\[4\baselineskip]
    {\Large\textsc{MELVIN CHIA}}

    \vspace{0.5\textheight}

    {\noindent Written on 9 October 2022}\\[\baselineskip]
    }

\end{titlepage}

\doublespacing{}
\tableofcontents
\singlespacing{}
\newpage

\begin{multicols}{2}

\end{multicols}
\end{document}