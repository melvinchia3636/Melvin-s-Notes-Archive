\documentclass{report}
\setlength{\parskip}{\baselineskip}%
\input{structure.tex}

\begin{document}
\pagestyle{fancy}
%... then configure it.
\fancyhead{} % clear all header fields
\fancyhead[RO,LE]{\thepage}
\fancyhead[LO,RE]{\leftmark}
\fancyfoot{} % clear all footer fields

\fancyfoot[LO,RE]{Dong Zong Addmath Textbook Senior 1 Volume II}
\fancyfoot[RO,RE]{\thepage}

\onehalfspacing
\setcounter{chapter}{7}

\chapter{Degree and Radian}

\section{Concept and Measurement of Arbitrary Angle}

\subsection*{Arbitrary Angle}

Back in junior high, we learned that the angle is the geometric object formed by rays from the same endpoint. As shown in the figure below, whether it is from ray $OA$ to ray $OB$, or from ray $OA$ to ray $OC$, the angle can be measured as $30^\circ$.

\begin{center}
	\includegraphics[scale=0.21]{assets/8-1.png}
\end{center}
\vspace{-1em}

In fact, angle can be seen as a rotation of a ray about its endpoint in a plane. As shown in the figure below, the endpoint $O$ of the ray is known as the \textbf{vertex} of the angle, the initial position $OA$ of the rotation is known as the \textbf{initial side} of the angle, and the final position $OB$ of the rotation is known as the \textbf{terminal side} of the angle, while the amount of rotation is $\angle AOB$. The rotation can be either clockwise or counterclockwise, as shown in the figure below, the angle formed counterclockwise is positive, while the angle formed clockwise is negative, $\theta > 0^\circ$ in the figure below.

\begin{center}
	\includegraphics[scale=0.23]{assets/8-2.png}
\end{center}

\begin{explore}[Exploration Activity 1]
	    
	\textbf{Aim:} To understand the definition of arbitrary angle.
	
	\textbf{Materials Needed:} A piece of paper, a protractor, a ruler
	
	\textbf{Steps:}
	\vspace{-1em}
	\begin{enumerate}
		\item Draw three parallel rays $OA$ (as shown below) on the paper. Hence, with $OA$ as the initial side of the angle, measure the following using a protractor:
		      \begin{enumerate}
		      	\item The terminal side $OB$ of the $160^\circ$.
		      	\item The terminal side $OC$ of the $-45^\circ$.
		      	\item The terminal side $OD$ of the $450^\circ$.
		      \end{enumerate}
		\item Write down clearly the respective direction and the measurement of the rotation in the three figures above.
	\end{enumerate}
	
	\textbf{Tool:} \underline{\url{https://www.geogebra.org/m/u7a3zxfd}}
\end{explore}

From exploration activity 1, this kind of directional rotational measurement that is not limited to $0^\circ$ to $360^\circ$ is known as the \textbf{arbitrary angle}. If the initial side doesn't rotate at all, the angle formed is called a \textbf{zero angle}.

\subsection*{Degree and Radian}

Back in junior high, the angle unit that we learned is formed by dividing a circle into $360$ equal parts, and the corresponding central angle of each part is known as 1 degree, denoted as $1^\circ$. Dividing the arc corresponding to 1 degree angle into $60$ equal parts, the corresponding central angle of each part is known as 1 minute, denoted as $1'$. Dividing the arc corresponding to 1 minute angle into $60$ equal parts, the corresponding central angle of each part is known as 1 second, denoted as $1''$.
\begin{align*}
	\therefore 1 \text{ full angle} & = 360^\circ \\
	1^\circ                         & = 60'       \\
	1'                              & = 60''      
\end{align*}
This form of base 60 measurement of angles is known as the \textbf{degree measure}.

\begin{explore}[Exploration Activity 2]
	
	\begin{enumerate}[label=\textbf{Aim:} ,leftmargin=4.5em]
		\item To inspect the relationship between arc length and radius, and to understand teh definition of radian measure.
	\end{enumerate}
	\vspace{-2em}
	\begin{enumerate}[label=\textbf{Materials Needed:} ,leftmargin=2em, align=left]
		\item A compass, a protractor, a string of length 10 cm, a scrap paper.
	\end{enumerate}
	\vspace{-1em}
	
	\textbf{Steps:}
	\vspace{-1em}
	\begin{enumerate}
		\item Form a group of 2 to 4 people, each group member is required to draw a circle with different radii. (Choose any arbitrary radius like 2cm, 3cm, 4cm, $\cdots$)
		              
		\item Each person uses a line to measure the arc $AB$ with the same length as the radius on his own arc, then marks it on the string, as shown in the figure below.
		              
		      \begin{center}
		      	\includegraphics[scale=0.1]{assets/8-3.png}
		      \end{center}
		              
		\item Use a protractor to measure the corresponding central angle of the arc $AB$.
	\end{enumerate}
	
	\textbf{Discuss and Discover:}
	\vspace{-1em}
	\begin{enumerate}
		\item Compare the diagram drawn by each group member, are the measured central angles the same?
		\item Use a string to measure how many parts with the same length as the $AB$ of the circumference can be divided.
		\item Is it possible to derive an accurate answer for the question above using the formula of the circumference of a circle?
	\end{enumerate}
\end{explore}

From the activities above, we have led out another form of angle measurement that is commonly used in higher mathematics and other fields of science, known as the \textbf{radian measure}. We call the central angle corresponding to the arc with the same length as the radius (i.e. $\angle AOB$ in the figure above) 1 radian, denoted as $1 \text{ rad}$. From discussion 2 and discussion 3, we know that the ratio between the circumference of a circle and its radius is $2\pi$, i.e. 1 full angle $= 360^\circ = 2\pi \text{ rad}$, hence we can derive the following conversion formula between degree and radian:

\begin{info}[Conversion between Degree and Radian]
	\begin{align*}
		\pi \text{ rad} & = 180^\circ                   \\
		1 \text{ rad}   & = \frac{180}{\pi}^\circ       \\
		1^\circ         & = \frac{\pi}{180} \text{ rad} 
	\end{align*}
\end{info}

\begin{question}
	Convert the following angles from degree to radian:
	\vspace{-1em}
	\begin{multicols}{2}
		\begin{enumerate}[label=(\alph*)]
			\item $218.84^\circ$
			\item $48^\circ 36'$
		\end{enumerate}
	\end{multicols}
	\vspace{-1em}
	\sol{}
	\begin{enumerate}[label=(\alph*)]
		\item \begin{align*}
		      218.84^\circ &= 218.84 \times \frac{\pi}{180}\\
		      &= 3.8195 \text{ rad}
		\end{align*}
		\item \begin{align*}
		      48^\circ 36' &= \left(48 + \frac{36}{60}\right)^{\circ}\\
		      & = 48.6 \times \frac{\pi}{180}\\
		      & = 0.8482 \text{ rad}
		\end{align*}
	\end{enumerate}
\end{question}

\begin{question}
	Convert 1.5 radian to degree (accurate to minute).
	
	\sol{}
	\begin{align*}
		1.5 \text{ rad} & = 1.5 \times \frac{180}{\pi}^\circ \\
		                & = 85^\circ 57'                     
	\end{align*}
\end{question}

\begin{question}
	Convert $\dfrac{\pi}{45}$ radian to degree.
	
	\sol{}
	\begin{align*}
		\frac{\pi}{45} \text{ rad} & = \frac{\pi}{45} \times \frac{180}{\pi}^\circ \\
		                           & = 4^\circ                                     
	\end{align*}
\end{question}

It is worth nothing that when we use the radian measure, the unit "radian" is usually omitted. We usually write $90^\circ = \dfrac{\pi}{2}$, but if we use degree measure to express the same angle, the unit "degree" cannot be omitted.

\practice{8.1}

\begin{enumerate}
	\item Complete the following table, express the angle in $pi$:
	          
	      \begin{tabular}{|l|c|c|c|c|c|c|c|c|c|c|}
	      	\hline Angle  & $0^{\circ}$ & $30^{\circ}$ & $45^{\circ}$ & $60^{\circ}$ & $90^{\circ}$ & $120^{\circ}$ & $135^{\circ}$ & $150^{\circ}$ & $180^{\circ}$ & $270^{\circ}$ \\
	      	\hline Radian &             &              &              &              &              &               &               &               &               &               \\
	      	\hline
	      \end{tabular}
	      \vspace{1em}
	      
	\item Convert the following angles from radian to degree (if the result is not an integer, round to 2 decimal places):
	      \vspace{-1em}
	      \begin{multicols}{2}
	      	\begin{enumerate}[label=(\alph*)]
	      		\item $0.5 \text{ rad}$
	      		\item $\dfrac{7\pi}{6} \text{ rad}$
	      	\end{enumerate}
	      \end{multicols}
\end{enumerate}

\exercise{8.1}

\begin{enumerate}
	\item Using the horizontal ray $OA$ in the figure below as the initial side, sketch the position of the terminal side $OB$ of the following angles:
	      \begin{center}
	      	\includegraphics[scale=0.15]{assets/8-4.png}
	      \end{center}
	      \vspace{-1em}
	      \begin{multicols}{3}
	      	\begin{enumerate}[label=(\alph*)]
	      		\item $\dfrac{5 \pi}{4}$
	      		\item $-\dfrac{6 \pi}{7}$
	      		\item $\dfrac{5 \pi}{2}$
	      	\end{enumerate}
	      \end{multicols}
	          
	\item Convert the following angles from degree to radian (accurate to 4 decimal places):
	      \vspace{-1em}
	      \begin{multicols}{3}
	      	\begin{enumerate}[label=(\alph*)]
	      		\item $68^\circ 93'$
	      		\item $139^\circ 12'$
	      		\item $-502^\circ 46'$
	      	\end{enumerate}
	      \end{multicols}
	      
	\item Convert the following angles from radian to degree (if the result is not an integer, round to minute):
	      \vspace{-1em}
	      \begin{multicols}{3}
	      	\begin{enumerate}[label=(\alph*)]
	      		\item $0.89 \text{ rad}$
	      		\item $-\dfrac{17\pi}{4} \text{ rad}$
	      		\item $3 \text{ rad}$
	      	\end{enumerate}
	      \end{multicols}
	      
	\item If the gear in the figure below rotates counterclockwise about its origin $O$, how many radians does the gear rotate through such that
	      \begin{center}
	      	\includegraphics[scale=0.2]{assets/8-5.png}
	      \end{center}
	      \begin{enumerate}
	      	\item The line segment $OB$ reaches the position of $OA$ in the figure above for the first time?
	      	\item The line segment $OB$ reaches the position of $OA$ in the figure above for the second time?
	      \end{enumerate}
\end{enumerate}

\newpage
\section{Arc Length and Sector Area}

\subsection*{Formula of Arc Length}

Back in junior high, we learned that in a circle with radius $r$ and central angle $\theta$,
\vspace{-1em}
\begin{enumerate}[label=(\arabic*)]
	\item if the angle is expressed in degree, the arc length $l$ is given by
	      \begin{cequation}
	      	l = \frac{\theta}{360^\circ} \times 2\pi r\qquad(\text{where }\theta \text{ is in degree})
	      \end{cequation}
	\item if the angle is expressed in radian, i.e. $360^\circ = 2\pi$, the arc length $l$ is given by
	      \begin{cequation}
	      	l = \dfrac{\theta}{2\pi} \times 2\pi r = \theta r\qquad(\text{where }\theta \text{ is in radian})
	      \end{cequation}
\end{enumerate}

From the definition of radians, we can see that if the central angle corresponding to the arc $AB$ is $\theta$ radian, then the arc length $AB$ is $\theta$ times the length of radius $r$. That is,
\begin{info}[Formula of Arc Length]
	\begin{cequation}
		\theta = \frac{l}{r}
	\end{cequation}
	\vspace{-1em}
	\begin{cequation}
		r = \frac{l}{\theta}
	\end{cequation}
	\vspace{-1em}
\end{info}

\begin{question}
	In a circle with radius 5 cm, find
	\vspace{-1em}
	\begin{enumerate}[label=(\alph*)]
		\item The arc length corresponding to the central angle of $1.5 \text{ rad}$.
		\item The central angle corresponding to the arc length of 17 cm.
	\end{enumerate}
	\sol{}
	\begin{enumerate}[label=(\alph*)]
		\item $\begin{aligned}[t]
		      \text{The corresponding arc length} &= 1.5 \times 5\\
		      &= 7.5 \text{ cm}
		\end{aligned}$
		\item Let $\theta$ be the central angle, then
		      \begin{align*}
		      	5\theta & = 17 &   \\
		      	\theta &= 3.4
		      \end{align*}
		      The central angle corresponding to the arc length of 17 cm is $3.4 \text{ rad}$.
	\end{enumerate}
\end{question}

\begin{question}
	\begin{center}
		\includegraphics[scale=0.09]{assets/8-6.png}
	\end{center}
	\noindent The figure above shows a sector with center $O$ and radius of 10 cm. If the length of chord $AB$ is 16 cm, find
	\vspace{-1em}
	\begin{multicols}{2}
		\begin{enumerate}[label=(\alph*)]
			\item $\angle AOB$ (in radian);
			\item The length of arc $AB$.
		\end{enumerate}
	\end{multicols}
	\vspace{-1em}
	\sol{}
	\begin{enumerate}[label=(\alph*)]
		\item \begin{multicols}{2}
		      Let $M$ be the midpoint of chord $AB$,
		              
		      $\therefore$ $OM \perp AB$ and $\angle MOB = \angle MOA$
		      \begin{flalign*}
		      	\text{In }\triangle OMB,&\text{$\qquad$$\begin{aligned}[t]
		      		\sin \angle BOM &= \dfrac{MB}{OB} &\\
		      		&= \dfrac{8}{10}\\
		      		\angle BOM &= 0.9273
		      		\end{aligned}$}\\
		      	\text{Hence},&\text{$\qquad$$\begin{aligned}[t]
		      		\angle AOB &= 2 \times 0.9273\\
		      		&= 1.8546 \text{ rad}
		      		\end{aligned}$}&
		      \end{flalign*}
		      \includegraphics[scale=0.12]{assets/8-7.png}
		\end{multicols}
		\item $\begin{aligned}[t]
		      \text{Arc length } AB &= 1.8546 \times 10\\
		      &= 18.546 \text{ cm}
		\end{aligned}$
	\end{enumerate}
\end{question}
\begin{info}[Knowledge Review]
	        
	\noindent The trigonometric functions that we learned in junior high are:
	\begin{multicols}{2}
		\vspace*{-3em}
		\begin{align*}
			\sin \theta & = \dfrac{y}{r} \\
			\cos \theta & = \dfrac{x}{r} \\
			\tan \theta & = \dfrac{y}{x} 
		\end{align*}
		\columnbreak
		\includegraphics[scale=0.12]{assets/8-8.png}
	\end{multicols}
	\vspace{-2em}
	Adding appropriate auxiliary lines to form a right angle triangle, we can use the trigonometric functions to find the relationship between the sides and the angles of the triangle.
\end{info}
\vspace{1em}
\begin{think}
	   
	\noindent Can we use degree to find the arc length $AB$?
\end{think}

\practice{8.2a}

Find the circumference of the following sectors:
\begin{multicols}{2}
	\begin{enumerate}[label=(\alph*)]
		\item \includegraphics[scale=0.12,valign=t]{assets/8-9.png}
		\item \includegraphics[scale=0.12,valign=t]{assets/8-10.png}
	\end{enumerate}
\end{multicols}

\subsection*{Formula of Sector Area}

\begin{explore}[Exploration Activity 3]
	        
	\begin{enumerate}[label=\textbf{Aim:} ,leftmargin=4.4em]
		\item To strengthen the understanding of the relationship between the ratio of radian and arc length, and the ratio of radian and sector area.
	\end{enumerate}
	\vspace{-2em}
	\begin{enumerate}[label=\textbf{Tool:} ,leftmargin=2em, align=left]
		\item \url{https://www.geogebra.org/m/a4brp8yg}
	\end{enumerate}
	\vspace{-1em}
	        
	\textbf{Steps:}
	\vspace{-1em}
	\begin{enumerate}
		\item Without moving point $B$, move the radius of the circle by moving the slider $L$, and inspect the changes of the following three ratios by the coloured section of the sector:
		      \begin{enumerate}[label=(\arabic*)]
		      	\item $\angle AOB$ and $2\pi$ rad (i.e. $360^\circ$).
		      	\item The arc length $AB$ and the circumference of the circle.
		      	\item The area of sector $OAB$ and the area of the circle.
		      \end{enumerate}
		\item Without changing the radius of the circle, move point $B$ to change the central angle, and inspect the changes of the same three ratios above.
	\end{enumerate}
	
	From the inspection above, Have you found any identical values?
\end{explore}

\newpage

Back in junior, we learned that the area of a sector is proportional to the central angle of the sector. Hence, we can derive that
\begin{cequation}
	\text{Area of sector } = \frac{\theta}{360^\circ} \times \pi r^2\qquad(\text{where }\theta \text{ is in degree})
\end{cequation}
Since $360^\circ = 2\pi$,
\begin{cequation}
	\therefore \text{Area of sector }S = \frac{\theta}{2\pi} \times \pi r^2 \qquad(\text{where }\theta \text{ is in radian})
\end{cequation}
That is,
\begin{info}[Formula of Sector Area]
	\begin{cequation}
		S = \frac{1}{2} r^2 \theta
	\end{cequation}
	\vspace{-1em}
\end{info}
Because arc length $l = r\theta$,
$\therefore$ the area formula of a sector can also be expressed as
\begin{info}[Formula of Sector Area]
	\begin{align*}
		S & = \frac{1}{2} \times \pi r \times r \\
		S & = \frac{1}{2} r l                   
	\end{align*}
\end{info}

\begin{question}
	There is a sector with radius 3cm and area 15 cm$^2$. Find the central and the circumference of the sector.
	
	\sol{}
	\vspace{-1em}
	\setlength{\columnsep}{-30em}
	\begin{multicols}{2}
		\vspace*{-2.6em}
		\begin{align*}
			S      & = \dfrac{1}{2}\pi r^2  \\
			15     & = \dfrac{1}{2}\pi(3)^2 \\
			\theta & = \dfrac{10}{3}        
		\end{align*}
		\begin{align*}
			S  & = \dfrac{1}{2}lr         \\
			15 & = \dfrac{1}{2}l \times 3 \\
			l  & = 10                     
		\end{align*}
	\end{multicols}
	\vspace{-4em}
	\begin{align*}
		\therefore\ \text{The central angle} & = \dfrac{10}{3} \text{ rad,} \\
		\text{The circumference}             & = l + 2r                     \\
		                                     & = 10 + 2 \times 3            \\
		                                     & = 16 \text{ cm}              
	\end{align*}
\end{question}
\newpage
\begin{question}
	\begin{center}
		\includegraphics[scale=0.12]{assets/8-11.png}
	\end{center}
	\noindent The figure above shows two circles with the same radius $r$ and the center being $A$ and $B$ respectively. Prove that the area of the overlapping region of these two circles is $\dfrac{2}{3} \pi r^2 - \dfrac{\sqrt{3}}{2}r^2$.
	
	\sol{}
	
	\includegraphics[scale=0.12]{assets/8-12.png}
	    
	\noindent Let the two circles intersect at point $P$ and $Q$,
	    
	\noindent Since the two circles are of the same size, the overlapping area is formed by two identical arches.
	\begin{flalign*}
		\because &\ \triangle ABP \text{ and} \triangle ABQ \text{ are equilateral triangles,}&\\
		\therefore &\ \angle PAB = \angle BAQ = \dfrac{\pi}{3} &
	\end{flalign*}
	\vspace*{-2em}
	\begin{vwcol}[widths={0.7,0.3},rule=0pt,sep=3em]
		Let $PQ$ and $AB$ intersect at point $M$, $M$ happens to be the midpoint of line segment $PQ$ and $AB$.
		
		\vspace{1em}
		\noindent$\therefore AM = \dfrac{1}{2}AB = \dfrac{1}{2}r$
		\begin{flalign*}
			\text{In }\triangle AMP,\ \sin\dfrac{\pi}{3} & = \dfrac{MP}{AP}\\
			MP &= \dfrac{\sqrt{3}}{2}r&
		\end{flalign*}
		\includegraphics[scale=0.16]{assets/8-13.png}
	\end{vwcol}
	\vspace*{-2em}
	\begin{vwcol}[widths={0.2,0.8},rule=0pt,sep=3em]
		\vspace{-2.8em}
		\begin{align*}
			  & \text{Area of sector } APQ                    &   \\
			  & = \dfrac{1}{2}r^2\left(\dfrac{2\pi}{3}\right) &   \\
			&= \dfrac{\pi r^2}{3}
		\end{align*}
		\begin{align*}
			  & \text{Area of } \triangle APQ                                                      &   \\
			  & = 2 \times \dfrac{1}{2}\left(\dfrac{1}{2}r\right)\left(\dfrac{\sqrt{3}}{2}r\right) &   \\
			&= \dfrac{\sqrt{3}}{4}r^2
		\end{align*}
	\end{vwcol}
	\vspace{-4em}
	\begin{flalign*}
		\text{Hence, the area of the overlapping region of the two circles } &= 2 \times \left(\text{Area of sector } APQ - \text{Area of } \triangle APQ\right)&\\
		&= 2 \times \left[\dfrac{1}{3}\pi r^2 - \dfrac{\sqrt{3}}{4}r^2\right]&\\
		&= \dfrac{2}{3}\pi r^2 - \dfrac{\sqrt{3}}{2}r^2&
	\end{flalign*}
\end{question}

\newpage
\begin{question}
	\begin{vwcol}[widths={0.6,0.4},rule=0pt,sep=-10em]
		As shown in the figure to the right, both the centers of the circles with arcs $\wideparen{AB}$ and $\wideparen{CD}$ is $O$. Given that $AC = d$, $\wideparen{AB} = l'$, and $\wideparen{CD} = l$. Prove that:
		        
		\vspace{1em}
		\noindent \hspace{1em}(a) $\angle AOB = \dfrac{l - l'}{d}$ radian;
		        
		\noindent \hspace{1em}(b) The shaded area $= \dfrac{1}{2}(l + l')d$.
		        
		\includegraphics[scale=0.2]{assets/8-14.png}
	\end{vwcol}
	        
	\sol{}
	\vspace{-1em}
	\begin{enumerate}[label=(\alph*)]
		\item Let $\begin{aligned}[t]
		      \angle AOB &= \theta,\\
		      l' &= (OA)\theta\\
		      l &= (OC)\theta\\
		      l - l' &= (OC - OA)\theta\\
		      & = d\theta\\
		      \theta &= \dfrac{l - l'}{d}
		\end{aligned}$
		        
		\item $\begin{aligned}[t]
		      \text{Area of shaded region} &= \text{Area of sector} OCD - \text{Area of sector} OAB\\
		      & = \dfrac{1}{2}(OC)^2\theta - \dfrac{1}{2}(OA)^2\theta\\
		      &= \dfrac{1}{2}\theta\left[(OC)^2 - (OA)^2\right]\\
		      &= \dfrac{1}{2}\left[OC + OA\right]\left[OC - OA\right]\theta \qquad \because l + l' = (OC + OA)\theta\\
		      & = \dfrac{1}{2}d(l + l')
		\end{aligned}$
	\end{enumerate}
\end{question}

\practice{8.2b}

Find the area of the shaded regions in the following figures:
\begin{multicols}{2}
	\begin{enumerate}[label=(\alph*)]
		\item \includegraphics[scale=0.12,valign=t]{assets/8-15.png}
		\item $OAC$ is a sector, $OABC$ is a rhombus
		              
		      \vspace{0.5em}
		      \includegraphics[scale=0.12]{assets/8-16.png}
	\end{enumerate}
\end{multicols}

\newpage

\exercise{8.2a}

\begin{enumerate}
	\item \begin{multicols}{2}
	      The right figure shows a semicircle with center $O$ and diameter $8$ cm. $C$ is a point on arc $AB$. Given that $\angle CAB=0.5$ rad, find the length of arc $BC$ and the area of the shaded region.
	      
	      \begin{center}
	      	\includegraphics[scale=0.14]{assets/8-17.png}
	      \end{center}
	\end{multicols}
	
	\item Given that the arc length of a sector is $8 \pi \mathrm{cm}$ and the area is $48 \pi \mathrm{cm}^2$, determine the radius and the central angle of the sector.
	      
	\item \begin{multicols}{2}
	      The circle in the right figure is formed by a major sector $OACB$ and a minor sector $OAB$. If the area of the major sector $OACB$ is $30 \pi \mathrm{cm}^2$, and the length of the minor arc $AB$ is $2 \pi \mathrm{cm}$, find the radius of the circle and the central angles of the two sectors.
	      
	      \begin{center}
	      	\includegraphics[scale=0.14]{assets/8-18.png}
	      \end{center}
	\end{multicols}
	
	\item \begin{multicols}{2}
	      As shown in the figure, a circle with center $C$ is tangent to the sector $OAB$ at points $P$, $Q$, and $R$. Given that $CP=4 \mathrm{~cm}$ and $\angle AOB=60^{\circ}$, without using a computer, find:
	      \begin{enumerate}
	      	\item the length of $OA$;
	      	\item the area of the major sector $CPRQ$;
	      	\item the area of the shaded region.
	      \end{enumerate}
	      
	      \begin{center}
	      	\includegraphics[scale=0.14]{assets/8-19.png}
	      \end{center}
	\end{multicols}
	
	\item \begin{multicols}{2}
	      In the right figure, $ABCD$ is a square with a side length of $a$. Four semicircles are drawn with the sides of the square as their diameters. These semicircles enclose the shaded region in the figure. Find the area of the shaded region.
	      
	      \begin{center}
	      	\includegraphics[scale=0.14]{assets/8-20.png}
	      \end{center}
	\end{multicols}
	
	\item \begin{multicols}{2}
	      In the right figure, $O_1$ and $O_2$ are the centers of the large and small circles respectively. The two circles are tangent to each other at point $A$, line $O_1CB$ intersects the large and small circles at points $B$ and $C$ respectively. Prove that $\wideparen{AB}=\wideparen{AC}$.
	      
	      \begin{center}
	      	\includegraphics[scale=0.14]{assets/8-21.png}
	      \end{center}
	\end{multicols}
\end{enumerate}

\newpage
\subsection*{Problems related to Arc Length and Sector}

\begin{question}
	\begin{center}
		\includegraphics[scale=0.14]{assets/8-22.png}
	\end{center}
	\noindent The figure above shows two gears of a bicycle.
	
	\noindent Given that the radius of the larger and the smaller gear are 12 cm and 4 cm respectively. If the distance between the center of the two gears is 41 cm, how long is the chain that connects the two gears?
	
	\sol{}
	
	\begin{vwcol}[widths={0.6,0.4},rule=0pt,sep=1em]
		\noindent Let the center of the two gears be point $A$ and point $B$ respectively, hence $AB = 41$ cm.
		    
		\noindent Let the tangent point of the chain on the gears be $P$, $Q$, $R$, and $S$ respectively, and draw a line from $M$ perpendicular to $AR$ to intersect $AR$ at point $M$.
		    
		\noindent $\therefore$ $MBSR$ is a rectangle, hence 
		    
		$BS = MR = 4$ cm, $AM = 8$ cm.
		
		\includegraphics[scale=0.12]{assets/8-23.png}
	\end{vwcol}
	\vspace{-3em}
	\begin{flalign*}
		\text{In }\triangle ABM,\ MB^2 &= 41^2 - 8^2 &\\
		MB & = \sqrt{1617}\text{ cm} = RS = PQ&\\
		\sin \angle A B M & =\frac{8}{41} \\ 
		\angle A B M & =0.1964 \\ 
		\angle M A B & =\frac{\pi}{2}-0.1964=1.3744
	\end{flalign*}
	\vspace{-2em}
	\begin{flalign*}
		\text { Major angle } P A R&=2 \pi-2 \times \angle M A B=3.5344 &\\ 
		\text { Length of major arc } P R &=12 \times 3.5344=42.4128 \mathrm{~cm} \\ 
		\angle Q B S&=2 \pi-2 \times\left(\frac{\pi}{2}+0.1964\right)=2.7488 \\ 
		\text { Arc length } Q S&=4 \times 2.7488=10.9952 \mathrm{~cm} \\ 
		\therefore \text { Length of the chain }&=P Q+\wideparen{Q S}+R S+\text { major } \wideparen{P R} \\ 
		&=\sqrt{1617}+10.9952+\sqrt{1617}+42.4128 \\ 
		&=133.83 \mathrm{~cm} 
	\end{flalign*}
\end{question}

\newpage

\begin{question}
	\begin{center}
		\includegraphics[scale=0.18]{assets/8-24.png}
	\end{center}
	\noindent Shown in the figure above is a tank truck, its shape can be approximated by a cylinder with a radius of 100 cm and length of 1200 cm. The tank truck is filled with gasoline with a surface width of 150 cm.
	
	\begin{vwcol}[widths={0.8,0.2},rule=0pt,sep=1em]
		        
		\vspace{1em}
		\noindent The right figure shows a standard-sized oil drum with a diameter of $58$ cm and a height of $89$ cm. How many oil drums' worth of gasoline does the amount in the tanker truck contains?
		
		\includegraphics[scale=0.18]{assets/8-25.png}
	\end{vwcol}
	
	\sol{}
	\begin{multicols}{2}
		\vspace*{-4em}
		\begin{flalign*}
			& P O=Q O=100 \mathrm{~cm} &\\
			& P M=M Q=75 \mathrm{~cm}
		\end{flalign*}
		\noindent In $\triangle P M O$,
		\begin{flalign*}
			\sin \angle P O M & =\frac{75}{100} & M O & =\sqrt{O P^2-P M^2} &&&&&&&&\\
			\angle P O M & =0.8481 & & =\sqrt{100^2-75^2} \\
			\angle P O Q & =1.6961 & & =\sqrt{4375} \mathrm{~cm}
		\end{flalign*}
		\includegraphics[scale=0.12]{assets/8-26.png}
	\end{multicols}
	\vspace{-3em}
	\begin{flalign*}
		\text {Cross-sectional area of gasoline} & =\text {Area of circle}-(\text {Area of sector } P O Q - \text {Area of }\triangle P O Q ) \\
		& =\pi(100)^2-\frac{1}{2}(100)^2(1.6961)+\frac{1}{2}(150)(\sqrt{4375}) &\\
		& =27896.21 \mathrm{~cm}^2
	\end{flalign*}
	\vspace{-3em}
	\begin{flalign*}
		\text{Amount of oil drum} &= \text{Total volume of gasoline} \div \text{Volume of an oil drum} &\\
		&=\frac{27896.71 \times 1200}{\pi \times 29^2 \times 89} \approx 142.36
	\end{flalign*}
	
\end{question}

\newpage

\exercise{8.2b}

\begin{enumerate}
	\item \begin{multicols}{2}
	      The school's dance club bought 15 fans with a radius of $20 \mathrm{~cm}$ each. Each fan can be opened up to a maximum of $160^{\circ}$, as shown in the image on the right. If the leader wants to add red ribbons with a width of $12 \mathrm{~cm}$ around the front and back fan surfaces, how long is the red ribbon needed for all the fans?
	      
	      \begin{center}
	      	\includegraphics[scale=0.14]{assets/8-27.png}
	      \end{center}
	\end{multicols}
	
	\item \begin{multicols}{2}
	      Kuala Lumpur, Malaysia, and Lijiang City, Yunnan Province, China, are located on the same meridian. Their latitudes are respectively $3^\circ 09'$ north and $26^\circ 51'$ north. Assuming the Earth is a sphere with a radius of $6371 \mathrm{~km}$, estimate the distance between Kuala Lumpur and Lijiang City.
	      
	      \begin{center}
	      	\includegraphics[scale=0.14]{assets/8-28.png}
	      \end{center}
	\end{multicols}
	
	\item \begin{multicols}{2}
	      The image on the right is a wall decoration made by a student by sticking three bottle caps together, each with a radius of $29.6 \mathrm{~mm}$ and a height of $19 \mathrm{~mm}$.
	              
	      \begin{enumerate}
	      	\item If he wants to fill the gap between the three bottle caps (the shaded region in the image) with clay, how much clay in $\mathrm{mm}^3$ does he need?
	      	                  
	      	\item Hence, if he wants to place a flower shaped decoration made up of 7 bottle caps each in the four corners of the wall decoration (as shown in the image on the right), and fill the gaps between the bottle caps with clay, how much clay in total does he need in $\mathrm{mm}^3$?    
	      	      \columnbreak
	      	                  
	      	      \begin{center}
	      	      	\includegraphics[scale=0.14]{assets/8-29.png}
	      	      \end{center}
	      	      \begin{center}
	      	      	\includegraphics[scale=0.14]{assets/8-30.png}
	      	      \end{center}
	      \end{enumerate}
	\end{multicols}
\end{enumerate}

\newpage

\revision{8}

\begin{enumerate}
	\item Convert the following angles to radians (express the answers in terms of $\pi$):
	      \begin{multicols}{3}
	      	\begin{enumerate}
	      		\item $75^\circ$
	      		\item $-22.5^\circ$
	      		\item $540^\circ$
	      	\end{enumerate}
	      \end{multicols}
	      
	\item Convert the following angles to radians:
	      \begin{multicols}{3}
	      	\begin{enumerate}
	      		\item $17^\circ 50'$
	      		\item $322^\circ 5'$
	      		\item $\pi^\circ$
	      	\end{enumerate}
	      \end{multicols}
	      
	\item Convert the following radians to degrees:
	      \begin{enumerate}
	      	\begin{multicols}{3}
	      		\item $\dfrac{7 \pi}{4}$
	      		\item $-2.5$
	      		\item $\dfrac{5 \pi}{9}$
	      	\end{multicols}
	      \end{enumerate}
	      
	\item Given a sector with a radius of $5 \mathrm{~cm}$ and an arc length of $12 \mathrm{~cm}$, express the central angle of this sector in radians. Hence, calculate its area.
	          
	\item \begin{multicols}{2}
	      As shown in the figure on the right, the center of the small circle is $A$, with a radius of 4; the center of the large circle is $B$, with a radius of 7. The two circles touch at point $P$ and are tangent to the line $CD$ at points $D$ and $C$, respectively. Find:
	      \begin{enumerate}
	      	\item $\angle ABC$ and $\angle BAD$, in radian;
	      	\item The area of the shaded region.
	      \end{enumerate}
	      
	      \begin{center}
	      	\includegraphics[scale=0.1]{assets/8-31.png}
	      \end{center}
	\end{multicols}
	
	\item \begin{multicols}{2}
	      The figure on the right shows two semicircles with centers at points $B$ and $C$. If $AB=a$,
	      \begin{enumerate}
	      	\item Find the perimeter of the shaded region;
	      	\item Prove that the area of the shaded region is $\dfrac{a^2}{12}(2 \pi+3 \sqrt{3})$.
	      \end{enumerate}
	      
	      \begin{center}
	      	\includegraphics[scale=0.1]{assets/8-32.png}
	      \end{center}
	\end{multicols}
	
	\item \begin{multicols}{2}
	      In the figure on the right, two circles with radii of $8 \mathrm{~cm}$ and $5 \mathrm{~cm}$ respectively intersect at points $A$ and $B$. $C$ and $D$ are the centers of the two circles.
	      \begin{enumerate}
	      	\item Find $\angle ACB$ and $\angle ADB$, in radian;
	      	\item Find the perimeter of the shaded region.
	      \end{enumerate}
	      \vfill\null
	      
	      \begin{center}
	      	\includegraphics[scale=0.1]{assets/8-33.png}
	      \end{center}
	\end{multicols}
	
	\newpage
	\item As shown in the diagram below, a sector with a radius of $20 \mathrm{~cm}$ and a central angle of $135^\circ$ forms a hollow cone.
	      \begin{center}
	      	\includegraphics[scale=0.12]{assets/8-34.png}
	      \end{center}
	      \begin{enumerate}
	      	\item Find the radius and volume of this cone.
	      	\item If the radii of the sectors are the same but the central angles are different, will the volumes of the cone formed by these sectors be the same?
	      \end{enumerate}
	          
	      
	\item John bought a cake with dimensions as shown in the diagram below. This cake is cut from a cylindrical cake with a height of $12 \mathrm{~cm}$, where $O$ is the center of the cross-section of the cake. If the shopkeeper sells a piece of cake cut into $n$ equal parts,
	      \begin{center}
	      	\includegraphics[scale=0.12]{assets/8-35.png}
	      \end{center}
	      \begin{enumerate}
	      	\item What is the possible values of $n$?
	      	\item What is the volume of the remaining cake that is less than one piece after being cut into $n$ equal parts?
	      	\item Based on the value of $n$ in part (a), if the entire cake were to be cut into $n$ equal parts, what should be the central angle of each piece of cake in radian?
	      \end{enumerate}
	      
	\item The design of a swimming pool in an apartment complex is shown below, divided into an adult area and a children's area. The adult area consists of a rectangular section $BCED$ and two semicircular sections, where $BC = 40$ meters, $CE = 20$ meters, and the depth is $1.3$ meters. The children's area consists of the a major arc $BFD$ with $A$ as the center and the semicircular arc $BAD$, with a depth of $0.6$ meters. To disinfect the swimming pool regularly, special disinfection tablets need to be added. It is known that 20 grams of disinfection tablets are required per cubic meter of water. How many grams of disinfection tablets are needed for the adult area and the children's area respectively?
	      \begin{center}
	      	\includegraphics[scale=0.15]{assets/8-36.png}
	      \end{center}
	      
\end{enumerate}

\end{document}