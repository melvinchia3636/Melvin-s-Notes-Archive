% chktex-file 2
% chktex-file 29
% chktex-file 13
\documentclass[12pt]{report}
\usepackage{setspace}
\usepackage[a4paper, total={7in, 10in}]{geometry}
\usepackage[fleqn]{amsmath}
\usepackage{empheq}
\usepackage{amssymb}
\usepackage{amsthm}
\usepackage{gensymb}
\usepackage[fleqn]{cases}
\usepackage{multicol}
\usepackage{color}
\usepackage{stix}
\usepackage{chngcntr}
\usepackage{tikz}
\usepackage{enumitem}
\usepackage{pgfplots}
\usepackage{etoolbox}
\usepackage{tkz-euclide}
\usepackage{graphicx}
\usepackage{enumitem}
\usepackage{multirow}
\usepackage{mathtools}
\usepackage{mdframed}
\usepackage{adjustbox}
\usepackage{xpatch}
\usepackage{nicematrix}
\usepackage{ifthen}

\def\nswe#1#2#3{#1\,$#2^\circ\,#3'$}
\graphicspath{ {./assets/} }
\usetikzlibrary{calc,trees,positioning,arrows,fit,shapes,calc, decorations.markings}
\newcommand{\midarrow}{\tikz \draw[-triangle 90] (0,0) -- +(.1,0);}

\newcommand\typel[2]{
    \mathbin{\mathop{#1\kern0pt}%
        \limits_{\raisebox{3.6ex}{\hbox to0pt{\hss\strut$\uparrow$\hss}}\hbox to0pt{\hss#2\hss}}}
}

\newcommand\typem[2]{
    \mathbin{\mathop{#1\kern0pt}%
        \limits^{\raisebox{3.6ex}{\hbox to0pt{\hss#2\hss}}\hbox to0pt{\hss\strut$\downarrow$\hss}}}
}

\counterwithout{equation}{chapter}

\newcommand{\pgfplotsdrawaxis}{\pgfplots@draw@axis}
\newcommand\perm[2][^n]{\prescript{#1\mkern-2.5mu}{}P_{#2}}
\newcommand\permtwo[2][^n]{{}_{#1}P_{#2}}
\newcommand\comb[2][^n]{{}_{#1}C_{#2}}
\newcommand\combtwo[2][^n]{\prescript{#1\mkern-2.5mu}{}C_{#2}}
\makeatother
\pgfplotsset{only axis on top/.style={axis on top=false, after end axis/.code={
                    \pgfplotsset{axis line style=opaque, ticklabel style=opaque, tick style={thick,opaque},
                        grid=none}\pgfplotsdrawaxis}}}

\newtheorem{theorem}{Theorem}

\makeatletter
\xpatchcmd{\endmdframed}
{\aftergroup\endmdf@trivlist\color@endgroup}
{\endmdf@trivlist\color@endgroup\@doendpe}
{}{}
\makeatother

\mdfdefinestyle{MyFrame}{%
    linecolor=black,
    linewidth=1pt,
    roundcorner=20pt, innertopmargin=20pt,innerbottommargin=20pt, innerrightmargin=12pt,
    innerleftmargin=12pt, skipbelow=20pt, skipabove=20pt
    %backgroundcolor=gray!50!white}
}

\newcommand{\newitem}[1]{%
    \refstepcounter{subenum}%
    \parbox{\dimexpr.5\linewidth-.5\columnsep}{
        \makebox[\labelwidth][r]{(\thesubenum)\hspace*{\labelsep}} #1}\hfill }%%%

\setcounter{chapter}{21}

\setlength{\arrayrulewidth}{1pt}
\setlength{\tabcolsep}{12pt}

\begin{document}

\newcommand{\sol}[1]{

    \noindent \textbf{Sol.}
}
\newcommand{\prooff}[1]{

    \noindent \textbf{Proof.}
}

\newcommand{\sxrightarrow}[2][]{%
    \mathrel{\text{$\xrightarrow[#1]{#2}$}}%
}

\newenvironment{cequation}{
    \makeatletter
    \setbool{@fleqn}{false}
    \makeatother
    \begin{equation*}
        }{\end{equation*}}

\begin{titlepage}
    \raggedleft{}
    \rule{1pt}{\textheight}
    \hspace{0.02\textwidth}
    \parbox[b]{0.75\textwidth}{

    {\fontsize{40}{60}\selectfont\bfseries Mathematics}\\[2\baselineskip]
    {\huge\textit{Senior 3 Part I}}\\[4\baselineskip]
    {\Large\textsc{Melvin Chia}}

    \vspace{0.5\textheight}

    {\noindent Started on 10 April 2023}\\[\baselineskip]
    {\noindent Finished on XX XX 2023}\\[\baselineskip]
    {\noindent Actual time spent: XX days}\\[\baselineskip]}

\end{titlepage}

\chapter*{Introduction}
\addcontentsline{toc}{chapter}{Introduction} \markboth{INTRODUCTION}{}

\doublespacing{}
\section*{Why this book?}

\section*{Disclaimer}
\section*{Acknowledgements}

\singlespacing{}

\doublespacing{}
\tableofcontents
\singlespacing{}
\newpage

\onehalfspacing

\section{Compound Interest and Annuity}

Simple interest and compound interest are two different methods of calculating
interest. Simple interest is calculated on the principal amount of a loan only.
Compound interest is calculated on the principal amount and also on the
accumulated interest of previous periods, and can thus be regarded as “interest
on interest.”

For example, a fund amounted to RM$p$ is deposited into a bank account with a
yearly interest rate of $r\%$.
\begin{flalign*}
    \text{Principal amount} & = \text{RM} p
\end{flalign*}
When $t = 1$,
\begin{flalign*}
    \text{Interest earned}    & = p \times r\% = \frac{pr}{100}                        \\
    \text{Accumulated amount} & = p + \frac{pr}{100} = p\left(1 + \frac{r}{100}\right)
\end{flalign*}
When $t = 2$,
\begin{flalign*}
    \text{Interest earned}    & = \left(p + \frac{pr}{100}\right) \times r\% = \frac{pr}{100}\left(1 + \frac{r}{100}\right) \\
    \text{Accumulated amount} & = p\left(1 + \frac{r}{100}\right) + \frac{pr}{100}\left(1 + \frac{r}{100}\right)            \\
                              & = p\left(1 + \frac{r}{100}\right)\left(1 + \frac{r}{100}\right)                             \\
                              & = p{\left(1 + \frac{r}{100}\right)}^{2}
\end{flalign*}
When $t = 3$,
\begin{flalign*}
    \text{Interest earned}    & = \left(p{\left(1 + \frac{r}{100}\right)}^{2}\right) \times r\% = \frac{pr}{100}{\left(1 + \frac{r}{100}\right)}^{2} \\
    \text{Accumulated amount} & = p{\left(1 + \frac{r}{100}\right)}^{2} + \frac{pr}{100}{\left(1 + \frac{r}{100}\right)}^{2}                         \\
                              & = p{\left(1 + \frac{r}{100}\right)}^{2}\left(1 + \frac{r}{100}\right)                                                \\
                              & = p{\left(1 + \frac{r}{100}\right)}^{3}
\end{flalign*}

In general, the accumulated amount after $t$ years is given by
\begin{mdframed}[style=MyFrame]
    \setlength{\abovedisplayshortskip}{0pt}
    \setlength{\belowdisplayshortskip}{0pt}
    \setlength{\abovedisplayskip}{0pt}
    \setlength{\belowdisplayskip}{0pt}
    \makeatletter
    \setbool{@fleqn}{false}
    \makeatother
    \begin{flalign*}
        A & = p{\left(1 + \frac{r}{100}\right)}^{t}
    \end{flalign*}
    \makeatletter
    \setbool{@fleqn}{true}
    \makeatother
\end{mdframed}
where $p$ is called the \textit{present value} of $A$.

\newpage
If the interest is compounded $m$ times per year, then the accumulated amount
is given by
\begin{mdframed}[style=MyFrame]
    \setlength{\abovedisplayshortskip}{0pt}
    \setlength{\belowdisplayshortskip}{0pt}
    \setlength{\abovedisplayskip}{0pt}
    \setlength{\belowdisplayskip}{0pt}
    \makeatletter
    \setbool{@fleqn}{false}
    \makeatother
    \begin{flalign*}
        A & = p{\left(1 + \frac{r}{100m}\right)}^{mt}
    \end{flalign*}
    \makeatletter
    \setbool{@fleqn}{true}
    \makeatother
\end{mdframed}

\subsection*{Annuity and Present Value of Annuity}

An annuity is a series of equal payments made at equal intervals of time
according to some kind of contract, standing order or the amount received. For
example, all sorts of instalment, insurance premiums, house rent, car loan,
etc. are annuities. In this book, we will only consider annuities with equal
payments made or received at equal intervals of time.

Note that the annuity is not limited to once a year.

We can compare which payment plan is better by comparing the present values of
the annuities. From the formula $A = p{\left(1 + r\%\right)}^{t}$, we can know
that the present value $p = \dfrac{A}{{\left(1 + r\%\right)}^{t}}$. If the
yearly interest rate is $r\%$, the annuity is RM$A$, the payment is made once
per year, then the present value of the amount paid after a year is $A(1 +
    r\%)^{-1}$, the present value of the amount paid after two years is $A(1 +
    r\%)^{-2}$, and so on. The present value of the amount paid after $n$ years is
$A{(1 + r\%)}^{-n}$. Hence, the sum of the present values of the amount paid
after $n$ years is
\begin{flalign*}
     & \dfrac{A}{1+r\%} + \dfrac{A}{{(1+r\%)}^2} + \cdots + \dfrac{A}{{(1+r\%)}^n}                 \\
     & = A\left[\dfrac{1}{1+r\%} + \dfrac{1}{{(1+r\%)}^2} + \cdots + \dfrac{1}{{(1+r\%)}^n}\right] \\
     & = A\left[\dfrac{1-\dfrac{1}{{(1+r\%)}^n}}{1-\dfrac{1}{1+r\%}}\right]                        \\
     & = \dfrac{A}{r\%}\left(1 - \dfrac{1}{{(1+r\%)}^n}\right)
\end{flalign*}

Annuity that is paid indefinitely is called \textit{perpetuity}, $n \to
    \infty$, $\dfrac{1}{{(1+r\%)}^n} \to 0$. From that, we can know that the
present value of perpetuity is $\dfrac{A}{r\%}$.

\newpage
\subsection*{Practice 8}

\begin{enumerate}
    \item Given that the principal amount is RM75,000, interest rate is 4.5\%. Using
          composite interest method, find the accumulated amount after 10 years. \sol{}
          \begin{flalign*}
              A & = p{\left(1 + \frac{r}{100}\right)}^{t}        \\
                & = 75000{\left(1 + \frac{4.5}{100}\right)}^{10} \\
                & = 75000{\left(\frac{104.5}{100}\right)}^{10}   \\
                & = 75000(1.045)^{10}                            \\
                & = \text{RM}116,472.71
          \end{flalign*}

    \item A person has deposited RM40,000 into a bank account. The bank pays 8\% interest
          per annum compounded half yearly. Using the compound interest method, find the
          amount in the account after 3 years. \sol{}
          \begin{flalign*}
              A & = p{\left(1 + \frac{r}{100m}\right)}^{mt}                 \\
                & = 40000{\left(1 + \frac{8}{100\times2}\right)}^{2\times3} \\
                & = 40000{\left(\frac{208}{200}\right)}^{6}                 \\
                & = 40000(1.04)^{6}                                         \\
                & = \text{RM}50,612.76
          \end{flalign*}

    \item Given that the interest rate is 6\%, the interest is compounded half yearly.
          Using the compound interest method, the accumulated amount after 5 years is
          RM4031.75, find the principal amount. \sol{}
          \begin{flalign*}
              A & = p{\left(1 + \frac{r}{100m}\right)}^{mt}                             \\
              p & = \frac{A}{{\left(1 + \dfrac{r}{100m}\right)}^{mt}}                   \\
                & = \frac{4031.75}{{\left(1 + \dfrac{6}{100\times2}\right)}^{2\times5}} \\
                & = \frac{4031.75}{1.03^{10}}                                           \\
                & = \text{RM}3000
          \end{flalign*}

    \item Given that the interest rate is 4\%, the annuity is RM3,500, the payment is
          made once per year. The payment has since been made for 15 years continuously.
          Find the present value. Hence, find the present value of the perpetuity. \sol{}
          \begin{flalign*}
              p & = \frac{A}{r\%}\left(1 - \frac{1}{{(1+r\%)}^n}\right)       \\
                & = \frac{3500}{4\%}\left(1 - \frac{1}{{(1+4\%)}^{15}}\right) \\
                & = \text{RM}38,914.36
          \end{flalign*}
          \vspace{-2em}
          \begin{flalign*}
              \text{Present value of perpetuity} & = \frac{3500}{4\%} \\
                                                 & = \text{RM}87,500
          \end{flalign*}
\end{enumerate}

\subsection*{Exercise 23.6}

\begin{enumerate}
    \item Given that the principal amount is RM90,000, the interest rate is 5\%.
          Compounding the interest once per year, find the accumulated amount after 10
          years. \sol{}
          \begin{flalign*}
              A & = p{\left(1 + \frac{r}{100}\right)}^{t}    \\
                & = 90000{\left(\frac{105}{100}\right)}^{10} \\
                & = 90000(1.05)^{10}                         \\
                & = \text{RM}146,600.52
          \end{flalign*}

    \item A person has deposited a fund into a bank account. The bank pays 8\% interest
          per annum compounded yearly. The amount in the account after 3 years has
          increased by RM779.14. Find the amount of the fund deposited. \sol{}
          \begin{flalign*}
              A - p                                   & = 779.14                     \\
              p\left(1 + \frac{8}{100}\right)^{3} - p & = 779.14                     \\
              p{\left(1.08\right)}^{3} - p            & = 779.14                     \\
              p(1.08^3 - 1)                           & = 779.14                     \\
              p                                       & = \dfrac{779.14}{1.08^3 - 1} \\
                                                      & = \text{RM}3,000.02
          \end{flalign*}

    \item RM80,000 was deposited into a financial institution. The interest rate is 8\%
          per annum compounded once per three months. Find the amount in the account
          after 5 years. \sol{}
          \begin{flalign*}
              A & = p{\left(1 + \frac{r}{100m}\right)}^{mt}                 \\
                & = 80000{\left(1 + \frac{8}{100\times4}\right)}^{4\times5} \\
                & = 80000{\left(\frac{408}{400}\right)}^{20}                \\
                & = 80000(1.02)^{20}                                        \\
                & = \text{RM}118,875.79
          \end{flalign*}

    \item Prove that the accumulated amount after being compounded with an interest of
          $5$ for 15 years will exceed twice the principal amount. \prooff{}
          \begin{flalign*}
              A & = p{\left(1 + \frac{r}{100}\right)}^{t}  \\
                & = p{\left(1 + \frac{5}{100}\right)}^{15} \\
                & = p(1.05)^{15}                           \\
                & \approx 2.078p > 2p \qed
          \end{flalign*}

    \item Given that the principal amount is RM15,000, the interest rate is 6\% being
          compounded once per year. How long does it take for the accumulated amount to
          be more than RM300,000? \sol{}
          \begin{flalign*}
              A           & = p{\left(1 + \frac{r}{100}\right)}^{t}     \\
                          & = 15000{\left(1 + \frac{6}{100}\right)}^{t} \\
                          & = 15000(1.06)^{t} > 300000                  \\
              1.06^t      & > 2                                         \\
              \log 1.06^t & > \log 20                                   \\
              t\log 1.06  & > \log 20                                   \\
              t           & > \frac{\log 20}{\log 1.06}                 \\
              t           & > 51.41                                     \\
              t           & = 52 \text{ years}
          \end{flalign*}

          \newpage
    \item Given that the principal amount is RM120,000, the interest rate is 5.5\% being
          compounded half yearly. How long does it take for the accumulated amount to be
          more than RM200,000? \sol{}
          \begin{flalign*}
              A             & = p{\left(1 + \frac{r}{100m}\right)}^{mt}              \\
                            & = 120000{\left(1 + \frac{5.5}{100\times2}\right)}^{2t} \\
                            & = 120000{\left(\frac{205.5}{200}\right)}^{2t}          \\
                            & = 120000(1.0275)^{2t} > 200000                         \\
              1.0275^{2t}   & > \frac{5}{3}                                          \\
              2t\log 1.0275 & > \log \frac{5}{3}                                     \\
              t             & > \frac{\log \frac{5}{3}}{2\log 1.0275}                \\
              t             & >  9.41                                                \\
              t             & = 9.5 \text{ years}
          \end{flalign*}

    \item A person deposited RM2,500 into his bank account at the beginning of every
          year, the interest rate is 4.5\% compounded once per year. Find the amount in
          the account after 15 years. \sol{}
          \begin{flalign*}
              A & = 2500\left(1 + \frac{4.5}{100}\right) + 2500\left(1 + \frac{4.5}{100}\right)^2 + \cdots + 15\cdot2500\left(1 + \frac{4.5}{100}\right)^{15} \\
                & = 2500(1.045 + 1.045^2 + \cdot + 1.045^{15})                                                                                                \\
                & = 1000\times\dfrac{1.045(1.045^{15} - 1)}{1.045 - 1}                                                                                        \\
                & = \text{RM}54,298.34
          \end{flalign*}

    \item If the present value is RM15,443.46, the interest rate is 5\%, find the annuity
          if the payment is made for 10 years. \sol{}
          \begin{flalign*}
              \text{Present value} & = \dfrac{A}{r\%}\left(1 - \frac{1}{\left(1 + r\%\right)^t}\right) \\
              15443.46             & = \dfrac{A}{5\%}\left(1 - \frac{1}{1.05^{10}}\right)              \\
              A                    & = \dfrac{15443.46 \times 5\%}{1 - \dfrac{1}{1.05^{10}}}           \\
                                   & = \text{RM}2,000
          \end{flalign*}

    \item Given that the annuity is RM5,000, the interest rate is 5\%, the payment is
          made once per year for 25 years. Find the present value. Hence, find the
          present value of the perpetuity. \sol{}
          \begin{flalign*}
              \text{Present value} & = \dfrac{A}{r\%}\left(1 - \frac{1}{\left(1 + r\%\right)^t}\right) \\
                                   & = \dfrac{5000}{5\%}\left(1 - \frac{1}{1.05^{25}}\right)           \\
                                   & = \text{RM}70,469.72
          \end{flalign*}
          \vspace{-2em}
          \begin{flalign*}
              \text{Present value of perpetuity} & = \dfrac{A}{r\%} = \dfrac{5000}{5\%} = \text{RM}100,000
          \end{flalign*}

    \item Given that the annuity is RM2,500, the interest rate is 4.5\%, the payment is
          made once per year. How many years does it take for the present value to exceed
          RM30,000? \sol{}
          \begin{flalign*}
              \text{Present value}                   & = \dfrac{A}{r\%}\left(1 - \frac{1}{\left(1 + r\%\right)^t}\right) \\
                                                     & = \dfrac{2500}{4.5\%}\left(1 - \frac{1}{1.045^t}\right) > 30000   \\
              2500\left(1 - \frac{1}{1.045^t}\right) & > 1350                                                            \\
              1 - \frac{1}{1.045^t}                  & > \frac{27}{50}                                                   \\
              1.045^t                                & > \frac{50}{23}                                                   \\
              t\log 1.045                            & > \log \frac{50}{23}                                              \\
              t                                      & > \frac{\log \dfrac{50}{23}}{\log 1.045} \approx  17.64           \\
              t                                      & = 18 \text{ years}
          \end{flalign*}

    \item If a bank has introduced an annuity scheme, the investors can receive RM1,000
          per year for life after paying RM20,000. If the annuity plan is considered
          approximately to be a perpetuity, find the interest rate. \sol{}
          \begin{flalign*}
              \text{Present value of perpetuity} & = \dfrac{A}{r\%}            \\
                                                 & = \dfrac{1000}{r\%}         \\
                                                 & = 20000                     \\
              r\%                                & = \dfrac{1000}{20000} = 5\%
          \end{flalign*}
\end{enumerate}

\newpage
\section*{Revision Exercise 23}
\begin{enumerate}
    \item Without using a calculator, find the value of the following:
          \begin{enumerate}
              \item ${\left({\frac{1}{2}}\right)}^{2}+{\left({\frac{1}{2}}\right)}^{6}+{\left(-{\frac{1}{2}}\right)}^{-2}$
              \item $5^{\frac{1}{2}}+5^{-{\frac{1}{2}}}-{\left(\dfrac{1}{5}\right)}^{\frac{1}{2}}+{\left(\dfrac{1}{5}\right)}^{-{\frac{1}{2}}}$
              \item ${\left({\frac{1}{3}}\right)}^{2}\times{\left(-{\frac{1}{3}}\right)}^{2}\times{\left({\frac{1}{2}}\right)}^{-2}$
              \item ${\sqrt{2{\sqrt[3]{3}}}}\div{\sqrt[3]{\dfrac{\sqrt{8}}{3}}}$
          \end{enumerate}

    \item Simplify the following expressions:
          \begin{enumerate}
              \item ${\left({\dfrac{b}{2a^{2}}}\right)}^{3}\div{\left({\dfrac{2b^{2}}{3a}}\right)}^{0}\times{\left(-{\dfrac{b}{a}}\right)}^{-3}$
              \item $\dfrac{3^{n+2}-2\times3^{n}}{5(3^{n+1})}$
              \item $\dfrac{\left(x^{-1}+y^{-1}\right)\left(x^{-1}-y^{-1}\right)}{x^{-2}y^{-2}}$
              \item $\left({x^{\frac{1}{4}}}-y^{-{\frac{1}{4}}}\right)\left(x^{\frac{1}{2}}+y^{-{\frac{1}{2}}}\right)\left(x^{\frac{1}{4}}+y^{-{\frac{1}{4}}}\right)$
              \item $\frac{{\left({\sqrt[4]{p^{3}}}\right)}^{\frac{1}{6}}{\sqrt[9]{p^{-3}}}}{{\left(\sqrt{p^{-7}}\right)}^{\frac{1}{6}}}$
              \item $\dfrac{{\left(a^{2}+a^{-2}+2\right)}^{2}}{{\left(a^{2}+1\right)}^{4}}$
          \end{enumerate}

    \item Without using a calculator, compare the value of the following:
          \begin{enumerate}
              \item $2.3^{-2}$ and $2.3^{-1}$
              \item $0.15^{-{\frac{1}{2}}}$ and ${0.15}^{-{\frac{1}{3}}}$
              \item ${\left({\dfrac{1}{3}}\right)}^{\frac{2}{5}}$ and $ 3^{-{\frac{5}{3}}}$
              \item ${\left({\dfrac{3}{5}}\right)}^{2}$ and ${\left({\dfrac{5}{3}}\right)}^{3.1}$
          \end{enumerate}

    \item Without using a calculator, compare the value of the following:
          \begin{enumerate}
              \item $\log_{3.2}3$ and $\log_{3.2}2$
              \item $\log_{0.5}5.3$ and $\log_{0.5}3.5$
              \item $\log_{3}2$ and $\log_{2}3$
              \item $\log_{2}2.3$ and $\log_{4}4.8$
          \end{enumerate}

    \item Find the domain of the following functions:
          \begin{enumerate}
              \item $y=\log_{0.5}\left(16-x^{2}\right)$
              \item $y=\log_{2}\left(2x^{2}-5x-12\right)$
              \item $y=\sqrt{3-3^{x}}$
              \item $y=\log_{2}{(x-3)}^{2}$
              \item $y=\log_{5}\left(x^{2}-2x\right)$
              \item $y=\log_{3}{\dfrac{2}{3-x}}$
              \item $y=\dfrac{1}{\log(x+1)-1}$
              \item $y=\dfrac{\log_3\left(2-x\right)}{\log_3\left(2+x\right)}$
              \item $y=\sqrt{\log_{3}(x-2)}$
              \item $y={\dfrac{2}{\sqrt{1-\log x}}}$
          \end{enumerate}

    \item Simplify the following expressions:
          \begin{enumerate}
              \item $\log_{3}27^{x}$
              \item $\log_{x}b^{a\log_{b}x}$
              \item $\log_{5}\left(25^{x}\cdot5^{y}\right)$
              \item $3^{2\log_{3}x - \log_{3}y}$
              \item $5^{-2\log_{25}x}$
          \end{enumerate}

    \item If $\log_2 5 = p$, express $\log_2 100$ in terms of $p$.
    \item If $\log_3 12 = a$, express the following in terms of $a$:
          \begin{enumerate}
              \item $\log_3 24$
              \item $\log_9 36$
          \end{enumerate}

    \item Without using a calculator, find the value of the following:
          \begin{enumerate}
              \item $\log_{c}{\dfrac{1}{5}}+\log_{c}5$
                    \sol{}
                    \begin{flalign*}
                        \log_{c}{\dfrac{1}{5}}+\log_{c}5 & =\log_{c}{\dfrac{1}{5}\times5} & \\
                                                         & =\log_{c}1                     & \\
                                                         & =0
                    \end{flalign*}

              \item $\log_{2}\left(2{\sqrt{2}}\right)-2\log_{2}{\sqrt{2}}$
                    \sol{}
                    \begin{flalign*}
                        \log_{2}\left(2{\sqrt{2}}\right)-2\log_{2}{\sqrt{2}} & =\log_{2}\left(2{\sqrt{2}}\right)-\log_{2}2  & \\
                                                                             & =\log_{2}\left(\dfrac{2{\sqrt{2}}}{2}\right) & \\
                                                                             & =\log_{2}\sqrt{2}                            & \\
                                                                             & =\frac{1}{2}
                    \end{flalign*}

              \item $\log_8{\dfrac{2}{7}}-\log_{8}{(-2)}^{2}-\log_{8}{\dfrac{1}{7}}$
                    \sol{}
                    \begin{flalign*}
                        \log_8{\dfrac{2}{7}}-\log_{8}{(-2)}^{2}-\log_{8}{\dfrac{1}{7}} & =\log_8{\dfrac{2}{7}}-\log_{8}4-\log_{8}{\dfrac{1}{7}}      & \\
                                                                                       & =\log_8{\left(\dfrac{2}{7}\times\dfrac{1}{4}\times7\right)} & \\
                                                                                       & =\log_8\dfrac{1}{2}                                         & \\
                                                                                       & =-\log_8{2}                                                 & \\
                                                                                       & =-\dfrac{1}{3}
                    \end{flalign*}

              \item $\log{\dfrac{5}{32}}-2\log{\dfrac{5}{6}}+\log{\dfrac{40}{9}}$
                    \sol{}
                    \begin{flalign*}
                        \log{\dfrac{5}{32}}-2\log{\dfrac{5}{6}}+\log{\dfrac{40}{9}} & =\log{\dfrac{5}{32}}-\log{{\left(\dfrac{5}{6}\right)}^{2}}+\log{\dfrac{40}{9}} & \\
                                                                                    & =\log{\dfrac{5}{32}}-\log{\dfrac{25}{36}}+\log{\dfrac{40}{9}}                  & \\
                                                                                    & =\log{\left(\dfrac{5}{32}\times\dfrac{36}{25}\times\dfrac{40}{9}\right)}       & \\
                                                                                    & =\log1                                                                           \\
                                                                                    & =0
                    \end{flalign*}

              \item ${\big(}\log_{2}3{\big)}{\big(}\log_{3}4{\big)}$
                    \sol{}
                    \begin{flalign*}
                        {\big(}\log_{2}3{\big)}{\big(}\log_{3}4{\big)} & =\log_2 3 \cdot \dfrac{\log_2 4}{\log_2 3}     & \\
                                                                       & =\dfrac{1}{\log_3 2} \cdot \dfrac{2}{\log_2 3} & \\
                                                                       & =2
                    \end{flalign*}
              \item $\dfrac{\log_{16}5}{\log_{32}5}$
                    \sol{}
                    \begin{flalign*}
                        \dfrac{\log_{5}32}{\log_{5}16} & = \dfrac{\log_{5}2^5}{\log_{5}2^4} \\
                                                       & = \dfrac{5\log_{5}2}{4\log_{5}2}   \\
                                                       & = \dfrac{5}{4}
                    \end{flalign*}

              \item $\log_{3}5\cdot\log_{5}7\cdot\log_{7}27$
                    \sol{}
                    \begin{flalign*}
                        \log_{3}5\cdot\dfrac{\log_{3}7}{\log_{3}5}\cdot\dfrac{\log_{3}27}{\log_{3}7} & = \log_{3}5\cdot\dfrac{\log_{3}7}{\log_{3}5}\cdot\dfrac{3\log_{3}3}{\log_{3}7} \\
                                                                                                     & = 3\log_{3}3                                                                   \\
                                                                                                     & = 3
                    \end{flalign*}

              \item $\log_{2}{\dfrac{1}{9}}\cdot\log_{3}{\dfrac{1}{25}}\cdot\log_{5}\sqrt8$
                    \sol{}
                    \begin{flalign*}
                        \log_{2}{\dfrac{1}{9}}\cdot\log_{3}{\dfrac{1}{25}}\cdot\log_{5}\sqrt8 & = \dfrac{\log_3 \dfrac{1}{9}}{\log_3 2} \cdot \dfrac{\log_3 \dfrac{1}{25}}{\log_3 3} \cdot \dfrac{\log_3 \sqrt{8}}{\log_3 5} & \\
                                                                                              & = \dfrac{-2\log_3 3}{\log_3 2} \cdot \dfrac{-2\log_3 5}{1} \cdot \dfrac{\dfrac{3}{2}\log_3 2}{\log_3 5}                      & \\
                                                                                              & = -2 \cdot (-2) \cdot \dfrac{3}{2}                                                                                           & \\
                                                                                              & = 6
                    \end{flalign*}

              \item $\dfrac{1}{3}\log_{2}8+\log_{3}27-\dfrac{1}{4}\log_{4}16$
                    \sol{}
                    \begin{flalign*}
                        \dfrac{1}{3}\log_{2}8+\log_{3}27-\dfrac{1}{4}\log_{4}16 & = \dfrac{1}{3}\cdot 3 + 3 - \dfrac{1}{4}\cdot 2 & \\
                                                                                & = 1 + 3 - \dfrac{1}{2}                          & \\
                                                                                & = \dfrac{7}{2}
                    \end{flalign*}
              \item $\log^{2}2+\log2\cdot\log5+\log5$
                    \sol{}
                    \begin{flalign*}
                        \log^{2}2+\log2\cdot\log5+\log5 & = \log2\log2+\log2\cdot\log5+\log5 \\
                                                        & = \log2(\log2+\log5)+\log5         \\
                                                        & = \log2\log10+\log5                \\
                                                        & = \log2+\log5                      \\
                                                        & = \log10                           \\
                                                        & = 1
                    \end{flalign*}

              \item $2\log_{3}15+3\log_{3}12-\log_{3}25-6\log_{3}2$
                    \sol{}
                    \begin{flalign*}
                        2\log_{3}15+3\log_{3}12-\log_{3}25-6\log_{3}2 & = \log_3 15^2 + \log_3 12^3 - \log_3 25 - \log_3 2^6                        & \\
                                                                      & = \log_3 \dfrac{15^2 \cdot 12^3}{5^2 \cdot 2^6}                             & \\
                                                                      & = \log_3 \dfrac{3^2 \cdot 5^2 \cdot 2^3 \cdot 3^3 \cdot 2^3}{5^2 \cdot 2^6} & \\
                                                                      & = \log_3 3^5
                                                                      & = 5
                    \end{flalign*}

              \item ${\dfrac{\log{\sqrt{27}}+\log{\sqrt{8}}-\log{\sqrt{125}}}{\log6-\log5}}$
              \item \resizebox{15.5em}{!}{$\log_{8}\left(\log_{2}{\sqrt{8+4{\sqrt{3}}}}+\log_{2}{\sqrt{8-4\sqrt{3}}}\right)$}
          \end{enumerate}

    \item If $\log_3 2 = a$, $\log_2 5 = b$, prove that $\log_5 3 = \dfrac{1}{a(b+1)}$.
    \item If $\log_2 3 = p$, $\log_3 7 = q$, prove that $\log_21 14 =
              \dfrac{pq+1}{p(q+1)}$.
    \item Given that $2\log_5(x+y) = 1 + \log_5 x + \log_5 y$. Prove that $x^2 + y^2 =
              3xy$.
    \item Given that $x = 5^k$ and $y = 5^n$. Express the following in terms of $k$ and
          $n$:
          \begin{enumerate}
              \item $\log_{5}{\dfrac{xy^3}{125}}$
              \item $\log_{25}\left(5{\sqrt{xy}}\right)$
          \end{enumerate}

    \item Given that $2 + \log_4 y = 2\log_16 x$. Express $x$ in terms of $y$.

    \item Solve the following exponential equations:
          \begin{enumerate}
              \begin{multicols}{2}
                  \item $3^{3x-2}=243$
                  \sol{}
                  \begin{flalign*}
                      3^{3x-2}      & =243          \\
                      \log 3^{3x-2} & =\log 3^5     \\
                      (3x-2)\log 3  & = 5\log 3     \\
                      3x-2          & =5            \\
                      3x            & =7            \\
                      x             & =\dfrac{7}{3}
                  \end{flalign*}
                  \vfill\null{}
                  \columnbreak{}
                  \item $4^{1-x}={\left({\dfrac{1}{8}}\right)}^{2x}$
                  \sol{}
                  \begin{flalign*}
                      4^{1-x}       & ={\left({\dfrac{1}{8}}\right)}^{2x}      \\
                      \log 4^{1-x}  & =\log {\left({\dfrac{1}{8}}\right)}^{2x} \\
                      (1-x)\log 2^2 & =2x\log 2^{-3}                           \\
                      2-2x          & =-6x                                     \\
                      -4x           & =2                                       \\
                      x             & =-\dfrac{1}{2}
                  \end{flalign*}
              \end{multicols}

              \begin{multicols}{2}
                  \item $2^{x^{2}}={\left(2^{x}\right)}^{2}$
                  \sol{}
                  \begin{flalign*}
                      2^{x^{2}}     & ={\left(2^{x}\right)}^{2} \\
                      x^{2}         & =2x                       \\
                      x^{2}-2x      & =0                        \\
                      x(x-2)        & =0                        \\
                      x         = 0 & \text{ or } x = 2
                  \end{flalign*}
                  \vfill\null{}
                  \columnbreak{}
                  \item $3^{5^x}=3$
                  \sol{}
                  \begin{flalign*}
                      3^{5^x}      & =3      \\
                      \log 3^{5^x} & =\log 3 \\
                      5^x\log 3    & =\log 3 \\
                      5^x          & =1      \\
                      \log 5^x     & =\log 1 \\
                      x\log 5      & =0      \\
                      x            & =0
                  \end{flalign*}
              \end{multicols}

              \begin{multicols}{2}
                  \item $5^{8^x}=625$
                  \sol{}
                  \begin{flalign*}
                      5^{8^x}      & =625                        \\
                      \log 5^{8^x} & =\log 625                   \\
                      8^x\log 5    & =\log 5^4                   \\
                      8^x          & =4                        & \\
                      \log 8^x     & =\log 4                     \\
                      x\log 8      & =\log 4                     \\
                      x            & =\dfrac{\log 4}{\log 8}     \\
                                   & =\dfrac{2\log 2}{3\log 2}   \\
                                   & =\dfrac{2}{3}
                  \end{flalign*}
                  \vfill\null{}
                  \columnbreak{}
                  \item $3^{x+1}=6^{x}$
                  \sol{}
                  \begin{flalign*}
                      3^{x+1}            & =6^{x}                                \\
                      \log 3^{x+1}       & =\log 6^{x}                         & \\
                      (x+1)\log 3        & =x\log 6                              \\
                      x\log 3+\log 3     & =x\log 6                              \\
                      x\log 3-x\log 6    & =-\log 3                              \\
                      x(\log 3-\log 6)   & =-\log 3                              \\
                      x\log \dfrac{1}{2} & =-\log 3                              \\
                      x                  & =\dfrac{-\log 3}{\log \dfrac{1}{2}}   \\
                                         & \approx 1.5850
                  \end{flalign*}
              \end{multicols}

              \newpage
              \begin{multicols}{2}
                  \item $7^{x}-7^{x-1}=6$
                  \sol{}
                  \begin{flalign*}
                      7^{x}-7^{x-1}        & = 6        \\
                      7^x - \dfrac{7^x}{7} & = 6        \\
                      \text{Let } y        & = 7^x      \\
                      y - \dfrac{y}{7}     & = 6      & \\
                      7y - y               & = 42       \\
                      6y                   & = 42       \\
                      y                    & = 7        \\
                      7^x                  & = 7        \\
                      \log 7^x             & = \log 7   \\
                      x\log 7              & = \log 7   \\
                      x                    & = 1
                  \end{flalign*}
                  \vfill\null{}
                  \columnbreak{}
                  \item $3^{x+1}=10\left(3^x\right)-3$
                  \sol{}
                  \begin{flalign*}
                      3^{2x+1}           & = 10\left(3^x\right)-3   \\
                      3^{2x} \cdot 3     & = 10\left(3^x\right)-3   \\
                      \text{Let } y      & = 3^x                    \\
                      3y^2               & = 10y - 3                \\
                      3y^2               & - 10y + 3 = 0            \\
                      (3y - 1)(y - 3)    & = 0                      \\
                      y = \dfrac{1}{3}   & \text{ or } y = 3        \\
                      \\
                      \text{When } y     & = \dfrac{1}{3},          \\
                      3^x                & = \dfrac{1}{3}           \\
                      \log 3^x           & = \log 3^{-1}            \\
                      x\log 3            & = -\log 3                \\
                      x                  & = -1                   & \\
                      \\
                      \text{When } y     & = 3,                     \\
                      3^x                & = 3                      \\
                      \log 3^x           & = \log 3                 \\
                      x\log 3            & = \log 3                 \\
                      x                  & = 1                      \\
                      \\
                      \therefore\ x = -1 & \text{ or } x = 1
                  \end{flalign*}
              \end{multicols}

              \newpage
              \begin{multicols}{2}
                  \item $2^{2x+1}=3\left(2^{x}\right)-1$
                  \sol{}
                  \begin{flalign*}
                      2^{2x+1}           & = 3\left(2^x\right)-1   \\
                      2^{2x} \cdot 2     & = 3\left(2^x\right)-1   \\
                      \text{Let } y      & = 2^x                   \\
                      2y^2               & = 3y - 1                \\
                      2y^2               & - 3y + 1 = 0            \\
                      (2y - 1)(y - 1)    & = 0                     \\
                      y = \dfrac{1}{2}   & \text{ or } y = 1       \\
                      \\
                      \text{When } y     & = \dfrac{1}{2},       & \\
                      2^x                & = \dfrac{1}{2}          \\
                      \log 2^x           & = \log 2^{-1}           \\
                      x\log 2            & = -\log 2               \\
                      x                  & = -1                    \\
                      \\
                      \text{When } y     & = 1,                    \\
                      2^x                & = 1                     \\
                      \log 2^x           & = \log 1                \\
                      x\log 2            & = 0                     \\
                      x                  & = 0                     \\
                      \\
                      \therefore\ x = -1 & \text{ or } x = 0
                  \end{flalign*}
                  \vfill\null{}
                  \columnbreak{}
                  \item $5^{2x+1}=26\left(5^{x}\right)-5$
                  \sol{}
                  \begin{flalign*}
                      5^{2x+1}           & = 26\left(5^x\right)-5   \\
                      5^{2x} \cdot 5     & = 26\left(5^x\right)-5   \\
                      \text{Let } y      & = 5^x                    \\
                      5y^2               & = 26y - 5                \\
                      5y^2               & - 26y + 5 = 0            \\
                      (5y - 1)(y - 5)    & = 0                      \\
                      y = \dfrac{1}{5}   & \text{ or } y = 5        \\
                      \\
                      \text{When } y     & = \dfrac{1}{5},        & \\
                      5^x                & = \dfrac{1}{5}           \\
                      \log 5^x           & = \log 5^{-1}            \\
                      x\log 5            & = -\log 5                \\
                      x                  & = -1                     \\
                      \\
                      \text{When } y     & = 5,                     \\
                      5^x                & = 5                      \\
                      \log 5^x           & = \log 5                 \\
                      x\log 5            & = \log 5                 \\
                      x                  & = 1                      \\
                      \\
                      \therefore\ x = -1 & \text{ or } x = 1
                  \end{flalign*}
              \end{multicols}

              \newpage
              \begin{multicols}{2}
                  \item $2^{2x+3}-2^{x}=1-2^{x+3}$
                  \sol{}
                  \begin{flalign*}
                      2^{2x+3}-2^{x}       & = 1-2^{x+3}                       \\
                      2^{2x} \cdot 8 - 2^x & = 1 - 2^x \cdot 8                 \\
                      \text{Let } y        & = 2^x                             \\
                      8y^2 - y             & = 1 - 8y                          \\
                      8y^2 + 7y - 1        & = 0                               \\
                      (8y - 1)(y + 1)      & = 0                               \\
                      y = \dfrac{1}{8}     & \text{ or } y = -1                \\
                      \\
                      \text{When } y       & = \dfrac{1}{8},                 & \\
                      2^x                  & = \dfrac{1}{8}                    \\
                      \log 2^x             & = \log 2^{-3}                     \\
                      x                    & = -3                              \\
                      \\
                      \text{When } y       & = -1,                             \\
                      2^x                  & = -1                              \\
                      \because\ 2^x        & > 0, \text{ no solution for } x   \\
                      \\
                      \therefore\ x        & = -3
                  \end{flalign*}
                  \vfill\null{}
                  \columnbreak{}
                  \item $2^{2x+8}-32\left(2^{x}\right)+1=0$
                  \sol{}
                  \begin{flalign*}
                      2^{2x+8}-32\left(2^{x}\right)+1     & = 0                \\
                      2^{2x} \cdot 256 - 32 \cdot 2^x + 1 & = 0                \\
                      \text{Let } y                       & = 2^x              \\
                      256y^2 - 32y + 1                    & = 0              & \\
                      {(16y - 1)}^2                       & = 0                \\
                      y                                   & = \dfrac{1}{16}    \\
                      \\
                      \text{When } y                      & = \dfrac{1}{16},   \\
                      2^x                                 & = \dfrac{1}{16}    \\
                      \log 2^x                            & = \log 2^{-4}      \\
                      x                                   & = -4
                  \end{flalign*}
              \end{multicols}
          \end{enumerate}

    \item If $\log_2 x + \log_4 x = \dfrac{9}{2}$, find the value of $x$. \sol{}
          \begin{flalign*}
              \log_2 x + \log_4 x             & = \dfrac{9}{2} \\
              \log_2 x + \dfrac{1}{2}\log_2 x & = \dfrac{9}{2} \\
              \dfrac{3}{2}\log_2 x            & = \dfrac{9}{2} \\
              \log_2 x                        & = 3            \\
              x                               & = 2^3          \\
                                              & = 8
          \end{flalign*}

          \newpage
    \item Solve the following logarithmic equations: \setlength{\columnsep}{2em}
          \begin{enumerate}
              \begin{multicols}{2}
                  \item $2\log x-3\log4=2$
                  \sol{}
                  \begin{flalign*}
                      2\log x - 3\log 4 & = 2                    \\
                      \log x^2          & = \log 4^3 + \log 100  \\
                      \log x^2          & = \log (4^3 \cdot 100) \\
                      x^2               & = 6400                 \\
                      x                 & = \pm 80
                  \end{flalign*}
                  After checking, $x = 80$ is the only solution, while $x = -80$ is an extraneous solution.
                  \vfill\null{}
                  \columnbreak{}
                  \item $2\log x=\log32+\log2$
                  \sol{}
                  \begin{flalign*}
                      2\log x  & = \log 32 + \log 2  \\
                      \log x^2 & = \log (32 \cdot 2) \\
                      x^2      & = 64                \\
                      x        & = \pm 8
                  \end{flalign*}
                  After checking, $x = 8$ is the only solution, while $x = -8$ is an extraneous solution.
              \end{multicols}

              \begin{multicols}{2}
                  \item $\log x+\log\left(x+3\right)=\log\left(x+8\right)$
                  \sol{}
                  \begin{flalign*}
                      \log x + \log (x + 3)        & = \log (x + 8)    \\
                      \log \dfrac{x(x + 3)}{x + 8} & = 0               \\
                      \dfrac{x(x + 3)}{x + 8}      & = 1               \\
                      x^2 + 3x                     & = x + 8           \\
                      x^2 + 2x - 8                 & = 0               \\
                      (x + 4)(x - 2)               & = 0               \\
                      x = -4                       & \text{ or } x = 2
                  \end{flalign*}
                  After checking, $x = 2$ is the only solution, while $x = -4$ is an extraneous solution.
                  \vfill\null{}
                  \columnbreak{}
                  \item ${\left(\log_{2}x\right)}^{2}=\log_{2}x+6$
                  \sol{}
                  \begin{flalign*}
                      \text{Let } y  & = \log_2 x           \\
                      y^2            & = y + 6              \\
                      y^2 - y - 6    & = 0                  \\
                      (y - 3)(y + 2) & = 0                  \\
                      y = 3          & \text{ or } y = -2   \\
                      \\
                      \text{When } y & = 3,                 \\
                      \log_2 x       & = 3                  \\
                      x              & = 2^3              & \\
                                     & = 8                  \\
                      \\
                      \text{When } y & = -2,                \\
                      \log_2 x       & = -2                 \\
                      x              & = 2^{-2}             \\
                                     & = \dfrac{1}{4}
                  \end{flalign*}
                  After checking, $x = 8$ and $x = \dfrac{1}{4}$ are both solutions.
              \end{multicols}

              \newpage
              \begin{multicols}{2}
                  \item $\log_{3}x+6\log_{x}3=5$
                  \sol{}
                  \begin{flalign*}
                      \log_3 x + 6\dfrac{1}{\log_3 x} & = 5                 \\
                      \text{Let } y                   & = \log_3 x          \\
                      y + \dfrac{6}{y}                & = 5                 \\
                      y^2 - 5y + 6                    & = 0               & \\
                      (y - 2)(y - 3)                  & = 0                 \\
                      y = 2                           & \text{ or } y = 3   \\
                      \\
                      \text{When } y                  & = 2,                \\
                      \log_3 x                        & = 2                 \\
                      x                               & = 3^2               \\
                                                      & = 9                 \\
                      \\
                      \text{When } y                  & = 3,                \\
                      \log_3 x                        & = 3                 \\
                      x                               & = 3^3               \\
                                                      & = 27
                  \end{flalign*}
                  After checking, $x = 9$ and $x = 27$ are both solutions.
                  \vfill\null{}
                  \columnbreak{}
                  \item $4^{\log x}=2^{\log x+1}$
                  \sol{}
                  \begin{flalign*}
                      4^{\log x}                    & = 2^{\log x + 1}         & \\
                      \log 4^{\log x}               & = \log 2^{\log x + 1}      \\
                      \log x \log 4                 & = (\log x + 1) \log 2      \\
                      \log x \log 4                 & = \log x \log 2 + \log 2   \\
                      \log x \log 4 - \log x \log 2 & = \log 2                   \\
                      \log x (\log 4 - \log 2)      & = \log 2                   \\
                      \log x \log 2                 & = \log 2                   \\
                      \log x                        & = 1                        \\
                      x                             & = 10
                  \end{flalign*}
              \end{multicols}

              \begin{multicols}{2}
                  \item $\log_{x+1}\left(x^{2}-5x-13\right)=2$
                  \sol{}
                  \begin{flalign*}
                      \log_{x + 1} (x^2 - 5x - 13) & = 2            \\
                      x^2 - 5x - 13                & = (x + 1)^2    \\
                      x^2 - 5x - 13                & = x^2 + 2x + 1 \\
                      -7x                          & = 14           \\
                      x                            & = -2
                  \end{flalign*}
                  After checking, $x = -2$ not a solution. Hence, no solution.
                  \vfill\null{}
                  \columnbreak{}
                  \item $\log_{x}{\sqrt{2x^{2}-5x+6}}=1$
                  \sol{}
                  \begin{flalign*}
                      \log_x \sqrt{2x^2 - 5x + 6} & = 1               \\
                      \sqrt{2x^2 - 5x + 6}        & = x               \\
                      2x^2 - 5x + 6               & = x^2             \\
                      x^2 - 5x + 6                & = 0               \\
                      (x - 2)(x - 3)              & = 0               \\
                      x = 2                       & \text{ or } x = 3
                  \end{flalign*}
                  After checking, $x = 2$ and $x = 3$ are both solutions.
              \end{multicols}

              \begin{multicols}{2}
                  \item $\log_{2}(x+1)+\log_{2}(x+3)=3+\log_{2}x$
                  \sol{}
                  \begin{flalign*}
                      \log_2 (x + 1) + \log_2 (x + 3) & = 3 + \log_2 x        \\
                      \log_2 (x + 1)(x + 3)           & = \log_2 8 + \log_2 x \\
                      \log_2 (x^2 + 4x + 3)           & = \log_2 8x           \\
                      x^2 + 4x + 3                    & = 8x                  \\
                      x^2 - 4x + 3                    & = 0                   \\
                      (x - 1)(x - 3)                  & = 0                   \\
                      x = 1                           & \text{ or } x = 3
                  \end{flalign*}
                  After checking, $x = 1$ and $x = 3$ are both solutions.
                  \vfill\null{}
                  \columnbreak{}
                  \item $\log_{2}\left[\log_{3}\left(\log_{5}x\right)\right]=0$
                  \sol{}
                  \begin{flalign*}
                      \log_2 \left[ \log_3 \left( \log_5 x \right) \right] & = 0   \\
                      \log_3 \left( \log_5 x \right)                       & = 1   \\
                      \log_5 x                                             & = 3   \\
                      x                                                    & = 5^3 \\
                                                                           & = 125
                  \end{flalign*}
              \end{multicols}

              \begin{multicols}{2}
                  \item $3\log_{8}x-2\log_{2}x+2=0$
                  \sol{}
                  \begin{flalign*}
                      3\log_8 x - 2 \log_2 x + 2                    & = 0     \\
                      \dfrac{3 \log_2 x}{\log_2 8} - 2 \log_2 x + 2 & = 0     \\
                      \text{Let } u = \log_2 x                      &         \\
                      \dfrac{3u}{3} - 2u + 2                        & = 0     \\
                      u - 2u + 2                                    & = 0   & \\
                      u                                             & = 2     \\
                      \\
                      \log_2 x                                      & = 2     \\
                      x                                             & = 2^2   \\
                                                                    & = 4
                  \end{flalign*}
                  \vfill\null{}
                  \columnbreak{}
                  \item $\log_{4}(x+4)+1=\log_{2}(x+1)$
                  \sol{}
                  \begin{flalign*}
                      \log_4 (x + 4) + 1               & = \log_2 (x + 1)                  \\
                      \dfrac{\log_2 (x + 4)}{\log_2 4} & = \log_2 (x + 1) + 1              \\
                      \log_2 (x + 4)                   & = \log_2 {(x + 1)}^2 - \log_2 4 & \\
                                                       & = \log_2 \dfrac{{(x + 1)}^2}{4}   \\
                      x + 4                            & = \dfrac{{(x + 1)}^2}{4}          \\
                      4x + 16                          & = {(x + 1)}^2                     \\
                      4x + 16                          & = x^2 + 2x + 1                    \\
                      x^2 - 2x - 15                    & = 0                               \\
                      (x - 5)(x + 3)                   & = 0                               \\
                      x                                & = 5 \text{ or } x = -3
                  \end{flalign*}
                  After checking, $x = 5$ is the only solution.
              \end{multicols}

              \newpage
              \item $2\log_{2}x\cdot\log_{8}x=\log_{2}x+\log_{8}x$
                    \sol{}
                    \begin{flalign*}
                        2 \log_2 x \cdot \log_8 x                       & = \log_2 x + \log_8 x                     \\
                        2 \log_2 x \cdot \dfrac{\log_2 x}{\log_2 8}     & = \log_2 x + \dfrac{\log_2 x}{\log_2 8}   \\
                        2 \log_2 x \cdot \dfrac{\log_2 x}{3}            & = \log_2 x + \dfrac{\log_2 x}{3}          \\
                        \text{Let } u = \log_2 x                        &                                         & \\
                        2u \cdot \dfrac{u}{3}                           & = u + \dfrac{u}{3}                        \\
                        \dfrac{2u^2}{3}                                 & = \dfrac{4u}{3}                           \\
                        2u^2                                            & = 4u                                      \\
                        u^2 - 2u                                        & = 0                                       \\
                        u(u - 2)                                        & = 0                                       \\
                        u                                           = 0 & \text{ or } u = 2                         \\
                        \\
                        \text{When } u                                  & = 0,                                      \\
                        \log_2 x                                        & = 0                                       \\
                        x                                               & = 2^0                                     \\
                                                                        & = 1                                       \\
                        \\
                        \text{When } u                                  & = 2,                                      \\
                        \log_2 x                                        & = 2                                       \\
                        x                                               & = 2^2                                     \\
                                                                        & = 4
                    \end{flalign*}
                    After checking, $x = 1$ and $x = 4$ are both solutions.
          \end{enumerate}
          \newpage

    \item A person has deposited a fund into a bank account that pays 5.5\% interest
          compounded annually. If the balance in the account has increased by RM1,432.95
          after 4 years, how much was deposited initially?

    \item Given that there is a principal of RM75,000 at an interest rate of 3.5\% per
          annum compounded once per three months. Find the accumulated value of the
          principal after 8 years.

    \item Given that there is a principal of RM150,000 at an interest rate of 5.25\% per
          annum compounded once per annum. How many years does it take to accumulate at
          least RM300,000?

    \item If the present value is RM24,924.44, the interest rate is 5\% per annum
          compounded once per annum, find the annuity payment if the payment is made for
          20 years.

    \item Given that the annuity payment is RM8,000, the interest rate is 4.5\% per annum
          compounded once per annum, and the payment is made for 15 years. Find the
          present value. Hence, find the present value of the perpetuity.

    \item Given that the annuity amount is RM4,500, the interest rate is 4.5\% per annum
          compounded once per annum. How many years does it take for the present value to
          exceed RM50,000?

    \item The price of a branded laptop is RM2,500, the payment can be paid in full or by
          instalment. If the payment is made by instalment, the monthly payment is RM110
          for 2 years. If the interest rate is 4\% per annum compounded once per month,
          which payment method is more economical considering the present value of the
          payment?
\end{enumerate}

\end{document}