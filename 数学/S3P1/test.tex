\documentclass[12pt]{report}
\usepackage{setspace}
\usepackage[a4paper, total={7in, 10in}]{geometry}
\usepackage[fleqn]{amsmath}
\usepackage{empheq}
\usepackage{amssymb}
\usepackage{amsthm}
\usepackage{gensymb}
\usepackage[fleqn]{cases}
\usepackage{multicol}
\usepackage{color}
\usepackage{stix}
\usepackage{chngcntr}
\usepackage{tikz}
\usepackage{enumitem}
\usepackage{pgfplots}
\usepackage{etoolbox}
\usepackage{tkz-euclide}
\usepackage{graphicx}
\usepackage{enumitem}
\usepackage{multirow}
\usepackage{mathtools}
\usepackage{mdframed}
\usepackage{adjustbox}
\usepackage{xpatch}
\usepackage{nicematrix}
\usepackage{ifthen}

\def\nswe#1#2#3{#1\,$#2^\circ\,#3'$}
\graphicspath{ {./assets/} }
\usetikzlibrary{calc,trees,positioning,arrows,fit,shapes,calc, decorations.markings}
\newcommand{\midarrow}{\tikz \draw[-triangle 90] (0,0) -- +(.1,0);}

\newcommand\typel[2]{
    \mathbin{\mathop{#1\kern0pt}%
        \limits_{\raisebox{3.6ex}{\hbox to0pt{\hss\strut$\uparrow$\hss}}\hbox to0pt{\hss#2\hss}}}
}

\newcommand\typem[2]{
    \mathbin{\mathop{#1\kern0pt}%
        \limits^{\raisebox{3.6ex}{\hbox to0pt{\hss#2\hss}}\hbox to0pt{\hss\strut$\downarrow$\hss}}}
}

\counterwithout{equation}{chapter}

\newcommand{\pgfplotsdrawaxis}{\pgfplots@draw@axis}
\newcommand\perm[2][^n]{\prescript{#1\mkern-2.5mu}{}P_{#2}}
\newcommand\permtwo[2][^n]{{}_{#1}P_{#2}}
\newcommand\comb[2][^n]{{}_{#1}C_{#2}}
\newcommand\combtwo[2][^n]{\prescript{#1\mkern-2.5mu}{}C_{#2}}
\makeatother
\pgfplotsset{only axis on top/.style={axis on top=false, after end axis/.code={
                    \pgfplotsset{axis line style=opaque, ticklabel style=opaque, tick style={thick,opaque},
                        grid=none}\pgfplotsdrawaxis}}}

\newtheorem{theorem}{Theorem}

\makeatletter
\xpatchcmd{\endmdframed}
{\aftergroup\endmdf@trivlist\color@endgroup}
{\endmdf@trivlist\color@endgroup\@doendpe}
{}{}
\makeatother

\mdfdefinestyle{MyFrame}{%
    linecolor=black,
    linewidth=1pt,
    roundcorner=20pt, innertopmargin=20pt,innerbottommargin=20pt, innerrightmargin=12pt,
    innerleftmargin=12pt, skipbelow=20pt, skipabove=20pt
    %backgroundcolor=gray!50!white}
}

\newcommand{\newitem}[1]{%
    \refstepcounter{subenum}%
    \parbox{\dimexpr.5\linewidth-.5\columnsep}{
        \makebox[\labelwidth][r]{(\thesubenum)\hspace*{\labelsep}} #1}\hfill }%%%

\setcounter{chapter}{21}

\setlength{\arrayrulewidth}{1pt}
\setlength{\tabcolsep}{12pt}

\begin{document}

\newcommand{\sol}[1]{

    \noindent \textbf{Sol.}
}
\newcommand{\prooff}[1]{

    \noindent \textbf{Proof.}
}

\newcommand{\sxrightarrow}[2][]{%
    \mathrel{\text{$\xrightarrow[#1]{#2}$}}%
}

\newenvironment{cequation}{
    \makeatletter
    \setbool{@fleqn}{false}
    \makeatother
    \begin{equation*}
        }{\end{equation*}}

\begin{titlepage}
    \raggedleft{}
    \rule{1pt}{\textheight}
    \hspace{0.02\textwidth}
    \parbox[b]{0.75\textwidth}{

    {\fontsize{40}{60}\selectfont\bfseries Mathematics}\\[2\baselineskip]
    {\huge\textit{Senior 3 Part I}}\\[4\baselineskip]
    {\Large\textsc{Melvin Chia}}

    \vspace{0.5\textheight}

    {\noindent Started on 10 April 2023}\\[\baselineskip]
    {\noindent Finished on XX XX 2023}\\[\baselineskip]
    {\noindent Actual time spent: XX days}\\[\baselineskip]}

\end{titlepage}

\chapter*{Introduction}
\addcontentsline{toc}{chapter}{Introduction} \markboth{INTRODUCTION}{}

\doublespacing{}
\section*{Why this book?}

\section*{Disclaimer}
\section*{Acknowledgements}

\singlespacing{}

\doublespacing{}
\tableofcontents
\singlespacing{}
\newpage

\onehalfspacing

\chapter{Differentiation}

\section{Gradient of Tangent Line on a Curve}

As shown in the diagram below, sketch the graph of curve $C$ on the Cartesian
plane, and pick two points $P$ and $Q$ on the curve, the straight line that
passes through $P$ and $Q$ is the secant line of the curve. When point $Q$
moves closer to point $P$ along the curve, the secant line $PQ$ rotates along
point $P$ and approaches a limit straight line $PT$. The line $PT$ is called
the \textit{tangent line} of the curve $C$ at point $P$.

We know the fact that a line can be determined by a point on the line and its
gradient. Therefore, in order to find the gradient of the tangent line of the
curve $C$ at point $P$, we have to first find the gradient of the tangent.

To find the gradient of the tangent line of curve $y=f (x)$ at point
$P\left(x_0, f (x_0)\right)$, we can pick a point $\left(x_0 + \Delta{x}, f
    (x_0 + \Delta{x})\right)$ near to point $P$ on the curve, as shown in the
diagram above. Denote the variable $PR$ on the horizontal axis as $\Delta{x}$,
and the corresponding variable $RQ$ on the vertical axis as $\Delta y$, we have
$\Delta y = f (x_0 + \Delta{x}) - f (x_0)$.

Hence, the gradient of the secant line $PQ$ is $\dfrac{\Delta y}{\Delta{x}} =
    \dfrac{f (x_0 + \Delta{x}) - f (x_0)}{\Delta{x}}$.

When $\Delta{x}\to{0}$, if the limit of the expression above exist, it is the
gradient of the tangent line of the curve $y=f (x)$ at point $P$, denoted as
$m$, that is
\begin{mdframed}[style=MyFrame]
    \begin{cequation}
        m = \lim\limits_{\Delta{x}\to{0}}{\dfrac{\Delta y}{\Delta{x}}} = \lim\limits_{\Delta{x}\to{0}}{\dfrac{f (x_0 + \Delta{x}) - f (x_0)}{\Delta{x}}}
    \end{cequation}
\end{mdframed}

\newpage

\subsection{Practice 1}

\begin{enumerate}
    \item Find the gradient of the tangent line of the curve $y = 2 - x^2$ at $x = 1$.
          \sol{}

          Let $f (x) = 2 - x^2$.

          When $x = 1$, $f (1) = 2 - 1^2 = 1$.

          Take nearby point $Q(1 + \Delta{x}, f (1 + \Delta{x}))$ of $x = 1$ on the
          curve, we have
          \begin{flalign*}
              \dfrac{\Delta y}{\Delta{x}} & = \dfrac{f (1 + \Delta{x}) - f (1)}{\Delta{x}}                \\
                                          & = \dfrac{2 - {(1 + \Delta{x})}^2 - 1}{\Delta{x}}              \\
                                          & = \dfrac{2 - 1 - 2\Delta{x} - {(\Delta{x})}^2 - 1}{\Delta{x}} \\
                                          & = \dfrac{-2\Delta{x} - {(\Delta{x})}^2}{\Delta{x}}            \\
                                          & = -2 - \Delta{x}
          \end{flalign*}
          $\therefore$ The gradient of the tangent line of the curve at $x = 1$ is $m = \lim\limits_{\Delta{x}\to{0}}{(-2 - \Delta{x})} = -2$.

    \item Find the gradient of the tangent line of the curve $y = x^2 + 3x$ at $x = 2$.
          \sol{}

          Let $f (x) = x^2 + 3x$. When $x = 2$, $f (2) = 2^2 + 3 \times 2 = 10$.

          Take nearby point $Q(2 + \Delta{x}, f (2 + \Delta{x}))$ of $x = 2$ on the
          curve, we have
          \begin{flalign*}
              \dfrac{\Delta y}{\Delta{x}} & = \dfrac{f (2 + \Delta{x}) - f (2)}{\Delta{x}}                              \\
                                          & = \dfrac{{(2 + \Delta{x})}^2 + 3(2 + \Delta{x}) - 10}{\Delta{x}}            \\
                                          & = \dfrac{4 + 4\Delta{x} + {(\Delta{x})}^2 + 6 + 3\Delta{x} - 10}{\Delta{x}} \\
                                          & = \dfrac{{(\Delta{x})}^2 + 7\Delta{x}}{\Delta{x}}                           \\
                                          & = \Delta{x} + 7
          \end{flalign*}
          $\therefore$ The gradient of the tangent line of the curve at $x = 2$ is $m = \lim\limits_{\Delta{x}\to{0}}{(\Delta{x} + 7)} = 7$.
\end{enumerate}

\newpage
\subsection{Exercise 25.1}

\begin{enumerate}
    \item Given the curve $y = x^2 + 1$. Assume that $P$ is the point on the curve at $x
              = 2$, $Q$ is a nearby point,
          \begin{enumerate}
              \item Find the variable $\Delta{x}$ and the corresponding variable $\Delta y$ when
                    the $x$ coordinates of point $Q$ is $2.5$, $2.25$, $2.1$, $2.05$, $2.01$, and
                    $2.001$ respectively. Hence, complete the table below.
                    \begin{center}
                        \begin{NiceTabular}{|c|c|c|c|c|}[hvlines,cell-space-limits=5pt]
                            $x$-coords of $Q$ & $y$-coords of $Q$ & $\Delta{x}$ & $\Delta y$ & Gradient of secant $PQ$ $\dfrac{\Delta y}{\Delta{x}}$ \\
                            $x = 2.5$         &                   &             &            &                                                       \\
                            $x = 2.25$        &                   &             &            &                                                       \\
                            $x = 2.1$         &                   &             &            &                                                       \\
                            $x = 2.05$        &                   &             &            &                                                       \\
                            $x = 2.01$        &                   &             &            &                                                       \\
                            $x = 2.001$       &                   &             &            &                                                       \\
                        \end{NiceTabular}
                    \end{center}
                    \sol{}

                    When $x = 2$, $y = 2^2 + 1 = 5$.
                    \begin{center}
                        \begin{NiceTabular}{|c|c|c|c|c|}[hvlines,cell-space-limits=5pt]
                            $x$-coords of $Q$ & $y$-coords of $Q$ & $\Delta{x}$ & $\Delta y$ & \Block{}{Gradient of secant \\ $PQ$ $\dfrac{\Delta y}{\Delta{x}}$} \\
                            $x = 2.5$         & $y = 7.25$        & $0.5$       & $2.25$     & $4.5$                       \\
                            $x = 2.25$        & $y = 6.0625$      & $0.25$      & $1.0625$   & $4.25$                      \\
                            $x = 2.1$         & $y = 5.41$        & $0.1$       & $0.41$     & $4.1$                       \\
                            $x = 2.05$        & $y = 5.2025$      & $0.05$      & $0.2025$   & $4.05$                      \\
                            $x = 2.01$        & $y = 5.0401$      & $0.01$      & $0.0401$   & $4.01$                      \\
                            $x = 2.001$       & $y = 5.004001$    & $0.001$     & $0.004001$ & $4.001$                     \\
                        \end{NiceTabular}
                    \end{center}

              \item Inspect the gradient of secant $PQ$ as point $Q$ approaches point $P$. Hence,
                    find the gradient of the tangent line of the curve at point $P$.\sol{}

                    As point $Q$ approaches point $P$, the gradient of secant $PQ$ approaches $4$.
                    Hence, the gradient of the tangent line of the curve at point $P$ is $m = 4$.
          \end{enumerate}

          \newpage
    \item Find the gradient of the tangent line of the curve $y = \dfrac{1}{3}x^2$ at $x
              = 2$.\sol{}

          Let $f (x) = \dfrac{1}{3}x^2$.

          When $x = 2$, $f (2) = \dfrac{1}{3} \times 2^2 = \dfrac{4}{3}$.

          Take nearby point $Q(2 + \Delta{x}, f (2 + \Delta{x}))$ of $x = 2$ on the
          curve, we have
          \begin{flalign*}
              \dfrac{\Delta y}{\Delta{x}} & = \dfrac{f (2 + \Delta{x}) - f (2)}{\Delta{x}}                                                         \\
                                          & = \dfrac{\dfrac{1}{3}{(2 + \Delta{x})}^2 - \dfrac{4}{3}}{\Delta{x}}                                    \\
                                          & = \dfrac{\dfrac{4}{3} + \dfrac{4}{3}\Delta{x} + \dfrac{1}{3}{(\Delta{x})}^2 - \dfrac{4}{3}}{\Delta{x}} \\
                                          & = \dfrac{\dfrac{4}{3}\Delta{x} + \dfrac{1}{3}{(\Delta{x})}^2}{\Delta{x}}                               \\
                                          & = \dfrac{4}{3} + \dfrac{1}{3}\Delta{x}
          \end{flalign*}
          $\therefore$ The gradient of the tangent line of the curve at $x = 2$ is $m = \lim\limits_{\Delta{x}\to{0}}{\left(\dfrac{4}{3} + \dfrac{1}{3}\Delta{x}\right)} = \dfrac{4}{3}$.

    \item Find the gradient of the tangent line of the curve $y = \dfrac{10}{x}$ at point
          $Q(2, 5)$.\sol{}

          Let $f (x) = \dfrac{10}{x}$.

          When $x = 2$, $f (2) = 5$.

          Take nearby point $P(2 + \Delta{x}, f (2 + \Delta{x}))$ of $x = 2$ on the
          curve, we have
          \begin{flalign*}
              \dfrac{\Delta y}{\Delta{x}} & = \dfrac{f (2 + \Delta{x}) - f (2)}{\Delta{x}}                      \\
                                          & = \dfrac{\dfrac{10}{2 + \Delta{x}} - 5}{\Delta{x}}                  \\
                                          & = \dfrac{\dfrac{10 - 5(2 + \Delta{x})}{(2 + \Delta{x})}}{\Delta{x}} \\
                                          & = \dfrac{- 5\Delta{x}}{(2 + \Delta{x})\Delta{x}}                    \\
                                          & = \dfrac{- 5}{(2 + \Delta{x})}
          \end{flalign*}
          $\therefore$ The gradient of the tangent line of the curve at $x = 2$ is $m = \lim\limits_{\Delta{x}\to{0}}{\dfrac{- 5}{(2 + \Delta{x})}} = -\dfrac{5}{2}$

          \newpage
    \item Find the gradient of the tangent line of the curve $y = \dfrac{4}{x} - 5$ at $x
              = 1$.\sol{}

          Let $f (x) = \dfrac{4}{x} - 5$.

          When $x = 1$, $f (1) = \dfrac{4}{1} - 5 = -1$.

          Take nearby point $Q(1 + \Delta{x}, f (1 + \Delta{x}))$ of $x = 1$ on the
          curve, we have
          \begin{flalign*}
              \dfrac{\Delta y}{\Delta{x}} & = \dfrac{f (1 + \Delta{x}) - f (1)}{\Delta{x}}                     \\
                                          & = \dfrac{\dfrac{4}{1 + \Delta{x}} - 5 - (-1)}{\Delta{x}}           \\
                                          & = \dfrac{\dfrac{4 - 4(1 + \Delta{x})}{(1 + \Delta{x})}}{\Delta{x}} \\
                                          & = \dfrac{- 4\Delta{x}}{(1 + \Delta{x})\Delta{x}}                   \\
                                          & = \dfrac{- 4}{(1 + \Delta{x})}
          \end{flalign*}
          $\therefore$ The gradient of the tangent line of the curve at $x = 1$ is $m = \lim\limits_{\Delta{x}\to{0}}{\dfrac{- 4}{(1 + \Delta{x})}} = -4$

    \item Find the gradient of the tangent line of the curve $y = \dfrac{1}{2}x^3 + 1$ at
          point $P(-2, -3)$.\sol{}

          Let $f (x) = \dfrac{1}{2}x^3 + 1$.

          When $x = -2$, $f (-2) = \dfrac{1}{2}{(-2)}^3 + 1 = -3$.

          Take nearby point $Q(-2 + \Delta{x}, f (-2 + \Delta{x}))$ of $x = -2$ on the
          curve, we have
          \begin{flalign*}
              \dfrac{\Delta y}{\Delta{x}} & = \dfrac{f (-2 + \Delta{x}) - f (-2)}{\Delta{x}}                       \\
                                          & = \dfrac{\dfrac{1}{2}{(-2 + \Delta{x})}^3 + 1 -(-3)}{\Delta{x}}        \\
                                          & = \dfrac{{(-2 + \Delta{x})}^3 + 8}{2\Delta{x}}                         \\
                                          & = \dfrac{{(\Delta{x})}^3 - 6{(\Delta{x})}^2 + 12\Delta{x}}{2\Delta{x}} \\
                                          & = \dfrac{{(\Delta{x})}^2 - 6\Delta{x} + 12}{2}
          \end{flalign*}
          $\therefore$ The gradient of the tangent line of the curve at $x = -2$ is $m = \lim\limits_{\Delta{x}\to{0}}{\dfrac{{(\Delta{x})}^2 - 6\Delta{x} + 12}{2}} = 6$
\end{enumerate}

\section{Gradient of Tangent Line and Derivative}

In the last section, we have learnt that, if $P\left(x_0, f (x_0)\right)$ and
$Q\left(x_0 + \Delta{x}, f (x_0 + \Delta{x})\right)$ are two points on the
curve $y=f (x)$, then the gradient of the secant line $PQ$ is $\dfrac{\Delta
        y}{\Delta{x}} = \dfrac{f (x_0 + \Delta{x}) - f (x_0)}{\Delta{x}}$. As point $Q$
approaches point $P$, that is, as $\Delta{x}\to{0}$, the gradient of the secant
line $PQ$ approaches the tangent line $PT$.

Hence, the gradient of the tangent line $PT$ is
\begin{cequation}
    m = \lim\limits_{\Delta{x}\to{0}}{\dfrac{\Delta y}{\Delta{x}}} = \lim\limits_{\Delta{x}\to{0}}{\dfrac{f (x_0 + \Delta{x})-f (x_0)}{\Delta{x}}}
\end{cequation}

\subsection*{Definition of Derivative}

Let the function $y = f (x)$ be defined at $x = x_0$ and its nearby points,
when there exists a variable $\Delta{x}$ of $x$ at $x_0$, there exist a
corresponding variable $\Delta y = f (x_0 + \Delta{x}) - f (x_0)$ of the
function $y$. When $\Delta{x}\to{0}$, the limit of $\dfrac{\Delta
        y}{\Delta{x}}$
\begin{cequation}
    \lim\limits_{\Delta{x}\to{0}}{\dfrac{\Delta y}{\Delta{x}}} = \lim\limits_{\Delta{x}\to{0}}{\dfrac{f (x_0 + \Delta{x})-f (x_0)}{\Delta{x}}}
\end{cequation}
exists, $y = f (x)$ is said to be \textit{differentiable} at $x =
    x_0$, and the limit is called the \textit{derivative} of $y = f (x)$ at $x =
    x_0$, denoted as $f' (x_0)$, that is
\begin{mdframed}[style=MyFrame]
    \begin{cequation}
        f' (x_0) = \lim\limits_{\Delta{x}\to{0}}{\dfrac{\Delta y}{\Delta{x}}} = \lim\limits_{\Delta{x}\to{0}}{\dfrac{f (x_0 + \Delta{x})-f (x_0)}{\Delta{x}}}
    \end{cequation}
\end{mdframed}
If the limit above does not exist, $y = f (x)$ is said to be \textit{non-differentiable} at $x = x_0$.

If function $y = f (x)$ is differentiable at every point of an interval $(a,
    b)$. Each defined value $x_0$ in the interval $(a, b)$ corresponds to a
derivative value $f' (x_0)$, thus forming a new function in the interval $(a,
    b)$. This new function is called the \textit{derivative function} of $f (x)$,
denoted as $f' (x)$ or $y'$.

From the definition of derivative, we can get the derivative
\begin{cequation}
    y' = f' (x) = \lim\limits_{\Delta{x}\to{0}}{\dfrac{\Delta y}{\Delta{x}}} = \lim\limits_{\Delta{x}\to{0}}{\dfrac{f (x + \Delta{x})-f (x)}{\Delta{x}}}
\end{cequation}

$f' (x)$ is often being denoted as $\dfrac{dy}{dx}$ as well, and is called \textit{differentiation of $y = f (x)$ with respect to $x$}.

From the definition of derivative, we can conclude the following steps to find
the derivative of a function $y = f (x)$:

\begin{enumerate}
    \item Find the variable $\Delta y = f (x + \Delta{x}) - f (x)$ of the function.
    \item Find the ratio of the variable $\Delta y$ to $\Delta{x}$, that is,
          $\dfrac{\Delta y}{\Delta{x}} = \dfrac{f (x + \Delta{x}) - f (x)}{\Delta{x}}$.
    \item Find the limit of the ratio above, that is, $\dfrac{dy}{dx} =
              \lim\limits_{\Delta{x}\to{0}}{\dfrac{\Delta y}{\Delta{x}}} =
              \lim\limits_{\Delta x \to 0}{\dfrac{f (x + \Delta{x}) - f (x)}{\Delta{x}}}$.
\end{enumerate}

The method above to find the derivative using the definition of derivative is
called the \textbf{first principle} of differentiation.

Note that derivative function is also called the derivative. Unless otherwise
stated, finding the derivative means finding the derivative function.

Also note that the derivative function $f' (x)$ of the function $y = f (x)$ is
different from the derivative $f' (x_0)$ of the function $y = f (x)$ at $x =
    x_0$. $f' (x)$ is a function, while $f' (x_0)$ is a value, but they are related
to each other. $f' (x_0)$ is the function value of the derivative function $f'
    (x)$ at $x = x_0$, and is called the \textit{derivative value} at $x = x_0$.

\subsection{Practice 2}

\begin{enumerate}
    \item Find the derivative of the function $y = 2x^2$, and find the derivative value
          at $x = 1$.\sol{}
          \begin{flalign*}
              \Delta y                                                   & = 2 {(x + \Delta{x})}^2 - 2x^2                    \\
                                                                         & = 2x^2 + 4x \Delta{x} + 2\Delta{x}^2 - 2x^2       \\
                                                                         & = 4x \Delta{x} + 2 \Delta{x}^2                    \\
              \\
              \dfrac{\Delta y}{\Delta{x}}                                & = \dfrac{4x \Delta{x} + 2 \Delta{x}^2}{\Delta{x}} \\
                                                                         & = 4x + 2 \Delta{x}                                \\
              \\
              \lim\limits_{\Delta{x}\to{0}}{\dfrac{\Delta y}{\Delta{x}}} & = \lim\limits_{\Delta{x}\to{0}}{4x + 2 \Delta{x}} \\
              \dfrac{dy}{dx}                                             & = 4x
          \end{flalign*}
          $\therefore$ The derivative value at $x = 1$ is $4(1) = 4$.

    \item Let function $y = 4 - 3x + x^2$. Find:
          \begin{enumerate}
              \item $\dfrac{\Delta y}{\Delta{x}}$
                    \sol{}
                    \begin{flalign*}
                        \Delta y                    & = 4 - 3 (x + \Delta{x}) + {(x + \Delta{x})}^2 - (4 - 3x + x^2)           \\
                                                    & = 4 - 3x - 3 \Delta{x} + x^2 + 2x \Delta{x} + \Delta{x}^2 - 4 + 3x - x^2 \\
                                                    & = -3 \Delta{x} + 2x \Delta{x} + \Delta{x}^2                              \\
                        \\
                        \dfrac{\Delta y}{\Delta{x}} & = \dfrac{-3 \Delta{x} + 2x \Delta{x} + \Delta{x}^2}{\Delta{x}}           \\
                                                    & = \Delta{x} + 2x - 3
                    \end{flalign*}

              \item $\dfrac{dy}{dx}$
                    \sol{}
                    \begin{flalign*}
                        \lim\limits_{\Delta{x}\to{0}}{\dfrac{\Delta y}{\Delta{x}}} & = \lim\limits_{\Delta{x}\to{0}}{\Delta{x} + 2x - 3} \\
                        \dfrac{dy}{dx}                                             & = 2x - 3
                    \end{flalign*}

              \item The derivative value of the function at $x = 2$.\sol{}
                    \begin{flalign*}
                        \dfrac{dy}{dx} & = 2x - 3   \\
                                       & = 2(2) - 3 \\
                                       & = 1
                    \end{flalign*}
          \end{enumerate}
\end{enumerate}

\subsection{Exercise 25.2}

Find teh derivative of the following functions using the definition of
derivative. Hence, find the derivative value at $x = 2$ (Question 1 to 12):
\begin{enumerate}
    \begin{multicols}{2}
        \item $f(x) = 3x + 2$
        \sol{}
        \begin{flalign*}
            \Delta y                                                   & = 3(x + \Delta{x}) + 2 - (3x + 2)  & \\
                                                                       & = 3x + 3 \Delta{x} + 2 - 3x - 2      \\
                                                                       & = 3 \Delta{x}                        \\
            \\
            \dfrac{\Delta y}{\Delta{x}}                                & = \dfrac{3 \Delta{x}}{\Delta{x}}     \\
                                                                       & = 3                                  \\
            \\
            \lim\limits_{\Delta{x}\to{0}}{\dfrac{\Delta y}{\Delta{x}}} & = \lim\limits_{\Delta{x}\to{0}}{3}   \\
            f'(x)                                                      & = 3
        \end{flalign*}
        $\therefore$ The derivative value at $x = 2$ is $3$.
        \columnbreak
        \item $f(x) = x^2 + 1$
        \sol{}
        \begin{flalign*}
            \Delta y                                                   & = {(x + \Delta{x})}^2 + 1 - (x^2 + 1)                        & \\
                                                                       & = x^2 + 2x \Delta{x} + \Delta{x}^2 - x^2 - 1                   \\
                                                                       & = 2x \Delta{x} + \Delta{x}^2                                   \\
            \\
            \dfrac{\Delta y}{\Delta{x}}                                & = \dfrac{2x \Delta{x} + \Delta{x}^2}{\Delta{x}}                \\
                                                                       & = 2x + \Delta{x}                                               \\
            \\
            \lim\limits_{\Delta{x}\to{0}}{\dfrac{\Delta y}{\Delta{x}}} & = \lim\limits_{\Delta{x}\to{0}}{\left(2x + \Delta{x}\right)}   \\
            f'(x)                                                      & = 2x
        \end{flalign*}
        $\therefore$ The derivative value at $x = 2$ is $2(2) = 4$.
    \end{multicols}

    \newpage
    \begin{multicols}{2}
        \item $f(x) = 4x^2 - 3$
        \sol{}
        \begin{flalign*}
            \Delta y                                                   & = 4{(x + \Delta{x})}^2 - 3 - (4x^2 - 3)                        & \\
                                                                       & = 4x^2 + 8x \Delta{x} + 4 \Delta{x}^2 - 3 - 4x^2 + 3             \\
                                                                       & = 8x \Delta{x} + 4 \Delta{x}^2                                   \\
            \\
            \dfrac{\Delta y}{\Delta{x}}                                & = \dfrac{8x \Delta{x} + 4 \Delta{x}^2}{\Delta{x}}                \\
                                                                       & = 8x + 4 \Delta{x}                                               \\
            \\
            \lim\limits_{\Delta{x}\to{0}}{\dfrac{\Delta y}{\Delta{x}}} & = \lim\limits_{\Delta{x}\to{0}}{\left(8x + 4 \Delta{x}\right)}   \\
            f'(x)                                                      & = 8x
        \end{flalign*}
        $\therefore$ The derivative value at $x = 2$ is $8(2) = 16$.
        \columnbreak{}
        \item $f(x) = 2 - x^2$
        \sol{}
        \begin{flalign*}
            \Delta y                                                   & = 2 - {(x + \Delta{x})}^2 - (2 - x^2)                         & \\
                                                                       & = 2 - x^2 - 2x \Delta{x} - \Delta{x}^2 - 2 + x^2                \\
                                                                       & = -2x \Delta{x} - \Delta{x}^2                                   \\
            \\
            \dfrac{\Delta y}{\Delta{x}}                                & = \dfrac{-2x \Delta{x} - \Delta{x}^2}{\Delta{x}}                \\
                                                                       & = -2x - \Delta{x}                                               \\
            \\
            \lim\limits_{\Delta{x}\to{0}}{\dfrac{\Delta y}{\Delta{x}}} & = \lim\limits_{\Delta{x}\to{0}}{\left(-2x - \Delta{x}\right)}   \\
            f'(x)                                                      & = -2x
        \end{flalign*}
        $\therefore$ The derivative value at $x = 2$ is $-2(2) = -4$.
    \end{multicols}

    \begin{multicols}{2}
        \item $y = x^2 - 3x$
        \sol{}
        \begin{flalign*}
            \Delta y                    & = {(x + \Delta{x})}^2 - 3(x + \Delta{x}) - (x^2 - 3x)         & \\
                                        & = x^2 + 2x \Delta{x} + \Delta{x}^2 - 3x - 3 \Delta{x} - x^2     \\
                                        & \ \ \ \ + 3x                                                    \\
                                        & = 2x \Delta{x} + \Delta{x}^2 - 3 \Delta{x}                      \\
            \\
            \dfrac{\Delta y}{\Delta{x}} & = \dfrac{2x \Delta{x} + \Delta{x}^2 - 3 \Delta{x}}{\Delta{x}}   \\
                                        & = 2x + \Delta{x} - 3
        \end{flalign*}
        \vspace{-1.5em}
        \begin{flalign*}
            \lim\limits_{\Delta{x}\to{0}}{\dfrac{\Delta y}{\Delta{x}}} & = \lim\limits_{\Delta{x}\to{0}}{\left(2x + \Delta{x} - 3\right)} & \\
            \dfrac{dy}{dx}                                             & = 2x - 3
        \end{flalign*}
        $\therefore$ The derivative value at $x = 2$ is $2(2) - 3 = 1$.
        \vfill{}\null{}
        \columnbreak{}
        \item $y = x^2 - x - 3$
        \sol{}
        \begin{flalign*}
            \Delta y                    & = {(x + \Delta{x})}^2 - (x + \Delta{x}) - 3 - (x^2 - x - 3)      & \\
                                        & = x^2 + 2x \Delta{x} + \Delta{x}^2 - x - \Delta{x} - 3 - x^2 + x   \\
                                        & \ \ \ \ + 3                                                        \\
                                        & = 2x \Delta{x} + \Delta{x}^2 - \Delta{x}                           \\
            \\
            \dfrac{\Delta y}{\Delta{x}} & = \dfrac{2x \Delta{x} + \Delta{x}^2 - \Delta{x}}{\Delta{x}}      & \\
                                        & = 2x + \Delta{x} - 1
        \end{flalign*}
        \vspace{-0.8cm}
        \begin{flalign*}
            \lim\limits_{\Delta{x}\to{0}}{\dfrac{\Delta y}{\Delta{x}}} & = \lim\limits_{\Delta{x}\to{0}}{\left(2x + \Delta{x} - 1\right)} & \\
            \dfrac{dy}{dx}                                             & = 2x - 1
        \end{flalign*}
        $\therefore$ The derivative value at $x = 2$ is $2(2) - 1 = 3$.
    \end{multicols}

    \newpage
    \begin{multicols}{2}
        \item $y = 2x^2 + 3x - 1$
        \sol{}
        \begin{flalign*}
            \Delta y                    & = 2{(x + \Delta{x})}^2 + 3(x + \Delta{x}) - 1                     \\
                                        & \ \ \ \ - (2x^2 + 3x - 1)                                       & \\
                                        & = 2(x^2 + 2x \Delta{x} + \Delta{x}^2) + 3x + 3 \Delta{x}          \\
                                        & \ \ \ \ - 1 - 2x^2                                                \\
                                        & \ \ \ \ - 3x + 1                                                  \\
                                        & = 4x \Delta{x} + 2 \Delta{x}^2 + 3 \Delta{x}                      \\
            \\
            \dfrac{\Delta y}{\Delta{x}} & = \dfrac{4x \Delta{x} + 2 \Delta{x}^2 + 3 \Delta{x}}{\Delta{x}} & \\
                                        & = 4x + 2 \Delta{x} + 3
        \end{flalign*}
        \vspace{-0.8cm}
        \begin{flalign*}
            \lim\limits_{\Delta{x}\to{0}}{\dfrac{\Delta y}{\Delta{x}}} & = \lim\limits_{\Delta{x}\to{0}}{\left(4x + 2 \Delta{x} + 3\right)} & \\
            \dfrac{dy}{dx}                                             & = 4x + 3
        \end{flalign*}
        $\therefore$ The derivative value at $x = 2$ is $4(2) + 3 = 11$.
        \vfill{}\null{}
        \columnbreak{}
        \item $y = x^3 + 1$
        \sol{}
        \begin{flalign*}
            \Delta y                    & = {(x + \Delta{x})}^3 + 1 - (x^3 + 1)                              & \\
                                        & = x^3 + 3x^2 \Delta{x} + 3x \Delta{x}^2 + \Delta{x}^3 + 1            \\
                                        & \ \ \ \ - x^3 - 1                                                    \\
                                        & = 3x^2 \Delta{x} + 3x \Delta{x}^2 + \Delta{x}^3                      \\
            \\
            \dfrac{\Delta y}{\Delta{x}} & = \dfrac{3x^2 \Delta{x} + 3x \Delta{x}^2 + \Delta{x}^3}{\Delta{x}} & \\
                                        & = 3x^2 + 3x \Delta{x} + \Delta{x}^2
        \end{flalign*}
        \vspace{-0.8cm}
        \begin{flalign*}
            \lim\limits_{\Delta{x}\to{0}}{\dfrac{\Delta y}{\Delta{x}}} & = \lim\limits_{\Delta{x}\to{0}}{\left(3x^2 + 3x \Delta{x} + \Delta{x}^2\right)} & \\
            \dfrac{dy}{dx}                                             & = 3x^2
        \end{flalign*}
        $\therefore$ The derivative value at $x = 2$ is $3{(2)}^2 = 12$.
    \end{multicols}

    \item $y = x^3 - 2x$
          \sol{}
          \begin{flalign*}
              \Delta y                    & = {(x + \Delta{x})}^3 - 2(x + \Delta{x}) - (x^3 - 2x)                            & \\
                                          & = x^3 + 3x^2 \Delta{x} + 3x \Delta{x}^2 + \Delta{x}^3                              \\
                                          & \ \ \ \ - 2x - 2 \Delta{x} - x^3 + 2x                                              \\
                                          & = 3x^2 \Delta{x} + 3x \Delta{x}^2 + \Delta{x}^3 - 2 \Delta{x}                      \\
              \\
              \dfrac{\Delta y}{\Delta{x}} & = \dfrac{3x^2 \Delta{x} + 3x \Delta{x}^2 + \Delta{x}^3 - 2 \Delta{x}}{\Delta{x}} & \\
                                          & = 3x^2 + 3x \Delta{x} + \Delta{x}^2 - 2
          \end{flalign*}
          \vspace{-0.8cm}
          \begin{flalign*}
              \lim\limits_{\Delta{x}\to{0}}{\dfrac{\Delta y}{\Delta{x}}} & = \lim\limits_{\Delta{x}\to{0}}{\left(3x^2 + 3x \Delta{x} + \Delta{x}^2 - 2\right)} & \\
              \dfrac{dy}{dx}                                             & = 3x^2 - 2
          \end{flalign*}
          $\therefore$ The derivative value at $x = 2$ is $3{(2)}^2 - 2 = 10$.

          \newpage
    \item $y = \sqrt{x + 2}$
          \sol{}
          \begin{flalign*}
              \Delta y                                                   & = \sqrt{x + \Delta{x} + 2} - \sqrt{x + 2}                                                                                                                                                  & \\
              \\
              \dfrac{\Delta y}{\Delta{x}}                                & = \dfrac{\sqrt{x + \Delta{x} + 2} - \sqrt{x + 2}}{\Delta{x}}                                                                                                                               & \\
              \\
              \lim\limits_{\Delta{x}\to{0}}{\dfrac{\Delta y}{\Delta{x}}} & = \lim\limits_{\Delta{x}\to{0}}{\dfrac{\sqrt{x + \Delta{x} + 2} - \sqrt{x + 2}}{\Delta{x}}}                                                                                                & \\
                                                                         & = \lim\limits_{\Delta{x}\to{0}}{\dfrac{\sqrt{x + \Delta{x} + 2} - \sqrt{x + 2}}{\Delta{x}}} \cdot \dfrac{\sqrt{x + \Delta{x} + 2} + \sqrt{x + 2}}{\sqrt{x + \Delta{x} + 2} + \sqrt{x + 2}} & \\
                                                                         & = \lim\limits_{\Delta{x}\to{0}}{\dfrac{x + \Delta{x} + 2 - x +-2}{\Delta{x} \left(\sqrt{x + \Delta{x} + 2} + \sqrt{x + 2}\right)}}                                                         & \\
                                                                         & = \lim\limits_{\Delta{x}\to{0}}{\dfrac{\Delta{x}}{\Delta{x} \left(\sqrt{x + \Delta{x} + 2} + \sqrt{x + 2}\right)}}                                                                         & \\
                                                                         & = \lim\limits_{\Delta{x}\to{0}}{\dfrac{1}{\sqrt{x + \Delta{x} + 2} + \sqrt{x + 2}}}                                                                                                        & \\
                                                                         & = \dfrac{1}{\sqrt{x + 2} + \sqrt{x + 2}}                                                                                                                                                   & \\
              \dfrac{dy}{dx}                                             & = \dfrac{1}{2\sqrt{x + 2}}
          \end{flalign*}
          $\therefore$ The derivative value at $x = 2$ is $\dfrac{1}{2\sqrt{2 + 2}} = \dfrac{1}{4}$.

          \newpage
          \begin{multicols}{2}
              \item $y = \dfrac{2}{x^2}$
              \sol{}
              \begin{flalign*}
                  \Delta y                                                   & = \dfrac{2}{{(x + \Delta{x})}^2} - \dfrac{2}{x^2}                                                          & \\
                                                                             & = \dfrac{2}{x^2 + 2x \Delta{x} + \Delta{x}^2} - \dfrac{2}{x^2}                                             & \\
                                                                             & = \dfrac{2x^2 - 2x^2 - 4x \Delta{x} - 2 \Delta{x}^2}{x^2 \left(x^2 + 2x \Delta{x} + \Delta{x}^2\right)}    & \\
                                                                             & = \dfrac{-4x \Delta{x} - 2 \Delta{x}^2}{x^2 \left(x^2 + 2x \Delta{x} + \Delta{x}^2\right)}                 & \\
                                                                             & = \dfrac{-4x \Delta{x} - 2 \Delta{x}^2}{x^4 + 2x^3 \Delta{x} + x^2 \Delta{x}^2}                            & \\
                  \\
                  \dfrac{\Delta y}{\Delta{x}}                                & = \dfrac{-4x \Delta{x} - 2 \Delta{x}^2}{x^4 + 2x^3 \Delta{x} + x^2 \Delta{x}^2} \cdot \dfrac{1}{\Delta{x}} & \\
                                                                             & = \dfrac{-4x - 2 \Delta{x}}{x^4 + 2x^3 \Delta{x} + x^2 \Delta{x}^2}                                        & \\
                  \\
                  \lim\limits_{\Delta{x}\to{0}}{\dfrac{\Delta y}{\Delta{x}}} & = \lim\limits_{\Delta{x}\to{0}}{\dfrac{-4x - 2 \Delta{x}}{x^4 + 2x^3 \Delta{x} + x^2 \Delta{x}^2}}         & \\
                  \dfrac{dy}{dx}                                             & = \dfrac{-4x}{x^4}                                                                                         & \\
                                                                             & = -\dfrac{4}{x^3}
              \end{flalign*}
              $\therefore$ The derivative value at $x = 2$ is $-\dfrac{4}{2^3} = -\dfrac{1}{2}$.
              \columnbreak{}
              \item $y = \dfrac{1}{1 - x}$
              \sol{}
              \begin{flalign*}
                  \Delta y                                                   & = \dfrac{1}{1 - (x + \Delta{x})} - \dfrac{1}{1 - x}                        & \\
                                                                             & = \dfrac{1}{1 - x - \Delta{x}} - \dfrac{1}{1 - x}                          & \\
                                                                             & = \dfrac{1 - x - 1 + x + \Delta{x}}{(1 - x - \Delta{x})(1 - x)}            & \\
                                                                             & = \dfrac{\Delta{x}}{(1 - x - \Delta{x})(1 - x)}                            & \\
                  \\
                  \dfrac{\Delta y}{\Delta{x}}                                & = \dfrac{\Delta{x}}{(1 - x - \Delta{x})(1 - x)} \cdot \dfrac{1}{\Delta{x}} & \\
                                                                             & = \dfrac{1}{(1 - x - \Delta{x})(1 - x)}                                    & \\
                  \\
                  \lim\limits_{\Delta{x}\to{0}}{\dfrac{\Delta y}{\Delta{x}}} & = \lim\limits_{\Delta{x}\to{0}}{\dfrac{1}{(1 - x - \Delta{x})(1 - x)}}     & \\
                  \dfrac{dy}{dx}                                             & = \dfrac{1}{(1 - x)(1 - x)}                                                & \\
                                                                             & = \dfrac{1}{{(1 - x)}^2}
              \end{flalign*}
              $\therefore$ The derivative value at $x = 2$ is $\dfrac{1}{{(1 - 2)}^2} = 1$.
          \end{multicols}

          \newpage
    \item Given the function $y = \dfrac{3}{x - 2}$, $x \neq 2$, find the following:
          \begin{enumerate}
              \item $\dfrac{\Delta y}{\Delta{x}}$
                    \sol{}
                    \begin{flalign*}
                        \Delta y                    & = \dfrac{3}{(x + \Delta{x}) - 2} - \dfrac{3}{x - 2}                           & \\
                                                    & = \dfrac{3}{x + \Delta{x} - 2} - \dfrac{3}{x - 2}                             & \\
                                                    & = \dfrac{3(x - 2) - 3(x + \Delta{x} - 2)}{(x + \Delta{x} - 2)(x - 2)}         & \\
                                                    & = \dfrac{3x - 6 - 3x - 3 \Delta{x} + 6}{(x + \Delta{x} - 2)(x - 2)}           & \\
                                                    & = \dfrac{-3 \Delta{x}}{(x + \Delta{x} - 2)(x - 2)}                            & \\
                        \\
                        \dfrac{\Delta y}{\Delta{x}} & = \dfrac{-3 \Delta{x}}{(x + \Delta{x} - 2)(x - 2)} \cdot \dfrac{1}{\Delta{x}} & \\
                                                    & = -\dfrac{3}{(x + \Delta{x} - 2)(x - 2)}                                      & \\
                    \end{flalign*}

              \item $\dfrac{dy}{dx}$
                    \sol{}
                    \begin{flalign*}
                        \dfrac{dy}{dx} & = \lim\limits_{\Delta{x}\to{0}}{\dfrac{\Delta y}{\Delta{x}}}            & \\
                                       & = \lim\limits_{\Delta{x}\to{0}}{-\dfrac{3}{(x + \Delta{x} - 2)(x - 2)}} & \\
                                       & = -\dfrac{3}{{(x - 2)}^2}
                    \end{flalign*}

              \item the derivative value at $x = 3$ \sol{}
                    \begin{flalign*}
                        \dfrac{dy}{dx} & = -\dfrac{3}{{(x - 2)}^2} & \\
                                       & = -\dfrac{3}{{(3 - 2)}^2} & \\
                                       & = -3
                    \end{flalign*}
          \end{enumerate}

          \newpage
    \item Given the function $f(x) = \dfrac{2}{\sqrt{x}}$, $x \neq 0$. Find (a) $f'(x)$;
          (b) $f'(1)$.\sol{}
          \begin{flalign*}
              \Delta y                                                   & = \dfrac{2}{\sqrt{x + \Delta{x}}} - \dfrac{2}{\sqrt{x}}                                                                                                                                                 & \\
                                                                         & = \dfrac{2\sqrt{x} - 2\sqrt{x + \Delta{x}}}{\sqrt{x(x + \Delta{x})}}                                                                                                                                    & \\
                                                                         & = \dfrac{2\left(\sqrt{x} - \sqrt{x + \Delta{x}}\right)}{\sqrt{x^2 + x\Delta{x}}}                                                                                                                        & \\
              \\
              \dfrac{\Delta y}{\Delta{x}}                                & = \dfrac{2\left(\sqrt{x} - \sqrt{x + \Delta{x}}\right)}{\Delta{x}\sqrt{x^2 + x\Delta{x}}}                                                                                                                 \\
              \lim\limits_{\Delta{x}\to{0}}{\dfrac{\Delta y}{\Delta{x}}} & = \lim\limits_{\Delta{x}\to{0}}{\dfrac{2\left(\sqrt{x} - \sqrt{x + \Delta{x}}\right)}{\Delta{x}\sqrt{x^2 + x\Delta{x}}}}                                                                                & \\
                                                                         & = \lim\limits_{\Delta{x}\to{0}}{\dfrac{2\left(\sqrt{x} - \sqrt{x + \Delta{x}}\right)}{\Delta{x}\sqrt{x^2 + x\Delta{x}}}} \cdot \dfrac{\sqrt{x} + \sqrt{x + \Delta{x}}}{\sqrt{x} + \sqrt{x + \Delta{x}}} & \\
                                                                         & = \lim\limits_{\Delta{x}\to{0}}{\dfrac{2\left(x - x - \Delta{x}\right)}{\Delta{x}\left(\sqrt{x^2 + x\Delta{x}}\right)\left(\sqrt{x} + \sqrt{x + \Delta{x}}\right)}}                                     & \\
                                                                         & = \lim\limits_{\Delta{x}\to{0}}{\dfrac{-2\Delta{x}}{\Delta{x}\left(\sqrt{x^2 + x\Delta{x}}\right)\left(\sqrt{x} + \sqrt{x + \Delta{x}}\right)}}                                                         & \\
                                                                         & = \lim\limits_{\Delta{x}\to{0}}{\dfrac{-2}{\left(\sqrt{x^2 + x\Delta{x}}\right)\left(\sqrt{x} + \sqrt{x + \Delta{x}}\right)}}                                                                           & \\
              \\
              f'(x)                                                      & = \lim\limits_{\Delta{x}\to{0}}{\dfrac{-2}{\left(\sqrt{x^2 + x\Delta{x}}\right)\left(\sqrt{x} + \sqrt{x + \Delta{x}}\right)}}                                                                           & \\
                                                                         & = \dfrac{-2}{x\left(\sqrt{x} + \sqrt{x}\right)}                                                                                                                                                         & \\
                                                                         & = -\dfrac{1}{\sqrt{x^3}}                                                                                                                                                                                & \\
              \\
              f'(1)                                                      & = -\dfrac{1}{\sqrt{1^3}}                                                                                                                                                                                & \\
                                                                         & = -1
          \end{flalign*}

          \newpage
    \item Given the curve $y = 3 - 4x - 5x^2$.
          \begin{enumerate}
              \item Find $\dfrac{dy}{dx}$ using the definition of derivative.\sol{}
                    \begin{flalign*}
                        \Delta y                    & = 3 - 4(x + \Delta{x}) - 5{(x + \Delta{x})}^2 - (3 - 4x - 5x^2)            & \\
                                                    & = 3 - 4x - 4\Delta{x} - 5x^2 - 10x\Delta{x} - 5\Delta{x}^2 - 3 + 4x + 5x^2 & \\
                                                    & = -4\Delta{x} - 10x\Delta{x} - 5\Delta{x}^2                                & \\
                        \\
                        \dfrac{\Delta y}{\Delta{x}} & = \dfrac{-4\Delta{x} - 10x\Delta{x} - 5\Delta{x}^2}{\Delta{x}}               \\
                                                    & = -4 - 10x - 5\Delta{x}                                                      \\
                        \\
                        \dfrac{dy}{dx}              & = \lim\limits_{\Delta{x}\to{0}}{\dfrac{\Delta y}{\Delta{x}}}               & \\
                                                    & = \lim\limits_{\Delta{x}\to{0}}{-4 - 10x - 5\Delta{x}}                     & \\
                                                    & = -4 - 10x
                    \end{flalign*}

              \item Find the gradient of the tangent to the curve at the point where $x = -2$.
                    \sol{}
                    \begin{flalign*}
                        \dfrac{dy}{dx} & = -4 - 10x    & \\
                                       & = -4 - 10(-2) & \\
                                       & = 16
                    \end{flalign*}

              \item If the gradient of the tangent to the curve at point $P$ is $6$, find the
                    coordinates of point $P$.\sol{}
                    \begin{flalign*}
                        \dfrac{dy}{dx} & = 6                     & \\
                        -4 - 10x       & = 6                     & \\
                        -10x           & = 10                    & \\
                        x              & = -1                    & \\
                        y              & = 3 - 4x - 5x^2         & \\
                                       & = 3 - 4(-1) - 5{(-1)}^2 & \\
                                       & = 3 + 4 - 5             & \\
                                       & = 2                       \\
                        \\
                        P              & = (-1, 2)
                    \end{flalign*}
          \end{enumerate}
\end{enumerate}

\section{Law of Differentiation}

Since the method of finding the derivative using the definition of derivative
is very complicated, we need to derive some formulas from the definition of the
derivative, then we can use these formulas to find the derivatives.

\subsection*{Derivative of Power Function}

Let $y = f (x) = c$, where $c$ is a constant.
\begin{flalign*}
    \dfrac{\Delta y}{\Delta{x}} & = \dfrac{f (x + \Delta{x}) - f (x)}{\Delta{x}}               \\
                                & = \dfrac{c - c}{\Delta{x}}                                   \\
                                & = 0                                                          \\
    \\
    \dfrac{d}{dx} (c)           & = \lim\limits_{\Delta{x}\to{0}}{\dfrac{\Delta y}{\Delta{x}}} \\
                                & = \lim\limits_{\Delta{x}\to{0}}{0}                           \\
                                & = 0
\end{flalign*}

\begin{mdframed}[style=MyFrame]
    \begin{cequation}
        \dfrac{d}{dx} (c) = 0
    \end{cequation}
\end{mdframed}

Let $y = f (x) = x^n$, where $n$ is a positive integer.
\begin{flalign*}
    \Delta y = f (x + \Delta{x}) - f (x) & = f (x - \Delta{x}) - f (x)                                                                                                                    \\
                                         & = {(x + \Delta{x})}^n - x^n                                                                                                                    \\
                                         & = \left[x^n + \comb[n]{1}x^{n - 1}\Delta{x} + \comb[n]{2}x^{n - 2}{(\Delta{x})}^2 + \cdots + \comb[n]{n}{(\Delta{x})}^n\right] - x^n           \\
                                         & = \comb[n]{1}x^{n - 1}\Delta{x} + \comb[n]{2}x^{n - 2}{(\Delta{x})}^2 + \cdots + \comb[n]{n}{(\Delta{x})}^n                                    \\
    \dfrac{\Delta y}{\Delta{x}}          & = \comb[n]{1}x^{n - 1} + \comb[n]{2}x^{n - 2}\Delta{x} + \cdots + \comb[n]{n}{(\Delta{x})}^{n - 1}                                             \\
    \\
    \dfrac{d}{dx} (x^n)                  & = \lim\limits_{\Delta{x}\to{0}}{\dfrac{\Delta y}{\Delta{x}}}                                                                                   \\
                                         & = \lim\limits_{\Delta{x}\to{0}}{\left[\comb[n]{1}x^{n - 1} + \comb[n]{2}x^{n - 2}\Delta{x} + \cdots + \comb[n]{n}{(\Delta{x})}^{n - 1}\right]} \\
                                         & = \comb[n]{1}x^{n - 1}
\end{flalign*}

\begin{mdframed}[style=MyFrame]
    \begin{cequation}
        \dfrac{d}{dx} (x^n) = nx^{n-1}
    \end{cequation}
\end{mdframed}
The derivation above only considered the case where $n$ is a positive integer. In fact, the formula above is also valid when $n$ is a negative integer or a rational number.

\subsection*{Derivative of Product of Function and Constant}

Let $y = cu$, where $c$ is a constant and $u$ is a differentiable function of
$x$.
\begin{flalign*}
    y + \Delta y                & = c(u + \Delta u) = cu + c\Delta u                                         \\
    \Delta y                    & = c\Delta u                                                                \\
    \dfrac{\Delta y}{\Delta{x}} & = c\dfrac{\Delta u}{\Delta{x}}                                             \\
    \\
    \dfrac{d}{dx} (cu)          & = \lim\limits_{\Delta{x}\to{0}}{\left(c\dfrac{\Delta u}{\Delta{x}}\right)} \\
                                & = c\lim\limits_{\Delta{x}\to{0}}{\dfrac{\Delta u}{\Delta{x}}}              \\
                                & = c\dfrac{du}{dx}
\end{flalign*}

\begin{mdframed}[style=MyFrame]
    \begin{cequation}
        \dfrac{d}{dx} (cu) = c\dfrac{du}{dx} \quad \text{(where $c$ is a constant, $u$ is a differentiable function of $x$)}
    \end{cequation}
\end{mdframed}

\subsection*{Derivative of Sum and Difference of Functions}

Let $y = u \pm v$, where $u$ and $v$ are differentiable functions of $x$.
\begin{flalign*}
    y + \Delta y                & = (u + \Delta u) \pm (v + \Delta v)                                                                                         \\
    \Delta y                    & = \Delta u \pm \Delta v                                                                                                     \\
    \dfrac{\Delta y}{\Delta{x}} & = \dfrac{\Delta u}{\Delta{x}} \pm \dfrac{\Delta v}{\Delta{x}}                                                               \\
    \\
    \dfrac{d}{dx} (u \pm v)     & = \lim\limits_{\Delta{x}\to{0}}{\left(\dfrac{\Delta u}{\Delta{x}} \pm \dfrac{\Delta v}{\Delta{x}}\right)}                   \\
                                & = \lim\limits_{\Delta{x}\to{0}}{\dfrac{\Delta u}{\Delta{x}}} \pm \lim\limits_{\Delta{x}\to{0}}{\dfrac{\Delta v}{\Delta{x}}} \\
                                & = \dfrac{du}{dx} \pm \dfrac{dv}{dx}
\end{flalign*}

\begin{mdframed}[style=MyFrame]
    \begin{cequation}
        \def\arraystretch{1.5}
        \begin{array}{l}
            \dfrac{d}{dx} (u + v) = \dfrac{du}{dx} + \dfrac{dv}{dx} \\
            \dfrac{d}{dx} (u - v) = \dfrac{du}{dx} - \dfrac{dv}{dx}
        \end{array} \quad \text{(where $u$ and $v$ are differentiable functions of $x$)}
    \end{cequation}
\end{mdframed}

The above formulae can be extended to the case where there are more than two
functions being added or subtracted. That is,
\begin{cequation}
    \dfrac{d}{dx} (u_1 \pm u_2 \pm \cdots \pm u_n) = \dfrac{du_1}{dx} \pm \dfrac{du_2}{dx} \pm \cdots \pm \dfrac{du_n}{dx}
\end{cequation}

\subsection*{Product Rule}

Let $y = uv$, where $u$ and $v$ are differentiable functions of $x$.
\begin{flalign*}
    y + \Delta y                & = (u + \Delta u)(v + \Delta v)                                                                       \\
                                & = uv + u\Delta v + v\Delta u + \Delta u \Delta v                                                     \\
    \Delta y                    & = u\Delta v + v\Delta u + \Delta u \Delta v                                                          \\
    \dfrac{\Delta y}{\Delta{x}} & = u\dfrac{\Delta v}{\Delta{x}} + v\dfrac{\Delta u}{\Delta{x}} + \Delta u \dfrac{\Delta v}{\Delta{x}}
\end{flalign*}
\ \ \ \ \ \ \ \ \ \ Given that $\lim\limits_{\Delta{x}\to{0}}{\dfrac{\Delta u}{\Delta{x}}} = \dfrac{du}{dx}$ and $\lim\limits_{\Delta{x}\to{0}}{\dfrac{\Delta v}{\Delta{x}}} = \dfrac{dv}{dx}$,
\begin{flalign*}
    \text{Hence, }\lim\limits_{\Delta{x}\to{0}}{\Delta v} & = \lim\limits_{\Delta{x}\to{0}}{\left(\dfrac{\Delta v}{\Delta{x}} \cdot \Delta{x}\right)}                                                                                                                                                   \\
                                                          & = \lim\limits_{\Delta{x}\to{0}}{\dfrac{\Delta v}{\Delta{x}}} \cdot \lim\limits_{\Delta{x}\to{0}}{\Delta{x}}                                                                                                                                 \\
                                                          & = v' (x) \cdot 0                                                                                                                                                                                                                            \\
                                                          & = 0                                                                                                                                                                                                                                         \\
    \\
    \therefore\ \lim\limits_{\Delta{x}\to{0}}{(uv)}       & = \lim\limits_{\Delta{x}\to{0}}{\left(u\dfrac{\Delta v}{\Delta{x}} + v\dfrac{\Delta u}{\Delta{x}} + \Delta v \dfrac{\Delta v}{\Delta{x}}\right)}                                                                                            \\
                                                          & = u\lim_{\Delta{x}\to{0}}{\dfrac{\Delta v}{\Delta{x}}} + v\lim_{\Delta{x}\to{0}}{\dfrac{\Delta u}{\Delta{x}}} + \left(\lim\limits_{\Delta{x}\to{0}}{\dfrac{\Delta u}{\Delta{x}}}\right)\left(\lim\limits_{\Delta{x}\to{0}}{\Delta v}\right) \\
                                                          & = u\dfrac{dv}{dx} + v\dfrac{du}{dx} + 0                                                                                                                                                                                                     \\
                                                          & = u\dfrac{dv}{dx} + v\dfrac{du}{dx}
\end{flalign*}

\begin{mdframed}[style=MyFrame]
    \begin{cequation}
        \dfrac{d}{dx} (uv) = u\dfrac{dv}{dx} + v\dfrac{du}{dx} \quad \text{(where $u$ and $v$ are differentiable functions of $x$)}
    \end{cequation}
\end{mdframed}

\noindent This has proven that if $\dfrac{dv}{dx}$ exists at $x = x_0$, then
\begin{flalign*}
     & \lim\limits_{\Delta{x}\to{0}}{v(x_0 + \Delta{x}) - v(x_0)} \\
     & = \lim_{\Delta{x}\to{0}}{\Delta v}                         \\
     & = 0
\end{flalign*}
From that, we can conclude that
\[
    \lim_{\Delta{x}\to{0}}{v(x_0 + \Delta{x})} = v(x_0)
\]
That is, $y = v(x)$ is continuous at $x = x_0$.

\noindent Note that $\dfrac{d}{dx} (uv) \neq \dfrac{d}{dx} (u) \cdot \dfrac{d}{dx} (v)$.

\newpage
\subsection{Practice 3}

\noindent Find the derivative of the following functions:
\begin{enumerate}
    \begin{multicols}{2}
        \item $y = 5x$
        \sol{}
        \begin{flalign*}
            y' & = 5
        \end{flalign*}
        \columnbreak{}
        \item $y={\sqrt{x}}$
        \sol{}
        \begin{flalign*}
            y' & = \dfrac{1}{2\sqrt{x}}
        \end{flalign*}
    \end{multicols}

    \begin{multicols}{2}
        \item $y=6x^{3}+3x^{2}-5$
        \sol{}
        \begin{flalign*}
            y' & = 18x^{2}+6x
        \end{flalign*}
        \columnbreak{}
        \item $y=(2x-3)\left(x^{2}+2\right)$
        \sol{}
        \begin{flalign*}
            y' & = (x^2 + 2)(2x - 3)' + (2x - 3)(x^2 + 2)' \\
               & = (x^2 + 2)(2) + (2x - 3)(2x)             \\
               & = 2x^2 + 4 + 4x^2 - 6x                    \\
               & = 6x^2 - 6x + 4
        \end{flalign*}
    \end{multicols}

    \begin{multicols}{2}
        \item $y={\dfrac{2}{x}}+{\dfrac{1}{x^{2}}}$
        \sol{}
        \begin{flalign*}
            y' & = \left(\dfrac{2x + 1}{x^2}\right)'          \\
               & = \dfrac{(2x + 1)'x^2 - (2x + 1)(x^2)'}{x^4} \\
               & = \dfrac{2x^2 - (2x + 1)(2x)}{x^4}           \\
               & = \dfrac{2x^2 - 4x^2 - 2x}{x^4}              \\
               & = \dfrac{-2x^2 - 2x}{x^4}                    \\
               & = \dfrac{-2x(x + 1)}{x^4}                    \\
               & = -\dfrac{2(x + 1)}{x^3}
        \end{flalign*}
        \columnbreak{}
        \item $y=2x^{3}-3x+{\dfrac{7}{\sqrt{x}}}$
        \sol{}
        \begin{flalign*}
            y' & = 6x^2 - 3 - \dfrac{7}{2\sqrt{x^3}}       \\
               & = 6x^2 - 3 - \dfrac{7}{2}x^{-\frac{3}{2}}
        \end{flalign*}
    \end{multicols}
\end{enumerate}

\newpage
\subsection{Exercise 25.3a}

\noindent Find the derivative of the following functions (Question 1 to 18):
\begin{enumerate}
    \begin{multicols}{2}
        \item $y=2x^{3}+2$
        \sol{}
        \begin{flalign*}
            y' & = 6x^2
        \end{flalign*}
        \columnbreak{}
        \item $y={\dfrac{1}{2}}x^{2}-{\dfrac{1}{3}}x+2$
        \sol{}
        \begin{flalign*}
            y' & = x - \dfrac{1}{3}
        \end{flalign*}
    \end{multicols}

    \begin{multicols}{2}
        \item $y=x^{5}-{\dfrac{1}{4}}x^{4}+3x^{2}-4$
        \sol{}
        \begin{flalign*}
            y' & = 5x^4 - x^3 + 6x
        \end{flalign*}

        \columnbreak{}
        \item $y=2x^{2}+4x^{3}-7x^{4}$
        \sol{}
        \begin{flalign*}
            y' & = 4x + 12x^2 - 28x^3
        \end{flalign*}

    \end{multicols}
    \begin{multicols}{2}
        \item $y=x^{2}-3x+{\dfrac{2}{x}}-{\dfrac{4}{x^{2}}}$
        \sol{}
        \begin{flalign*}
            y' & = 2x - 3 + \dfrac{2}{x^2} - \dfrac{8}{x^3}
        \end{flalign*}

        \columnbreak{}
        \item $y=2x^{3}-4x-{\dfrac{3}{x^{2}}}+{\dfrac{4}{x^{3}}}$
        \sol{}
        \begin{flalign*}
            y' & = 6x^2 - 4 + \dfrac{6}{x^3} - \dfrac{12}{x^4}
        \end{flalign*}

    \end{multicols}
    \begin{multicols}{2}
        \item $y={\sqrt{x}}+{\dfrac{2}{\sqrt{x}}}+3$
        \sol{}
        \begin{flalign*}
            y' & = \left(x^\frac{1}{2} + 2x^{-\frac{1}{2}} + 3\right)'         \\
               & = \dfrac{1}{2}x^{-\frac{1}{2}} - \dfrac{1}{2}x^{-\frac{3}{2}}
        \end{flalign*}

        \columnbreak{}
        \item $y=3{\sqrt{x}}-5{\sqrt[3]{x^{2}}}$
        \sol{}
        \begin{flalign*}
            y' & = \left(3x^\frac{1}{2} - 5x^\frac{2}{3}\right)'                \\
               & = \dfrac{3}{2}x^{-\frac{1}{2}} - \dfrac{10}{3}x^{-\frac{1}{3}}
        \end{flalign*}

    \end{multicols}
    \begin{multicols}{2}
        \item $y={(2x-5)}^{2}$
        \sol{}
        \begin{flalign*}
            y' & = (4x^2 - 20x + 25)' \\
               & = 8x - 20
        \end{flalign*}
        \vfill{}\null{}
        \columnbreak{}
        \item $y={\left(x+{\dfrac{1}{x}}\right)}^{2}$
        \sol{}
        \begin{flalign*}
            y' & = \left(x^2 + 2 + \dfrac{1}{x^2}\right)' \\
               & = \left(x^2 + 2 + x^{-2}\right)'         \\
               & = 2x - 2x^{-3}                           \\
               & = 2x - \dfrac{2}{x^3}
        \end{flalign*}

    \end{multicols}
    \newpage
    \begin{multicols}{2}
        \item $y=\dfrac{x^{3}+4x^{2}-x+2}{x}$
        \sol{}
        \begin{flalign*}
            y' & = \left(x^2 + 4x - 1 + 2x^{-1}\right)' \\
               & = 2x + 4 - \dfrac{2}{x^2}
        \end{flalign*}
        \columnbreak{}
        \item $y=\dfrac{2x^{4}+3x^{2}-6}{x^{2}}$
        \sol{}
        \begin{flalign*}
            y' & = \left(2x^2 + 3 - 6x^{-2}\right)' \\
               & = 4x + 12x^{-3}                    \\
               & = 4x + \dfrac{12}{x^3}
        \end{flalign*}

    \end{multicols}
    \begin{multicols}{2}
        \item $y=\dfrac{\sqrt[3]{x}-2}{\sqrt{x}}$
        \sol{}
        \begin{flalign*}
            y' & = \left(x^{-\frac{1}{6}} - 2x^{-\frac{1}{2}}\right)' \\
               & = -\dfrac{1}{6}x^{-\frac{7}{6}} + x^{-\frac{3}{2}}
        \end{flalign*}

        \columnbreak{}
        \item $y=\left(x+2\right)\left(x^{2}+3x-8\right)$
        \sol{}
        \begin{flalign*}
            y' & = \left(x^3 + 3x^2 - 8x + 2x^2 + 6x - 16\right)' \\
               & = \left(x^3 + 5x^2 - 2x - 16\right)'             \\
               & = 3x^2 + 10x - 2
        \end{flalign*}

    \end{multicols}
    \begin{multicols}{2}
        \item $y=\left(2+3x\right)\left(1+x-x^{2}\right)$
        \sol{}
        \begin{flalign*}
            y' & = (1 + x - x^2)(2 + 3x)' + (2 + 3x)   & \\
               & \ \ \ \ (1 + x - x^2)'                  \\
               & = (1 + x - x^2)(3) + (2 + 3x)(1 - 2x)   \\
               & = 3 + 3x - 3x^2 + 2 - x - 6x^2          \\
               & = 5 + 2x - 9x^2
        \end{flalign*}

        \columnbreak{}
        \item $y=\left(x+2\right)\left(x-2\right)\left(x^{2}+4\right)$
        \sol{}
        \begin{flalign*}
            y' & = \left[(x^2 - 4)(x^2 + 4)\right]' \\
               & = \left(x^4 - 16\right)'           \\
               & = 4x^3
        \end{flalign*}

    \end{multicols}
    \begin{multicols}{2}
        \item $y={(x-1)}^{2}(3x+5)$
        \sol{}
        \begin{flalign*}
            y' & = \left[(x^2 - 2x + 1)(3x + 5)\right]' & \\
               & = (x^2 - 2x + 1)(3x + 5)'                \\
               & \ \ \ \ + (3x + 5)(x^2 - 2x + 1)'        \\
               & = (x^2 - 2x + 1)(3) + (3x + 5)(2x - 2)   \\
               & = 3x^2 - 6x + 3 + 6x^2 + 4x - 10         \\
               & = 9x^2 - 2x - 7
        \end{flalign*}

        \columnbreak{}
        \item $y=\left(x^{2}+1\right)\left(3x-1\right)\left(1-x^{2}\right)$
        \sol{}
        \begin{flalign*}
            y' & = \left[(1 - x^4)(3x - 1)\right]'         \\
               & = (1 - x^4)'(3x - 1) + (3x - 1)'(1 - x^4) \\
               & = (-4x^3)(3x - 1) + (3)(1 - x^4)          \\
               & = -12x^4 + 4x^3 + 3 - 3x^4                \\
               & = -15x^4 + 4x^3 + 3
        \end{flalign*}

    \end{multicols}
    \newpage
    \begin{multicols}{2}
        \item If the gradient of the curve $y = x^3 + 6x^2 \\+ 45x + 12$ at point $A$ is
        $36$, find the \\coordinates of $A$.\sol{}
        \begin{flalign*}
            y'     & = \left(x^3 + 6x^2 + 45x + 12\right)' \\
            36     & = 3x^2 + 12x + 45                     \\
            12     & = x^2 + 4x + 15                       \\
            0      & = x^2 + 4x + 3                        \\
            0      & = (x + 3)(x + 1)                      \\
            x = -3 & \text{ or } x = -1
        \end{flalign*}
        \vspace{-1.2cm}
        \begin{flalign*}
            y(-3) & = -27 + 54 - 135 + 12 \\
                  & = -96                 \\
            y(-1) & = -1 + 6 - 45 + 12    \\
                  & = -28
        \end{flalign*}
        Therefore, the coordinates of $A$ are $(-3, -96)$ and $(-1, -28)$.
        \columnbreak{}
        \item Given the functions $f (x) = f (x) = x^2 + 3x + 4$ and $g(x) = x^3 + x^2 + 7$.
        If $f' (a) = g' (a)$, find the value of $a$.\sol{}
        \begin{flalign*}
            f'(x)    & = 2x + 3    \\
            g'(x)    & = 3x^2 + 2x \\
            \\
            f'(a)    & = g'(a)     \\
            2a + 3   & = 3a^2 + 2a \\
            3a^2 - 3 & = 0         \\
            a^2      & = 1         \\
            a        & = \pm 1
        \end{flalign*}
    \end{multicols}
\end{enumerate}

\newpage
\subsection*{Quotient Rule}

Let $y = \dfrac{u}{v}$, where $u$ and $v$ are differentiable functions of $x$.
\begin{flalign*}
    y + \Delta y                & = \dfrac{u + \Delta u}{v + \Delta v}                                                   \\
    \Delta y                    & = \dfrac{u + \Delta u}{v + \Delta v} - \dfrac{u}{v}                                    \\
                                & = \dfrac{v\Delta u - u\Delta v}{v(v + \Delta v)}                                       \\
    \dfrac{\Delta y}{\Delta{x}} & = \dfrac{v\dfrac{\Delta u}{\Delta{x}} - u\dfrac{\Delta v}{\Delta{x}}}{v(v + \Delta v)}
\end{flalign*}
When $\dfrac{\Delta{x}}{\Delta{x}}$ exists, $\lim\limits_{\Delta{x}\to{0}}{\Delta v} = 0$.
\begin{flalign*}
    \therefore\ \dfrac{d}{dx}\left(\dfrac{u}{v}\right) & = \lim\limits_{\Delta{x}\to{0}}{\dfrac{\Delta y}{\Delta{x}}}                                                                                                                                                       \\
                                                       & = \lim\limits_{\Delta{x}\to{0}}{\dfrac{v\dfrac{\Delta u}{\Delta{x}} - u\dfrac{\Delta v}{\Delta{x}}}{v(v + \Delta v)}}                                                                                              \\
                                                       & = \dfrac{v\lim\limits_{\Delta{x}\to{0}}{\dfrac{\Delta u}{\Delta{x}}} - u\lim\limits_{\Delta{x}\to{0}}{\dfrac{\Delta v}{\Delta{x}}}}{\lim\limits_{\Delta{x}\to{0}}{v^2} + v\lim\limits_{\Delta{x}\to{0}}{\Delta v}} \\
                                                       & = \dfrac{v\dfrac{du}{dx} - u\dfrac{dv}{dx}}{v^2}
\end{flalign*}

\begin{mdframed}[style=MyFrame]
    \begin{cequation}
        \dfrac{d}{dx}\left(\dfrac{u}{v}\right) = \dfrac{v\dfrac{du}{dx}-u\dfrac{dv}{dx}}{v^2} \quad \text{($v \neq 0$, $u$ and $v$ are differentiable functions of $x$)}
    \end{cequation}
\end{mdframed}

\newpage
\subsection{Practice 4}
\noindent Find the derivative of the following functions:
\begin{enumerate}
    \begin{multicols}{2}
        \item $y={\dfrac{2x}{x+2}}$
        \sol{}
        \begin{flalign*}
            y' & = \dfrac{(x + 2)(2x)' - 2x(2x + 2)'}{{(x + 2)}^2} \\
               & = \dfrac{(x + 2)(2) - 2x(2)}{{(x + 2)}^2}         \\
               & = \dfrac{2x + 4 - 4x}{{(x + 2)}^2}                \\
               & = \dfrac{-2x + 4}{{(x + 2)}^2}
        \end{flalign*}
        \columnbreak{}
        \item $y={\dfrac{3x-2}{x+2}}$
        \sol{}
        \begin{flalign*}
            y' & = \dfrac{(x + 2)(3x)' - (3x - 2)(x + 2)'}{{(x + 2)}^2} \\
               & = \dfrac{(x + 2)(3) - (3x - 2)(1)}{{(x + 2)}^2}        \\
               & = \dfrac{3x + 6 - 3x + 2}{{(x + 2)}^2}                 \\
               & = \dfrac{8}{{(x + 2)}^2}
        \end{flalign*}
    \end{multicols}

    \begin{multicols}{2}
        \item $y={\dfrac{x}{x^{2}-5}}$
        \sol{}
        \begin{flalign*}
            y' & = \dfrac{(x^2 - 5)(x)' - x(x^2 - 5)'}{{(x^2 - 5)}^2} \\
               & = \dfrac{(x^2 - 5)(1) - x(2x)}{{(x^2 - 5)}^2}        \\
               & = \dfrac{x^2 - 5 - 2x^2}{{(x^2 - 5)}^2}              \\
               & = -\dfrac{x^2 + 5}{{(x^2 - 5)}^2}
        \end{flalign*}
        \columnbreak{}
        \item $y={\dfrac{x^{2}-1}{3x-2}}$
        \sol{}
        \begin{flalign*}
            y' & = \dfrac{(3x - 2)(x^2 - 1)' - (x^2 - 1)(3x - 2)'}{{(3x - 2)}^2} \\
               & = \dfrac{(3x - 2)(2x) - (x^2 - 1)(3)}{{(3x - 2)}^2}             \\
               & = \dfrac{6x^2 - 4x - 3x^2 + 3}{{(3x - 2)}^2}                    \\
               & = \dfrac{3x^2 - 4x + 3}{{(3x - 2)}^2}
        \end{flalign*}
    \end{multicols}
\end{enumerate}

\newpage\subsection{Exercise 25.3b}
\noindent Find the derivative of the following functions:
\begin{enumerate}
    \begin{multicols}{2}
        \item $y={\dfrac{x-2}{x+2}}$
        \sol{}
        \begin{flalign*}
            y' & = \dfrac{(x + 2)(x - 2)' - (x - 2)(x + 2)'}{{(x + 2)}^2} \\
               & = \dfrac{(x + 2)(1) - (x - 2)(1)}{{(x + 2)}^2}           \\
               & = \dfrac{x + 2 - x + 2}{{(x + 2)}^2}                     \\
               & = \dfrac{4}{{(x + 2)}^2}
        \end{flalign*}
        \columnbreak{}
        \item $y={\dfrac{x-a}{2x+a}}$here $a$ is a constant
        \sol{}
        \begin{flalign*}
            y' & = \dfrac{(2x + a)(x - a)' - (x - a)(2x + a)'}{{(2x + a)}^2} \\
               & = \dfrac{(2x + a)(1) - (x - a)(2)}{{(2x + a)}^2}            \\
               & = \dfrac{2x + a - 2x + 2a}{{(2x + a)}^2}                    \\
               & = \dfrac{3a}{{(2x + a)}^2}
        \end{flalign*}
    \end{multicols}

    \begin{multicols}{2}
        \item $y={\dfrac{2x^{3}}{x+2}}$
        \sol{}
        \begin{flalign*}
            y' & = \dfrac{(x + 2)(2x^3)' - 2x^3(x + 2)'}{{(x + 2)}^2} \\
               & = \dfrac{(x + 2)(6x^2) - 2x^3(1)}{{(x + 2)}^2}       \\
               & = \dfrac{6x^3 + 12x^2 - 2x^3}{{(x + 2)}^2}           \\
               & = \dfrac{4x^3 + 12x^2}{{(x + 2)}^2}
        \end{flalign*}
        \columnbreak{}
        \item $y={\dfrac{2x+3}{x^{2}+1}}$
        \sol{}
        \begin{flalign*}
            y' & = \dfrac{(x^2 + 1)(2x + 3)' - (2x + 3)(x^2 + 1)'}{{(x^2 + 1)}^2} \\
               & = \dfrac{(x^2 + 1)(2) - (2x + 3)(2x)}{{(x^2 + 1)}^2}             \\
               & = \dfrac{2x^2 + 2 - 4x^2 - 6x}{{(x^2 + 1)}^2}                    \\
               & = \dfrac{-2x^2 - 6x + 2}{{(x^2 + 1)}^2}
        \end{flalign*}
    \end{multicols}

    \begin{multicols}{2}
        \item $y={\dfrac{3x}{x^{2}-4x}}$
        \sol{}
        \begin{flalign*}
            y' & = \dfrac{(x^2 - 4x)(3x)' - 3x(x^2 - 4x)'}{{(x^2 - 4x)}^2} \\
               & = \dfrac{(x^2 - 4x)(3) - 3x(2x - 4)}{{(x^2 - 4x)}^2}      \\
               & = \dfrac{3x^2 - 12x - 6x^2 + 12x}{{(x^2 - 4x)}^2}         \\
               & = \dfrac{-3x^2}{{(x^2 - 4x)}^2}                           \\
               & = -\dfrac{3}{{(x - 4)}^2}
        \end{flalign*}
        \item $y={\dfrac{3x^{2}+x-1}{2x-1}}$
        \sol{}
        \begin{flalign*}
            y' & = \dfrac{(2x - 1)(6x + 1) - (3x^2 + x - 1)(2)}{{(2x - 1)}^2} \\
               & = \dfrac{12x^2 - 4x - 1 - 6x^2 - 2x + 2}{{(2x - 1)}^2}       \\
               & = \dfrac{6x^2 - 6x + 1}{{(2x - 1)}^2}
        \end{flalign*}
    \end{multicols}

    \begin{multicols}{2}
        \item $y={\dfrac{2x^{4}}{{\left(x+3\right)}^{2}}}$
        \sol{}
        \begin{flalign*}
            y' & = \dfrac{{(x + 3)}^2(2x^4)' - 2x^4\bigl[{(x + 3)}^2\bigr]'}{{(x + 3)}^4} \\
               & = \dfrac{{(x + 3)}^2(8x^3) - 2x^4(2)(x + 3)}{{(x + 3)}^4}                \\
               & = \dfrac{(x + 3)\left[(x + 3)(8x^3) - 4x^4\right]}{{(x + 3)}^4}          \\
               & = \dfrac{8x^4 + 24x^3 - 4x^4}{{(x + 3)}^3}                               \\
               & = \dfrac{4x^4 + 24x^3}{{(x + 3)}^3}
        \end{flalign*}
        \columnbreak{}
        \item $y={\dfrac{2}{1+x}}+{\dfrac{2}{1-x}}$
        \sol{}
        \begin{flalign*}
            y' & = \dfrac{2(1 - x + 1 + x)}{(1 - x)(1+x)}             \\
               & = \dfrac{4}{1 - x^2}                                 \\
               & = \dfrac{(1 - x^2)(4)' - 4(1 - x^2)'}{{(1 - x^2)}^2} \\
               & = \dfrac{- 4(-2x)}{{(1 - x^2)}^2}                    \\
               & = \dfrac{8x}{{(1 - x^2)}^2}
        \end{flalign*}
    \end{multicols}

    \begin{multicols}{2}
        \item $y={\dfrac{4-x}{3-2x+x^{2}}}$
        \sol{}
        \begin{flalign*}
            y' & = \dfrac{(3 - 2x + x^2)(-1) - (4 - x)(-2 + 2x)}{{(3 - 2x + x^2)}^2} & \\
               & = \dfrac{-3 + 2x - x^2 + 8 - 2x + 2x^2}{{(3 - 10x + x^2)}^2}          \\
               & = \dfrac{x^2 - 8x + 5}{{(3 - 2x + x^2)}^2}
        \end{flalign*}
        \columnbreak{}
        \item $y={\dfrac{1+x-x^{2}}{1-x+x^{2}}}$
        \sol{}
        \begin{flalign*}
            y' & = \dfrac{(1 - x + x^2)(1 - 2x) - (1 + x - x^2)(-1 + 2x)}{{(1 - x + x^2)}^2} & \\
               & = \dfrac{(1 - 2x)(1 - x + x^2 + 1 + x - x^2)}{{(1 - x + x^2)}^2}              \\
               & = \dfrac{(1 - 2x)(2)}{{(1 - x + x^2)}^2}                                      \\
               & = \dfrac{-4x + 2}{{(1 - x + x^2)}^2}
        \end{flalign*}
    \end{multicols}
\end{enumerate}

\newpage
\section{Chain Rule - Differentiation of Composite Functions}

If $y = f (u)$ and $u = g(x)$ are both differentiable functions of $x$, then $y
    = f\bigl(g(x)\bigr)$ is also a differentiable function of $x$. In a composite
function $y = f\bigl(g(x)\bigr)$, when $x$ changes by $\Delta{x}$, $u$ changes
by $\Delta u$ and $y$ also changes by $\Delta y$, where $\dfrac{\Delta
        y}{\Delta{x}} = \dfrac{\Delta y}{\Delta u} \cdot \dfrac{\Delta u}{\Delta{x}}$.

If the derivative of the functions $y = f (u)$ and $u = g(x)$ are
$\dfrac{dy}{du}$ and $\dfrac{du}{dx}$ respectively, then when
$\Delta{x}\to{0}$, $\Delta u \to 0$, and
$\lim\limits_{\Delta{x}\to{0}}{\dfrac{\Delta u}{\Delta{x}}} = \dfrac{du}{dx}$,
$\lim\limits_{\Delta u \to 0}{\dfrac{\Delta y}{\Delta u}} = \dfrac{dy}{du}$.
\begin{flalign*}
    \text{Hence, } \lim\limits_{\Delta{x}\to{0}}{\dfrac{\Delta y}{\Delta{x}}} & = \lim\limits_{\Delta{x}\to{0}}{\left(\dfrac{\Delta y}{\Delta u} \cdot \dfrac{\Delta u}{\Delta{x}}\right)}                   \\
                                                                              & = \lim\limits_{\Delta{x}\to{0}}{\dfrac{\Delta y}{\Delta u}} \cdot \lim\limits_{\Delta{x}\to{0}}{\dfrac{\Delta u}{\Delta{x}}} \\
                                                                              & = \dfrac{dy}{du} \cdot \dfrac{du}{dx}
\end{flalign*}
Therefore, the derivative of the composite function $y = f\bigl(g(x)\bigr)$ is
\begin{mdframed}[style=MyFrame]
    \begin{cequation}
        \dfrac{dy}{dx} = \dfrac{dy}{du} \cdot \dfrac{du}{dx} = f'\bigl(g (x)\bigr) \cdot g' (x)
    \end{cequation}
\end{mdframed}
Further extending the chain rule, if $y = f (v)$, $v = g(u)$, and $u = h(x)$, then
\begin{mdframed}[style=MyFrame]
    \begin{cequation}
        \dfrac{dy}{dx} = \dfrac{dy}{dv} \cdot \dfrac{dv}{du} \cdot \dfrac{du}{dx}
    \end{cequation}
\end{mdframed}
The rule above is called the \textit{chain rule}.

\newpage
\subsection{Practice 5}
\noindent Find the derivative of the following functions:
\begin{enumerate}
    \begin{multicols}{2}
        \item $y={\left(2x^{3}-4\right)}^{3}$
        \sol{}
        \begin{flalign*}
            y' & = 3{\left(2x^{3}-4\right)}^{2} \cdot \left(2x^{3}-4\right)' & \\
               & = 3{\left(2x^{3}-4\right)}^{2} \cdot 6x^{2}                   \\
               & = 18x^{2}{\left(2x^{3}-4\right)}^{2}                          \\
               & = 72x^2{\left(x^3-2\right)}^2
        \end{flalign*}
        \columnbreak{}
        \item $y={\dfrac{6}{\sqrt{2x-3}}}$
        \sol{}
        \begin{flalign*}
            y' & = 6{(2x - 3)}^{-\frac{1}{2}}                                         \\
               & = 6\cdot \left(-\dfrac{1}{2}\right){(2x - 3)}^{-\frac{3}{2}} \cdot 2 \\
               & = -6{(2x - 3)}^{-\frac{3}{2}}                                        \\
               & = -\dfrac{6}{\sqrt{{(2x - 3)}^3}}
        \end{flalign*}
    \end{multicols}

    \begin{multicols}{2}
        \item $y=(x+1){(x-3)}^{3}$
        \sol{}
        \begin{flalign*}
            y' & = (x+1)'{(x-3)}^{3} + (x+1)\left[{(x-3)}^{3}\right]' & \\
               & = {(x-3)}^{3} + (x+1)\cdot 3{(x-3)}^{2} \cdot (x-3)'   \\
               & = {(x-3)}^{3} + 3{(x-3)}^{2}(x + 1)                    \\
               & = {(x-3)}^{2}\left[{(x-3)} + 3(x + 1)\right]           \\
               & = {(x-3)}^{2}(x - 3 + 3x + 3)                          \\
               & = 4x{(x-3)}^2
        \end{flalign*}
        \vfill{}\null{}
        \columnbreak{}
        \item $y=\left(x^{2}-2\right){\sqrt{1+x}}$
        \sol{}
        \begin{flalign*}
            y' & = \left[(x^2 - 2){(1 + x)}^{\frac{1}{2}}\right]'                                       \\
               & = (x^2 - 2)'{(1 + x)}^{\frac{1}{2}} + (x^2 - 2)\left[{(1 + x)}^{\frac{1}{2}}\right]' & \\
               & = 2x{(1 + x)}^{\frac{1}{2}} + (x^2 - 2)\cdot \dfrac{1}{2}{(1 + x)}^{-\frac{1}{2}}      \\
               & = 2x{(1 + x)}^{\frac{1}{2}} + \dfrac{x^2 - 2}{2{(1 + x)}^{\frac{1}{2}}}                \\
               & = \dfrac{4x(1 + x) + (x^2 - 2)}{2{(1 + x)}^{\frac{1}{2}}}                              \\
               & = \dfrac{4x^2 + 4x + x^2 - 2}{2\sqrt{1 + x}}                                           \\
               & = \dfrac{5x^2 + 4x - 2}{2\sqrt{x+1}}
        \end{flalign*}
    \end{multicols}
    \newpage
    \item $y={\dfrac{x}{{(2x+3)}^{3}}}$
          \sol{}
          \begin{flalign*}
              y' & = \dfrac{{(2x + 3)}^3 \cdot x' - x\left[{(2x + 3)}^3\right]'}{{(2x + 3)}^3}  \\
                 & = \dfrac{{(2x + 3)}^3 - x \cdot 3{(2x + 3)}^2 \cdot (2x + 3)'}{{(2x + 3)}^6} \\
                 & = \dfrac{{(2x + 3)}^3 - 6x{(2x + 3)}^2}{{(2x + 3)}^6}                        \\
                 & = \dfrac{{(2x + 3)}^2(2x + 3 - 6x)}{{(2x + 3)}^6}                            \\
                 & = \dfrac{-4x + 3}{{(2x + 3)}^4}                                              \\
          \end{flalign*}
    \item $y={\dfrac{{(3x+4)}^{2}}{{(x-5)}^{3}}}$
          \sol{}
          \begin{flalign*}
              y' & = \dfrac{{(x - 5)}^3\left[{(3x + 4)}^2\right]' - {(3x + 4)}^2\left[{(x-5)}^3\right]'}{{(x-5)}^6}              \\
                 & = \dfrac{{(x - 5)}^3\cdot 2(3x + 4) \cdot (3x + 4)' - {(3x + 4)}^2\cdot 3{(x - 5)}^2 \cdot (x-5)'}{{(x-5)}^6} \\
                 & = \dfrac{6{(x - 5)}^3(3x + 4) - 3{(3x + 4)}^2{(x - 5)}^2}{{(x-5)}^6}                                          \\
                 & = \dfrac{{(x - 5)}^2\left[6(x - 5)(3x + 4) - 3{(3x + 4)}^2\right]}{{(x-5)}^6}                                 \\
                 & = \dfrac{6(3x^2 - 11x - 20) - 3(9x^2 - 24x + 16)}{{(x-5)}^4}                                                  \\
                 & = \dfrac{18x^2 - 66x - 120 - 27x^2 - 72x - 48}{{(x-5)}^4}                                                     \\
                 & = \dfrac{-9x^2 - 138x - 168}{{(x-5)}^4}                                                                       \\
                 & = -\dfrac{3(3x^2 + 46x + 56)}{{(x-5)}^4}                                                                      \\
                 & = -\dfrac{3(3x + 4)(x + 14)}{{(x-5)}^4}
          \end{flalign*}
\end{enumerate}

\newpage
\subsection{Exercise 25.4}
\noindent Find the derivative of the following functions:
\begin{enumerate}
    \begin{multicols}{2}
        \item $y={(2x-3)}^{8}$
        \sol{}
        \begin{flalign*}
            y' & = 8{(2x - 3)}^7 \cdot (2x - 3)' \\
               & = 16{(2x - 3)}^7
        \end{flalign*}
        \vfill{}\null{}
        \item $y={\left(3x^{2}-6x+5\right)}^{4}$
        \sol{}
        \begin{flalign*}
            y' & = 4{(3x^2 - 6x + 5)}^3 \cdot (3x^2 - 6x + 5)' \\
               & =  4(6x - 6){(3x^2 - 6x + 5)}^3               \\
               & = 24(x - 1){(3x^2 - 6x + 5)}^3
        \end{flalign*}
        \vfill{}\null{}
    \end{multicols}

    \begin{multicols}{2}
        \item $y={\sqrt{2x^{4}-4x^{2}+5}}$
        \sol{}
        \begin{flalign*}
            y' & = {\left(2x^4 - 4x^2 + 5\right)}^{\frac{1}{2}}                                                 & \\
               & = \frac{1}{2}{\left(2x^4 - 4x^2 + 5\right)}^{-\frac{1}{2}} \cdot \left(2x^4 - 4x^2 + 5\right)'   \\
               & = \frac{1}{2}{\left(2x^4 - 4x^2 + 5\right)}^{-\frac{1}{2}} \cdot (8x^3 - 8x)                     \\
               & = (4x^3 - x){\left(2x^4 - 4x^2 + 5\right)}^{-\frac{1}{2}}                                        \\
               & = \dfrac{4x^3 - x}{\sqrt{2x^4 - 4x^2 + 5}}
        \end{flalign*}
        \vfill{}\null{}
        \item $y={\sqrt[3]{3x^{2}-3x+2}}$
        \sol{}
        \begin{flalign*}
            y' & = \left[{\left(3x^{2}-3x+2\right)}^{\frac{1}{3}}\right]'              \\
               & = \dfrac{1}{3}{(3x^2 - 3x + 2)}^{-\frac{2}{3}} \cdot (3x^2 - 3x + 2)' \\
               & = \dfrac{1}{3}{(3x^2 - 3x + 2)}^{-\frac{2}{3}} \cdot (6x - 3)         \\
               & = {(3x^2 - 3x + 2)}^{-\frac{2}{3}}(2x - 1)                            \\
               & = \dfrac{2x - 1}{\sqrt[3]{{(2x^4 - 4x^2 + 5)}^2}}
        \end{flalign*}
        \vfill{}\null{}
    \end{multicols}

    \begin{multicols}{2}
        \item $y={\dfrac{3}{6x^{2}-4}}$
        \sol{}
        \begin{flalign*}
            y' & = \dfrac{(6x^2 - 4)(3)' - 3(6x^2 - 4)'}{{(6x^2 - 4)}^2} \\
               & = \dfrac{-3(12x)}{{(6x^2 - 4)}^2}                       \\
               & = \dfrac{-36x}{{(6x^2 - 4)}^2}                          \\
               & = -\dfrac{36x}{4{(3x^2 - 2)}^2}                         \\
               & = \dfrac{9x}{{(3x^2 - 2)}^2}
        \end{flalign*}
        \vfill{}\null{}
        \item $y={\dfrac{9}{\sqrt{6x^{3}-9x}}}$
        \sol{}
        \begin{flalign*}
            y' & = \dfrac{(9')\left(\sqrt{6x^3 - 9x}\right) - 9\left(\sqrt{6x^3 - 9x}\right)'}{6x^3 - 9x}            \\
               & = \dfrac{- 9\left[\dfrac{1}{2}{\left(6x^3 - 9x\right)}^{-\frac{1}{2}}(18x^2 - 9)\right]}{6x^3 - 9x} \\
               & = \dfrac{-9(18x^2 - 9)}{2(6x^3 - 9x){\left(6x^3 - 9x\right)}^{\frac{1}{2}}}                         \\
               & = -\dfrac{81(2x^2 - 1)}{2\sqrt{{\left(6x^3 - 9x\right)}^3}}                                         \\
        \end{flalign*}
        \vfill{}\null{}
    \end{multicols}

    \begin{multicols}{2}
        \item $y={\dfrac{3x}{{(x+5)}^{2}}}$
        \sol{}
        \begin{flalign*}
            y' & = \dfrac{{(3x)'(x+5)}^2 - 3x\left[{(x+5)}^2\right]'}{{(x+5)}^4} & \\
               & = \dfrac{3{(x+5)}^2 - 3x\cdot2(x+5) \cdot (x+5)'}{{(x+5)}^4}      \\
               & = \dfrac{3{(x+5)}^2 - 6x(x+5)}{{(x+5)}^4}                         \\
               & = \dfrac{(x+5)\left[3(x+5) - 6x\right]}{{(x+5)}^4}                \\
               & = \dfrac{3x+15 - 6x}{{(x+5)}^3}                                   \\
               & = \dfrac{-3x+15}{{{(x+5)}}^3}                                     \\
               & = -\dfrac{3(x-5)}{{(x+5)}^3}
        \end{flalign*}
        \item $y={\dfrac{3x-1}{\sqrt{x-1}}}$
        \sol{}
        \begin{flalign*}
            y' & = \dfrac{(3x - 1)'\sqrt{x - 1} - (3x - 1)\left(\sqrt{x - 1}\right)'}{x - 1}        & \\
               & = \dfrac{3\sqrt{x - 1} - (3x - 1)\cdot\dfrac{1}{2}{(x - 1)}^{-\frac{1}{2}}}{x - 1}   \\
               & = \dfrac{3\sqrt{x - 1} - \dfrac{3x - 1}{2\sqrt{x - 1}}}{x - 1}                       \\
               & = \dfrac{6(x - 1) - (3x - 1)}{2(x - 1){\sqrt{x - 1}}}                                \\
               & = \dfrac{3x - 5}{2{\sqrt{{(x - 1)}^3}}}
        \end{flalign*}
        \vfill{}\null{}
    \end{multicols}

    \begin{multicols}{2}
        \item $y={\sqrt{x}}{\left(x-3\right)}^{5}$
        \sol{}
        \begin{flalign*}
            y' & = \left(\sqrt{x}\right)'{(x-3)}^5 + \sqrt{x}\left[{(x-3)}^5\right]' & \\
               & = \dfrac{1}{2\sqrt{x}}{(x-3)}^5 + \sqrt{x}\cdot5{(x-3)}^4             \\
               & = \dfrac{{(x-3)}^5 + 10x{(x-3)}^4}{2\sqrt{x}}                         \\
               & = \dfrac{{(x-3)}^4\left(x - 3 + 10x\right)}{2\sqrt{x}}                \\
               & = \dfrac{{(x-3)}^4(11x - 3)}{2\sqrt{x}}
        \end{flalign*}

        \item $y=x^{2}(x-3){(x+2)}^{2}$
        \sol{}
        \begin{flalign*}
            y' & = \left[(x^3 - 3x^2){(x + 2)}^2\right]'                            & \\
               & = (x^3 - 3x^2)'{(x + 2)}^2 + (x^3 - 3x^2)\left[{(x + 2)}^2\right]'   \\
               & = (3x^2 - 6x){(x + 2)}^2 + (x^3 - 3x^2)\cdot2(x + 2)                 \\
               & = (x+2)\left[(3x^2 - 6x)(x+2) + 2\left(x^3 - 3x^2\right)\right]      \\
               & = (x+2)\left(3x^3 - 6x^2 + 6x^2 - 12x + 2x^3 - 6x^2\right)           \\
               & = (x+2)\left(5x^3 - 6x^2 - 12x\right)                                \\
               & = x(x+2)\left(5x^2 - 6x - 12\right)
        \end{flalign*}
    \end{multicols}

    \newpage
    \item $y={(2x+1)}^{2}{(x+1)}^{3}$
          \sol{}
          \begin{flalign*}
              y' & = {(2x+1)}^2\left[{(x+1)}^3\right]' + \left[{(2x+1)}^2\right]'{(x+1)}^3 \\
                 & = {(2x+1)}^2\cdot3{(x+1)}^2 + (2)(2)(2x+1){(x+1)}^3                     \\
                 & = 3{(2x+1)}^2{(x+1)}^2 + 4(2x+1){(x+1)}^3                               \\
                 & = (2x+1){(x+1)}^2\left[3(2x+1) + 4(x+1)\right]                          \\
                 & = (2x+1){(x+1)}^2(10x + 7)
          \end{flalign*}

    \item $y={\left({\dfrac{1+x}{1-x}}\right)}^{2}$
          \sol{}
          \begin{flalign*}
              y' & = 2{\left(\dfrac{1+x}{1-x}\right)}\cdot\dfrac{(1-x)(1+x)' - (1+x)(1-x)'}{{(1-x)}^2} \\
                 & = \dfrac{2(1 + x)}{(1 - x)}\cdot\dfrac{(1-x) + (1+x)}{{(1-x)}^2}                    \\
                 & = \dfrac{2(1 + x)}{(1 - x)}\cdot\dfrac{2}{(1-x)^2}                                  \\
                 & = \dfrac{4 + 4x)}{{(1-x)}^3}
          \end{flalign*}

    \item $y=x^{2}{\sqrt{1+x^{2}}}$
          \sol{}
          \begin{flalign*}
              y' & = \left(x^2\right)'{\sqrt{1+x^2}} + x^2\left[\sqrt{1+x^2}\right]' \\
                 & = 2x\sqrt{1+x^2} + x^2\cdot\dfrac{2x}{2\sqrt{1+x^2}}              \\
                 & = 2x\sqrt{1+x^2} + \dfrac{x^3}{\sqrt{1+x^2}}                      \\
                 & = \dfrac{2x(1+x^2) + x^3}{\sqrt{1+x^2}}                           \\
                 & = \dfrac{2x + 2x^3 + x^3}{\sqrt{1+x^2}}                           \\
                 & = \dfrac{3x^3 + 2x}{\sqrt{1+x^2}}
          \end{flalign*}

          \newpage
    \item $y={\sqrt{\dfrac{1+x^{2}}{1-x^{2}}}}$
          \sol{}
          \begin{flalign*}
              y' & = \dfrac{1}{2}\left(\dfrac{1+x^2}{1-x^2}\right)^{-\frac{1}{2}}\cdot\dfrac{(1+x^2)'(1-x^2) - (1+x^2)(1-x^2)'}{{(1-x^2)}^2} \\
                 & = \dfrac{1}{2}\left(\dfrac{1+x^2}{1-x^2}\right)^{-\frac{1}{2}}\cdot\dfrac{2x(1-x^2) + (1+x^2)(2x)}{{(1-x^2)}^2}           \\
                 & = \dfrac{1}{2}\left(\dfrac{1+x^2}{1-x^2}\right)^{-\frac{1}{2}}\cdot\dfrac{2x - 2x^3 + 2x + 2x^3}{{(1-x^2)}^2}             \\
                 & = \dfrac{1}{2}\left(\dfrac{1-x^2}{1+x^2}\right)^{\frac{1}{2}}\cdot\dfrac{4x}{{(1-x^2)}^2}                                 \\
                 & = \dfrac{2x\sqrt{1-x^2}}{\sqrt{1+x^2}{(1-x^2)}^2}                                                                         \\
                 & = \dfrac{2x\sqrt{1-x^4}}{(1+x^2){(1-x^2)}^2}                                                                              \\
                 & = \dfrac{2x\sqrt{1-x^4}}{(1-x^4)(1-x^2)}                                                                                  \\
                 & = \dfrac{2x}{(1-x^2)\sqrt{1-x^4}}
          \end{flalign*}
\end{enumerate}

\newpage

\section{Higher Order Derivatives}

If the derivative $f' (x)$ of a function $y = f (x)$ is differentiable at $x =
    x_0$, then the derivative of $f'$ at $x = x_0$ is called the \textit{second
    derivative} of $y = f (x)$ at $x = x_0$, and is denoted by $f'' (x_0)$, that
is,
\begin{cequation}
    \lim\limits_{\Delta{x} \to{0}}{\dfrac{f' (x_0 + \Delta{x})-f' (x_0)}{\Delta{x}}} = f'' (x_0)
\end{cequation}

The derivative of the derivative $f' (x)$ of a function $y = f (x)$ is called
the \textit{second derivative} of $y = f (x)$, and is denoted by $(f')' (x)$ or
$\dfrac{d}{dx} \left(\dfrac{dy}{dx}\right) = \dfrac{d^2y}{dx^2} =
    \dfrac{d^2\bigl(f (x)\bigr)}{dx^2}$; while $f' (x)$ is called the \textit{first
    derivative} of $y = f (x)$. Using the same way, we can define the third
derivative $f''' (x)$, the fourth derivative $f^{(4)}(x)$, up to the $n$th
derivative $f^{(n)}(x)$.

The way of finding higher order derivatives is similar to the way of finding
the first derivative, as we just need differentiate the function again and
again until we get the desired order of derivative.

\subsection{Practice 6}
\begin{enumerate}[leftmargin=*]
    \item Find the first to sixth derivatives of $y=3x^{5}+2x^{4}-5x^{2}-8x+9$ \sol{}
          \begin{flalign*}
              y'      & = 15x^4 + 8x^3 - 10x - 8 \\
              y''     & = 60x^3 + 24x^2 - 10     \\
              y'''    & = 180x^2 + 48x           \\
              y^{(4)} & = 360x + 48              \\
              y^{(5)} & = 360                    \\
              y^{(6)} & = 0
          \end{flalign*}
\end{enumerate}
\noindent Find the third derivative of the following functions (Question 2 to 5):
\begin{enumerate}
    \setcounter{enumi}{1}
    \begin{multicols}{2}
        \item $y={\dfrac{3}{x^{4}}}$
        \sol{}
        \begin{flalign*}
            y'   & = (3x^{-4})'        \\
                 & = -12x^{-5}         \\
            y''  & = (-12x^{-5})'      \\
                 & = 60x^{-6}          \\
            y''' & = (60x^{-6})'       \\
                 & = -360x^{-7}        \\
                 & = -\dfrac{360}{x^7}
        \end{flalign*}

        \item $y=3x^{3}+6x^{2}-5x$
        \sol{}
        \begin{flalign*}
            y'   & = 9x^2 + 12x - 5 \\
            y''  & = 18x + 12       \\
            y''' & = 18
        \end{flalign*}
    \end{multicols}

    \begin{multicols}{2}
        \item $y={\sqrt{2x-3}}$
        \sol{}
        \begin{flalign*}
            y'   & = \left[\left(2x - 3\right)^{\frac{1}{2}}\right]'        \\
                 & = \dfrac{1}{2}\left(2x - 3\right)^{-\frac{1}{2}}\cdot 2  \\
                 & = \left(2x - 3\right)^{-\frac{1}{2}}                     \\
            y''  & = -\dfrac{1}{2}\left(2x - 3\right)^{-\frac{3}{2}}\cdot 2 \\
                 & = -\left(2x - 3\right)^{-\frac{3}{2}}                    \\
            y''' & = \dfrac{3}{2}\left(2x - 3\right)^{-\frac{5}{2}}\cdot 2  \\
                 & = 3\left(2x - 3\right)^{-\frac{5}{2}}                    \\
                 & = \dfrac{3}{\sqrt{(2x - 3)^5}}
        \end{flalign*}

        \item $y={\dfrac{1}{{(2x+5)}^{3}}}$
        \sol{}
        \begin{flalign*}
            y'   & = \left[(2x + 5)^{-3}\right]' \\
                 & = -3(2x + 5)^{-4}\cdot 2      \\
                 & = -6(2x + 5)^{-4}             \\
            y''  & = 24(2x + 5)^{-5}\cdot 2      \\
                 & = 48(2x + 5)^{-5}             \\
            y''' & = -240(2x + 5)^{-6}\cdot 2    \\
                 & = -480(2x + 5)^{-6}           \\
                 & = -\dfrac{480}{(2x + 5)^6}
        \end{flalign*}
    \end{multicols}
\end{enumerate}

\newpage
\subsection{Exercise 25.5}
\begin{enumerate}
    \item Given the function $y=a x^{6}+2b x^{4}-3c x^{3}+4x^{2}-5$ where $a$, $b$ and
          $c$ are constants, find the first to seventh derivatives of $y$. \sol{}
          \begin{flalign*}
              y'      & = 6ax^5 + 8bx^3 - 9cx^2 + 8x \\
              y''     & = 30ax^4 + 24bx^2 - 18cx + 8 \\
              y'''    & = 120ax^3 + 48bx - 18c       \\
              y^{(4)} & = 360ax^2 + 48b              \\
              y^{(5)} & = 720ax                      \\
              y^{(6)} & = 720a                       \\
              y^{(7)} & = 0
          \end{flalign*}
\end{enumerate}
\noindent Find the second derivative of the following functions (Question 2 to 5):
\begin{enumerate}
    \setcounter{enumi}{1}
    \begin{multicols}{2}
        \item $y=2x-{\dfrac{x^{3}}{2}}$
        \sol{}
        \begin{flalign*}
            y'  & = 2 - \dfrac{3x^2}{2} \\
            y'' & = -3x
        \end{flalign*}
        \vfill{}\null{}
        \item $y={\left(2+x^{2}\right)}^{3}$
        \sol{}
        \begin{flalign*}
            y'  & = 3\left(2 + x^2\right)^2\cdot 2x                     \\
                & = 6x\left(2 + x^2\right)^2                            \\
            y'' & = 6\left(2 + x^2\right)^2 + 24x^2\left(2 + x^2\right) \\
                & = 6\left(4 + 4x^2 + x^4\right) + 48x^2 + 24x^4        \\
                & = 24 + 24x^2 + 6x^4 + 48x^2 + 24x^4                   \\
                & = 30x^4 + 72x^2 + 24
        \end{flalign*}
    \end{multicols}

    \begin{multicols}{2}
        \item $y=x+{\dfrac{1}{x}}+{\dfrac{1}{x^{2}}}$
        \sol{}
        \begin{flalign*}
            y'  & = \left(1 + x^{-1} + x^{-2}\right)' \\
                & = -x^{-2} - 2x^{-3}                 \\
            y'' & = 2x^{-3} + 6x^{-4}                 \\
                & = \dfrac{2}{x^3} + \dfrac{6}{x^4}
        \end{flalign*}
        \item $y=x^{2}-{\dfrac{1}{x^{2}}}$
        \sol{}
        \begin{flalign*}
            y'  & = \left(x^2 - x^{-2}\right)' \\
                & = 2x + 2x^{-3}               \\
            y'' & = 2 - 6x^{-4}                \\
                & = 2 - \dfrac{6}{x^4}
        \end{flalign*}
    \end{multicols}

    \newpage
    \begin{multicols}{2}
        \item $y={\sqrt{a^{2}-x^{2}}}$ where $a$ is a constant
        \sol{}
        \begin{flalign*}
            y'  & = \left[\left(a^2 - x^2\right)^{\frac{1}{2}}\right]'                                                  \\
                & = -\dfrac{1}{2}\left(a^2 - x^2\right)^{-\frac{1}{2}}\cdot 2x                                          \\
                & = -x\left(a^2 - x^2\right)^{-\frac{1}{2}}                                                             \\
            y'' & = -\left(a^2 - x^2\right)^{-\frac{1}{2}} - \dfrac{1}{2}x\left(a^2 - x^2\right)^{-\frac{3}{2}}\cdot 2x \\
                & = -\dfrac{1}{\sqrt{a^2 - x^2}} - \dfrac{x^2}{\sqrt{\left(a^2 - x^2\right)^{3}}}                       \\
                & = \dfrac{-a^2 + x^2 - x^2}{\sqrt{\left(a^2 - x^2\right)^{3}}}                                         \\
                & = -\dfrac{a^2}{\sqrt{\left(a^2 - x^2\right)^{3}}}
        \end{flalign*}

        \item $y={\dfrac{2}{\sqrt{x-2}}}$
        \sol{}
        \begin{flalign*}
            y'  & = 2\left[\left(x - 2\right)^{-\frac{1}{2}}\right]'       \\
                & = -2 \cdot \dfrac{1}{2}\left(x - 2\right)^{-\frac{3}{2}} \\
                & = -\left(x - 2\right)^{-\frac{3}{2}}                     \\
            y'' & = \dfrac{3}{2}\left(x - 2\right)^{-\frac{5}{2}}          \\
                & = \dfrac{3}{2\sqrt{\left(x - 2\right)^5}}
        \end{flalign*}
    \end{multicols}

    \begin{multicols}{2}
        \item $y={\sqrt[3]{3x^{2}+2}}$
        \sol{}
        \begin{flalign*}
            y'  & = \left[\left(3x^2 + 2\right)^{\frac{1}{3}}\right]'                                                   \\
                & = \dfrac{1}{3}(3x^2 + 2)^{-\frac{2}{3}}\cdot 6x                                                       \\
                & = 2x\left(3x^2 + 2\right)^{-\frac{2}{3}}                                                              \\
            y'' & = 2\left(3x^2 + 2\right)^{-\frac{2}{3}} - \dfrac{4x}{3}\left(3x^2 + 2\right)^{-\frac{5}{3}}\cdot 6x & \\
                & = 2\left(3x^2 + 2\right)^{-\frac{2}{3}} - 8x^2\left(3x^2 + 2\right)^{-\frac{5}{3}}                    \\
                & = \dfrac{2}{\sqrt[3]{\left(3x^2 + 2\right)^2}} - \dfrac{8x^2}{\sqrt[3]{\left(3x^2 + 2\right)^5}}      \\
                & = \dfrac{2(3x^2 + 2) - 8x^2}{\sqrt[3]{\left(3x^2 + 2\right)^5}}                                       \\
                & = \dfrac{4-2x^2}{\sqrt[3]{\left(3x^2 + 2\right)^5}}
        \end{flalign*}
        \vfill{}\null{}
        \item $y={\dfrac{x}{\sqrt{1-x^{2}}}}$
        \sol{}
        \begin{flalign*}
            y'  & = \left[x\left(1 - x^2\right)^{-\frac{1}{2}}\right]'                                                  \\
                & = \left(1 - x^2\right)^{-\frac{1}{2}} - \dfrac{1}{2}x\left(1 - x^2\right)^{-\frac{3}{2}}\cdot (-2x) & \\
                & = \left(1 - x^2\right)^{-\frac{1}{2}} + x^2\left(1 - x^2\right)^{-\frac{3}{2}}                        \\
                & = \dfrac{1}{\sqrt{1 - x^2}} + \dfrac{x^2}{\sqrt{\left(1 - x^2\right)^3}}                              \\
                & = \dfrac{1 - x^2 + x^2}{\sqrt{\left(1 - x^2\right)^3}}                                                \\
                & = \dfrac{1}{\sqrt{\left(1 - x^2\right)^3}}                                                            \\
                & = \left(1 - x^2\right)^{-\frac{3}{2}}                                                                 \\
            y'' & = -\dfrac{3}{2}\left(1 - x^2\right)^{-\frac{5}{2}}\cdot (-2x)                                         \\
                & = \dfrac{3x}{\sqrt{\left(1 - x^2\right)^5}}
        \end{flalign*}
    \end{multicols}

    \newpage
    \item Given the function $y = 2x^3 - 3x^2 - 6x + 10$, find the value of $y'$, $y''$
          and $y'''$ at $x = 1$. \sol{}
          \begin{flalign*}
              y'      & = 6x^2 - 6x - 6 \\
              y'(1)   & = 6 - 6 - 6     \\
                      & = -6            \\
              y''     & = 12x - 6       \\
              y''(1)  & = 12 - 6        \\
                      & = 6             \\
              y'''    & = 12            \\
              y'''(1) & = 12
          \end{flalign*}

    \item Given the function $f (x) = \dfrac{5}{{(2x + 1)}^2}$, find the value of $f''
              (2)$. \sol{}
          \begin{flalign*}
              f'(x)  & = \left[5\left(2x + 1\right)^{-2}\right]' \\
                     & = -20\left(2x + 1\right)^{-3}             \\
              f''(x) & = 120\left(2x + 1\right)^{-4}             \\
              f''(2) & = 120\left(2\cdot 2 + 1\right)^{-4}       \\
                     & = 120\left(5\right)^{-4}                  \\
                     & = \dfrac{120}{625}                        \\
                     & = \dfrac{24}{125}
          \end{flalign*}

    \item Find the second derivative $y''$, of the function $y = 2x^3 + 3x^2 - 72x + 15$,
          and find the value of $x$ at $y'' = 0$. \sol{}
          \begin{flalign*}
              y'  & = 6x^2 + 6x - 72 \\
              y'' & = 12x + 6        \\
              0   & = 12x + 6        \\
              x   & = -\dfrac{1}{2}
          \end{flalign*}
\end{enumerate}

\newpage
\section{Implicit Differentiation}

Consider the function $x^2 + y^2 = x^3$ and $y^2 - 2xy - 3 = 0$. For these kind
of functions, $y$ is not expressed in terms of $x$ explicitly, hence we say
that $y$ is an \textit{implicit function} of $x$. In some cases, the equations
above are hard if not impossible to be rewritten in the form of $y = f (x)$. To
find the derivative of implicit functions, we can differentiate both sides of
the equation with respect to $x$ and use the chain rule at the same time to get
an equation in terms of $x$, $y$ and $\dfrac{dy}{dx}$, then we can solve for
$\dfrac{dy}{dx}$.

\subsection{Practice 7}

\noindent Find $\dfrac{dy}{dx}$ for the following implicit functions:
\begin{enumerate}
    \begin{multicols}{2}
        \item $x^{2}+y^{2}-6x+8y-9=0$
        \sol{}
        \begin{flalign*}
            x^{2}+y^{2}-6x+8y-9                             & = 0                  \\
            2x+2y\cdot\dfrac{dy}{dx}-6+8\cdot\dfrac{dy}{dx} & = 0                  \\
            2y \cdot \dfrac{dy}{dx}+8 \cdot \dfrac{dy}{dx}  & = 6-2x               \\
            \left(2y+8\right)\cdot\dfrac{dy}{dx}            & = 6-2x               \\
            \dfrac{dy}{dx}                                  & = \dfrac{6-2x}{2y+8} \\
                                                            & = \dfrac{3-x}{y+4}
        \end{flalign*}

        \item ${\dfrac{x^{2}}{a^{2}}}-{\dfrac{y^{2}}{b^{2}}}=5$ where $a$ and $b$ are constants
        \sol{}
        \begin{flalign*}
            {\dfrac{x^{2}}{a^{2}}}-{\dfrac{y^{2}}{b^{2}}}          & =5                      \\
            \dfrac{2x}{a^{2}}-\dfrac{2y}{b^{2}}\cdot\dfrac{dy}{dx} & =0                      \\
            \dfrac{2y}{b^{2}}\cdot\dfrac{dy}{dx}                   & =\dfrac{2x}{a^{2}}      \\
            \dfrac{dy}{dx}                                         & =\dfrac{b^{2}x}{a^{2}y}
        \end{flalign*}
    \end{multicols}

    \begin{multicols}{2}
        \item $2x^{2}-4xy+2y^{2}=5$
        \sol{}
        \begin{flalign*}
            2x^{2}-4xy+2y^{2}     & = 5                             \\
            4x-4x'y - 4xy' + 4yy' & = 0                           & \\
            4x - 4y - 4xy' + 4yy' & = 0                             \\
            -4xy' + 4yy'          & = - 4x + 4y                     \\
            y'(-4x + 4y)          & = - 4x + 4y                     \\
            y'                    & = \dfrac{- 4x + 4y}{-4x + 4y}   \\
                                  & = 1
        \end{flalign*}
        \vfill{}\null{}
        \item $x^{3}+2x^{2}y^{3}+y^{4}=1$
        \sol{}
        \begin{flalign*}
            x^{3}+2x^{2}y^{3}+y^{4}                             & = 1   \\
            3x^{2}+2\left[(x^2)'y^3 + x^2(y^3)'\right] + 4y^3y' & = 0   \\
            3x^{2}+2\left(2xy^3 + 3x^2y^2y'\right) + 4y^3y'     & = 0 & \\
            3x^{2}+4xy^3 + 6x^2y^2y' + 4y^3y'                   & = 0
        \end{flalign*}
        \vspace{-1cm}
        \begin{flalign*}
            6x^2y^2y' + 4y^3y' & = -5x^4 - 4xy^3                          \\
            y'(6x^2y^2 + 4y^3) & = -3x^2 - 4xy^3                          \\
            y'                 & = -\dfrac{3x^2 + 4xy^3}{6x^2y^2 + 4y^3}  \\
                               & = -\dfrac{x(3x + 4y^3)}{2y^2(3x^2 + 2y)}
        \end{flalign*}
    \end{multicols}
\end{enumerate}

\subsection{Exercise 25.6}

\noindent Find $\dfrac{dy}{dx}$ for the following implicit functions:
\begin{enumerate}
    \begin{multicols}{2}
        \item $x^{2}+y^{3}=3$
        \sol{}
        \begin{flalign*}
            x^{2}+y^{3} & = 3                 \\
            2x + 3y^2y' & = 0                 \\
            3y^2y'      & = -2x               \\
            y'          & = -\dfrac{2x}{3y^2}
        \end{flalign*}
        \vfill{}\null{}
        \columnbreak{}
        \item ${\sqrt{x}}+{\sqrt{y}}={\sqrt{a}}$ where $a$ is a constant
        \sol{}
        \begin{flalign*}
            {\sqrt{x}}+{\sqrt{y}}                                          & = {\sqrt{a}}                    \\
            \dfrac{1}{2\sqrt{x}} + \dfrac{1}{2\sqrt{y}}\cdot\dfrac{dy}{dx} & = 0                             \\
            \dfrac{1}{2\sqrt{y}}\cdot\dfrac{dy}{dx}                        & = -\dfrac{1}{2\sqrt{x}}         \\
            \dfrac{dy}{dx}                                                 & = -\dfrac{2\sqrt{y}}{2\sqrt{x}} \\
                                                                           & = -\sqrt{\dfrac{y}{x}}
        \end{flalign*}
    \end{multicols}

    \begin{multicols}{2}
        \item $x^{3}+y^{3}=2xy+3$
        \sol{}
        \begin{flalign*}
            x^{3}+y^{3}   & = 2xy+3                        \\
            3x^2 + 3y^2y' & = 2(x'y + xy')                 \\
            3x^2 + 3y^2y' & = 2y + 2xy'                    \\
            2xy' - 3y^2y' & = 3x^2 - 2y                    \\
            y'(2x - 3y^2) & = 3x^2 - 2y                    \\
            y'            & = \dfrac{3x^2 - 2y}{2x - 3y^2}
        \end{flalign*}
        \vfill{}\null{}
        \columnbreak{}
        \item $y^{2}(x+1)=3x^{2}$
        \sol{}
        \begin{flalign*}
            y^{2}(x+1)             & = 3x^{2}                      \\
            xy^2 + y^2             & = 3x^2                        \\
            x'y^2 + x(y^2)' + 2yy' & = 6x                          \\
            y^2 + 2xyy' + 2yy'     & = 6x                          \\
            2xyy' + 2yy'           & = 6x - y^2                    \\
            y'(2xy + 2y)           & = 6x - y^2                    \\
            y'                     & = \dfrac{6x - y^2}{2y(x + 1)}
        \end{flalign*}
        \vfill{}\null{}
    \end{multicols}

    \vspace{-1cm}
    \begin{multicols}{2}
        \item $3x^{2}-6x y+3y^{2}=25$
        \sol{}
        \begin{flalign*}
            3x^{2}-6x y+3y^{2}    & = 25    \\
            6x - 6y - 6xy' + 6yy' & = 0     \\
            x - y - xy' + yy'     & = 0     \\
            xy' - yy'             & = x - y \\
            y'(x - y)             & = x - y \\
            y'                    & = 1
        \end{flalign*}
        \vfill{}\null{}
        \columnbreak{}
        \item $xy^{3}=2x^{2}-2y^{2}$
        \sol{}
        \begin{flalign*}
            y^{3}+3xy^{2}y'        & = 4x - 4yy'                    \\
            3xy^{2}y'       + 4yy' & = 4x - y^3                     \\
            y'(3xy^2 + 4y)         & = 4x - y^3                     \\
            y'                     & = \dfrac{4x - y^3}{3xy^2 + 4y}
        \end{flalign*}
    \end{multicols}
    \begin{multicols}{2}
        \item ${\dfrac{1}{x}}+{\dfrac{1}{y}}=2x$
        \sol{}
        \begin{flalign*}
            {\dfrac{1}{x}}+{\dfrac{1}{y}}      & = 2x                          \\
            -\dfrac{1}{x^2} - \dfrac{1}{y^2}y' & = 2                           \\
            \dfrac{1}{y^2}y'                   & = - 2 - \dfrac{1}{x^2}        \\
            y'                                 & = -\dfrac{y^2(2x^2 + 1)}{x^2}
        \end{flalign*}
        \vfill{}\null{}
        \columnbreak{}
        \item ${\dfrac{3x^{2}}{y}}+{\dfrac{3y^{2}}{x}}=1$
        \sol{}
        \begin{flalign*}
            {\dfrac{3x^{2}}{y}}+{\dfrac{3y^{2}}{x}}           & = 1 & \\
            3x^2y^{-1} + 3y^2x^{-1}                           & = 1   \\
            6xy^{-1} - 3x^2y^{-2}y' + 6yx^{-1}y' - 3y^2x^{-2} & = 0
        \end{flalign*}
        \vspace{-1.5cm}
        \begin{flalign*}
            x^2y^{-2}y' - 2yx^{-1}y' & = 2xy^{-1} - y^2x^{-2}                                                       & \\
            y'(x^2y^{-2} - 2yx^{-1}) & = 2xy^{-1} - y^2x^{-2}                                                         \\
            y'                       & = \dfrac{2xy^{-1} - y^2x^{-2}}{x^2y^{-2} - 2yx^{-1}}                           \\
                                     & = \dfrac{\dfrac{2x}{y} - \dfrac{y^2}{x^2}}{\dfrac{x^2}{y^2} - \dfrac{2y}{x}}   \\
                                     & = \dfrac{\dfrac{2x^3 - y^3}{x^2y}}{\dfrac{x^3 - 2y^3}{xy^2}}                   \\
                                     & = \dfrac{2x^3 - y^3}{x^2y} \cdot \dfrac{xy^2}{x^3 - 2y^3}                      \\
                                     & = \dfrac{y(2x^3 - y^3)}{x(x^3 - 2y^3)}
        \end{flalign*}
    \end{multicols}

    \begin{multicols}{2}
        \item $xy-x+xy^{3}=10$
        \sol{}
        \begin{flalign*}
            xy-x+xy^{3}                 & = 10                              & \\
            y + xy' - 1 + y^3 + 3xy^2y' & = 0                                 \\
            xy' + 3xy^2y'               & = - y^3 - y + 1                     \\
            y'(x + 3xy^2)               & = - (y^3 + y - 1)                   \\
            y'                          & = -\dfrac{y^3 + y - 1}{3xy^2 + x}
        \end{flalign*}
        \vfill{}\null{}
        \columnbreak{}
        \item ${4x^{3}}+2x y^{2}-x y=0$
        \sol{}
        \begin{flalign*}
            4x^{3}+2x y^{2}-x y            & = 0                                 & \\
            12x^2 + 2y^2 + 4xyy' - y - xy' & = 0                                   \\
            4xyy' - xy'                    & = y - 12x^2 - 2y^2                    \\
            y'(4xy - x)                    & = y - 12x^2 - 2y^2                    \\
            y'                             & = \dfrac{y - 12x^2 - 2y^2}{4xy - x}   \\
                                           & = \dfrac{12x^2 + 2y^2 - y}{x - 4xy}
        \end{flalign*}
    \end{multicols}
\end{enumerate}

\section{Two Basic Limits}

\subsection*{A. $\lim\limits_{x \to 0} \dfrac{\sin x}{x}$}

Consider the value of $\dfrac{\sin x}{x}$ when $x$ approaches $0$. When $x =
    0$, the value of $\sin x$ and $x$ are both $0$, hence $\dfrac{\sin x}{x} =
    \dfrac{0}{0}$, which is undefined. However, the value of $\lim\limits_{x \to 0}
    \dfrac{\sin x}{x}$ is defined, as shown in the diagram below:
\begin{center}
    \begin{NiceTabular}{|c|c|c|}[hvlines,cell-space-limits=5pt]
        x (in rad) & $\sin x$      & $\dfrac{\sin x}{x}$ \\
        1          & 0.84147098    & 0.84147098          \\
        0.500      & 0.479425539   & 0.958851077         \\
        0.250      & 0.247403959   & 0.989615837         \\
        0.100      & 0.099833417   & 0.99833417          \\
        0.010      & 0.0099998333  & 0.99998333          \\
        0.001      & 0.00099999983 & 0.99999983          \\
        \vdots     & \vdots        & \vdots              \\
    \end{NiceTabular}
\end{center}
Note that the trigonometric independent variable $x$ is in radian.

By looking at the table above, as $x$ approaches $0$, the value of $\dfrac{\sin
        x}{x}$ approaches $1$. Hence, we can speculate that
\begin{mdframed}[style=MyFrame]
    \begin{cequation}
        \lim\limits_{x \to{0}} \dfrac{\sin{x}}{x} = 1
    \end{cequation}
\end{mdframed}

We can also prove the speculation above.
\begin{center}
    \begin{tikzpicture}[scale=1.8]
        \draw[->] (-2,0) -- (2,0);
        \draw[->] (0,-2) -- (0,2);
        \node at (2.1,0) {$\scriptstyle{x}$};
        \node at (0,2.1) {$\scriptstyle{y}$};
        \draw [thick] (0,0) circle (1cm);
        \draw [thick] (0,0) -- (1.5,1);
        \draw [thick] (1,0) -- (1, 0.666666);
        \draw [thick] (0.832,0.555) -- (1, 0);
        \draw (0.2cm,0cm) arc (0:33:0.2cm);
        \node at (0.33, 0.33) {$\scriptstyle{1}$};
        \node at (0.28, 0.08) {$\scriptstyle{x}$};
        \node at (0, 0) [below right] {$\scriptstyle{O}$};
        \node at (1, 0) [below right] {$\scriptstyle{A}$};
        \node at (0.832, 0.555) [above=0.8pt, left] {$\scriptstyle{P}$};
        \node at (1, 0.666666) [above] {$\scriptstyle{N}$};
    \end{tikzpicture}
\end{center}
In the unit circle above, the central angle $x$ from the initial arm $OA$, its terminal side intersects with the circle at point $P$, $AN$ is the tangent line that passes through point $A$, $AN$ and $OP$ intersects at point $N$.
\begin{flalign*}
    \text{From the diagram above, we can see that } \overset{\LARGE\frown}{PA} & = x      \\
    AN                                                                         & = \tan x
\end{flalign*}
and the area of $\Delta OAP$ $<$ the area of sector $OAP$ $<$ the area of $\Delta OAN$.
\begin{flalign*}
    \therefore\ \dfrac{1}{2} \sin x < \dfrac{1}{2} & {(1)}^2x < \dfrac{1}{2} (1) \tan x \\
    \sin x <                                       & \ x < \dfrac{\sin x}{\cos x}
\end{flalign*}

When $x$ is a positive angle that is smaller than $90^{\circ}$, $\sin x > 0$.
Dividing the inequality above by $\sin x$, we get
\begin{flalign*}
    1 <      & \ \dfrac{x}{\sin x} < \dfrac{1}{\cos x} \\
    \cos x < & \ \dfrac{\sin x}{x} < 1
\end{flalign*}

When $x < 0$, $\dfrac{\sin x}{x} = \dfrac{\sin (-x)}{-x}$, $\cos(-x) = \cos x$.
\begin{flalign*}
    \therefore\ \lim\limits_{x \to 0} \dfrac{\sin x}{x} \text{ is in between } \cos x \text{ and } 1
    \because\ \lim\limits_{x \to 0} \cos x = 1
    \therefore\ \lim\limits_{x \to 0} \dfrac{\sin x}{x} = 1
\end{flalign*}
The statement above uses the Squeeze Theorem. From the proof above, we can also conclude that
\begin{cequation}
    \lim\limits_{x \to{0}} \dfrac{\sin{x}}{x} =
    \lim\limits_{x \to{0}} \dfrac{1}{\dfrac{\sin{x}}{x}} = 1
\end{cequation}

\subsection{Practice 8}

Find the limit of the following:
\begin{enumerate}
    \begin{multicols}{2}
        \item $\lim\limits_{x\to0}{\dfrac{\sin4x}{5x}}$
        \sol{}
        \begin{flalign*}
            \lim\limits_{x\to0}{\dfrac{\sin4x}{5x}} & = \dfrac{1}{5}\lim\limits_{x\to0}{\dfrac{\sin4x}{x}}   \\
                                                    & = \dfrac{1}{5}\lim\limits_{x\to0}{\dfrac{4\sin4x}{4x}} \\
                                                    & = \dfrac{4}{5}\lim\limits_{x\to0}{\dfrac{\sin4x}{4x}}  \\
                                                    & = \dfrac{4}{5}
        \end{flalign*}

        \item $\lim\limits_{x\to0}{\dfrac{7x}{\sin5x}}$
        \sol{}
        \begin{flalign*}
            \lim\limits_{x\to0}{\dfrac{7x}{\sin5x}} & = 7\lim\limits_{x\to0}{\dfrac{x}{\sin5x}}             \\
                                                    & = \dfrac{7}{5}\lim\limits_{x\to0}{\dfrac{5x}{\sin5x}} \\
                                                    & = \dfrac{7}{5}
        \end{flalign*}
    \end{multicols}

    \begin{multicols}{2}
        \item $\lim\limits_{x\to0}{\dfrac{\sin 5x}{\sin 4x}}$
        \sol{}
        \begin{flalign*}
            \lim\limits_{x\to0}{\dfrac{\sin 5x}{\sin 4x}} & = \lim\limits_{x\to0}{\left(\dfrac{\sin 5x}{5x}\cdot\dfrac{4x}{\sin 4x} \cdot \dfrac{5}{4}\right)} \\
                                                          & = 1\cdot1\cdot\dfrac{5}{4}                                                                         \\
                                                          & = \dfrac{5}{4}
        \end{flalign*}
        \vfill{}\null{}
        \columnbreak{}
        \item $\lim\limits_{x\to0}{\dfrac{1-\cos2x}{3x^{2}}}$
        \sol{}
        \begin{flalign*}
            \lim\limits_{x\to0}{\dfrac{1-\cos2x}{3x^{2}}} & = \dfrac{1}{3}\lim\limits_{x\to0}{\dfrac{1-1 + 2\sin^2 x}{x^2}}     \\
                                                          & = \dfrac{2}{3}\lim\limits_{x\to0}{\dfrac{\sin^2 x}{x^2}}            \\
                                                          & = \dfrac{2}{3}\lim\limits_{x\to0}{\left(\dfrac{\sin x}{x}\right)^2} \\
                                                          & = \dfrac{2}{3}
        \end{flalign*}
    \end{multicols}

    \begin{multicols}{2}
        \item $\lim\limits_{x\to0}{\dfrac{\tan9x}{\sin 5x}}$
        \sol{}
        \begin{flalign*}
             & \lim\limits_{x\to0}{\dfrac{\tan9x}{\sin 5x}}                                                          \\
             & = \lim\limits_{x\to0}{\dfrac{\sin9x}{\sin 5x\cos9x}}                                                  \\
             & = \lim\limits_{x\to0}{\left(\dfrac{\sin9x}{9x}\cdot\dfrac{5x}{\sin 5x}\cdot\dfrac{9}{5\cos9x}\right)} \\
             & = \dfrac{9}{5}\lim\limits_{x\to0}{\dfrac{1}{\cos9x}}                                                  \\
             & = \dfrac{9}{5}
        \end{flalign*}
        \vfill{}\null{}
        \columnbreak{}
        \item $\lim\limits_{x\to0}{\dfrac{\tan2x\sin3x}{x^{2}}}$
        \sol{}
        \begin{flalign*}
             & \lim\limits_{x\to0}{\dfrac{\tan2x\sin3x}{x^{2}}}                                                                     \\
             & = \lim\limits_{x\to0}{\dfrac{\sin2x\sin3x}{\cos2x \cdot x^{2}}}                                                      \\
             & = \lim\limits_{x\to0}{\left(\dfrac{\sin2x}{2x} \cdot \dfrac{\sin3x}{3x} \cdot \dfrac{6x^2}{\cos2x \cdot x^2}\right)} \\
             & = 6\lim\limits_{x\to0}{\dfrac{1}{\cos2x}}                                                                            \\
             & = 6
        \end{flalign*}
    \end{multicols}
\end{enumerate}

\subsection{Exercise 25.7a}

Find the limit of the following:
\begin{enumerate}
    \item $\lim\limits_{x\to0}{\dfrac{x\cos x}{\sin2x}}$
          \sol{}
          \begin{flalign*}
              \lim\limits_{x\to0}{\dfrac{x\cos x}{\sin2x}} & = \dfrac{1}{2}\lim\limits_{x\to0}{\left(\dfrac{2x}{\sin2x}\cdot \cos x\right)} \\
                                                           & = \dfrac{1}{2}
          \end{flalign*}

    \item $\lim\limits_{x\to0}(x\cot x)$
          \sol{}
          \begin{flalign*}
              \lim\limits_{x\to0}(x\cot x) & = \lim\limits_{x\to0}{\dfrac{x\cdot \cos x}{\sin x}}               \\
                                           & = \lim\limits_{x\to0}{\left(\dfrac{x}{\sin x} \cdot \cos x\right)} \\
                                           & = 1
          \end{flalign*}

    \item $\lim\limits_{x\to0}{\dfrac{2x}{\sin7x}}$
          \sol{}
          \begin{flalign*}
              \lim\limits_{x\to0}{\dfrac{2x}{\sin7x}} & = 2\lim\limits_{x\to0}{\dfrac{x}{\sin7x}}             \\
                                                      & = \dfrac{2}{7}\lim\limits_{x\to0}{\dfrac{7x}{\sin7x}} \\
                                                      & = \dfrac{2}{7}
          \end{flalign*}

    \item $\lim\limits_{x\to0}{\dfrac{\sin{\dfrac{5}{2}}x}{2x}}$
          \sol{}
          \begin{flalign*}
              \lim\limits_{x\to0}{\dfrac{\sin{\dfrac{5}{2}}x}{2x}} & = \lim\limits_{x\to0}{\dfrac{\sin{\dfrac{5}{2}}x}{\dfrac{4}{5}\cdot\dfrac{5}{2}x}} \\
                                                                   & = \dfrac{5}{4}\lim\limits_{x\to0}{\dfrac{\sin{\dfrac{5}{2}}x}{\dfrac{5}{2}x}}      \\
                                                                   & = \dfrac{5}{4}
          \end{flalign*}
    \item $\lim\limits_{x\to0}{\dfrac{\sin(2x^{2})}{x}}$
          \sol{}
          \begin{flalign*}
              \lim\limits_{x\to0}{\dfrac{\sin(2x^{2})}{x}} & = 2\lim\limits_{x\to0}{\dfrac{x^2\sin(2x^{2})}{2x^2}} \\
                                                           & = 2\lim\limits_{x\to0}{x^2}                           \\
                                                           & = 0
          \end{flalign*}

    \item $\lim\limits_{x\to0}{\dfrac{x^{2}}{\sin^{2}b x}}$
          \sol{}
          \begin{flalign*}
              \lim\limits_{x\to0}{\dfrac{x^{2}}{\sin^{2}b x}} & = \lim\limits_{x\to0}{\left(\dfrac{x}{\sin bx}\right)^2}                \\
                                                              & = \dfrac{1}{b^2}\lim\limits_{x\to0}{\left(\dfrac{bx}{\sin bx}\right)^2} \\
                                                              & = \dfrac{1}{b^2}
          \end{flalign*}

    \item $\lim\limits_{x\to0}{\dfrac{\sin^{2}3x}{2x^{2}}}$
          \sol{}
          \begin{flalign*}
              \lim\limits_{x\to0}{\dfrac{\sin^{2}3x}{2x^{2}}} & = \dfrac{1}{2}\lim\limits_{x\to0}{\dfrac{\sin^{2}3x}{x^{2}}}         \\
                                                              & = \dfrac{1}{2}\lim\limits_{x\to0}{\left(\dfrac{\sin3x}{x}\right)^2}  \\
                                                              & = \dfrac{9}{2}\lim\limits_{x\to0}{\left(\dfrac{\sin3x}{3x}\right)^2} \\
                                                              & = \dfrac{9}{2}
          \end{flalign*}

    \item $\lim\limits_{x\to0}{\dfrac{4x}{\tan8x}}$
          \sol{}
          \begin{flalign*}
              \lim\limits_{x\to0}{\dfrac{4x}{\tan8x}} & = 4\lim\limits_{x\to0}{\dfrac{x\cos8x}{\sin8x}}                                 \\
                                                      & = \dfrac{1}{2}\lim\limits_{x\to0}{\left(\dfrac{8x}{\sin8x} \cdot \cos8x\right)} \\
                                                      & = \dfrac{1}{2}
          \end{flalign*}
    \item $\lim\limits_{x\to0}{\dfrac{\sin5x}{\tan3x}}$
    \item $\lim\limits_{x\to0}{\dfrac{1-\cos4x}{x\sin5x}}$
    \item $\lim\limits_{x\to0}{\dfrac{5x^{2}}{2\tan^{2}2x}}$
    \item $\lim\limits_{x\to0}{\dfrac{\cos3x-1}{3x^{2}}}$
\end{enumerate}

\subsection*{B. $\lim\limits_{x\to\infty}{{\left(1 + \dfrac{1}{x}\right)}^{x}}$}

Consider the limit of ${\left(1 + \dfrac{1}{x}\right)}^{x}$ as $x$ approaches
infinity. As $x \to \infty$, the changes of the function $f (x) = {\left(1 +
    \dfrac{1}{x}\right)}^{x}$ is shown in the table below:
\begin{center}
    \begin{NiceTabular}{|c|c|}[hvlines,cell-space-limits=5pt]
        $x$      & $f (x) = {\left(1 + \dfrac{1}{x}\right)}^{x}$             \\
        1        & ${\left(1 + \dfrac{1}{1}\right)}^{1} = 2$                 \\
        10       & ${\left(1 + \dfrac{1}{10}\right)}^{10} = 2.59374$         \\
        100      & ${\left(1 + \dfrac{1}{100}\right)}^{100} = 2.70481$       \\
        1000     & ${\left(1 + \dfrac{1}{1000}\right)}^{1000} = 2.71692$     \\
        10000    & ${\left(1 + \dfrac{1}{10000}\right)}^{10000} = 2.71815$   \\
        100000   & ${\left(1 + \dfrac{1}{100000}\right)}^{100000} = 2.71827$ \\
        $\vdots$ & $\vdots$                                                  \\
    \end{NiceTabular}
\end{center}

By looking at the table above, we can see that as $x$ approaches infinity, the
value of ${\left(1 + \dfrac{1}{x}\right)}^{x}$ approaches a constant value that
is denoted as $e$. $e$ is an irrational number and is approximately equal to
$2.71828182846$. It is called the \textbf{natural base}, and is one of the most
important numbers in mathematics and appears in many applications.

\begin{mdframed}[style=MyFrame]
    \begin{cequation}
        \lim\limits_{x\to\infty}{{\left(1 + \dfrac{1}{x}\right)}^{x}} = e
    \end{cequation}
\end{mdframed}

If $y = \dfrac{1}{x}$, then $x = \dfrac{1}{y}$. As $x \to \infty$, $y \to 0$.

Therefore, $\lim\limits_{x\to\infty}{{\left(1 + \dfrac{1}{x}\right)}^{x}} =
    \lim\limits_{y\to0}{{\left(1 + y\right)}^{\frac{1}{y}}} = e$.

\subsection{Practice 9}

Find the limit of the following: \setlength{\columnseprule}{1pt}
\setlength{\columnsep}{24pt}
\begin{multicols}{2}
    \begin{enumerate}
        \item $\lim\limits_{x\to\infty}{\left(1+{\dfrac{1}{x}}\right)}^{x+2}$
        \item $\lim\limits_{x\to\infty}{\left(1-3x\right)}^{\frac{2}{x}}$
        \item $\lim\limits_{x\to\infty}{\left(1+\dfrac{3}{x}\right)}^{\frac{x}{2}}$
        \item $\lim\limits_{x\to\infty}\biggl{(\dfrac{2-3x}{2}\biggr)}^{2}$
    \end{enumerate}
\end{multicols}

\subsection{Exercise 25.7b}

Find the limit of the following: \setlength{\columnseprule}{1pt}
\setlength{\columnsep}{24pt}
\begin{multicols}{2}
    \begin{enumerate}
        \item $\lim\limits_{x\to\infty}{\left(1+\dfrac{1}{x}\right)}^{\frac{x}{2}}$
        \item $\lim\limits_{x\to\infty}{\left(1+\dfrac{3}{x}\right)}^{2x}$
        \item $\lim\limits_{x\to\infty}{\left({\dfrac{x+5}{x}}\right)}^{2x}$
        \item $\lim\limits_{x\to\infty}{\left(1+{\dfrac{1}{2x}}\right)}^{-3x}$
        \item $\lim\limits_{x\to0}{\left({\dfrac{3-x}{3}}\right)}^{-{\frac{3}{x}}}$
        \item $\lim\limits_{x\to\infty}{\left(1+{\dfrac{5}{x}}\right)}^{2-x}$
        \item $\lim\limits_{x\to\infty}{\left(1+\dfrac{3}{2x}\right)}^{3+x}$
        \item $\lim\limits_{x\to0}{\left(1-4x\right)}^{-\frac{3}{x}}$
        \item $\lim\limits_{x\to0}{\left(1+2x\right)}^{1-\frac{2}{x}}$
        \item $\lim\limits_{x\to\infty}{\left({\dfrac{x+2}{x+1}}\right)}^{x}$
    \end{enumerate}
\end{multicols}

\section{Derivatives of Trigonometric Functions}

Here we will derive the derivatives of the sine, cosine, and tangent functions.

Let $y = \sin x$.
\begin{flalign*}
    y + \Delta y                & = \sin(x + \Delta{x})                                                                     \\
    \Delta y                    & = \sin(x + \Delta{x}) - \sin x                                                            \\
                                & = \sin x \cos \Delta{x} + \cos x \sin \Delta{x} - \sin x                                  \\
                                & = \sin x (\cos \Delta{x} - 1) + \cos x \sin \Delta{x}                                     \\
    \dfrac{\Delta y}{\Delta{x}} & = \sin x \dfrac{\cos \Delta{x} - 1}{\Delta{x}} + \cos x \dfrac{\sin \Delta{x}}{\Delta{x}}
\end{flalign*}
\begin{flalign*}
    \because\ \lim\limits_{\Delta{x}\to{0}}{\dfrac{\cos \Delta{x} - 1}{\Delta{x}}} & = \lim\limits_{\Delta{x}\to{0}}{\dfrac{-2 \sin^{2} \dfrac{\Delta{x}}{2}}{\Delta{x}}}                                               \\
                                                                                   & = -\dfrac{1}{2}\lim\limits_{\Delta{x}\to{0}}{\left(\dfrac{\sin \dfrac{\Delta{x}}{2}}{\dfrac{\Delta{x}}{2}}\right) \cdot \Delta{x}} \\
                                                                                   & = 0                                                                                                                                \\
    \lim\limits_{\Delta{x}\to{0}}{\dfrac{\sin \Delta{x}}{\Delta{x}}}               & = 1
\end{flalign*}
\begin{flalign*}
    \therefore\ \dfrac{d}{dx} (\sin x) & = \lim\limits_{\Delta{x}\to{0}}{\dfrac{\Delta y}{\Delta{x}}} \\
                                       & = \sin x \cdot 0 + \cos x \cdot 1                            \\
                                       & = \cos x
\end{flalign*}

\begin{mdframed}[style=MyFrame]
    \begin{cequation}
        \dfrac{d}{dx} (\sin{x}) = \cos{x}
    \end{cequation}
\end{mdframed}
\newpage
\begin{flalign*}
    \because\ \cos x = \sin\left(\dfrac{\pi}{2} - x\right)
\end{flalign*}
\begin{flalign*}
    \therefore\ \dfrac{d}{dx} (\cos x) & = \dfrac{d}{dx}\left(\sin\left(\dfrac{\pi}{2} - x\right)\right) \\
                                       & = \cos\left(\dfrac{\pi}{2} - x\right)(-1)                       \\
                                       & = -\sin{x}
\end{flalign*}

\begin{mdframed}[style=MyFrame]
    \begin{cequation}
        \dfrac{d}{dx} (\cos{x}) = -\sin{x}
    \end{cequation}
\end{mdframed}

\begin{flalign*}
    \because\ \tan x = \dfrac{\sin x}{\cos x}
\end{flalign*}
\begin{flalign*}
    \therefore\ \dfrac{d}{dx} (\tan x) & = \dfrac{d}{dx}\left(\dfrac{\sin x}{\cos x}\right)                                  \\
                                       & = \dfrac{\cos x \dfrac{d}{dx} (\sin x) - \sin x \dfrac{d}{dx} (\cos x)}{\cos^{2} x} \\
                                       & = \dfrac{\cos x \cos x - \sin x (-\sin x)}{\cos^{2} x}                              \\
                                       & = \dfrac{\cos^{2} x + \sin^{2} x}{\cos^{2} x}                                       \\
                                       & = \dfrac{1}{\cos^{2} x}                                                             \\
                                       & = \sec^{2} x
\end{flalign*}

\begin{mdframed}[style=MyFrame]
    \begin{cequation}
        \dfrac{d}{dx} (\tan{x}) = \sec^{2}{x}
    \end{cequation}
\end{mdframed}

Below are the derivatives of the remaining trigonometric functions.

\begin{mdframed}[style=MyFrame]
    \begin{cequation}
        \dfrac{d}{dx} (\csc{x}) = -\csc{x} \cot{x}
    \end{cequation}
    \begin{cequation}
        \dfrac{d}{dx} (\sec{x}) = \sec{x} \tan{x}
    \end{cequation}
    \begin{cequation}
        \dfrac{d}{dx} (\cot{x}) = -\csc^{2}{x}
    \end{cequation}
\end{mdframed}

\subsection{Practice 10}

Find the derivatives of the following functions:
\setlength{\columnseprule}{1pt} \setlength{\columnsep}{24pt}
\begin{multicols}{2}
    \begin{enumerate}
        \item $y=\cos\left(5x^{2}\right)$
        \item $y=\sin^{2}(2x-3)$
        \item $y=\tan^{4}3x$
    \end{enumerate}
\end{multicols}

\subsection{Exercise 25.8}

Find the derivatives of the following functions:
\setlength{\columnseprule}{1pt} \setlength{\columnsep}{24pt}
\begin{multicols}{2}
    \begin{enumerate}
        \item $y=\sin^{2}x$
        \item $y=\sin{3x}-\cos{3x}$
        \item $y=\tan4x-\cot5x$
        \item $y=\sec3x+\csc5x$
        \item $y=2x\sec x$
        \item $y=\dfrac{2x}{\sin x}$
        \item $y=\cos3x^{\circ}$
        \item $y=\cos^{4}(1-2x)$
        \item $y=\tan^{3}\left(2x^{2}\right)$
        \item $y=\cos^{2}x-\sin^{2}x$
        \item $y=4\sin x\cos x$
        \item $y=\sin3x\tan6x$
        \item $y=\sqrt{\cos2x}$
        \item $y=\sin^{2}\sqrt{1+x^{2}}$
        \item $y=x\tan^{2}(3x-2)$
        \item $y=\dfrac{3\sin2x}{\cos x}$
        \item $y={\dfrac{2}{3\tan^{3}x}}$
        \item $y={\dfrac{1+\cos x}{1-\cos x}}$
        \item $y={\dfrac{2\tan x}{1-\tan^{2}x}}$
        \item $y={\dfrac{\sin\left(2x-{\dfrac{\pi}{4}}\right)}{\sin\left(2x+{\dfrac{\pi}{4}}\right)}}$
    \end{enumerate}
\end{multicols}

\begin{enumerate}
    \setcounter{enumi}{20}
    \item Given the function $y = \dfrac{x}{2 + \cos x}$. Find the derivative value when
          $x = 0$, $x = \dfrac{\pi}{2}$, and $x = \pi$.
    \item If the function $y = \sec x + \tan x$, prove that $\dfrac{dy}{dx} = y\sec x$.
    \item If the function $y = \sqrt{\dfrac{1 - \cos x}{1 + \cos x}}$, find
          $\dfrac{dy}{dx}$.
    \item If the function $r = \sin 3t - 2\cos t$, find $\dfrac{d^2r}{dt^2}$ when $t =
              \dfrac{\pi}{3}$.
\end{enumerate}

\section{Derivatives of Exponential Functions}

\subsection*{Derivative of Natural Logarithmic Functions}

Let $y = \ln x$ where $x > 0$.
\begin{flalign*}
    y + \Delta y                & = \ln (x + \Delta{x})                                                                                              \\
    \Delta y                    & = \ln (x + \Delta{x}) - \ln x                                                                                      \\
                                & = \ln \left(\frac{x + \Delta{x}}{x}\right)                                                                         \\
                                & = \ln \left(1 + \dfrac{\Delta{x}}{x}\right)                                                                        \\
    \dfrac{\Delta y}{\Delta{x}} & = \dfrac{1}{\Delta{x}}\ln \left(1 + \dfrac{\Delta{x}}{x}\right)                                                    \\
                                & = \dfrac{1}{\Delta{x}}\cdot \dfrac{x}{\Delta{x}} \cdot \ln \left(1 + \dfrac{\Delta{x}}{x}\right)                   \\
                                & = \dfrac{1}{x}\ln {\left(1 + \dfrac{\Delta{x}}{x}\right)}^{\frac{x}{\Delta{x}}}                                    \\
    \dfrac{dy}{dx}              & = \lim_{\Delta{x}\to{0}} \dfrac{\Delta y}{\Delta{x}}                                                               \\
                                & = \dfrac{1}{x}\lim_{\Delta{x}\to{0}} \ln {\left(1 + \dfrac{\Delta{x}}{x}\right)}^{\frac{x}{\Delta{x}}}             \\
                                & = \dfrac{1}{x}\ln\left(\lim_{\Delta{x}\to{0}} {\left(1 + \dfrac{\Delta{x}}{x}\right)}^{\frac{x}{\Delta{x}}}\right) \\
                                & = \dfrac{1}{x}\ln e                                                                                                \\
                                & = \dfrac{1}{x}
\end{flalign*}

\begin{mdframed}[style=MyFrame]
    \begin{cequation}
        \dfrac{d}{dx} (\ln{x}) = \dfrac{1}{x}
    \end{cequation}
\end{mdframed}

If $x < 0$, then $-x > 0$.
\begin{flalign*}
    \dfrac{d}{dx} (\ln -x) & = \dfrac{1}{-x}\dfrac{d}{dx} (-x) \\
                           & = \dfrac{1}{x}
\end{flalign*}

\begin{mdframed}[style=MyFrame]
    \begin{cequation}
        \dfrac{d}{dx} (\ln\vert{x}\vert) = \dfrac{1}{x}
    \end{cequation}
\end{mdframed}

\subsection*{Derivative of Other Logarithmic Functions}

\begin{flalign*}
    \text{Let } y  & = \log_{a}x                                       \\
                   & = \dfrac{\ln x}{\ln a}                            \\
                   & = \dfrac{1}{\ln a}\ln x                           \\
    \dfrac{dy}{dx} & = \dfrac{1}{\ln a}\cdot \dfrac{d}{dx} (\ln x)     \\
                   & = \dfrac{1}{\ln a}\cdot \left(\dfrac{1}{x}\right) \\
                   & = \dfrac{1}{x\ln a}
\end{flalign*}

\begin{mdframed}[style=MyFrame]
    \begin{cequation}
        \dfrac{d}{dx} (\log_{a}x) = \dfrac{1}{x\ln{a}}
    \end{cequation}
\end{mdframed}

Generally, when solving for the derivative of a logarithmic function, we can
first convert it to a natural logarithmic function and then solve for the
derivative.

\subsection{Practice 11}

Find the derivative of the following functions: \setlength{\columnseprule}{1pt}
\setlength{\columnsep}{24pt}
\begin{multicols}{2}
    \begin{enumerate}
        \item $y=\ln\left(x^{2}-2x+1\right)$
        \item $y=\log_{2}\sin3x$
        \item $y=\ln(\sec x)$
        \item $y=x\ln x$
        \item $y=\ln{\frac{1+x}{1-x}}$
        \item $y=\log_{2}{\sqrt{1+x^{2}}}$
        \item $y=\ln{\dfrac{1+x}{1-x}}$
        \item $y=\log_{2}{\sqrt{1+x^{2}}}$
    \end{enumerate}
\end{multicols}

\subsection{Exercise 25.9}

Find the derivative of the following functions: \setlength{\columnseprule}{1pt}
\setlength{\columnsep}{24pt}
\begin{multicols}{2}
    \begin{enumerate}
        \item $y=\ln\left(5x-3\right)$
        \item $y=\ln\left(a x^{3}\right)$
        \item $y=\ln{\dfrac{3}{x^{2}}}$
        \item $y=\ln\left(2x^{2}-3x+4\right)$
        \item $y=\log_{5}(3x)$
        \item $y=\log_{2}\left(x^{2}-4x\!+\!3\right)$
        \item $y=\log_{a}\left(2a x^{2}-4a x\right)$
        \item $y=\log_{a}\left(\ln x\right)$
        \item $y=3x^{2}\ln\left(5x\right)$
        \item $y=\ln\left(\cos^{2}x\right)$
        \item $y=\ln{(4x+3)}^{2}$
        \item $y=\log_{5}\left(2x^{2}-3\right)$
        \item $y=\log_{8}{\sqrt{x^{2}-2}}$
        \item $y=\log_{b}(\sin5x)$
        \item $y=\log{(5x)}+\ln{(\tan{x})}$
        \item $y=\ln^{2}(\sec x)$
        \item $y=\log_{3}{\dfrac{2}{x^{2}-1}}$
        \item $y={\dfrac{1+\ln x}{1-\ln x}}$
        \item $y=\ln{\dfrac{2+x^{2}}{2-x^{2}}}$
        \item $y=\ln{\dfrac{2\sin x}{\sec x}}$
    \end{enumerate}
\end{multicols}

\section{Derivatives of Logarithmic Functions}

\subsection*{Derivative of exponential functions $y = e^{x}$}

If $y = e^{x}$, then $\ln y = e^{x} = x$.
\begin{flalign*}
    \text{Differentiate both sides with respect to $x$, we get } \dfrac{1}{y}\cdot\dfrac{dy}{dx} & = 1     & \\
    \dfrac{dy}{dx}                                                                               & = y       \\
                                                                                                 & = e^{x}
\end{flalign*}

\begin{mdframed}[style=MyFrame]
    \begin{cequation}
        \dfrac{d}{dx} (e^{x}) = e^{x}
    \end{cequation}
\end{mdframed}

\subsection*{Derivative of other exponential functions}

Let $y = a^{x}$ where $a > 0$.
\begin{flalign*}
    \text{Take the natural logarithm of both sides, we get } \ln y                               & = \ln a^{x}  & \\
                                                                                                 & = x\ln a       \\
    \\
    \text{Differentiate both sides with respect to $x$, we get } \dfrac{1}{y}\cdot\dfrac{dy}{dx} & = \ln a        \\
    \dfrac{dy}{dx}                                                                               & = y\ln a       \\
                                                                                                 & = a^{x}\ln a
\end{flalign*}

Generally, when solving for the derivative of an exponential function, we can
first convert it to a natural exponential function and then solve for the
derivative.

\subsection{Practice 12}

Find the derivative of the following functions: \setlength{\columnseprule}{1pt}
\setlength{\columnsep}{24pt}
\begin{multicols}{2}
    \begin{enumerate}
        \item $y=e^{-3x}$
        \item $y=-e^{-{\frac{1}{3}x}}$
        \item $y=2a^{5x}$
        \item $y=a^{2x}-e^{-x}+x^{a}$
        \item $y={\left(e^{x}+{\frac{1}{e^{x}}}\right)}^{2}$
        \item $y={\frac{e^{3x}-e^{-3x}}{e^{x}}}$
    \end{enumerate}
\end{multicols}

\subsection{Exercise 25.10}

Find the derivative of the following functions: \setlength{\columnseprule}{1pt}
\setlength{\columnsep}{24pt}
\begin{multicols}{2}
    \begin{enumerate}
        \item $y=e^{-5x}$
        \item $y=3e^{-{\frac{1}{3}}x}$
        \item $y=e^{x^{2}+3x-1}$
        \item $y=e^{5x}\cdot e^{-6x}\cdot4e^{2}$
        \item $y=5^{2x}$
        \item $y={(3b)}^{7x}$
        \item $y=4^{3-2x^2}$
        \item $y=4^{2x}-4^{-2x}$
        \item $y=3e^{\tan x}$
        \item $y=e^{x}\ln x$
        \item $y=3^{2x}\cos5x$
        \item $y={\dfrac{e^{x}+1}{e^{x}-1}}$
        \item $y={\dfrac{e^{2x}-e^{-2x}}{2e^{x}}}$
        \item $y={\dfrac{3e^{5x}-2e^{-2x}}{6e^{x}}}$
    \end{enumerate}
\end{multicols}

\section{Revision Exercise 25}

\begin{enumerate}
    \item Find the gradient of the tangent to the curve $y = 2x^2 + 1$ at the point where
          $x = 2$.
    \item Find the gradient of the curve $y = 3x^2 - 1$ at the point $A(-1, 2)$.
    \item Find the gradient of the curve $y = 2x - x^3$ at the point $B(-1, -1)$.
    \item Find the derivative of the following functions using the definition of the
          derivative, and find the value of the derivative at the point where $x = 1$:
          \begin{enumerate}
              \item $f (x) = x^2 + 2x$
              \item $g(x) = x^3$
              \item $h(x) = \dfrac{5}{x}$
              \item $k(x) = \sqrt{x + 3}$
          \end{enumerate}
    \item Find the derivative of the following functions:
          \begin{enumerate}
              \item $y=2x^{4}-3x^{3}+5x-8$
              \item $y=2x+{\dfrac{2}{x}}-{\dfrac{3}{x^{2}}}$
              \item $y=\sqrt[3]{x}-{\dfrac{1}{\sqrt{3x}}}$
              \item $y=\left(x^{3}-4x\right)\left(x^{2}+3x-1\right)$
              \item $y={(x-1)}^{5}{\sqrt{x+2}}$
              \item $y=\left(2x+5\right)\left(x^{2}-2\right)\left(x^{3}-1\right)$
              \item $y={\dfrac{2x^{3}-3x^{2}+4}{x^{2}}}$
              \item $y={\dfrac{x^{2}+4}{x+1}}$
              \item $y={\dfrac{x+2}{x^{2}+5x+6}}$
              \item $y={\dfrac{x^{2}}{{\left(x^{2}-1\right)}^{3}}}$
          \end{enumerate}
    \item Find the derivative of the following functions:
          \begin{enumerate}
              \item $y={\left(x^{3}-1\right)}^{4}$
              \item $y={(5x+3)}^{6}$
              \item $y={\left(x^{3}-3x\right)}^{5}$
              \item $y={\sqrt{x^{2}-2x}}$
              \item $y={\dfrac{1}{\sqrt[3]{2x^{2}-1}}}$
              \item $y={\dfrac{2x-1}{\sqrt{1-2x}}}$
          \end{enumerate}

    \item Find the second derivative of the following functions:
          \begin{enumerate}
              \item $y=x^{2}(3x-4)$
              \item $y=2x^{5}-6x^{4}-3x+5$
              \item $y={\dfrac{3}{x^{5}}}$
              \item $y={\sqrt{2x+1}}$
          \end{enumerate}

    \item If the function $y = \dfrac{x^3}{{(x-1)}^2}$, find $y'$ and $y''$.
    \item Given the function $y = 2x^3 + 3x^2 - 72x + 21$, find the value of $x$ when
          $\dfrac{dy}{dx} = 0$.
    \item Given the function $y = {(2-3x^2)}^4$, find the value of $\dfrac{d^2y}{dx^2}$
          when $x = 1$.
    \item If the function $f (x) = \dfrac{x^3}{3} - \dfrac{5x^2}{2} + x + 1$, find
          \begin{enumerate}
              \item $f' (1)$ and $f'' (2)$
              \item the value of $x$ when $f'' (x) = 0$
          \end{enumerate}
    \item If the function $f (x) = \sqrt{x\sqrt{x\sqrt{x}}}$, find $f' (1)$, $f'' (1)$,
          $f''' (1)$ and $f^{(4)}(1)$.
    \item Find the derivative $\dfrac{dy}{dx}$ of the following implicit functions:
          \begin{enumerate}
              \item $x^{2}+2y=2x+3$
              \item $x^{2}+3x=y^{2}-5y$
              \item $3x^{2}+7x y-9y^{2}=2$
              \item $x^{5}y+x y^{5}=3x y$
          \end{enumerate}
    \item Find the gradient of the tangent to the curve $x^2 + xy + y^2 = 4$ at the point
          $A(2, -2)$.
    \item Find the limit of the following:
          \begin{enumerate}
              \item $\lim\limits_{x\to0}{\dfrac{\sin{4x}}{x}}$
              \item $\lim\limits_{x\to0}{\dfrac{\tan2x}{\tan5x}}$
              \item $\lim\limits_{x\to0}\dfrac{x^{2}}{\tan^{2}3x}$
              \item $\lim\limits_{x\to\infty}{\left(1+{\dfrac{1}{x}}\right)}^{4x+1}$
              \item $\lim\limits_{x\to0}{\left(1-3x\right)}^{\frac{2}{x}}$
              \item $\lim\limits_{x\to\infty}{\left({\dfrac{x+3}{x+1}}\right)}^{x}$
          \end{enumerate}
    \item Find the derivative of the following functions:
          \begin{enumerate}
              \item $y=\tan^{2}3x$
              \item $y=\cos^{4}2x$
              \item $y=\sec^{3}2x$
              \item $y=\sec^{2}(3x+5)$
              \item $y={(1+\sin x)}^{3}$
              \item $y=\sin\left(\cos x\right)$
              \item $y=\sin2x\cos2x$
              \item $y={\dfrac{1}{\sin x+\cos x}}$
              \item $y={\dfrac{\cos 5x}{\sin 3x}}$
              \item $y={\dfrac{1+\cos x}{\sin x}}$
          \end{enumerate}
    \item Find the derivative of the following functions:
          \begin{enumerate}
              \item $y=5^{3x-2}$
              \item $y=3e^{2x^{2}}$
              \item $y=a^{3x}+e^{-3x}$
              \item $y={\dfrac{e^{3x}-e^{2x}+e^{5x}}{e^{2x}}}$
              \item $y=x^{a}-2a^{x}$
              \item $y=e^{2x}\csc2x$
          \end{enumerate}
    \item If the function $y = \dfrac{\sin 2x}{e^x}$, prove that $\dfrac{d^2y}{dx^2} +
              2\dfrac{dy}{dx} + 5y = 0$.
    \item Find the derivative of the following functions:
          \begin{enumerate}
              \item $y=\ln\left(x^{2}+5\right)$
              \item $y=\ln\left(3x^{2}+6x\right)$
              \item $y=\ln\left(e^{x}+2\right)$
              \item $y=\ln\left(\,\sin^{2}4x\right)$
              \item $y=\log\left(x^{3}+3x-4\right)$
              \item $y=\log_{5}\left(3x+7\right)$
              \item $y=\log_{2}{\dfrac{x}{x+3}}$
              \item $y={\dfrac{1+\log x}{1+\ln x}}$
          \end{enumerate}
    \item If the function $y = \ln(x+1)$, find the value of $\dfrac{d^2y}{dx^2}$ when $x
              = 1$.
\end{enumerate}

\end{document}