% chktex-file 2
% chktex-file 29
% chktex-file 13
\documentclass[12pt]{report}
\usepackage{setspace}
\usepackage[a4paper, total={7in, 10in}]{geometry}
\usepackage[fleqn]{amsmath}
\usepackage{empheq}
\usepackage{amssymb}
\usepackage{amsthm}
\usepackage{gensymb}
\usepackage[fleqn]{cases}
\usepackage{multicol}
\usepackage{color}
\usepackage{stix}
\usepackage{chngcntr}
\usepackage{tikz}
\usepackage{enumitem}
\usepackage{pgfplots}
\usepackage{etoolbox}
\usepackage{tkz-euclide}
\usepackage{graphicx}
\usepackage{enumitem}
\usepackage{multirow}
\usepackage{mathtools}
\usepackage{mdframed}
\usepackage{adjustbox}
\usepackage{xpatch}
\usepackage{nicematrix}
\usepackage{ifthen}

\def\nswe#1#2#3{#1\,$#2^\circ\,#3'$}
\graphicspath{ {./assets/} }
\usetikzlibrary{calc,trees,positioning,arrows,fit,shapes,calc, decorations.markings}
\newcommand{\midarrow}{\tikz \draw[-triangle 90] (0,0) -- +(.1,0);}

\newcommand\typel[2]{
    \mathbin{\mathop{#1\kern0pt}%
        \limits_{\raisebox{3.6ex}{\hbox to0pt{\hss\strut$\uparrow$\hss}}\hbox to0pt{\hss#2\hss}}}
}

\newcommand\typem[2]{
    \mathbin{\mathop{#1\kern0pt}%
        \limits^{\raisebox{3.6ex}{\hbox to0pt{\hss#2\hss}}\hbox to0pt{\hss\strut$\downarrow$\hss}}}
}

\counterwithout{equation}{chapter}

\newcommand{\pgfplotsdrawaxis}{\pgfplots@draw@axis}
\newcommand\perm[2][^n]{\prescript{#1\mkern-2.5mu}{}P_{#2}}
\newcommand\permtwo[2][^n]{{}_{#1}P_{#2}}
\newcommand\comb[2][^n]{{}_{#1}C_{#2}}
\newcommand\combtwo[2][^n]{\prescript{#1\mkern-2.5mu}{}C_{#2}}
\makeatother
\pgfplotsset{only axis on top/.style={axis on top=false, after end axis/.code={
                    \pgfplotsset{axis line style=opaque, ticklabel style=opaque, tick style={thick,opaque},
                        grid=none}\pgfplotsdrawaxis}}}

\newtheorem{theorem}{Theorem}

\makeatletter
\xpatchcmd{\endmdframed}
{\aftergroup\endmdf@trivlist\color@endgroup}
{\endmdf@trivlist\color@endgroup\@doendpe}
{}{}
\makeatother

\mdfdefinestyle{MyFrame}{%
    linecolor=black,
    linewidth=1pt,
    roundcorner=20pt, innertopmargin=20pt,innerbottommargin=20pt, innerrightmargin=12pt,
    innerleftmargin=12pt, skipbelow=20pt, skipabove=20pt
    %backgroundcolor=gray!50!white}
}

\newcommand{\newitem}[1]{%
    \refstepcounter{subenum}%
    \parbox{\dimexpr.5\linewidth-.5\columnsep}{
        \makebox[\labelwidth][r]{(\thesubenum)\hspace*{\labelsep}} #1}\hfill }%%%

\setcounter{chapter}{21}

\setlength{\arrayrulewidth}{1pt}
\setlength{\tabcolsep}{12pt}

\begin{document}

\newcommand{\sol}[1]{
    \vspace{0.5em}
    \\
    \noindent \textbf{Sol.}
}
\newcommand{\prooff}[1]{

    \noindent \textbf{Proof.}
}

\newcommand{\sxrightarrow}[2][]{%
    \mathrel{\text{$\xrightarrow[#1]{#2}$}}%
}

\newenvironment{cequation}{
    \makeatletter
    \setbool{@fleqn}{false}
    \makeatother
    \begin{equation*}
        }{\end{equation*}}

\begin{titlepage}
    \raggedleft{}
    \rule{1pt}{\textheight}
    \hspace{0.02\textwidth}
    \parbox[b]{0.75\textwidth}{

    {\fontsize{40}{60}\selectfont\bfseries Mathematics}\\[2\baselineskip]
    {\huge\textit{Senior 3 Part I}}\\[4\baselineskip]
    {\Large\textsc{Melvin Chia}}

    \vspace{0.5\textheight}

    {\noindent Started on 10 April 2023}\\[\baselineskip]
    {\noindent Finished on XX XX 2023}\\[\baselineskip]
    {\noindent Actual time spent: XX days}\\[\baselineskip]}

\end{titlepage}

\chapter*{Introduction}
\addcontentsline{toc}{chapter}{Introduction} \markboth{INTRODUCTION}{}

\doublespacing{}
\section*{Why this book?}

\section*{Disclaimer}
\section*{Acknowledgements}

\singlespacing{}

\doublespacing{}
\tableofcontents
\singlespacing{}
\newpage

\onehalfspacing

\chapter{Limits}

\section{Concept of Limits}

Limit is a fundamental concept of calculus. It is the theoretical basis for
studies on the changes and trends of functions. We will first introduce an
example related to the idea of limits.

\subsection*{Cyclotomic Method by Liu Wei}

The circle is not a shape with straight edges, so where does its area formula
$A = \pi r^2$ come from?

Let the side length of a regular $n$ -gon inscribed in a circle be $a_n$,
length from the center to the side be $r_n$, as shown in the diagram below, the
area of the regular $n$-gon is $A_n = n \cdot \dfrac{1}{2}a_n\cdot r_n$, while
its circumference is $P_n = na_n$, then $A_n = \dfrac{1}{2} P_n \cdot r_n$.

When the value of $n$ becomes larger and larger, the area $A_n$ of the $n$-gon
is indefinitely close to the area $A$ of the circle, denoted as $A =
    \lim\limits_{n\to \infty} A_n$, the limit of $A_n$ is said to be $A$ when $n$
approaches infinity.

When $n \to \infty$, the circumference $P_n$ and the length between center and
side $r_n$ of the inscribed regular $n$-gon, approaches the circumference $P$
and the radius $r$ of the circle respectively. That is to say, $\lim\limits_{n
        \to \infty} P_n = P$, $\lim\limits_{n \to \infty} r_n = r$.
\begin{flalign*}
    \therefore\ A & = \lim\limits_{n \to \infty}A_n                        & \\
                  & = \lim\limits_{n \to \infty}\dfrac{1}{2} P_n \cdot r_n & \\
                  & = \dfrac{1}{2}Pr
\end{flalign*}
from $P = 2\pi r$. we get $A = \pi r^2$.

The concept above uses the idea of "replacing curves with straight lines",
which treats the area of a circle as the limit of the area of a regular $n$-gon
when $n$ approaches infinity. This way of calculating the area of a circle
using the limit is invented by Liu Hui, a mathematician back in the 3rd
century, and is called the "Cyclotomic Method". Quoted from his own words, "The
smaller the circle is divided, the lesser the error is; divide and divide,
until the circle is unable to be divided, then the error will be negligible and
it will be the same as the circle."

\section{Limits of Functions}

If the value of a variable $x$ approaches a certain constant $a$, we say that
$x$ tends to $a$.

When $x$ approaches $x_0$ (but not equal to $x_0$), if the value of $f (x)$
approaches a certain constant $A$, we say that as $x$ approaches $x_0$, the
limit of $f (x)$ is $A$, denoted as $\lim\limits_{x \to x_0} f (x) = A$.

In the definition of the limit $\lim\limits_{x \to x_0} f (x) = A$, when
approaches $x_0$ from the left ($x < x_0$) and right ($x > x_0$), the limit of
$f (x)$ approaches the same constant $A$.

When $x$ approaches $x_0$ from the left ($x < x_0$), denoted as $x \to x_0^-$,
if the value of $f (x)$ approaches a certain constant $A$, then $A$ is the left
limit of $f (x)$ when $x$ approaches $x_0$, denoted as $\lim\limits_{x \to
        x_0^-} f (x) = A$.

Similarly, when $x$ approaches $x_0$ from the right ($x > x_0$), denoted as $x
    \to x_0^+$, if the value of $f (x)$ approaches a certain constant $A$, then $A$
is the right limit of $f (x)$ when $x$ approaches $x_0$, denoted as
$\lim\limits_{x \to x_0^+} f (x) = A$.

If $x$ approaches $x_0$ from the left (or right), but the value of $f (x)$ does
not approach a certain constant, then the left (or right) limit does not exist.

From the definition of the limit, left limit and right limit, we can conclude
the following theorem:
\begin{mdframed}[style=MyFrame]
    When $x \to x_0$, if the left and right limit of the function $f (x)$ exist and are equal, then the limit of $f (x)$ exists, and $\lim\limits_{x \to x_0} f (x) = \lim\limits_{x \to x_0^-} f (x) = \lim\limits_{x \to x_0^+} f (x)$.

    Contrarily, if the left limit or right limit of the function $f (x)$ does not
    exist, or the left and right limit are not equal, then the limit of $f (x)$
    does not exist.
\end{mdframed}

Note that the limit $\lim\limits_{x \to x_0} f (x)$ and the function value $f
    (x_0)$ are two different concepts. When $\lim\limits_{x \to x_0} f (x)$ exists,
it does not mean that $f (x_0)$ exists. Even if $f (x_0)$ exists, it does not
guarantee to be equal to $\lim\limits_{x \to x_0} f (x)$.

\subsection{Practice 1}

Complete the following table, state the changes of the value of the function $f
    (x) = 2x + 1$ when $x \to 1$, and find $\lim\limits_{x \to 1} f (x)$.

\begin{center}
    \begin{tabular}{|c|c|c|c|c|c|}
        \hline
        $x$              & 0.9 & 0.99 & 0.999 & 0.9999 & 0.99999 \\
        \hline
        $f (x) = 2x + 1$ &     &      &       &        &         \\
        \hline
    \end{tabular}
\end{center}
\begin{center}
    \begin{tabular}{|c|c|c|c|c|c|}
        \hline
        $x$              & 1.1 & 1.01 & 1.001 & 1.0001 & 1.00001 \\
        \hline
        $f (x) = 2x + 1$ &     &      &       &        &         \\
        \hline
    \end{tabular}
\end{center}

If $x$ approaches positive infinity, the value of function $f (x)$ approaches a
certain constant $A$, then $A$ is the limit of $f (x)$ when $x \to \infty$,
denoted as $\lim\limits_{x \to \infty} f (x) = A$.

Similarly, if $x$ approaches negative infinity, the value of function $f (x)$
approaches a certain constant $B$, then $B$ is the limit of $f (x)$ when $x \to
    -\infty$, denoted as $\lim\limits_{x \to -\infty} f (x) = B$.

If $x$ approaches positive infinity (or negative infinity), but the value of $f
    (x)$ does not approach a certain constant, then the limit of $f (x)$ does not
exist when $x \to \infty$ (or $x \to -\infty$).

\subsection{Practice 2}

Complete the following table, state the changes of the value of the function $f
    (x) = \dfrac{1}{x}$ when $x \to \infty$, and find $\lim\limits_{x \to \infty} f
    (x)$.

\begin{center}
    \begin{NiceTabular}{|c|c|c|c|c|c|c|}[hvlines,cell-space-limits=5pt]
        $x$                   & -1 & -10 & -100 & -1000 & -10000 & $\cdots$ \\
        $f (x) = \dfrac{1}{x}$ &    &     &      &       &        &          \\
    \end{NiceTabular}
\end{center}

\subsection{Exercise 24.2}

\begin{enumerate}
    \item Complete the following table. Hence, find the left limit, right limit, and
          limit of the function $f (x) = 3x - 1$ and $x = 1$.
          \begin{center}
              \begin{tabular}{|c|c|c|c|c|c|}
                  \hline
                  $x$              & 0.9 & 0.99 & 0.999 & 0.9999 & 0.99999 \\
                  \hline
                  $f (x) = 3x - 1$ &     &      &       &        &         \\
                  \hline
              \end{tabular}
              \vskip 0.2cm
              \begin{tabular}{|c|c|c|c|c|c|}
                  \hline
                  $x$              & 1.1 & 1.01 & 1.001 & 1.0001 & 1.00001 \\
                  \hline
                  $f (x) = 3x - 1$ &     &      &       &        &         \\
                  \hline
              \end{tabular}
          \end{center}

    \item Complete the following tables, state the changes of the value of the function
          $f (x)$ when $x \to x_0$, and find $\lim\limits_{x \to x_0} f (x)$.
          \begin{enumerate}
              \item $f (x) = x^3 + 1$, $x_0 = 0$
                    \begin{center}
                        \begin{tabular}{|c|c|c|c|c|c|}
                            \hline
                            $x$               & 0.1 & 0.01 & 0.001 & 0.0001 & 0.00001 \\
                            \hline
                            $f (x) = x^3 + 1$ &     &      &       &        &         \\
                            \hline
                        \end{tabular}
                        \vskip 0.2cm
                        \begin{tabular}{|c|c|c|c|c|c|}
                            \hline
                            $x$               & -0.1 & -0.01 & -0.001 & -0.0001 & -0.00001 \\
                            \hline
                            $f (x) = x^3 + 1$ &      &       &        &         &          \\
                            \hline
                        \end{tabular}
                    \end{center}

              \item $f (x) = \dfrac{x^2 - 4}{x + 2}$, $x_0 = -2$
                    \begin{center}
                        \begin{tabular}{|c|c|c|c|c|c|}
                            \hline
                            $x$                              & -2.1 & -2.01 & -2.001 & -2.0001 & -2.00001 \\
                            \hline
                            $f (x) = \dfrac{x^2 - 4}{x + 2}$ &      &       &        &         &          \\
                            \hline
                        \end{tabular}
                        \vskip 0.2cm
                        \begin{tabular}{|c|c|c|c|c|c|}
                            \hline
                            $x$     & -1.9 & -1.99 & -1.999 & -1.9999 & -1.99999 \\
                            \hline
                            $f (x)$ &      &       &        &         &          \\
                            \hline
                        \end{tabular}
                    \end{center}

              \item $f (x) = \left\{\begin{array}{rl}
                            4 - x,   & x < 1    \\
                            x^2 + 2, & x \geq 1
                        \end{array}\right.$, $x_0 = 1$
                    \begin{center}
                        \begin{tabular}{|c|c|c|c|c|c|}
                            \hline
                            $x$     & 0.9 & 0.99 & 0.999 & 0.9999 & 0.99999 \\
                            \hline
                            $f (x)$ &     &      &       &        &         \\
                            \hline
                        \end{tabular}
                        \vskip 0.2cm
                        \begin{tabular}{|c|c|c|c|c|c|}
                            \hline
                            $x$     & 1.1 & 1.01 & 1.001 & 1.0001 & 1.00001 \\
                            \hline
                            $f (x)$ &     &      &       &        &         \\
                            \hline
                        \end{tabular}
                    \end{center}
          \end{enumerate}

    \item Complete the following table, state the changes of the function $f (x) =
              \dfrac{2x + 1}{x + 2}$ when $x \to x_0$, and find $\lim\limits_{x \to \infty} f
              (x)$.
          \begin{center}
              \begin{tabular}{|c|c|c|c|c|c|c|}
                  \hline
                  $x$     & 1 & 10 & 100 & 1000 & 10000 & 100000 \\
                  \hline
                  $f (x)$ &   &    &     &      &       &        \\
                  \hline
              \end{tabular}
          \end{center}

    \item Complete the following table, state the changes of the function $f (x) = 2^x$
          when $x \to -\infty$, and find $\lim\limits_{x \to -\infty} f (x)$.
          \begin{center}
              \begin{tabular}{|c|c|c|c|c|c|c|}
                  \hline
                  $x$     & -1 & -2 & -3 & -4 & -5 & -10 \\
                  \hline
                  $f (x)$ &    &    &    &    &    &     \\
                  \hline
              \end{tabular}
          \end{center}
\end{enumerate}

\section{Arithmetic Rules of Limits of Functions}

If $\lim\limits_{x \to x_0} f (x)$ and $\lim\limits_{x \to x_0} g(x)$ exist,
then
\begin{enumerate}
    \item $\lim\limits_{x \to x_0} [f (x) \pm g(x)] = \lim\limits_{x \to x_0} f (x) \pm
              \lim\limits_{x \to x_0} g(x)$
    \item $\lim\limits_{x \to x_0} [f (x) \cdot g(x)] = \lim\limits_{x \to x_0} f (x) \cdot \lim\limits_{x \to x_0} g(x)$
    \item $\lim\limits_{x \to x_0} \dfrac{f (x)}{g(x)} = \dfrac{\lim\limits_{x \to x_0} f (x)}{\lim\limits_{x \to x_0} g(x)}$,
          provided that $\lim\limits_{x \to x_0} g(x) \neq 0$
    \item $\lim\limits_{x \to x_0} k \cdot f (x) = k \cdot \lim\limits_{x \to x_0} f (x)$, where $k$ is a constant
    \item $\lim\limits_{x \to x_0} [f (x)]^n = \left[\lim\limits_{x \to x_0} f (x)\right]^n$, where $n$ is a positive integer
    \item $\lim\limits_{x \to x_0} k = k$, where $k$ is a constant
\end{enumerate}

\noindent The rules above also applied to $x \to \infty$ and $x \to -\infty$. Obviously,
$\lim\limits_{x \to x_0} x = x$, where $k$ is a constant.

Using the arithmetic rules above, we can calculate the limit of relatively
complicated functions using given limit of simpler functions. For example, from
the rules above, we can get
\begin{flalign*}
     & \lim_{x \to x_0}\left(a_nx^n + a_{n - 1}x^{n - 1} + \cdots + a_1x + a_0\right) \\
     & = a_nx_0^n + a_{n - 1}x_0^{n - 1} + \cdots + a_1x_0 + a_0
\end{flalign*}

\subsection{Practice 3}

Find the limit of the followings:
\begin{enumerate}
    \item $\lim\limits_{x\to-1}{\dfrac{\sqrt{x+2}-3}{x+4}}$
    \item $\lim\limits_{x\to1}{\dfrac{x^{2}+2x-3}{x^{2}-3x+2}}$
    \item $\lim\limits_{x\to \infty}{\dfrac{x^{2}+1}{2x^{2}+x+1}}$
\end{enumerate}

\subsection{Practice 4}

State whether the left limit, right limit, and limit of the function $f (x) =
    \dfrac{1}{x}$ exist when $x \to 0$.

\subsection{Exercise 24.3}

\setlength{\columnseprule}{1pt}
\setlength{\columnsep}{24pt}
\begin{multicols}{2}
    Find the limit of the followings (Question 1 to 31):

    \begin{enumerate}
        \item $\lim\limits_{x\to-2}\left(x^{2}+x-3\right)$
        \item $\lim\limits_{x\to4}x^{2}(x-1)$
        \item $\lim\limits_{x\to-2}x\left(9-x^{2}\right)$
        \item $\lim\limits_{x\to-1}(x+3)\left(x-1\right)$
        \item $\lim\limits_{x\to{\frac{1}{2}}}\left(2x-1\right)\left(x^{2}+3x+4\right)$
        \item $\lim\limits_{x\to-2}\left(4x^{3}+2x^{2}+3x+1\right)$
        \item $\lim\limits_{x\to2}{\dfrac{x^{2}+2}{x-5}}$
        \item $\lim\limits_{x\to-1}{\dfrac{(x+2)(x-3)}{x-1}}$
        \item $\lim\limits_{x\to0}{\dfrac{2x^{2}+3x-4}{x-4}}$
        \item $\lim\limits_{x\to-3}\dfrac{(x+5)(x+3)}{x+3}$
        \item $\lim\limits_{x\to2}{\dfrac{x^{2}-5x+6}{x-2}}$
        \item $\lim\limits_{x\to-2}{\dfrac{x^{3}+8}{x+2}}$
        \item $\lim\limits_{x\to1}{\dfrac{x^{2}-2x+1}{x^{2}-1}}$
        \item $\lim\limits_{x\to0}{\dfrac{2x^{5}+3x}{x^{2}+2x}}$
        \item $\lim\limits_{x\to0}{\dfrac{4x^{3}+x^{2}}{x^{2}-3x}}$
        \item $\lim\limits_{x\to1}{\dfrac{x^{3}+3x^{2}-2x-2}{2x^{2}+x-3}}$
        \item $\lim\limits_{x\to2}{\sqrt{2x^{2}+1}}$
        \item $\lim\limits_{x\to2}{\dfrac{{\sqrt{x+2}}-3}{x^{2}-9}}$
        \item $\lim\limits_{x\to4}{\dfrac{{\sqrt{x}}-2}{x-4}}$
        \item $\lim\limits_{x\to0}{\dfrac{\sqrt{3x+4}-2}{x}}$
        \item $\lim\limits_{x\to1}{\dfrac{2-{\sqrt{x+3}}}{x^{2}-1}}$
        \item $\lim\limits_{x\to2}{\dfrac{{\sqrt{3x-2}}-{\sqrt{x+2}}}{x-2}}$
        \item $\lim\limits_{x\to1}\left({\dfrac{x+3}{x^{2}-1}}-{\dfrac{x+1}{x^{2}-x}}\right)$
        \item $\lim\limits_{x\to\infty}{\dfrac{2x^{3}-1}{x^{3}-x+2}}$
        \item $\lim\limits_{x\to\infty}{\dfrac{3x^{3}-2x+1}{2x^{3}+x^{2}-5}}$
        \item $\lim\limits_{x\to\infty}{\dfrac{2x+7}{x^{3}+2x^{2}-4}}$
        \item $\lim\limits_{x\to\infty}{\dfrac{x^{4}+2x^{3}-x^{2}+1}{x^{5}+x^{3}-2}}$
        \item $\lim\limits_{x\to\infty}\left({\dfrac{x^{2}+3x}{2_x+1}}-{\dfrac{x}{2}}\right)$
        \item $\lim\limits_{x\to\infty}2^{x}$
        \item $\lim\limits_{x\to\infty}{\dfrac{x^{2}}{x+1}}$
        \item $\lim\limits_{x\to\infty}{\dfrac{x^{3}-3x+1}{2x^{2}+x+1}}$
        \item $f (x) = \left\{\begin{array}{rl}
                      -1, & x < 0    \\
                      2x, & x \geq 0
                  \end{array}\right.$, find $\lim\limits_{x\to0}f (x)$
        \item $f (x) = \left\{\begin{array}{rl}
                      -x+1,   & x < 1 \\
                      3,      & x = 1 \\
                      2x - 2, & x > 1
                  \end{array}\right.$, find $\lim\limits_{x\to1}f (x)$
    \end{enumerate}
    Determine if the limit of the following functions exists at $x = x_0$. If it
    exists, find their limit.
    \begin{enumerate}
        \setcounter{enumi}{33}
        \item $f (x) = \left\{\begin{array}{rl}
                      x+1, & x < 0 \\
                      0,   & x = 0 \\
                      x-1, & x > 0
                  \end{array}\right.$, $x_0 = 0$

        \item $f (x) = \left\{\begin{array}{rl}
                      -x + 1, & x \leq 2 \\
                      x - 3,  & x > 2
                  \end{array}\right.$, $x_0 = 0$, $x_0 = 2$

        \item $f (x) = \dfrac{1}{x+3}$, $x_0 = -3$
    \end{enumerate}
\end{multicols}

\section{Revision Exercise 24}

\begin{enumerate}
    \item Complete the following table, state the changes of the function $f (x) = x^3 -
              1$ at $x = 1$. Hence, find the left limit, right limit and limit of the
          function at $x = 1$.
          \begin{center}
              \begin{tabular}{|c|c|c|c|c|c|c|c|c|c|}
                  \hline
                  $x$     & $0.9$ & $0.99$ & $0.999$ & $0.9999$ & $1$ & $1.0001$ & $1.001$ & $1.01$ & $1.1$ \\
                  \hline
                  $f (x)$ &       &        &         &          &     &          &         &        &       \\
                  \hline
              \end{tabular}
          \end{center}
\end{enumerate}
\vskip 1em

\setlength{\columnseprule}{1pt}
\setlength{\columnsep}{24pt}
\begin{multicols}{2}
    \noindent Find the limit of the followings (Question 2 to 13):
    \begin{enumerate}
        \setcounter{enumi}{1}
        \item $\lim\limits_{x\to1}\left(x^{2}+x-2\right)$
        \item $\lim\limits_{x\to2}\left(x^{2}-1\right){\sqrt{x+2}}$
        \item $\lim\limits_{x\to-2}{\dfrac{2x^{2}+3x-2}{3x^{2}+4x-4}}$
        \item $\lim\limits_{x\to2}{\dfrac{3-{\sqrt{x+7}}}{x^{2}-1}}$
        \item $\lim\limits_{x\to0}{\dfrac{2-{\sqrt{3x+4}}}{x^{2}+x}}$
        \item $\lim\limits_{x\to-1}{\dfrac{{\sqrt{2x+5}}-{\sqrt{3}}}{x+1}}$
        \item $\lim\limits_{x\to3}{\dfrac{{\sqrt{x+1}}-2}{x^{2}-x-6}}$
        \item $\lim\limits_{x\to\infty}{\dfrac{4x+3}{x^{2}+2x-1}}$
        \item $\lim\limits_{x\to\infty}{\dfrac{2x^{4}-x^{3}+3x+1}{3x^{4}+4x^{2}-x-2}}$
        \item $\lim\limits_{x\to\infty}{\dfrac{x^{3}-2x^{2}+3}{2x^{2}-1}}$
        \item $\lim\limits_{x\to1}\left({\dfrac{1}{x-1}}-{\dfrac{2}{x^{2}-1}}\right)$
        \item $\lim\limits_{x\to\infty}\left(\dfrac{x^{3}}{2x^{2}+1}-\dfrac{x^{2}}{2x+3}\right)$
    \end{enumerate}

    \noindent Determine if the limit of the following functions exists at $x = x_0$. If it exists, find their limit. (Question 14 to 17)
    \begin{enumerate}
        \setcounter{enumi}{13}
        \item $f (x) = \left\{\begin{array}{rl}
                      1-3x, & x < 0    \\
                      2x+1, & x \geq 0 \\
                  \end{array}\right.$, $x_0 = 0$
        \item $f (x) = \left\{\begin{array}{rl}
                      \sqrt{x+3}, & x < -2 \\
                      x+1,        & x > -2 \\
                  \end{array}\right.$, $x_0 = -2$
        \item $f (x) = \left\{\begin{array}{rl}
                      \dfrac{x^2 - 1}{x-1}, & x \neq 1 \\
                      \dfrac{1}{2},         & x = 1    \\
                  \end{array}\right.$, $x_0 = 1$
        \item $f (x) = \left\{\begin{array}{rl}
                      2x + 1,  & x \leq 1     \\
                      x^2 + 1, & 1 < x \leq 2 \\
                      3x - 1,  & x > 2        \\
                  \end{array}\right.$, $x_0 = 1$, $x_0 = 2$
    \end{enumerate}
\end{multicols}

\end{document}