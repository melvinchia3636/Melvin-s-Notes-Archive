% Options for packages loaded elsewhere
\PassOptionsToPackage{unicode}{hyperref}
\PassOptionsToPackage{hyphens}{url}
%
\documentclass[
]{article}
\usepackage{amsmath,amssymb}
\usepackage{iftex}
\usepackage{ctex}
\usepackage{tabularx}
\usepackage{multicol}
\usepackage[total={7in,9in}]{geometry}
\newcolumntype{Z}{>{\centering\let\newline\\\arraybackslash\hspace{0pt}}X}
\ifPDFTeX
  \usepackage[T1]{fontenc}
  \usepackage[utf8]{inputenc}
  \usepackage{textcomp} % provide euro and other symbols
\else % if luatex or xetex
  \usepackage{unicode-math} % this also loads fontspec
  \defaultfontfeatures{Scale=MatchLowercase}
  \defaultfontfeatures[\rmfamily]{Ligatures=TeX,Scale=1}
\fi
\usepackage{lmodern}
\ifPDFTeX\else
  % xetex/luatex font selection
\fi
% Use upquote if available, for straight quotes in verbatim environments
\IfFileExists{upquote.sty}{\usepackage{upquote}}{}
\IfFileExists{microtype.sty}{% use microtype if available
  \usepackage[]{microtype}
  \UseMicrotypeSet[protrusion]{basicmath} % disable protrusion for tt fonts
}{}
\makeatletter
\@ifundefined{KOMAClassName}{% if non-KOMA class
  \IfFileExists{parskip.sty}{%
    \usepackage{parskip}
  }{% else
    \setlength{\parindent}{0pt}
    \setlength{\parskip}{6pt plus 2pt minus 1pt}}
}{% if KOMA class
  \KOMAoptions{parskip=half}}
\makeatother
\usepackage{xcolor}
\usepackage{longtable,booktabs,array}
\usepackage{calc} % for calculating minipage widths
% Correct order of tables after \paragraph or \subparagraph
\usepackage{etoolbox}
\makeatletter
\patchcmd\longtable{\par}{\if@noskipsec\mbox{}\fi\par}{}{}
\makeatother
% Allow footnotes in longtable head/foot
\IfFileExists{footnotehyper.sty}{\usepackage{footnotehyper}}{\usepackage{footnote}}
\makesavenoteenv{longtable}
\setlength{\emergencystretch}{3em} % prevent overfull lines
\providecommand{\tightlist}{%
  \setlength{\itemsep}{0pt}\setlength{\parskip}{0pt}}
\setcounter{secnumdepth}{-\maxdimen} % remove section numbering
\ifLuaTeX
  \usepackage{selnolig}  % disable illegal ligatures
\fi

\usepackage{titlesec}

\titleformat*{\section}{\centering\LARGE\bfseries}
\titleformat*{\subsection}{\Large\bfseries}
\titleformat*{\subsubsection}{\large\bfseries}
\titleformat*{\paragraph}{\large\bfseries}
\titleformat*{\subparagraph}{\large\bfseries}

\begin{document}

\section{2024
年高三年中考考试范围}\label{2024-ux5e74ux9ad8ux4e09ux5e74ux4e2dux8003ux8003ux8bd5ux8303ux56f4}

\subsection{华文}\label{ux534eux6587}

\subsubsection{高三上册}\label{ux9ad8ux4e09ux4e0aux518c}

\begin{tabularx}{\textwidth}{lX}
\toprule
章节 & 内容 \\
\bottomrule
第6、7课 & 注释、课文翻译、课文问答 \\
\midrule
第5、6、7、8、9、10、11、12、13、15、16、17课 & 课文作者知识、学习提示 \\
\bottomrule
\end{tabularx}

\subsubsection{高三下册}\label{ux9ad8ux4e09ux4e0bux518c}

\begin{tabularx}{\textwidth}{lX}
\toprule
章节 & 内容 \\
\bottomrule
第1、2课 & 课文翻译、课文问答 \\
\midrule
第1、2、3课 & 课文作者知识、学习提示、注释 \\
\bottomrule
\end{tabularx}

\subsubsection{文学与文化常识 \&
理解文}\label{ux6587ux5b66ux4e0eux6587ux5316ux5e38ux8bc6--ux7406ux89e3ux6587}

\begin{tabularx}{\textwidth}{lX}
\toprule
内容 & 范围 \\
\bottomrule
语文知识 & 生字、词汇、读音、修辞、病句修改 \\
\midrule
文学史 & 第三、四、五章 \\
\midrule
诗歌默写 &
\vtop{\hbox{\strut 高三:《琵琶行》第二段;}\hbox{\strut 高一上册古诗:\parbox[t]{0.65\textwidth}{《夜雨寄北》、《从军行》、《示儿》、《草》、《登高》、《山居秋暝》、《黄鹤楼》、《和子由渑池怀旧》}}\vspace*{-1.1em}} \\
\midrule
现代文阅读 &
\vtop{\hbox{\strut 课外现代文阅读:两篇;}\hbox{\strut 课内文言文一则:}\hbox{\strut 高一上册:第三课《先妣事略》、第十九课《五柳先生传》;}\hbox{\strut 高一下册:第三课《醉翁亭记》}\vspace*{-1.5em}} \\
\midrule
古诗文阅读 &
课外文言文:一则;课外古诗词:一则 \\
\midrule
应用文 & 启事、通告、公函 \\
\bottomrule
\end{tabularx}

\newpage
\subsection{Bahasa Malaysia}\label{bahasa-malaysia}

\begin{longtable}[]{@{}
  >{\raggedright\arraybackslash}p{(\columnwidth - 2\tabcolsep) * \real{0.5000}}
  >{\raggedright\arraybackslash}p{(\columnwidth - 2\tabcolsep) * \real{0.5000}}@{}}
\toprule
题型 & 内容 \\
\bottomrule
Rumusan (20\%) & \vtop{\hbox{\strut Pendahuluan, }\hbox{\strut Isi
tersurat (6 isi), }\hbox{\strut Isi tersirat (2 isi),
}\hbox{\strut Penutup}} \\
\midrule
Petikan Umum (20\%) & \vtop{\hbox{\strut 1. Maksud rangkai kata,
}\hbox{\strut 2. Maksud frasa, }\hbox{\strut 3. Dari petikan,
}\hbox{\strut 4. Dari petikan, }\hbox{\strut 5. KBAT}} \\
\midrule
Komsas (20\%) & \vtop{\hbox{\strut Baca nota }\parbox{0.48\textwidth}{Petikan
daripada novel(Cempaka Berdarah / Samudera / Pantun Empat Kerat)}} \\
\midrule
Bina Ayat (10\%) & \vtop{\hbox{\strut Kata pinjaman (ms 99), Kata
transitif}\hbox{\strut Baca nota bina ayat}} \\
\midrule
Sintaksis (10\%) & \vtop{\hbox{\strut Ayat aktif dan
pasif}\hbox{\strut Baca nota}} \\
\midrule
Analisis (10\%) & Salah -\textgreater{} Betul (Kesalahan ejaan,
Kesalahan imbuhan, Kesalahan kata transitif) \\
\midrule
Kosa Kata (10\%) & \vtop{\hbox{\strut Isi tempat kosong
}\hbox{\strut Baca nota kosa kata}} \\
\bottomrule
\end{longtable}

\subsection{English}\label{english}

\subsubsection{Section A: Multiple Choice Question (UEC
Format)}\label{section-a-multiple-choice-question-uec-format}

\begin{tabularx}{\textwidth}{cXc}
\toprule
Part & Particulars & Marks \\
\bottomrule
I & Comprehension (10 objective questions) & 10\% \\
\midrule
II & Vocabulary (10 objective questions) & 10\% \\
\midrule
III & Matching Paragraph Information (10 questions) & 10\% \\
\midrule
IV & Error Identification (12 questions) & 12\% \\
\midrule
\end{tabularx}

\subsubsection{Section B - E}\label{section-b---e}

\begin{tabularx}{\textwidth}{cXc}
\toprule
Section & Particulars & Marks \\
\bottomrule
B & Comprehension (5 subjective questions) & 10\% \\
\midrule
C & Vocabulary (12 questions) & 12\% \\
\midrule
D & Verb Forms (12 questions) & 12\% \\
\midrule
E & Word Formation (10 questions in passage form, 10 questions in
sentence form) & 20\% \\
\bottomrule
\end{tabularx}

\subsubsection{Revision Materials}\label{ux590dux4e60ux8d44ux6599}

\begin{itemize}
\item
  Dongzong Workbook, Unit 3, 4, and 5
\item
  UEC Test Practices, Unit 3 and 4
\item
  Reading Passages, Unit 2 and 3
\item
  All verb forms (active and passive)
\end{itemize}

\subsection{数学}\label{ux6570ux5b66}

\begin{itemize}
\item
  第22章 函数
\item
  第23章 指数与对数
\item
  第24章 极限
\item
  第25章 微分(到 25.7 两个基本极限)
\end{itemize}

\subsection{商业学}\label{ux5546ux4e1aux5b66}

\begin{tabularx}{\textwidth}{cZ}
\toprule
册 & 章节 \\
\bottomrule
第一册 & CH 7, 8, 9 \\
\midrule
第二册 & CH 1, 2, 3 \\
\midrule
第三册 & CH 1 \\
\bottomrule
\end{tabularx}

\newpage
\subsection{会计学}\label{ux4f1aux8ba1ux5b66}

\begin{itemize}
\item
  Inventory Valuation
\item
  Investment in Own Shares
\item
  Redemption / Purchases of Shares and Redemption of Loan Notes
\item
  Capital Reduction
\item
  Business Combinations
\end{itemize}

\subsection{经济学}\label{ux7ecfux6d4eux5b66}

\subsubsection{范围}\label{ux8303ux56f4}

\begin{tabularx}{\textwidth}{lX}
\toprule
册 & 章节 \\
\bottomrule
上册 & 第六章 生产与成本 \\
\midrule
下册 & \vtop{\hbox{\strut 第三章 经济循环与发展}\hbox{\strut 第四章 失业、物价与通货膨胀}\hbox{\strut 第五章 货币与存款货币的创造}\vspace*{-1.5em}} \\
\bottomrule
\end{tabularx}

\vspace{2em}

\begin{multicols}{2}
\subsubsection{出题方式}\label{ux51faux9898ux65b9ux5f0f}
  \begin{itemize}
    \item
      选择题(UEC及课本选择题)
    \item
      数据分析题
    \item
      问答题(包含列举)
    \end{itemize}
    
    \subsubsection{数据分析题}\label{ux6570ux636eux5206ux6790ux9898}
    
    \begin{itemize}
    \item
      会计成本及经济成本的计算
    \item
      各类成本的计算
    \item
      经济成长率的计算
    \item
      失业率及物价的计算
    \item
      存款货币创造的计算
    \end{itemize}
    \columnbreak
    
    \subsubsection{问答题}\label{ux95eeux7b54ux9898}
    
    \begin{enumerate}
    \def\labelenumi{\arabic{enumi}.}
    \item
      绘图说明包络曲线?并写出其四项特质?
    \item
      何谓边际报酬递减法则?试举例说明之?
    \item
      何谓规模经济(规模不经济)?
    \item
      绘图说明经济成循环
    \item
      试区别经济成长记与经济发展
    \item
      说明影响经济成长的因素(思维题)
    \item
      不同种类失业的定义及对策(思维题)
    \item
      何谓通货膨胀,原因及影响(思维题)
    \item
      何谓通货紧缩,原因及影响(思维题)
    \item
      试写出存款货币创造的三个假设
    \end{enumerate}
\end{multicols}

\end{document}
