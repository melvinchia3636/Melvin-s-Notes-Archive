\documentclass{article}
\usepackage[total={7in, 9in}]{geometry}
\usepackage{ctex}
\usepackage{booktabs}
\usepackage{tabularx}
\usepackage{siunitx}
\usepackage{enumitem}
\usepackage[autostyle]{csquotes}
\usepackage{mdframed}
\usepackage{xcolor}

\title{\Large{2024年高三年中考考前复习}\\\LARGE{\textbf{- 会计学 -}}}
\date{}

\begin{document}

\maketitle

\vspace*{-4em}
\section{股票及债券赎回\\Redemption / Purchase of Shares And Loan Notes}

\subsection{股票赎回\\Redemption / Purchase of Shares}

\noindent 股票赎回就像你把之前借出去的东西拿回来。想象一下,你借给朋友一本书。过了一段时间,你想要回这本书,于是你跟朋友说:“能把书还给我吗?”这就是“赎回”。在股票市场里,公司有时候会向大众“借钱”,也就是发行股票。买了股票的人就相当于借钱给公司了。公司承诺会给这些股票持有者分红,或者股票价格上涨后,持有者可以卖出股票赚取差价。但是,公司有时候也会想要把这些股票拿回来。他们会宣布一个“赎回计划”,按照一定的价格从股票持有者手里买回股票。这就叫做“股票赎回”。

\subsubsection{股票赎回的原因\\Reasons for Share Redemption}

\begin{itemize}[labelindent=1cm]
\item \textbf{市场层面 (Market Considerations):}
    \begin{itemize}
        \item \enquote{提振股价} (Boost Stock Price): 减少市场流通股数,增加股票稀缺性,从而推高股价。
        \item \enquote{抵御恶意收购} (Fend off Hostile Takeovers): 提高收购成本,增加收购难度,保护公司独立性。
    \end{itemize}
\item \textbf{财务层面 (Financial Considerations):}
    \begin{itemize}
        \item \enquote{优化资本结构} (Optimize Capital Structure): 调整债务和股权比例,降低财务风险。
        \item \enquote{提高每股收益} (Increase Earnings Per Share): 减少总股本,提高每股收益,提升公司形象。
        \item \enquote{有效利用资金} (Utilize Excess Cash): 在没有更好投资机会时,将多余资金用于回馈股东。
    \end{itemize}
\item \textbf{股东层面 (Shareholder Considerations):}
    \begin{itemize}
        \item \enquote{回报股东} (Reward Shareholders): 将资金返还给股东,让他们有机会获得回报。
        \item \enquote{解决内部纠纷} (Resolve Internal Conflicts): 通过赎回部分股东股票来解决内部矛盾。
        \item \enquote{摆脱麻烦股东} (Remove Troublesome Shareholders): 减少激进投资者或恶意收购者的影响力。
    \end{itemize}
\item \textbf{其他 (Other):}
    \begin{itemize}
        \item \enquote{员工激励} (Employee Incentives): 将赎回的股票用于员工激励计划,提高员工积极性。
    \end{itemize}
\end{itemize}

\subsubsection{赎回和收购股票的区别\\Difference Between Redemption and Purchase of Shares}

\vspace{1em}
\begin{tabularx}{0.955\linewidth}{lXX}
    \toprule
    \textbf{特点} & \textbf{赎回 (Redemption)} & \textbf{收购 (Purchase)} \\
    \midrule
    股票类型 (Share Type) & 
    适用于优先股或可赎回股票 \newline
    Usually applies to preferred or redeemable shares & 
    适用于普通股或优先股 \newline
    Applies to common or preferred shares \\
    \midrule
    强制性 (Compulsory) & 
    公司强制要求股东出售股票 \newline
    Company forces shareholders to sell shares & 
    公司在公开市场或直接从股东手中购买股票 \newline
    Company buys shares in the open market or directly from shareholders \\
    \midrule
    价格 (Price) & 
    通常有预先确定的赎回价格,可能高于市场价 \newline
    Usually has a predetermined redemption price, which may be higher than the market price & 
    按照市场价格或协商价格进行交易 \newline
    Traded at market price or negotiated price \\
    \midrule
    目的 (Purpose) & 
    改变资本结构、降低股息支出、应对财务压力等 \newline
    Change capital structure, reduce dividend payments, respond to financial pressure, etc. & 
    提振股价、回报股东、优化资本结构、抵御恶意收购等 \newline
    Boost stock price, reward shareholders, optimize capital structure, fend off hostile takeovers, etc. \\
    \midrule
    股东选择 (Shareholder Choice) & 
    没有选择权,必须出售股票 \newline
    Shareholders usually have no choice but to sell shares & 
    可以自由选择是否出售股票 \newline
    Shareholders can freely choose whether to sell shares \\
    \bottomrule
\end{tabularx}

\subsubsection{公司赎回/购回股份的条件\\Conditions for Share Redemption/Repurchase}

\noindent 在公司赎回/购回 (Share Redemption/Repurchase) 自己的股份之前和之后,必须满足以下条件:

\begin{enumerate}[label=\alph*)]
    \item 公司的《公司章程》 (Articles of Association) 授权进行赎回/购回;
    \item 只有在股份完全缴清 (Fully Paid-up) 后才能赎回/购回;
    \item 公司可以发行新股来取代 (Replace) 已赎回的股份,或者拥有足够的留存利润 (Retained Profits) 来进行赎回/购回;
    \item 赎回/购回后,公司必须持有不可赎回 (Non-redeemable) 的已发行股份,以避免最终没有股东的情况。
\end{enumerate}

\subsubsection{股票赎回的会计处理\\Accounting Treatment of Share Redemption}

\begin{enumerate}
    \item \textbf{筹措资金\\Raising Funds}
    
    公司需要足够的资金来赎回股票,可以通过内部留存利润 (Retained Profits) 或发行新股 (Issuing New Shares) 筹措。

    如果题目中说明需要发行新股,那么在进行股票赎回前,公司需要先发行新股,然后再用新发行的股票来赎回原有的股票。

    \begin{mdframed}[backgroundcolor=gray!10]
    \textbf{Dr} Ordinary Share Capital

    \hspace{1.7em}\textbf{Cr} Application and Allotment

    \textbf{Dr} Application and Allotment

    \hspace{1.7em}\textbf{Cr} Ordinary Share Capital

    \hspace{1.7em}\textbf{Cr} Share Premium
    \end{mdframed}

    \item \textbf{将要赎回的股票从资本账户中转出\\Transferring Shares to be Redeemed from Capital Account}

    在赎回股票之前,需要将要赎回的股票从资本账户 (Capital Account) 中转去赎回股票账户 (Share Redemption Account),让股东知道这部分股票已经不再属于他们。

    这个赎回股票账户是一个临时账户,用来记录赎回的股票数量和金额,以及使用哪个方面的资金来赎回。

    如果赎回的股票原本是按面值 (At Par Value) 发行的,那么赎回时也按照面值来处理。如果是按溢价 (At Premium) 发行的,那么赎回时也按照溢价来处理。

    如果是按面值赎回股票 (At Par Value):
    \begin{mdframed}[backgroundcolor=gray!10]
    \textbf{Dr} Ordinary / Preferred Share Capital

    \hspace{1.7em}\textbf{Cr} Ordinary / Preferred Share Redemption
    \end{mdframed}

    如果是按溢价赎回股票 (At Premium):
    \begin{mdframed}[backgroundcolor=gray!10]
    \textbf{Dr} Ordinary / Preferred Share Capital

    \hspace{1.7em}\textbf{Cr} Ordinary / Preferred Share Redemption

    \hspace{1.7em}\textbf{Cr} Share Premium
    \end{mdframed}

    \newpage
    如果赎回的股票原本是按面值发行,但是赎回时按溢价处理,那么溢价被视为公司的亏损,需要转入保留利润 (Retained Profits) 账户。
    \begin{mdframed}[backgroundcolor=gray!10]
        \textbf{Dr} Ordinary / Preferred Share Capital

        \textbf{Dr} Retained Profits - Premium on Redemption

        \hspace{1.7em}\textbf{Cr} Ordinary / Preferred Share Redemption
    \end{mdframed}

    如果赎回的股票原本是按溢价发行,但是赎回时按更高的溢价处理,那么则是将溢价的差额视为公司的亏损,需要转入保留利润 (Retained Profits) 账户。
    \begin{mdframed}[backgroundcolor=gray!10]
        \textbf{Dr} Ordinary / Preferred Share Capital

        \textbf{Dr} Share Premium

        \textbf{Dr} Retained Profits - Premium on Redemption (赎回的溢价 - 发行时溢价)

        \hspace{1.7em}\textbf{Cr} Ordinary / Preferred Share Redemption
    \end{mdframed}

    公司通常不能以低于股票面值的价格赎回股票 (Redeem at discount) ,因为这会违反法律规定、损害股东利益、影响公司形象并带来潜在法律风险。

    \item \textbf{支付赎回款项\\Payment of Redemption Amount}
    
    公司需要支付给股东赎回款项,这部分资金可以来自公司的现金账户 (Cash Account) 或者银行账户 (Bank Account)。无论资金来源是来自内部还是外筹,最终都会通过银行账户来支付。

    \begin{mdframed}[backgroundcolor=gray!10]
    \textbf{Dr} Ordinary / Preferred Share Redemption

    \hspace{1.7em}\textbf{Cr} Bank / Cash
    \end{mdframed}
    
    \item \textbf{计算赎回资本储备金\\Calculation of Capital Redemption Reserve (CRR)}
    
    公司赎回股票会减少公司的可用资金。为了确保公司在赎回后仍有足够的资金维持运营和偿还债务,马来西亚法律规定公司需要设立资本赎回储备金 (Capital Redemption Reserve, CRR),以弥补赎回股票所减少的资本。所以,公司需要将这部分资金从留存利润 (Retained Profits) 中转入资本赎回储备金账户。

    \begin{mdframed}[backgroundcolor=gray!10]
    \textbf{Dr} CRR (Total Par Value of Shares Redeemed - Total New Shares Issued (if any))

    \hspace{1.7em}\textbf{Cr} Retained Profits
    \end{mdframed}
\end{enumerate}

\newpage

\subsection{债券赎回\\Redemption of Loan Notes}

\noindent 债券赎回就像公司提前还清欠款。想象一下,你借钱给朋友,约定几年后还清,每年支付利息。但有一天,朋友突然有钱了,想提前把欠你的钱还清,这就是“债券赎回”。在债券市场里,公司或政府发行债券,相当于向投资者借钱。投资者购买债券,就相当于借钱给发行方。发行方承诺在一定期限内还本付息。但有时候,发行方可能因为各种原因,比如利率下降、财务状况改善等,想要提前还债。他们会宣布一个“债券赎回计划”,按照一定的价格从投资者手中买回债券。

\subsubsection{债券赎回的原因\\Reasons for Loan Notes Redemption}

\begin{itemize}[labelindent=1cm]
    \item \textbf{降低利息成本 (Reduce Interest Costs):} 当市场利率下降时,发行方可以以更低的利率发行新债券,从而降低利息成本。
    \item \textbf{优化债务结构 (Optimize Debt Structure):} 通过赎回部分债券,发行方可以调整债务的期限和利率结构,降低财务风险。
    \item \textbf{改善信用评级 (Improve Credit Rating):} 提前偿还债务可以展示发行方的良好财务状况,提高信用评级,从而更容易获得融资。
\end{itemize}

\subsubsection{债券赎回的价格\\Prices of Loan Notes Redemption}

\begin{itemize}
    \item \textbf{At Par (平价赎回):}\ 赎回价格等于债券的面值 (Face Value)。这是最常见的情况,通常发生在债券到期时 (Maturity)。
    \item \textbf{At Premium (溢价赎回):}\ 赎回价格高于债券的面值。发行方通常会在债券合同中预先规定溢价赎回的条款和时间。溢价赎回可以补偿投资者因提前赎回而损失的利息收入。
    \item \textbf{At Discount (折价赎回):}\ 赎回价格低于债券的面值。这种情况比较少见,通常发生在发行方遇到财务困难时。折价赎回可以帮助发行方减少债务负担,但会给投资者带来损失。
\end{itemize}

\subsubsection{债券赎回的资金来源\\Sources of Funds for Loan Notes Redemption}

\begin{enumerate}[label=\alph*)]
    \item 债券面值 (Nominal Value) 的资金来源可以是:
        \begin{enumerate}[label=\roman*)]
            \item 发行新股/债券 (Issue of New Shares/Debentures) 所得款项;或/和
            \item 出售投资/有形资产 (Sale of Investments/Tangible Assets) 所得款项;或/和
            \item 可分配利润 (Distributable Profits)(法律不要求)。
        \end{enumerate}
    \item 债券溢价赎回 (Premium on Redemption) 的资金来源可以是:
        \begin{enumerate}[label=\roman*)]
            \item 股本溢价 (Share Premium);或/和
            \item 可分配利润 (Distributable Profits)。
        \end{enumerate}
\end{enumerate}

\subsubsection{债券赎回储备金\\Redemption Reserve for Loan Notes}

\noindent 即使没有发行新股/债券 (Issue of New Shares/Loan Notes) 的收益,法律也没有强制要求 (Mandatory) 通过将等同于待赎回债券面值 (Nominal Value) 的金额从留存收益 (Retained Profits) 转入债券赎回准备金 (Loan Notes Redemption Reserve) 来替换债券。

然而,出于谨慎性原则 (Prudent) 和良好的会计实务 (Good Accounting Practice),强烈建议 (Strongly Recommended) 设置赎回准备金。

\subsubsection{债券赎回的会计处理\\Accounting Treatment of Loan Notes Redemption}

\begin{enumerate}
    \item \textbf{筹措资金\\Raising Funds}
    
    公司需要足够的资金来赎回债券,可以通过内部留存利润 (Retained Profits) 、发行新股 (Issuing New Shares) 或者发行新债券 (Issuing New Loan Notes) 筹措。

    \item \textbf{将要赎回的债券从债务账户中转出\\Transferring Loan Notes to be Redeemed from Debt Account}

    在赎回债券之前,需要将要赎回的债券从债券账户 (Loan Notes Account) 中转去赎回债券账户 (Loan Notes Redemption Account),让借款人知道这部分债券即将被赎回。

    这个赎回债券账户是一个临时账户,用来记录赎回的债券数量和金额,以及使用哪个方面的资金来赎回。

    \begin{mdframed}[backgroundcolor=gray!10]
    \textbf{Dr} Loan Notes

    \hspace{1.7em}\textbf{Cr} Loan Notes Redemption
    \end{mdframed}

    \item \textbf{支付赎回款项\\Payment of Redemption Amount}
    
    公司需要支付给投资者赎回款项,这部分资金可以来自公司的现金账户 (Cash Account) 或者银行账户 (Bank Account)。

    \begin{mdframed}[backgroundcolor=gray!10]
    \textbf{Dr} Loan Notes Redemption

    \hspace{1.7em}\textbf{Cr} Bank / Cash
    \end{mdframed}

    \item \textbf{计算赎回资本储备金\\Calculation of Loan Notes Redemption Reserve}
    
    如果题目要求设立债券赎回准备金 (Loan Notes Redemption Reserve),需要将等同于待赎回债券面值 (Nominal Value) 的金额从留存利润 (Retained Profits) 转入赎回准备金账户。
\end{enumerate}

\newpage

\section{股本的增加和减少\\Increase and Reduction of Share Capital}

\subsection{股本的增加\\Increase of Share Capital}

    \subsubsection{发行新股\\Issue of New Shares}

    \noindent 我们在之前的章节中已经讨论过发行新股的会计处理。发行新股是公司通过向公众出售新股来筹集资金的过程。公司可以通过发行新股来增加股本,以支持业务扩张、投资项目或偿还债务。

    \subsubsection{发行红股\\Bonus Issue / Scrip Issue}

    \noindent 红股是公司从其任何类型的储备中向现有股东发行普通股。公司不会收到任何现金,股东也无需支付现金。红股通常按面值发行,作为现金股息 (Dividends) 的替代或补充。发行红股通常使用不可分配储备 (Non-Distributable Reserves),如资本公积金 (Capital Reserve)、再估价储备金 (Revaluation Reserve) 等。如果不可分配储备不足,可以使用可分配储备,如普通储备金 (General Reserve) 和留存利润 (Retained Profits)。股东收到的红股数量通常与他们在红股发行时的持股数量成比例。

    公司发行红股的目的主要有以下几点:克服现金短缺,而不是分发现金股利;保留现金用于业务扩张;将不可分配储备金转化为股本。

    当董事会宣布红股时,会计处理如下:
    \begin{mdframed}[backgroundcolor=gray!10]
    \textbf{Dr} Reserves

    \noindent \hspace{1.7em}\textbf{Cr} Bonus Shares / Bonus Issue
    \end{mdframed}

    当发行红股时,会计处理如下:
    \begin{mdframed}[backgroundcolor=gray!10]
    \textbf{Dr} Bonus Shares / Bonus Issue

    \noindent \hspace{1.7em}\textbf{Cr} Ordinary Share Capital
    \end{mdframed}

    \subsubsection{发行附加股\\Rights Issue}

    \noindent 附加股发行是指公司以低于当前市场价格的价格向现有股东发行普通股。股东获得的普通股数量通常与他们在优先认购权股发行时的持股数量成比例。股东有两种选择:行使优先认购权购买股份,或将“权利”出售给第三方。

    附加股发行是公司筹集额外资金的另一种方式,可用于公司运营、资本支出、业务扩张、偿还借款、解决财务问题等。

    附加股并不需要像发行新股那样需要使用申请及配发账户 (Application and Allotment Account)。当公司附加股时,会计处理如下:

    \newpage
    若按照面值附加股 (At Par Value):
    \begin{mdframed}[backgroundcolor=gray!10]
    \textbf{Dr} Bank / Cash

    \noindent \hspace{1.7em}\textbf{Cr} Preference Share Capital
    \end{mdframed}

    若按溢价发行附加股 (At Premium):
    \begin{mdframed}[backgroundcolor=gray!10]
    \textbf{Dr} Bank / Cash

    \noindent \hspace{1.7em}\textbf{Cr} Preference Share Capital

    \noindent \hspace{1.7em}\textbf{Cr} Share Premium
    \end{mdframed}

\subsection{股本的减少\\Reduction of Share Capital}

\noindent 当公司的资本超过其需求或者面临严重的财务损失 (Financial Losses) 时,有限责任公司 (Limited Company) 可能需要减少其资本 (Capital)(即减少股本面值 (Par Value))。

\subsubsection{股本减少的步骤\\Reasons for Reduction of Share Capital}

\begin{enumerate}
    \item 减少实缴股本 (Paid-up Capital)。
    \item 核销累计亏损 (Accumulated Losses)。
    \item 调整资产 (Assets) 和负债 (Liabilities) 的账面价值 (Book Value)。
    \item 发行额外的新股 (New Shares) 或获取贷款资本 (Loan Capital),以继续经营和扩张。
    \item 采取积极措施赚取利润,并可能支付股息 (Dividends),等等。
\end{enumerate}

\subsubsection{削减资本的情况\\Situations for Reduction of Share Capital}

\begin{enumerate}
    \item \textbf{公司核销/减少未催缴股本\\Cancellation/Reduction of Uncalled Capital}
    
    当公司资本过剩,超过其需求时,可以核销或减少未催缴股本。例如,Alpha Bhd 发行了 50,000 股面值 1 令吉的普通股,每股已催缴 0.80 令吉。由于资本过剩,公司决定核销未催缴的 0.20 令吉。此时,每股实缴股本变为 0.80 令吉,股份总数不变。由于未催缴股本是股东尚未支付的金额,因此无需进行会计分录,只需更改股份面值即可。

    \item \textbf{公司退还盈余股本\\Return of Surplus Capital}
    
    当公司活动收缩,存在未利用的过多资金时,可以退还盈余股本。例如,Alpha Bhd 拥有 50,000 股 1 令吉的实缴普通股。由于公司业务收缩,存在未利用的过多资金。因此,公司将每股面值减少至 0.50 令吉,并将盈余现金退还给股东。此时,股本减少至 25,000 令吉,股份总数不变。

    \item \textbf{公司核销不以资产为代表的已缴股本\\Cancellation of Paid-up Capital not Represented Asset}
    
    当公司发生重大交易亏损,导致累计亏损侵蚀了实缴股本,且公司无法向股东支付股息时,可以考虑核销不以资产为代表的已缴股本。

    当公司发生重大交易亏损 (Trading Losses),导致累计亏损 (Accumulated Losses) 侵蚀了实缴股本 (Paid-up Capital),且公司无法向股东支付股息 (Dividends) 时,可以考虑核销不以资产为代表的已缴股本。

    例如,Alpha Bhd 在第一年年末的财务状况表显示,由于 40,000 令吉的重大交易亏损,100,000 令吉的实缴股本与 60,000 令吉的净资产不相等。
    
    为反映实际情况,公司决定将每股面值 (Par Value) 从 2 令吉减少到 1.20 令吉,股本相应减少 40,000 令吉,与净资产相等。股份总数 (Number of Shares) 保持不变。
    
    减少股本后的财务状况表显示,实缴股本减少以反映净资产,股东承担了核销的亏损。由于没有留存利润 (Retained Profits) 的借方余额 (Debit Balance),未来年度的利润可能会作为股息分配给股东。
    
\end{enumerate}

\subsubsection{股本减少的法定要求\\Statutory Requirements for Reduction of Share Capital}

\begin{enumerate}
    \item \textbf{公司章程 (Articles of Association) 须有规定:} 公司章程是规定公司运营的内部规则。公司只有在章程中有相关条款的情况下才能减少股本。
    \item \textbf{通过特别决议 (Special Resolution):} 减少股本会导致股本减少,这会影响股东的利益。因此,在减少股本之前,应通过特别决议获得股东的同意。
        \begin{itemize}
            \item \textbf{注意:} 当公司减少其股本和股份面值 (Par Value) 时,其法定股本 (Authorised Share Capital) 也会减少。因此,当通过减少股本的决议时,还需要通过另一项特别决议来恢复法定股本。
        \end{itemize}
    \item \textbf{获得法院批准 (Court Approval):} 法院只有在债权人 (Creditors) 的债权得到满足、解决或担保,并获得他们的同意后,才会批准减少股本。
\end{enumerate}

\subsubsection{股本减少的会计处理\\Accounting Treatment for Reduction of Share Capital}

在处理股本减少时,需要新增资本减少账户 (Capital Reduction Account)。这个账户在减少股本的过程中起到了一个中转站的作用。它记录了股本减少的金额,并根据法律规定和公司决策,将这部分资金用于核销亏损、调整资产负债表、退还给股东或其他法定用途。

\begin{enumerate}
    \item \textbf{減少已繳股本的面值\\Reduction of Par Value of Paid-up Capital}
    \begin{mdframed}[backgroundcolor=gray!10]
    \textbf{Dr} Ordinary / Preference Share Capital

    \hspace{1.7em}\textbf{Cr} Capital Reduction
    \end{mdframed}

    \newpage
    \item \textbf{核销/利用储备金\\Write-off/Utilization of Reserves}
    \begin{mdframed}[backgroundcolor=gray!10]
    \textbf{Dr} Reserves

    \hspace{1.7em}\textbf{Cr} Capital Reduction
    \end{mdframed}

    \item \textbf{核销累计亏损\\Write-off of Accumulated Losses}
    \begin{mdframed}[backgroundcolor=gray!10]
    \textbf{Dr} Retained Profits - Accumulated Losses

    \hspace{1.7em}\textbf{Cr} Capital Reduction
    \end{mdframed}

    \item \textbf{资产重估\\Revaluation of Assets}
    
    若资产价值增加,
    \begin{mdframed}[backgroundcolor=gray!10]
    \textbf{Dr} Assets

    \hspace{1.7em}\textbf{Cr} Capital Reduction
    \end{mdframed}

    若资产价值减少,
    \begin{mdframed}[backgroundcolor=gray!10]
    \textbf{Dr} Capital Reduction

    \hspace{1.7em}\textbf{Cr} Assets
    \end{mdframed}

    \item \textbf{减少负债\\Reduction of Liabilities}
    \begin{mdframed}[backgroundcolor=gray!10]
    \textbf{Dr} Liabilities

    \hspace{1.7em}\textbf{Cr} Capital Reduction
    \end{mdframed}

    \item \textbf{发行新股\\Issue of New Shares}
    
    为了偿还负债
    \begin{mdframed}[backgroundcolor=gray!10]
    \textbf{Dr} Liabilities

    \hspace{1.7em}\textbf{Cr} Ordinary / Preference Share Capital
    \end{mdframed}

    为了偿还优先股股息 (Preferred Dividends)
    \begin{mdframed}[backgroundcolor=gray!10]
    \textbf{Dr} Capital Reduction

    \hspace{1.7em}\textbf{Cr} Ordinary / Preference Share Capital
    \end{mdframed}

    \newpage
    \item \textbf{偿还削减资本所产生的费用\\Payment of Costs Arising from Reduction of Share Capital}
    
    \begin{mdframed}[backgroundcolor=gray!10]
    \textbf{Dr} Capital Reduction

    \hspace{1.7em}\textbf{Cr} Bank / Cash
    \end{mdframed}

    \item \textbf{结算股本削减账户的贷方余额\\Close Capital Reduction Account of Credit Balance}
    
    贷方余额可以用来冲销低估资产(例如,厂房和机器)或高估存货的价值。
    
    在大多数情况下,股本削减账户会在法院批准前结清,不留余额。这是因为减少股本通常是为了在法院批准前核销亏损。股本削减账户的贷方余额也可以转入资本公积金 (Capital Reserve) 下的“股本削减准备金” (Capital Reduction Reserve) 账户。
    \begin{mdframed}[backgroundcolor=gray!10]
    \textbf{Dr} Capital Reduction

    \hspace{1.7em}\textbf{Cr} Capital Reduction Reserve
    \end{mdframed}

    \item \textbf{恢复法定股本\\Restore Authorised Share Capital}
    
    这个步骤不需要会计分录,只需要在财务报表中的法定资本 (Authorised Share Capital) 项下记录即可。

\end{enumerate}

\newpage

\section{有限公司的创设合并和吸收合并\\Amalgamation and Absorption of Limited Companies}

\subsection{有限公司合并的种类\\Types of Amalgamation of Limited Companies}

\subsubsection{有限公司的创设合并\\Amalgamation of Limited Companies}

\noindent 当两家或多家公司(即出售公司)合并时,会成立一家新公司(即购买公司)来接管出售公司的资产和负债。合并后,新公司成立,原出售公司将被清盘 (Liquidated)。


\subsubsection{有限公司的吸收合并\\Absorption of Limited Companies}

\noindent 一家现有公司(即购买公司)接管一家或多家公司(即出售公司)的资产和负债。吸收合并后,购买公司继续存在,而出售公司将被清盘。

\subsection{合并的会计处理\\Accounting Treatment for Amalgamation and Absorption}

\noindent 企业合并与收购的会计处理相似。在合并过程中,需要将出售公司的资产和负债转移到购买公司的资产和负债中。以下是合并的会计处理:

\subsubsection{对于买方公司\\For Buyer Company}

\begin{enumerate}

    \item \textbf{筹措收购资金\\Raising Funds for Acquisition}

    \item \textbf{资产和负债的重估\\Revaluation of Assets and Liabilities}
    
    购买公司往往需要评估出售公司的资产和负债,以确定其公允价值 (Fair Value)。

    \item \textbf{记录出售公司的资产和负债的账面价值\\Recording Book Value of Assets and Liabilities of Vendor Company}
    \begin{mdframed}[backgroundcolor=gray!10]
    \textbf{Dr} Assets of Vendor Company

    \hspace{1.7em}\textbf{Cr} Business Purchase
    \end{mdframed}
    \begin{mdframed}[backgroundcolor=gray!10]
    \textbf{Dr} Business Purchase

    \hspace{1.7em}\textbf{Cr} Liabilities of Vendor Company
    \end{mdframed}

    \item \textbf{以卖方公司的名义支付清盘费用\\Payment of Liquidation Expenses on Behalf of Vendor Company}
    这个清盘费用是收购价 (Purchase Price) 的一部分,是除了购买对价 (Purchase Consideration) 之外的额外费用。
    \begin{mdframed}[backgroundcolor=gray!10]
    \textbf{Dr} Business Purchase - Liquidation Expenses

    \hspace{1.7em}\textbf{Cr} Bank / Cash
    \end{mdframed}

    \item \textbf{记录购买对价\\Recording Purchase Consideration}
    
    \begin{mdframed}[backgroundcolor=gray!10]
    \textbf{Dr} Business Purchase

    \hspace{1.7em}\textbf{Cr} Vendor 
    \end{mdframed}

    \item \textbf{记录商誉或廉价购买获利\\Recording Goodwill or Gain on Bargain Purchase}
    
    商誉/(廉价购买获利) = 购买对价 + 清盘费用(若由购买公司支付) - 净资产公允价值 (Fair Value of Net Assets)

    净资产公允价值 (Fair Value of Net Assets) = 资产公允价值 (Fair Value of Assets) - 负债公允价值 (Fair Value of Liabilities)

    \begin{mdframed}[backgroundcolor=gray!10]
    \textbf{Dr} Goodwill

    \hspace{1.7em}\textbf{Cr} Business Purchase
    \end{mdframed}
    或者
    \begin{mdframed}[backgroundcolor=gray!10]
    \textbf{Dr} Business Purchase

    \hspace{1.7em}\textbf{Cr} Retained Profits - Gain on Bargain Purchase
    \end{mdframed}

    \item \textbf{支付购买费用\\Payment of Purchase Price}
    
    购买费用可以通过现金 (Cash)、股票 (Shares) 或者债券 (Loan Notes) 来支付。

    \begin{mdframed}[backgroundcolor=gray!10]
    \textbf{Dr} Vendor

    \hspace{1.7em}\textbf{Cr} Bank / Cash / Loan Notes / Share Capital / Share Premium
    \end{mdframed}

    \item \textbf{支付前期费用/成立费用\\Payment of Preliminary Expenses/Incorporation/Formation Expenses}

    \begin{mdframed}[backgroundcolor=gray!10]
    \textbf{Dr} Preliminary Expenses

    \hspace{1.7em}\textbf{Cr} Bank / Cash
    \end{mdframed}
\end{enumerate}

\subsubsection{对于卖方公司\\For Vendor Company}

\begin{enumerate}
    \item \textbf{将持股人资金转入持股人账户\\Transfer of Shareholders' Funds to Shareholders' Account}
    \begin{mdframed}[backgroundcolor=gray!10]
    \textbf{Dr} Share capital

    \textbf{Dr} Reserves

    \hspace{1.7em}\textbf{Cr} Shareholders
    \end{mdframed}

    \item \textbf{将所有资产转入清盘账户\\Transfer of All Assets to Liquidation Account}
    \begin{mdframed}[backgroundcolor=gray!10]
    \textbf{Dr} Liquidation

    \hspace{1.7em}\textbf{Cr} Assets
    \end{mdframed}

    \item \textbf{关闭所有负债账户\\Closing of All Liabilities Accounts}
    
    若债务由购买公司承担,需要将负债的账面价值转入清盘账户。
    \begin{mdframed}[backgroundcolor=gray!10]
    \textbf{Dr} Liabilities

    \hspace{1.7em}\textbf{Cr} Liquidation
    \end{mdframed}

    若债务被还清,则需将所得债务折扣转入清盘账户。
    \begin{mdframed}[backgroundcolor=gray!10]
    \textbf{Dr} Liabilities

    \hspace{1.7em}\textbf{Cr} Bank / Cash
    \end{mdframed}
    \begin{mdframed}[backgroundcolor=gray!10]
    \textbf{Dr} Liabilities

    \hspace{1.7em}\textbf{Cr} Liquidation - Discounts Received
    \end{mdframed}

    \item \textbf{支付清盘费用\\Payment of Liquidation Expenses}
    \begin{mdframed}[backgroundcolor=gray!10]
    \textbf{Dr} Liquidation - Liquidation Expenses

    \hspace{1.7em}\textbf{Cr} Bank / Cash
    \end{mdframed}
    
    \newpage
    \item \textbf{记录收购价\\Recording Purchase Price}
    \begin{mdframed}[backgroundcolor=gray!10]
    \textbf{Dr} Buyer

    \hspace{1.7em}\textbf{Cr} Liquidation
    \end{mdframed}

    \item \textbf{将清盘账户的余额转入持股人账户\\Transfer of Balance of Liquidation Account to Shareholders' Account}
    
    清盘利润/(亏损) = 收购价 - 清盘费用(若由购卖方公司支付) - 净资产账面价值 

    \begin{mdframed}[backgroundcolor=gray!10]
    \textbf{Dr} Liquidation - Gain on Liquidation

    \hspace{1.7em}\textbf{Cr} Shareholders
    \end{mdframed}
    或者
    \begin{mdframed}[backgroundcolor=gray!10]
    \textbf{Dr} Shareholders

    \hspace{1.7em}\textbf{Cr} Liquidation - Loss on Liquidation
    \end{mdframed}

    \item \textbf{将买方公司支付的现金/股票/债券分摊给持股人\\Distribution of Cash/Shares/Loan Notes Paid by Buyer Company to Shareholders}
    
    \begin{mdframed}[backgroundcolor=gray!10]
    \textbf{Dr} Shareholders

    \hspace{1.7em}\textbf{Cr} Buyer - Bank / Cash / Loan Notes / Share Capital / Share Premium
    \end{mdframed}
    
\end{enumerate}

\end{document}
